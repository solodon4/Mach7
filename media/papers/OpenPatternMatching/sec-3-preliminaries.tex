\section{\Cpp{} Concepts} %%%%%%%%%%%%%%%%%%%%%%%%%%%%%%%%%%%%%%%%%%%%%%%%%%%%%%
\label{sec:prelim}

A constraint is an unconstrained, constexpr function template that takes no 
arguments and returns bool. constexpr functions are syntactically restricted 
functions that can be evaluated at compile time. Consider for example the 
definition of \code{Equality_comparable} constraint:

\begin{lstlisting}
template<typename T>
constexpr bool Equality_comparable()
{
    return requires (T a, T b) {
                bool = {a == b};
                bool = {a != b};
    };
}
\end{lstlisting}

\noindent
The \code{requires} clause introduces a set of syntactic constraints that the 
type \code{T} must satisfy. For convenience, it may introduce some formal 
parameters to avoid using \code{declval<T>}. Here is an incomplete list of some 
commonly-used syntactic requirements:

\begin{lstlisting}
e;        // e is a valid expression
T = {e}   // decltype(e) is convertible to T
T == {e}; // decltype(e) is the same as T
typename T::X; // X is an associated type of T
\end{lstlisting}

The rest of this section is to be completed. Meanwhile, please refer to Sutton 
et al~\cite{N3580} for the details on the syntax.

%TODO: Describe everything we use from concepts:
%
%\begin{itemize}
%\item concept definition
%\item refinement of another concept
%\item associated functions
%\item associated types
%\item associated templates (we have those, e.g. accepted\_type\_for)
%\item default implementation of an associated function
%\item requires
%\item concept-based overloading
%\item decltype
%\item std::declval
%\item template aliases
%\end{itemize}
%
%The following concepts can be used as examples since we later refer to them in 
%the paper: \code{CopyConstructible}, \code{Convertible}.
