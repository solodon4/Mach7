\section{Related Work} %%%%%%%%%%%%%%%%%%%%%%%%%%%%%%%%%%%%%%%%%%%%%%%%%%%%%%%%%
\label{sec:rw}

\emph{Lookup caches} have been long used to reduce the overhead of 
dynamically-bound message passing in Smalltalk~\cite{UngarPatterson83}. 
\emph{Inline caching} improves on that by noting that the type of the receiver at a given 
call site rarely varies so that the call instruction can be speculatively 
modified to jump directly to a previously looked up method~\cite{Deutsch84}. 
In this case, the method must ensure that the type of the receiver has 
not changed and redirect the call to generic lookup otherwise. The effects of 
inline caching on modern architectures can be seen through hardware call target 
prediction, which in our case is exemplified by repetitive benchmark: both 
virtual function calls and the underlying jump-table implementation of the 
\code{Match}-statement are about twice as fast as usual.

\emph{Polymorphic Inline Caches}~\cite{Holzle:Chambers:Ungar:91} generalize the 
idea of inline caches further by building a decision tree in the method prologue 
that caches all lookup results. The main difference of this approach from our 
work is that it requires code generation at run time, while we do not require
re-compilation, re-linking or any computations in case of dynamic linking.
The reason for this is the difference in the initial setting: they map an
arbitrary number of receiver types to an arbitrary number of implementations, while 
we map an arbitrary number of receivers to a fixed number of jump 
targets. This lets us generate code at compile time that incorporates both the 
initial and memoized execution.

\emph{Extensible Visitors with Defaults}~\cite[\textsection 4.2]{Zenger:2001} 
offer a solution to the extensibility problem of visitors. The visitation interface 
hierarchy can easily be grown linearly, but independent extensions by different  
programmers require manual coordination. In addition to the double dispatch, the 
solution incurs two additional virtual calls and a dynamic cast for each 
level of visitor extension. The solution is simpler with virtual inheritance, 
which adds even more indirections.

L\"{o}h and Hinze proposed to extend Haskell's type system with open data types 
and open functions~\cite{LohHinze2006}. The solution allows top-level data types 
and functions to be marked as open with concrete variants and overloads defined 
anywhere in the program. The semantics of open extension is given by 
transformation into a single module, which assumes a whole-program view and thus 
is not an open solution by our definition. Besides, open data types are extensible but not 
hierarchical, which avoids the problems discussed here.

Andrew Kennedy et al~\cite{GADTOOP05} considered encoding of generalized 
algebraic data types~\cite{SPJ06} using visitor design patterns in C\#.  That 
translation made essential use of generic methods, the equivalent of ``virtual 
function template'' (as they would be called if such functionalities existed in 
C++) to handle some of the open set aspects of GADTs.

Polymorphic variants in OCaml~\cite{garrigue-98} allow the addition of new 
variants, while defining subtyping on them. The subtyping, however, is not 
defined between the variants, but between combinations of them. 
This maintains disjointness between values from different variants and makes an 
important distinction between \emph{extensible sum types} like polymorphic 
variants and \emph{extensible hierarchical sum types} like classes. Our 
memoization device can be used to implement pattern matching on polymorphic 
variants as well.

\emph{Tom} is a pattern-matching compiler that can be used together with Java, C or 
Eiffel to bring a common pattern matching and term rewriting syntax into the 
languages~\cite{Moreau:2003}. Its type patterns and \code{\%match}
statement can be used as a type switch, however, their implementation is based 
on decision trees and an \code{instanceof}-like predicate.

Pattern matching in Scala~\cite{Scala2nd} also supports type switching through 
type patterns. The language supports extensible and hierarchical data types, but
their handling in a type switching constructs varies. Sealed classes are handled 
with an efficient switch over all tags, while extensible classes are  
handled with a combination of an \code{InstanceOf} operator and a decision 
tree~\cite{EmirThesis}.