%\documentclass[preprint]{sigplanconf}
\documentclass{sigplanconf}

\usepackage{amssymb}
\usepackage{amsthm}
\usepackage{breakurl}             % Not needed if you use pdflatex only.
\usepackage{color}
\usepackage{epsfig}
\usepackage{esvect}
\usepackage{listings}
\usepackage{mathpartir}
\usepackage{MnSymbol}
\usepackage{multirow}
\usepackage{rotating}

\lstdefinestyle{Caml}{language=Caml,%
  literate={when}{{{\bf when}}}4
}

\lstdefinestyle{C++}{language=C++,%
showstringspaces=false,
  columns=fullflexible,
  escapechar=@,
  basicstyle=\sffamily,
%  commentstyle=\rmfamily\itshape,
  moredelim=**[is][\color{white}]{~}{~},
  literate={[<]}{{\textless}}1      {[>]}{{\textgreater}}1 %
           {<}{{$\langle$}}1        {>}{{$\rangle$}}1 %
           {<=}{{$\leq$}}1          {>=}{{$\geq$}}1          
           {==}{{$==$}}2            {!=}{{$\neq$}}1 %
           {=>}{{$\Rightarrow\;$}}1 {->}{{$\rightarrow{}$}}1 %
           {<:}{{$\subtype{}\ $}}1  {<-}{{$\leftarrow$}}1 %
           {s1;}{{$s_1$;}}3 {s2;}{{$s_2$;}}3 {s3;}{{$s_3$;}}3 {s4;}{{$s_4$;}}3 {s5;}{{$s_5$;}}3 {s6;}{{$s_6$;}}3 {s7;}{{$s_7$;}}3 {sn;}{{$s_n$;}}3 {si;}{{$s_i$;}}3%
           {P1}{{$P_1$}}2 {P2}{{$P_2$}}2 {P3}{{$P_3$}}2 {P4}{{$P_4$}}2 {P5}{{$P_5$}}2 {P6}{{$P_6$}}2 {P7}{{$P_7$}}2 {Pn}{{$P_n$}}2 {Pi}{{$P_i$}}2%
           {D1}{{$D_1$}}2 {D2}{{$D_2$}}2 {D3}{{$D_3$}}2 {D4}{{$D_4$}}2 {D5}{{$D_5$}}2 {D6}{{$D_6$}}2 {D7}{{$D_7$}}2 {Dn}{{$D_n$}}2 {Di}{{$D_i$}}2%
           {e1}{{$e_1$}}2 {e2}{{$e_2$}}2 {e3}{{$e_3$}}2 {e4}{{$e_4$}}2%
           {E1}{{$E_1$}}2 {E2}{{$E_2$}}2 {E3}{{$E_3$}}2 {E4}{{$E_4$}}2%
           {m_e1}{{$m\_e_1$}}4 {m_e2}{{$m\_e_2$}}4 {m_e3}{{$m\_e_3$}}4 {m_e4}{{$m\_e_4$}}4%
           {Divide}{{Divide}}6 %
           {Match}{{\emph{Match}}}5 %
           {Case}{{\emph{Case}}}4 %
           {Que}{{\emph{Que}}}3 %
           {Otherwise}{{\emph{Otherwise}}}9 %
           {EndMatch}{{\emph{EndMatch}}}8 %
           {CM}{{\emph{CM}}}2 {KS}{{\emph{KS}}}2 {KV}{{\emph{KV}}}2 
}
\lstset{style=C++}
\DeclareRobustCommand{\Cpp}{C\texttt{++}}
\DeclareRobustCommand{\code}[1]{{\lstinline[breaklines=false,escapechar=@]{#1}}}
\DeclareRobustCommand{\codebr}[1]{{\lstinline[breaklines=true]{#1}}}
\DeclareRobustCommand{\codehaskell}[1]{{\lstinline[breaklines=false,language=Haskell]{#1}}}
\DeclareRobustCommand{\codeocaml}[1]{{\lstinline[breaklines=false,language=Caml]{#1}}}
\DeclareRobustCommand{\concept}[1]{{\small\textsc{#1}}}
\newcommand{\exclude}[1]{}
\newcommand{\halfline}{\vspace{-1.5ex}}

\newtheorem{lemma}{Lemma}
\newtheorem{theorem}{Theorem}
\newtheorem{corollary}{Corollary}

%% grammar commands
\newcommand{\Rule}[1]{{\rmfamily\itshape{#1}}}
\newcommand{\Alt}{\ensuremath{|}}
\newcommand{\is}{$::=$}
\newcommand{\subtype}{\textless:}
\newcommand{\evals}{\Rightarrow}
\newcommand{\evalspp}{\Rightarrow^+}
\newcommand{\DynCast}[2]{\ensuremath{dc\langle{#1}\rangle({#2})}}
\newcommand{\nullptr}{\ensuremath{\bot}}

\newcommand{\f}[1]{{ {\bf \textcolor{blue}{#1\%}}}}
\newcommand{\s}[1]{{ {\em \textcolor{cyan}{#1\%}}}}
\newcommand{\n}[1]{{ {\bf ~ ~ ~ ~ }}}
\newcommand{\Opn}{{\scriptsize {\bf Open}}}
\newcommand{\Cls}{{\scriptsize {\bf Tag}}}
\newcommand{\Unn}{{\scriptsize {\bf Union}}}

%\newcommand{\gwNGPp}{\n{}}
%\newcommand{\gwNGKp}{\n{}}
 \newcommand{\gwNGUp}{\n{}}
%\newcommand{\gwNSPp}{\n{}}
%\newcommand{\gwNSKp}{\n{}}
 \newcommand{\gwNSUp}{\n{}}
%\newcommand{\vwNGPp}{\n{}}
%\newcommand{\vwNGKp}{\n{}}
 \newcommand{\vwNGUp}{\n{}}
%\newcommand{\vwNSPp}{\n{}}
%\newcommand{\vwNSKp}{\n{}}
 \newcommand{\vwNSUp}{\n{}}
%\newcommand{\vxNGPp}{\n{}}
%\newcommand{\vxNGKp}{\n{}}
 \newcommand{\vxNGUp}{\n{}}
%\newcommand{\vxNSPp}{\n{}}
%\newcommand{\vxNSKp}{\n{}}
 \newcommand{\vxNSUp}{\n{}}

%\newcommand{\gwNGPq}{\n{}}
%\newcommand{\gwNGKq}{\n{}}
 \newcommand{\gwNGUq}{\n{}}
%\newcommand{\gwNSPq}{\n{}}
%\newcommand{\gwNSKq}{\n{}}
 \newcommand{\gwNSUq}{\n{}}
%\newcommand{\vwNGPq}{\n{}}
%\newcommand{\vwNGKq}{\n{}}
 \newcommand{\vwNGUq}{\n{}}
%\newcommand{\vwNSPq}{\n{}}
%\newcommand{\vwNSKq}{\n{}}
 \newcommand{\vwNSUq}{\n{}}
%\newcommand{\vxNGPq}{\n{}}
%\newcommand{\vxNGKq}{\n{}}
 \newcommand{\vxNGUq}{\n{}}
%\newcommand{\vxNSPq}{\n{}}
%\newcommand{\vxNSKq}{\n{}}
 \newcommand{\vxNSUq}{\n{}}

%\newcommand{\gwNGPn}{\n{}}
%\newcommand{\gwNGKn}{\n{}}
 \newcommand{\gwNGUn}{\n{}}
%\newcommand{\gwNSPn}{\n{}}
%\newcommand{\gwNSKn}{\n{}}
 \newcommand{\gwNSUn}{\n{}}
%\newcommand{\vwNGPn}{\n{}}
%\newcommand{\vwNGKn}{\n{}}
 \newcommand{\vwNGUn}{\n{}}
%\newcommand{\vwNSPn}{\n{}}
%\newcommand{\vwNSKn}{\n{}}
 \newcommand{\vwNSUn}{\n{}}
%\newcommand{\vxNGPn}{\n{}}
%\newcommand{\vxNGKn}{\n{}}
 \newcommand{\vxNGUn}{\n{}}
%\newcommand{\vxNSPn}{\n{}}
%\newcommand{\vxNSKn}{\n{}}
 \newcommand{\vxNSUn}{\n{}}


%\newcommand{\gwYGPp}{\n{}}
% \newcommand{\gwYGKp}{\n{}}
 \newcommand{\gwYGUp}{\n{}}
%\newcommand{\gwYSPp}{\n{}}
% \newcommand{\gwYSKp}{\n{}}
 \newcommand{\gwYSUp}{\n{}}
%\newcommand{\vwYGPp}{\n{}}
% \newcommand{\vwYGKp}{\n{}}
 \newcommand{\vwYGUp}{\n{}}
%\newcommand{\vwYSPp}{\n{}}
% \newcommand{\vwYSKp}{\n{}}
 \newcommand{\vwYSUp}{\n{}}
%\newcommand{\vxYGPp}{\n{}}
% \newcommand{\vxYGKp}{\n{}}
 \newcommand{\vxYGUp}{\n{}}
%\newcommand{\vxYSPp}{\n{}}
% \newcommand{\vxYSKp}{\n{}}
 \newcommand{\vxYSUp}{\n{}}

%\newcommand{\gwYGPq}{\n{}}
% \newcommand{\gwYGKq}{\n{}}
 \newcommand{\gwYGUq}{\n{}}
%\newcommand{\gwYSPq}{\n{}}
% \newcommand{\gwYSKq}{\n{}}
 \newcommand{\gwYSUq}{\n{}}
%\newcommand{\vwYGPq}{\n{}}
% \newcommand{\vwYGKq}{\n{}}
 \newcommand{\vwYGUq}{\n{}}
%\newcommand{\vwYSPq}{\n{}}
% \newcommand{\vwYSKq}{\n{}}
 \newcommand{\vwYSUq}{\n{}}
%\newcommand{\vxYGPq}{\n{}}
% \newcommand{\vxYGKq}{\n{}}
 \newcommand{\vxYGUq}{\n{}}
%\newcommand{\vxYSPq}{\n{}}
% \newcommand{\vxYSKq}{\n{}}
 \newcommand{\vxYSUq}{\n{}}

%\newcommand{\gwYGPn}{\n{}}
% \newcommand{\gwYGKn}{\n{}}
 \newcommand{\gwYGUn}{\n{}}
%\newcommand{\gwYSPn}{\n{}}
% \newcommand{\gwYSKn}{\n{}}
 \newcommand{\gwYSUn}{\n{}}
%\newcommand{\vwYGPn}{\n{}}
% \newcommand{\vwYGKn}{\n{}}
 \newcommand{\vwYGUn}{\n{}}
%\newcommand{\vwYSPn}{\n{}}
% \newcommand{\vwYSKn}{\n{}}
 \newcommand{\vwYSUn}{\n{}}
%\newcommand{\vxYGPn}{\n{}}
% \newcommand{\vxYGKn}{\n{}}
 \newcommand{\vxYGUn}{\n{}}
%\newcommand{\vxYSPn}{\n{}}
% \newcommand{\vxYSKn}{\n{}}
 \newcommand{\vxYSUn}{\n{}}

 \newcommand{\GwNGPp}{\n{}}
 \newcommand{\GwNGKp}{\n{}}
 \newcommand{\GwNGUp}{\n{}}
 \newcommand{\GwNSPp}{\n{}}
 \newcommand{\GwNSKp}{\n{}}
 \newcommand{\GwNSUp}{\n{}}
%\newcommand{\VwNGPp}{\n{}}
%\newcommand{\VwNGKp}{\n{}}
 \newcommand{\VwNGUp}{\n{}}
%\newcommand{\VwNSPp}{\n{}}
%\newcommand{\VwNSKp}{\n{}}
 \newcommand{\VwNSUp}{\n{}}
%\newcommand{\VxNGPp}{\n{}}
%\newcommand{\VxNGKp}{\n{}}
 \newcommand{\VxNGUp}{\n{}}
%\newcommand{\VxNSPp}{\n{}}
%\newcommand{\VxNSKp}{\n{}}
 \newcommand{\VxNSUp}{\n{}}

 \newcommand{\GwNGPq}{\n{}}
 \newcommand{\GwNGKq}{\n{}}
 \newcommand{\GwNGUq}{\n{}}
 \newcommand{\GwNSPq}{\n{}}
 \newcommand{\GwNSKq}{\n{}}
 \newcommand{\GwNSUq}{\n{}}
%\newcommand{\VwNGPq}{\n{}}
%\newcommand{\VwNGKq}{\n{}}
 \newcommand{\VwNGUq}{\n{}}
%\newcommand{\VwNSPq}{\n{}}
%\newcommand{\VwNSKq}{\n{}}
 \newcommand{\VwNSUq}{\n{}}
%\newcommand{\VxNGPq}{\n{}}
%\newcommand{\VxNGKq}{\n{}}
 \newcommand{\VxNGUq}{\n{}}
%\newcommand{\VxNSPq}{\n{}}
%\newcommand{\VxNSKq}{\n{}}
 \newcommand{\VxNSUq}{\n{}}

 \newcommand{\GwNGPn}{\n{}}
 \newcommand{\GwNGKn}{\n{}}
 \newcommand{\GwNGUn}{\n{}}
 \newcommand{\GwNSPn}{\n{}}
 \newcommand{\GwNSKn}{\n{}}
 \newcommand{\GwNSUn}{\n{}}
%\newcommand{\VwNGPn}{\n{}}
%\newcommand{\VwNGKn}{\n{}}
 \newcommand{\VwNGUn}{\n{}}
%\newcommand{\VwNSPn}{\n{}}
%\newcommand{\VwNSKn}{\n{}}
 \newcommand{\VwNSUn}{\n{}}
%\newcommand{\VxNGPn}{\n{}}
%\newcommand{\VxNGKn}{\n{}}
 \newcommand{\VxNGUn}{\n{}}
%\newcommand{\VxNSPn}{\n{}}
%\newcommand{\VxNSKn}{\n{}}
 \newcommand{\VxNSUn}{\n{}}


 \newcommand{\GwYGPp}{\n{}}
 \newcommand{\GwYGKp}{\n{}}
 \newcommand{\GwYGUp}{\n{}}
 \newcommand{\GwYSPp}{\n{}}
 \newcommand{\GwYSKp}{\n{}}
 \newcommand{\GwYSUp}{\n{}}
%\newcommand{\VwYGPp}{\n{}}
% \newcommand{\VwYGKp}{\n{}}
 \newcommand{\VwYGUp}{\n{}}
%\newcommand{\VwYSPp}{\n{}}
% \newcommand{\VwYSKp}{\n{}}
 \newcommand{\VwYSUp}{\n{}}
%\newcommand{\VxYGPp}{\n{}}
% \newcommand{\VxYGKp}{\n{}}
 \newcommand{\VxYGUp}{\n{}}
%\newcommand{\VxYSPp}{\n{}}
% \newcommand{\VxYSKp}{\n{}}
 \newcommand{\VxYSUp}{\n{}}

 \newcommand{\GwYGPq}{\n{}}
 \newcommand{\GwYGKq}{\n{}}
 \newcommand{\GwYGUq}{\n{}}
 \newcommand{\GwYSPq}{\n{}}
 \newcommand{\GwYSKq}{\n{}}
 \newcommand{\GwYSUq}{\n{}}
%\newcommand{\VwYGPq}{\n{}}
% \newcommand{\VwYGKq}{\n{}}
 \newcommand{\VwYGUq}{\n{}}
%\newcommand{\VwYSPq}{\n{}}
% \newcommand{\VwYSKq}{\n{}}
 \newcommand{\VwYSUq}{\n{}}
%\newcommand{\VxYGPq}{\n{}}
% \newcommand{\VxYGKq}{\n{}}
 \newcommand{\VxYGUq}{\n{}}
%\newcommand{\VxYSPq}{\n{}}
% \newcommand{\VxYSKq}{\n{}}
 \newcommand{\VxYSUq}{\n{}}

 \newcommand{\GwYGPn}{\n{}}
 \newcommand{\GwYGKn}{\n{}}
 \newcommand{\GwYGUn}{\n{}}
 \newcommand{\GwYSPn}{\n{}}
 \newcommand{\GwYSKn}{\n{}}
 \newcommand{\GwYSUn}{\n{}}
%\newcommand{\VwYGPn}{\n{}}
% \newcommand{\VwYGKn}{\n{}}
 \newcommand{\VwYGUn}{\n{}}
%\newcommand{\VwYSPn}{\n{}}
% \newcommand{\VwYSKn}{\n{}}
 \newcommand{\VwYSUn}{\n{}}
%\newcommand{\VxYGPn}{\n{}}
% \newcommand{\VxYGKn}{\n{}}
 \newcommand{\VxYGUn}{\n{}}
%\newcommand{\VxYSPn}{\n{}}
% \newcommand{\VxYSKn}{\n{}}
 \newcommand{\VxYSUn}{\n{}}

% This file defines variables with performance numbers for the table in Evaluation section
% Data from 2011-08-30 
\newcommand{\vwYGKp}{\s{3}}
\newcommand{\vwYGKn}{\s{8}}
\newcommand{\vwYGKq}{\s{11}}
\newcommand{\vwYGPp}{\f{10}}
\newcommand{\vwYGPn}{\f{14}}
\newcommand{\vwYGPq}{\s{0}}
\newcommand{\vwYSKp}{\s{7}}
\newcommand{\vwYSKn}{\s{7}}
\newcommand{\vwYSKq}{\s{10}}
\newcommand{\vwYSPp}{\f{10}}
\newcommand{\vwYSPn}{\f{14}}
\newcommand{\vwYSPq}{\s{0}}
\newcommand{\vwNGKp}{\f{35}}
\newcommand{\vwNGKn}{\s{6}}
\newcommand{\vwNGKq}{\s{5}}
\newcommand{\vwNGPp}{\f{1}}
\newcommand{\vwNGPn}{\s{1}}
\newcommand{\vwNGPq}{\s{10}}
\newcommand{\vwNSKp}{\f{133}}
\newcommand{\vwNSKn}{\f{25}}
\newcommand{\vwNSKq}{\f{59}}
\newcommand{\vwNSPp}{\f{1}}
\newcommand{\vwNSPn}{\s{1}}
\newcommand{\vwNSPq}{\s{8}}

\newcommand{\vxYGKp}{\s{61}}
\newcommand{\vxYGKn}{\s{24}}
\newcommand{\vxYGKq}{\s{25}}
\newcommand{\vxYGPp}{\s{24}}
\newcommand{\vxYGPn}{\s{24}}
\newcommand{\vxYGPq}{\s{36}}
\newcommand{\vxYSKp}{\s{79}}
\newcommand{\vxYSKn}{\s{25}}
\newcommand{\vxYSKq}{\s{35}}
\newcommand{\vxYSPp}{\s{9}}
\newcommand{\vxYSPn}{\s{23}}
\newcommand{\vxYSPq}{\f{133}}
\newcommand{\vxNGKp}{\s{8}}
\newcommand{\vxNGKn}{\s{5}}
\newcommand{\vxNGKq}{\s{0}}
\newcommand{\vxNGPp}{\s{33}}
\newcommand{\vxNGPn}{\s{47}}
\newcommand{\vxNGPq}{\s{43}}
\newcommand{\vxNSKp}{\f{38}}
\newcommand{\vxNSKn}{\f{12}}
\newcommand{\vxNSKq}{\f{3}}
\newcommand{\vxNSPp}{\s{27}}
\newcommand{\vxNSPn}{\s{44}}
\newcommand{\vxNSPq}{\s{45}}

\newcommand{\gwYGKp}{\f{88}}
\newcommand{\gwYGKn}{\f{32}}
\newcommand{\gwYGKq}{\f{250}}
\newcommand{\gwYGPp}{\f{67}}
\newcommand{\gwYGPn}{\f{28}}
\newcommand{\gwYGPq}{\f{87}}
\newcommand{\gwYSKp}{\f{79}}
\newcommand{\gwYSKn}{\f{31}}
\newcommand{\gwYSKq}{\f{259}}
\newcommand{\gwYSPp}{\f{67}}
\newcommand{\gwYSPn}{\f{27}}
\newcommand{\gwYSPq}{\f{90}}
\newcommand{\gwNGKp}{\f{116}}
\newcommand{\gwNGKn}{\f{29}}
\newcommand{\gwNGKq}{\f{43}}
\newcommand{\gwNGPp}{\f{55}}
\newcommand{\gwNGPn}{\s{0}}
\newcommand{\gwNGPq}{\f{1}}
\newcommand{\gwNSKp}{\f{216}}
\newcommand{\gwNSKn}{\f{542}}
\newcommand{\gwNSKq}{\f{520}}
\newcommand{\gwNSPp}{\f{55}}
\newcommand{\gwNSPn}{\f{1}}
\newcommand{\gwNSPq}{\f{3}}

\newcommand{\VwYGKp}{\f{16}}
\newcommand{\VwYGKn}{\f{11}}
\newcommand{\VwYGKq}{\f{168}}
\newcommand{\VwYGPp}{\f{10}}
\newcommand{\VwYGPn}{\f{19}}
\newcommand{\VwYGPq}{\f{153}}
\newcommand{\VwYSKp}{\f{31}}
\newcommand{\VwYSKn}{\f{24}}
\newcommand{\VwYSKq}{\f{185}}
\newcommand{\VwYSPp}{\f{10}}
\newcommand{\VwYSPn}{\f{18}}
\newcommand{\VwYSPq}{\f{153}}
\newcommand{\VwNGKp}{\f{61}}
\newcommand{\VwNGKn}{\f{18}}
\newcommand{\VwNGKq}{\f{13}}
\newcommand{\VwNGPp}{\f{4}}
\newcommand{\VwNGPn}{\s{17}}
\newcommand{\VwNGPq}{\s{9}}
\newcommand{\VwNSKp}{\f{124}}
\newcommand{\VwNSKn}{\f{43}}
\newcommand{\VwNSKq}{\f{34}}
\newcommand{\VwNSPp}{\f{4}}
\newcommand{\VwNSPn}{\s{18}}
\newcommand{\VwNSPq}{\f{3}}
               
\newcommand{\VxYGKp}{\s{5}}
\newcommand{\VxYGKn}{\s{2}}
\newcommand{\VxYGKq}{\f{132}}
\newcommand{\VxYGPp}{\s{5}}
\newcommand{\VxYGPn}{\s{5}}
\newcommand{\VxYGPq}{\f{130}}
\newcommand{\VxYSKp}{\s{9}}
\newcommand{\VxYSKn}{\s{10}}
\newcommand{\VxYSKq}{\f{118}}
\newcommand{\VxYSPp}{\s{6}}
\newcommand{\VxYSPn}{\s{6}}
\newcommand{\VxYSPq}{\f{145}}
\newcommand{\VxNGKp}{\f{20}}
\newcommand{\VxNGKn}{\f{7}}
\newcommand{\VxNGKq}{\f{34}}
\newcommand{\VxNGPp}{\s{14}}
\newcommand{\VxNGPn}{\s{27}}
\newcommand{\VxNGPq}{\f{2}}
\newcommand{\VxNSKp}{\f{47}}
\newcommand{\VxNSKn}{\f{16}}
\newcommand{\VxNSKq}{\f{14}}
\newcommand{\VxNSPp}{\s{0}}
\newcommand{\VxNSPn}{\s{27}}
\newcommand{\VxNSPq}{\f{1}}

% This file defines variables with performance numbers for the table in Evaluation section
% Data from 2011-11-04 collected on Sierra for Linux under g++ (GCC) 4.4.5 20101112 (Red Hat 4.4.5-2)

\newcommand{\glYGKp}{\f{45}}
\newcommand{\glYGKn}{\f{82}}
\newcommand{\glYGKq}{\f{77}}
\newcommand{\glYGPp}{\f{36}}
\newcommand{\glYGPn}{\f{76}}
\newcommand{\glYGPq}{\f{53}}
\newcommand{\glYSKp}{\f{53}}
\newcommand{\glYSKn}{\f{88}}
\newcommand{\glYSKq}{\f{86}}
\newcommand{\glYSPp}{\f{33}}
\newcommand{\glYSPn}{\f{78}}
\newcommand{\glYSPq}{\f{55}}
\newcommand{\glNGKp}{\f{54}}
\newcommand{\glNGKn}{\f{97}}
\newcommand{\glNGKq}{\f{109}}
\newcommand{\glNGPp}{\f{19}}
\newcommand{\glNGPn}{\f{57}}
\newcommand{\glNGPq}{\f{62}}
\newcommand{\glNSKp}{\f{124}}
\newcommand{\glNSKn}{\f{603}}
\newcommand{\glNSKq}{\f{640}}
\newcommand{\glNSPp}{\f{16}}
\newcommand{\glNSPn}{\f{56}}
\newcommand{\glNSPq}{\f{56}}


\newsavebox{\sembox}
\newlength{\semwidth}
\newlength{\boxwidth}

\newcommand{\Sem}[1]{%
\sbox{\sembox}{\ensuremath{#1}}%
\settowidth{\semwidth}{\usebox{\sembox}}%
\sbox{\sembox}{\ensuremath{\left[\usebox{\sembox}\right]}}%
\settowidth{\boxwidth}{\usebox{\sembox}}%
\addtolength{\boxwidth}{-\semwidth}%
\left[\hspace{-0.3\boxwidth}%
\usebox{\sembox}%
\hspace{-0.3\boxwidth}\right]%
}

\newcommand{\authormodification}[2]{{\color{#1}#2}}
\newcommand{\ys}[1]{\authormodification{blue}{#1}}
\newcommand{\bs}[1]{\authormodification{red}{#1}}
\newcommand{\gdr}[1]{\authormodification{magenta}{#1}}

\begin{document}

%\conferenceinfo{PLDI 2012}{Beijing, China} 
%\copyrightyear{2012} 
%\copyrightdata{[to be supplied]} 

\titlebanner{Draft}        % These are ignored unless
%\preprintfooter{Y.Solodkyy, G.Dos Reis, B.Stroustrup: Open and Efficient Type Switch for C++}   % 'preprint' option specified.
\preprintfooter{Open and Efficient Type Switch for C++}   % 'preprint' option specified.

\title{Open and Efficient Type Switch for C++}
%\subtitle{your \code{visit}, Jim, is not \code{accept}able anymore}
%\subtitle{\code{accepting} aint no \code{visit}ors}

\authorinfo{Omitted for Submission}
           {Affiliation}
           {anonymous@anonymous.com}

\maketitle

\begin{abstract}
Selecting operations based on a type of an object determined at run-time is key 
to many object-oriented and functional programming techniques. We present 
techniques that can implement efficient type switching, type testing, pattern 
matching, predicate dispatch, multi-methods in a compiler or a library. The 
techniques are general and cope well with C++ multiple inheritance.

Our library-only implementation provides a functional programming style notation 
to the programmer. It outperforms the visitor design pattern, as commonly used 
for type-casing scenarios in C++. For many use cases it equals or outperforms 
equivalent code in languages with built-in type switching constructs, such as 
OCaml. We find the pattern-based library code easier to read and write and more 
expressive than hand-coded visitors. The library is non-intrusive and does not have 
extensibility restrictions. It also avoids control inversion characteristic to 
visitors.
 
The library was motivated by and is used for applications involving large, 
typed, abstract syntax trees. Being a library only solution allows us to use 
production quality compilers and tool chains for our experiments and our 
intended applications.

%We present techniques that can be used in a compiler or library setting to 
%efficiently implement type switching, type testing, pattern matching, predicate 
%dispatch, multi-methods and other facilities that depend on the run-time type 
%of an argument. The techniques cope well with multiple inheritance in C++, 
%however they are not specific to C++ and can be adapted to implement 
%similar facilities in other languages.
%
%Our library-only implementation of a type switch based on these techniques equals 
%or outperforms the visitor design pattern, as commonly used for type-casing 
%scenarios in C++. For many use cases, it equals or outperforms equivalent 
%code in languages with built-in type-switching constructs. While remaining a 
%library-only solution, such a facility better addresses the expression problem 
%than the visitor design pattern does: it is non-intrusive and does not have 
%extensibility restrictions. It also avoids the control inversion characteristic to 
%visitors, thereby making the code significantly more concise and easier to comprehend.
%The library was motivated by and is used for applications involving large, 
%typed abstract syntax trees.
\end{abstract}

\category{CR-number}{subcategory}{third-level}

\terms
Languages, Design

\keywords
Type Switching, Visitor Design Pattern, Pattern Matching, Expression Problem, C++

\section{Introduction} %%%%%%%%%%%%%%%%%%%%%%%%%%%%%%%%%%%%%%%%%%%%%%%%%%%%%%%%%
\label{sec:intro}

%Motivate the problem
%Give a summary of the paper: what you did and how
%Explicitly state your contribution

Algebraic data types as seen in functional languages are closed and their 
variants are disjoint. This allows for easy addition of new functions through 
case analysis as well as their efficient implementation. Data types in 
object-oriented languages are extensible and hierarchical (non-disjoint), which 
significantly complicates the addition of new functions. Existing approaches to 
case analysis on such types are either efficient or open, but not both. Truly 
open approaches, which allow for independent extensibility, modular 
type-checking and dynamic linking, rely on type testing through an 
\code{instanceof}-like predicate combined with a decision tree. Efficient 
approaches rely on sealing either the class hierarchy or the set of functions, 
which looses extensibility. Consider a simple expression language: 

\begin{lstlisting}
@$exp$ \is{} $val$ \Alt{} $exp+exp$ \Alt{} $exp-exp$ \Alt{} $exp*exp$ \Alt{} $exp/exp$@
\end{lstlisting}

\noindent 
In an object-oriented language without direct support for algebraic data types, 
the type representing an expression-tree in the language will typically be 
encoded as an abstract base class, listing the (sealed set of) allowed virtual 
functions, with derived classes representing variants:

\begin{lstlisting}[keepspaces,columns=flexible]
struct Expr { virtual @$\sim$@Expr() {} };
struct Value  : Expr { int value; };
struct Plus   : Expr { Expr* e1; Expr* e2; }; // ...
\end{lstlisting}

\noindent
A simple evaluator for this language can be implemented with the aid of a
virtual function \code{eval()} declared in the base class \code{Expr}. 
The approach is \emph{intrusive}, as we will have to modify the base class every 
time we would like to add a function. A less-intrusive solution to the 
extensibility of functions can be achieved with \emph{visitor design 
pattern}~\cite{DesignPatterns1993}. That solution, however, is not open as it 
restricts extensibility of classes.

Our solution overcomes extensibility problems of both functions and 
types by allowing external introspection of objects with a type switch:

\begin{lstlisting}[keepspaces,columns=flexible]
int eval(const Expr* e)
{
  Match(e)
    Case(Value,  n)    return n;
    Case(Plus,   a, b) return eval(a) + eval(b);
    Case(Minus,  a, b) return eval(a) - eval(b);
    Case(Times,  a, b) return eval(a) * eval(b);
    Case(Divide, a, b) return eval(a) / eval(b);
  EndMatch
}
\end{lstlisting}

\noindent
The syntax is provided without any external tool support. Instead we rely on a 
few C++0x features~\cite{C++0x}, template meta-programming, and macros. It runs 
about as fast as the OCaml version (\textsection\ref{sec:ocaml}), and, depending 
on the usage scenario, compiler and underlying hardware, comes close or 
outperforms the handcrafted C++ code based on the \emph{visitor design pattern} 
(\textsection\ref{sec:eval}). The mapping of members to matching positions that 
the user has to provide to make the above example fully functional is 
omitted~\cite{TR}.

%\subsection{Summary}

We present techniques based on memoization (\textsection\ref{sec:copc}) and 
class precedence list (\textsection\ref{sec:cotc}) that can be used to
implement type switching efficiently based on the run-time type of the
argument:
  \begin{itemize}
  \setlength{\itemsep}{0pt}
  \setlength{\parskip}{0pt}
  \item The techniques come close and often outperform its de facto contender -- 
        visitor design pattern -- without sacrificing extensibility (\textsection\ref{sec:eval}).
  \item They work in the presence of multiple inheritance, including repeated and 
        virtual inheritance, as well as in generic code (\textsection\ref{sec:vtblmem}).
  \item The solution is open by construction (\textsection\ref{sec:poets}), 
        non-intrusive, and avoids the control inversion typical for visitors.
  \item It applies to polymorphic (\textsection\ref{sec:vtp}-\ref{sec:vtblmem}) and 
        tagged (\textsection\ref{sec:cotc}) class hierarchies through a unified  
        syntax~\cite{AP}.
  \item Our memoization device (\textsection\ref{sec:memdev}) generalizes to 
        other languages and can be used to implement type switching 
        (\textsection\ref{sec:vtblmem}), type testing 
        (\textsection\ref{sec:poets},\cite[\textsection 4.7]{TR}), predicate dispatch 
        (\textsection\ref{sec:memdev}), and multiple dispatch 
        (\textsection\ref{sec:cc}) efficiently.
  \item We list conditions under which virtual table pointers, commonly used in 
        C++ implementations, uniquely identify the exact subobject within the 
        most derived type (\textsection\ref{sec:vtp}).
  \item We also build an efficient cache indexing function for virtual table 
        pointers that minimizes the amount of conflicts 
        (\textsection\ref{sec:sovtp},\ref{sec:moc},\cite[\textsection 4.3.5]{TR}).
  \end{itemize}

\noindent
A practical benefit of our solution is that it can be used right away with any 
compiler with a descent support of C++0x without requiring the installation of 
any additional tools or preprocessors. The solution is a proof of concept that 
sets a minimum threshold for the performance, brevity, clarity and usefulness of 
a language solution for open type switching in C++.

Due to page restrictions, we will only deal here with the part of the library 
concerned with efficient type testing, type identification, and type switching. 
These primitive operations are essential in providing support for type patterns 
and constructor patterns, and their efficiency relative to the visitor design 
pattern might become a major argument for their adoption. We refer the reader to 
companion paper~\cite{AP} and technical report~\cite{TR} for details on the 
rest of our pattern-matching library.

%Section~\ref{sec:probl} will motivate the problem as well as discuss some 
%existing solutions and their limitations. Section~\ref{sec:copc} will outline 
%the three techniques that comprise our solution, while the 
%section~\ref{sec:eval} will evaluate their performance. Related work is 
%discussed in Section~\ref{sec:rw}, followed by concluding remarks in 
%Section~\ref{sec:cc}.


\section{Problem Description} %%%%%%%%%%%%%%%%%%%%%%%%%%%%%%%%%%%%%%%%%%%%%%%%%%
\label{sec:probl}

Pattern matching has been closely related to \emph{algebraic data types} and 
\emph{equational reasoning} since the early days of functional programming.
In languages like ML and Haskell an \emph{Algebraic Data Type} is a data type 
each of whose values is picked from a disjoint sum of (possibly recursive) data 
types, called \emph{variants}. Each of the variants is marked with a unique 
symbolic constant called \emph{constructor}. Constructors provide a convenient 
way of creating a value of its variant type as well as a way of discriminating 
its variant type from the algebraic data type through pattern matching.

C++ does not have direct support of algebraic data types, but they can usually 
be encoded with classes in a number of ways. One common such encoding is to 
introduce an abstract base class representing an algebraic data type with 
several derived classes representing variants. The variants can then be 
discriminated with either run-time type information (further referred to as 
\emph{polymorphic encoding}) or a dedicated member of a base class (further 
referred to as \emph{tagged encoding}). Object-oriented purists might argue that 
discrimination between the variants should be avoided at all costs through 
encapsulation. We disagree, however, as variants in traditional applications of 
algebraic data types are not implementation classes, but rather more-specialized 
interfaces.

By encoding algebraic data types with classes we alter their semantics in two 
important ways: we make them \emph{extensible} as new variants can be added by 
simply deriving from the base class, as well as \emph{hierarchical} as variants 
can be inherited from other variants and thus form a subtyping relation between 
themselves~\cite{Glew99}. This is not the case with traditional algebraic data 
types in functional languages, where the set of variants is \emph{closed}, while 
the variants are \emph{disjoint}. Some functional languages e.g. 
ML2000~\cite{ML2000} and Moby~\cite{Moby} were experimenting with 
\emph{hierarchical extensible sum types}, which are closer to object-oriented 
classes then algebraic data types are, but, interestingly, they did not provide 
pattern matching facilities on them. Working within a multi-paradigm  
programming language like C++, we will not be looking at algebraic data types in
the closed form they are present in functional languages, but rather in an 
open/extensible form discussed by Zenger~\cite{Zenger:2001}, Emir~\cite{EmirThesis}, 
L\"oh~\cite{LohHinze2006}, Glew~\cite{Glew99} and others. We will thus 
assume an object-oriented setting where new variants can be added later and form
subtyping relations between each other including those through multiple 
inheritance.

\subsection{Type Switch}

Consider a class \code{B} and a set of classes \code{Di} directly or indirectly 
inherited from it. An object is said to be of the \emph{most derived type} 
\code{D} if it was created by explicitly calling a constructor of that type.
The inheritance relation on classes induces a subtyping relation on them, which in 
turn allows objects of a derived class to be used in places where an object of a 
base class is expected. The type of variable or parameter referencing such an
object is called the \emph{static type} of the object. When object is passed by 
reference or by pointer, we might end up in a situation where the static type of an 
object is different from its most derived type, with the latter necessarily 
being a subtype of the former. The most derived class along with all its base classes 
that are not base classes of the static type are typically referred to as the 
\emph{dynamic types} of an object. At each program point the compiler knows the 
static type of an object, but not its dynamic types.

By \emph{type switch} we will refer to a programming language construct capable of 
uncovering a reference or a pointer to the dynamic type(s) of an object present in 
a given list of types.

Consider an object of (most derived) type \code{D}, pointed to by a variable of 
static type \code{B*}: e.g. \code{B* base = new D;}. A hypothetical type switch 
statement, not currently supported by C++, can look as following:

\begin{lstlisting}
switch (base)
{
case D1: s1;
 ...
case Dn: sn;
}
\end{lstlisting}

\noindent and can be given numerous plausible semantics:

\begin{itemize}
\setlength{\itemsep}{0pt}
\setlength{\parskip}{0pt}
\item \emph{First-fit} semantics will evaluate the first statement $s_i$ such 
      that $D_i$ is a base class of $D$
\item \emph{Best-fit} semantics will evaluate the statement corresponding to the 
      most derived base class $D_i$ of $D$ if it is unique (subject to 
      ambiguity)
\item \emph{The-only-fit} semantics will only evaluate statement $s_i$ if $D_i=D$.
\item \emph{All-fit} semantics will evaluate all statements $s_i$ whose guard 
      type $D_i$ is a subtype of $D$ (order of execution has to be defined)
\item \emph{Any-fit} semantics might choose non-deterministically one of the 
      statements enabled by all-fit
\end{itemize}

\noindent
The list is not exhaustive and depending on a language, any of these semantics 
or their combination might be a plausible choice. Functional languages, for 
example, often prefer first-fit, while object-oriented languages would typically 
be inclined to best-fit semantics. The-only-fit semantics is traditionally seen 
in procedural languages like C and Pascal to deal with discriminated union types. 
All-fit and any-fit semantics might be seen in languages based on predicate 
dispatching~\cite{ErnstKC98} or guarded commands~\cite{EWD:EWD472}, where a 
predicate can be seen as a characteristic function of a type, while logical 
implication can be seen as subtyping.

\subsection{Open and Efficient Type Switching}
\label{sec:poets}

The fact that algebraic data types in functional languages are closed allows for 
their efficient implementation. The traditional compilation scheme assigns unique 
tags to every variant of the algebraic data type and pattern matching is then 
simply implemented with a jump table over all tags. A number of issues in 
object-oriented languages makes this extremely efficient approach infeasible:

\begin{itemize}
\setlength{\itemsep}{0pt}
\setlength{\parskip}{0pt}
\item Extensibility
\item Subtyping
\item Multiple inheritance
\item Separate compilation
\item Dynamic linking 
\end{itemize}

\noindent
Unlike functional style algebraic data types, classes are \emph{extensible} 
whereby new variants can be arbitrarily added to the base class in the form of 
derived classes. Such extension can happen in a different translation unit or a
static library (subject to \emph{separate compilation}) or a dynamically linked 
module (subject to \emph{dynamic linking}). Separate compilation effectively 
implies that all the derived classes of a given class will only be known at link 
time, postponing thus any tag-allocation related decisions until then. The 
presence of dynamic linking effectively requires the compiler to assume that the
exact derived classes will only be known at run time, and not even at start-up 
time.

%and thus any tag allocation scheme should on one hand assume presence of 
%unknown tags and on the other -- the necessity of maintaing the same tags for 
%the commonly seen classes of each dynamic module.  

The \emph{subtyping} relation that comes along with extensibility through 
subclassing effectively gives every class multiple types -- its own and the 
types of all its base classes. In such a scenario it is natural to require that 
type switching can be done not only against the exact dynamic type of an object, 
but also against any of its base classes (subject to our substitutability 
requirement). This in itself is not a problem for functional-style tag 
allocation as long as the set of all derived classes is known, since the 
compiler can partition tags of all the derived classes according to chosen 
semantics based on classes mentioned in case clauses.
Unfortunately this will not work in the presence of dynamic linking as there 
might be new derived classes with tags not known at the time of partitioning and 
thus not mentioned in the generated jump table.

\emph{Multiple inheritance} complicates things further by making each class 
potentially belong to numerous unrelated hierarchies. Any tag allocation scheme 
capable of dealing with multiple inheritance will either have to assure that 
generated tags satisfy properties of each subhierarchy independently or use 
different tags for different subhierarchies. Multiple inheritance also 
introduces such a phenomenon as \emph{cross-casting}, whereby a user may request 
to cast pointers between unrelated classes, since they can potentially become 
base classes of a later defined class. From an implementation point of view this 
means that not only do we have to be able to check that a given object belongs 
to a given class (type testing), but also be able to find a correct offset to it 
from a given base class (type casting).

While looking at various schemes for implementing type switching we noted down a 
few questions that might help evaluate and compare solutions: 

\begin{enumerate}
\setlength{\itemsep}{0pt}
\setlength{\parskip}{0pt}
\item Can the solution handle base classes in case clauses?
\item Will it handle the presence of base and derived classes in the same match statement?
\item Will it work with derived classes coming from a DLL?
\item Can it cope with multiple inheritance (repeated, virtual)?
\item Can independently developed DLLs that either extend classes involved in 
      type switching or do type switching themselves be loaded together without 
      any integration efforts?
\item Are there any limitations on the number and or shape of class extensions?
\item What is the complexity of performing matching, based on the number of case clauses and 
      the number of possible types?
\end{enumerate}

The number of possible types in the last question refers to the number of subtypes 
of the static type of the subject, not all the types in the program. Several 
solutions discussed below depend on the number of case clauses in the match 
statement, which raises the question of how many such clauses a typical program 
might have. The C++ pretty-printer for Pivot we implemented using our pattern 
matching techniques originally had 8 match statements with 5, 7, 8, 10, 15, 17, 30 
and 63 case clauses each. While experimenting 
with probability distributions of various classes to minimize the number of 
conflicts (see \textsection\ref{sec:moc}), we had to associate probabilities 
with classes and implemented it with a match statement over all 160 nodes in the 
Pivot's class hierarchy. With Pivot having the smallest number of node kinds 
among the compiler frameworks we had a chance to work with, we expect a similar 
or larger number of case clauses in other compiler applications.

An obvious solution that will pass the above checklist can look like the following:

\begin{lstlisting}
if (D1* derived = dynamic_cast<D1*>(base)) { s1; } else
if (D2* derived = dynamic_cast<D2*>(base)) { s2; } else
...
if (Dn* derived = dynamic_cast<Dn*>(base)) { sn; }
\end{lstlisting}

\noindent
Despite the obvious simplicity, its main drawback is performance: a typical 
implementation of \code{dynamic_cast} might take time proportional to the 
distance between base and derived classes in the inheritance tree~\cite{XXXXX}.
What is worse, is that the time to uncover the type in the $i^{th}$ case clause 
is proportional to $i$, while failure to match will always take the longest. 
This linear increase can be seen in the Figure~\ref{fig:DCastVis1}, where 
the above cascading-if was applied to a flat hierarchy encoding an algebraic 
data type with 100 variants. The same type-switching functionality implemented 
with the visitor design pattern took only 28 cycles regardless of the case.
\footnote{Each case $i$ was timed multiple times to avoid fluctuations, turning 
the experiment into a repetitive benchmark described in 
\textsection\ref{sec:eval}. In a realistic setting the cost of double dispatch 
was varying between 52 and 55 cycles.}
This is more than 3 times faster than the 93 cycles it took to uncover even the 
first case with \code{dynamic_cast}, while it took 22760 cycles to uncover the 
last.

\begin{figure}[htbp]
  \centering
    \includegraphics[width=0.47\textwidth]{DCast-vs-Visitors1.png}
  \caption{Type switching based on na\"ive techniques}
  \label{fig:DCastVis1}
\end{figure}

When the class hierarchy is not flat and has several levels, the above 
cascading-if can be replaced with a decision tree that tests base classes first 
and thus eliminates many of the derived classes from consideration. This 
approach is used by Emir to deal with type patterns in Scala~\cite[\textsection 
4.2]{EmirThesis}. The intent is to replace a sequence of 
independent dynamic casts between classes that are far from each other in the 
hierarchy with nested dynamic casts between classes that are close to each 
other. Another advantage is the possibility to fail early: if the type of the subject 
does not match any of the clauses, we will not have to try all the cases. 
A flat hierarchy, which will likely be formed by the leaves in even a multi-level 
hierarchy, will not be able to benefit from this optimization and 
will effectively degrade to the above cascading-if. Nevertheless, when 
applicable, the optimization can be very useful and its benefits can be seen in
Figure~\ref{fig:DCastVis1} under ``Decision-Tree + dynamic\_cast''. The class 
hierarchy for this timing experiment formed a perfect binary tree with 
classes number 2*N and 2*N+1 derived from a class with number N. The structure 
of the hierarchy also explains the repetitive pattern of timings.

The above solution either in a form of cascading-if or as a decision tree can be 
significantly improved by lowering the cost of a single \code{dynamic_cast}. 
We devise an asymptotically constant version of this operator that we call
\code{memoized_cast} in \textsection\ref{sec:memcast}. As can be seen from the graph 
titled ``Cascading-If + memoized\_cast'', it speeds up the above cascading-if 
solution by a factor of 18 on average, as well as outperforms the decision-tree 
based solution with dynamic\_cast for a number of case clauses way beyond those that can happen in 
a reasonable program. We leave the discussion of the technique until 
\textsection\ref{sec:memcast}, while we keep it in the chart to give perspective on 
an even faster solution to dynamic casting. The slowest implementation in the 
chart based on exception handling facilities of C++ is discussed in 
\textsection\ref{sec:xpm}.

The approach of Gibbs and Stroustrup~\cite{FastDynCast} employs divisibility of numbers to obtain a 
tag allocation scheme capable of performing type testing in constant time. 
Extended with a mechanism for storing offsets required for this-pointer 
adjustments, the technique can be used for extremely fast dynamic casting on 
quite large class hierarchies. The idea is to allocate tags 
for each class in such a way that tag of a class D is divisible by a tag of a 
class B if and only if class D is derived from class B. For comparison purposes 
we handcrafted this technique on the above flat and binary-tree hierarchies and 
then redid the timing experiments from Figure~\ref{fig:DCastVis1} using the fast 
dynamic cast. The results are presented in Figure~\ref{fig:DCastVis2}. For 
reference purposes we retained ``Visitor Design Pattern'' and ``Cascading-If + 
memoized\_cast'' timings from Figure~\ref{fig:DCastVis1} unchanged. Note that 
the Y-axis has been scaled-up 140 times, which is why the slope of 
``Cascading-If + memoized\_cast'' timings is so much steeper.

\begin{figure}[htbp]
  \centering
    \includegraphics[width=0.47\textwidth]{DCast-vs-Visitors2.png}
  \caption{Type switching based on the fast dynamic cast of Gibbs and Stroustrup~\cite{FastDynCast}}
  \label{fig:DCastVis2}
\end{figure}

As can be seen from the figure the use of our memoized\_cast implementation can 
get close in terms of performance to the fast dynamic cast, especially 
when combined with decision trees. An important difference that cannot be seen 
from the chart, however, is that the performance of memoized\_cast is 
asymptotic, while the performance of fast dynamic cast is guaranteed. This 
happens because the implementation of memoized\_cast will incur an overhead of 
a regular dynamic\_cast call on every first call with a given most derived type. 
Once that class is memoized, the performance will remain as shown. Averaged over 
all calls with a given type we can only claim we are asymptotically as good as 
fast dynamic cast.

Unfortunately fast dynamic casting is not truly open to fully satisfy our 
checklist. The structure of tags required by the scheme limits the number of 
classes it can handle. A 32-bit integer is estimated to be able to represent 7 
levels of a class hierarchy that forms a binary tree (255 classes), 6 levels of 
a similar ternary tree hierarchy (1093 classes) or just one level of a hierarchy 
with 9 base classes -- multiple inheritance is the worst case scenario of the 
scheme that quickly drains its allocation possibilities. Besides, similarly to 
other tag allocation schemes, presence of class extensions in DLLs will likely 
require an integration effort to make sure different DLLs are not reusing prime 
numbers in a way that might result in an incorrect dynamic cast.

In view of the predictably-constant dispatching overhead of the visitor design pattern, 
it is clear that any open solution that will have a non-constant dispatching 
overhead will have a poor chance of being adopted. Multi-way switch on 
sequentially allocated tags~\cite{Spuler94} was one of the few techniques that 
could achieve constant overhead, and thus compete with and even outperform visitors. 
Unfortunately the scheme has problems of its own that make it unsuitable for 
truly open type-switching and here is why.

%To better understand the problem let us look at some existing solutions to type 
%switching that we found to be used in practice. 

%From our experience on this project we have noticed that we can only compete 
%with visitors when switch statements are implemented with a jump table. As soon 
%as compiler was putting even a single branch into the decision tree of cases, 
%the performance was degraded significantly. From this perspective we do not 
%regard solutions based on decision trees as efficient, since they do not let us 
%compete compete with the visitors solution.

The simple scheme of assigning a unique tag per variant (instantiatable class 
here) will not pass our first question because the tags of base and derived 
classes will have to be different if the base class can be instantiated on its 
own. In other words we will not be able to land on a case label of a base class, while 
having a derived tag only. The already mentioned partitioning of tags of derived 
classes based on the classes in case clauses also will not help as it assumes 
knowledge of all the classes and thus fails extensibility through DLLs.

In practical implementations hand crafted for a specific class hierarchy, tags 
often are not chosen arbitrarily, but to reflect the subtyping relation of the 
underlain hierarchy. Switching on base classes in such a setting will typically 
involve a call to some function $f$ that converts derived class' tag into a base 
class' tag. An example of such a scheme would be having a certain bit in the tag 
set for all the classes derived from a given base class. Unfortunately this 
solution creates more problems than it solves.

First of all the solution will not be able to recognize an exceptional case 
where most of the derived classes should be handled as a base class, while a few 
should be handled specifically. Applying the function $f$ puts several different 
types into an equivalence class with their base type, making them 
indistinguishable from each other.

Secondly, the assumed structure of tags is likely to make the set of tags 
sparse, effectively forcing the compiler to use a decision tree instead of a jump 
table to implement the switch. Even though conditional jump is reported to be 
faster than indirect jump on many computer architectures~\cite[\textsection 
4]{garrigue-98}, this did not seem to be the case in our experiments. Splitting 
of a jump table into two with a condition, that was sometimes happening because 
of our case label allocation scheme, was resulting in a noticeable degradation of 
performance in comparison to a single jump table.

Besides, as was seen in the scheme of Gibbs and Stroustrup, the assumed 
structure of tags can also significantly decrease the number of classes a given 
allocation scheme can handle. It is also interesting to note that even though 
their scheme can be easily adopted for type switching with decision trees, it is 
not easily adoptable for type switching with jump tables: in order to obtain 
tags of base classes we will have to decompose the derived tag into primes and 
then find all the dividers of the tag present in case clauses.

To summarize, truly open and efficient type switching seems to be a non-trivial 
problem. The implementations we found in the literature were either open or 
efficient, but not both. Efficient implementation was typically achieved by 
sealing the class hierarchy and using a jump table. Without sealing, the implementation 
was resorting to decision trees and type testing, which was not efficient.
We are unaware of any efficient tag allocation scheme that can be used in a 
truly open scenario.


\section{Type Switch}
\label{sec:copc}

C++ does not have direct support of algebraic data types, but they can be 
encoded with classes in a number of ways. One common such encoding is to 
introduce an abstract base class representing an algebraic data type with 
several derived classes representing variants. The variants can then be 
discriminated with either run-time type information (referred to as 
\emph{polymorphic encoding}) or a dedicated member of a base class 
(referred to  as \emph{tagged encoding}).

While our library supports both encodings, it handles them differently to let 
the user choose between openness and efficiency. The type switch for tagged 
encoding is simpler and more efficient for many typical use cases, however, 
making it open will eradicate its performance advantages. The difference in 
performance is the price we pay for keeping the solution open. We describe pros 
and cons of each approach in \textsection\ref{sec:cmp}.

%The core of the proposal relies on two key aspects of C++ implementations:
%\begin{enumerate}
%\item a constant-time access to the virtual table pointer embedded in an object of
%  dynamic class type;
%\item injectivity of the relation between an object's inheritance path
%  and the virtual table pointer extracted from that object.
%\end{enumerate}

\subsection{An Attractive Non-Solution}
\label{sec:cotc}

%The memoization device outlined in \textsection\ref{sec:memdev} can, in principle, also be 
%applied to tagged classes. The dynamic cast will be replaced by a small 
%compile-time template meta-program that checks whether the class associated with 
%the given tag is derived from the target type of the case clause. If so, a static 
%cast can be used to obtain the offset.

%Despite its straightforwardness, we felt that it should be possible to do better 
%than the general solution, given that each class is already identified with a 
%dedicated constant known at compile time.

While Wirth' linked list encoding was considered slow for subtype testing, it can 
be adopted for quite efficient type switching on a class hierarchy with no 
repeated inheritance. The idea is to combine fast switching on closed 
algebraic datatypes with a loop that tries the tags of base classes when 
switching on derived tags fails.

%The nominal subtyping of \Cpp{} effectively gives every class multiple types. The 
%idea is thus to associate with the type not only its most-derived tag, but also 
%the list of tags of all its base classes. In a compiler implementation such a 
%list can be stored inside the virtual table of a class, while in our library 
%solution it is shared between all the instances with the same most-derived tag 
%in a less efficient global map, associating the tag to its tag list.

For simplicity of presentation we assume a pointer to an array of tags be available 
directly through the subject's \code{taglist} data member. The array is of 
variable size: its first element is always the tag of the subject's dynamic 
type, while its end is marked with a dedicated \code{end_of_list} marker, 
distinct from all the tags. The tags in between are topologically sorted 
according to the subtyping relation with incomparable siblings listed in 
\emph{local precedence order} -- the order of the direct base classes used in 
the class definition. The list resembles the \emph{class precedence list} of 
object-oriented descendants of Lisp (e.g. Dylan, Flavors, LOOPS, and CLOS) used 
there for \emph{linearization} of class hierarchies. 
We also assume the tag-constant associated with a class \code{Di} is accessible 
through a static member \code{Di::class_tag}. These simplifications are not 
essential and the library does not rely on any of them.
%Instead, the user can retroactively narrate to the library the specific tag 
%encoding used through a trait-like class.

A type switch, below, %, built on top of a hierarchy of tagged classes, 
proceeds as 
a regular switch on the subject's tag. If the jump succeeds, we found an exact 
match; otherwise, we get into a default clause that obtains the next tag in the
list and jumps back %to the beginning of the switch statement 
for a rematch:

\begin{lstlisting}[keepspaces]
    size_t attempt = 0; 
    size_t tag = subject->taglist[attempt];
ReMatch:
    switch (tag) {
    default:
        tag = subject->taglist[++attempt];
        goto ReMatch;
    case end_of_list: 
        break;
    case D1::class_tag: 
        D1& match = static_cast<D1&>(*subject); s1;
        break;
        ...
    case Dn::class_tag: 
        Dn& match = static_cast<Dn&>(*subject); sn;
        break;
    }
\end{lstlisting}

\noindent
The above structure, which we call a \emph{tag switch}, implements a variation of 
best-fit semantics based on local precedence order. It lets us dispatch to the case 
clause of the most-specialized class with an overhead of initializing two 
local variables, compared to an efficient switch used on algebraic data types. 
Dispatching to a case clause of a base class will take time roughly proportional 
to the distance between the matched base class and the derived class in the 
inheritance graph, thus the technique is not constant. When none of the base 
class tags was matched, we will necessarily reach the end\_of\_list marker %in the list 
and exit the loop. %As mentioned before, 
The default clause, %of the type switch 
again, can be implemented with a case clause on the subject type's tag: \code{case S::class_tag:}

The efficiency of the above code crucially depends on the set of tags 
being small and sequential to justify the use of a jump table instead of a
decision tree to implement the switch. This is usually not a problem in closed 
hierarchies based on tag encoding since the designer of the hierarchy handpicks 
the tags herself. The use of a static cast %to obtain proper reference once the most specialized derived class has been established, 
however, essentially limits the use of 
this mechanism to non-repeated inheritance only. This only refers to the way target 
classes inherit from the subject type -- they can freely inherit from other classes. 
%as long as they inherit the subject type through non-repeated inheritance only. 
Due to these restrictions, the technique is not open because it may  
violate independent extensibility. We discuss in \textsection\ref{sec:cmp} that 
making the technique more open will also eradicate its performance advantages.


\subsection{Open Type Switching}
\label{sec:poets}

Instead of starting with an efficient solution and trying to make it open, we 
start with an open solution and try to make it efficient. The following 
cascading-if statement implements the first-fit semantics for our type switch in 
a truly open fashion:

\begin{lstlisting}
if (T1* match = dynamic_cast<T1*>(subject)) { s1; } else
if (T2* match = dynamic_cast<T2*>(subject)) { s2; } else
...
if (Tn* match = dynamic_cast<Tn*>(subject)) { sn; }
\end{lstlisting}

\noindent
Its main drawback is performance: a typical 
implementation of \code{dynamic_cast} takes time proportional to the 
distance between base and derived classes in the inheritance tree.
What is worse, is that the time to uncover the type in the $i^{th}$ case clause 
is proportional to $i$, while failure to match will always take the longest. 
This linear increase can be seen in the Figure~\ref{fig:DCastVis1}, where 
the above cascading-if was applied to a flat hierarchy encoding an algebraic 
data type with 100 variants. The same type-switching functionality implemented 
with the visitor design pattern took only 28 cycles regardless of the 
case.\footnote{Each case $i$ was timed multiple times, thus turning the experiment 
into a repetitive benchmark described in \textsection\ref{sec:eval}. In a more
realistic setting, represented by random and sequential benchmarks, the cost of 
double dispatch was varying between 52 and 55 cycles.}
This is more than 3 times faster than the 93 cycles it took to uncover even the 
first case with \code{dynamic_cast}, while it took 22760 cycles to uncover the 
last.

\begin{figure}[htbp]
  \centering
    \includegraphics[width=0.47\textwidth]{DCast-vs-Visitors1.png}
  \caption{Type switching based on na\"ive techniques}
  \label{fig:DCastVis1}
\end{figure}

%Seeing several solutions whose time increases with the position of the case 
%clause in the type switch, one may wonder how many such clauses a typical 
%program might have. A program dealing with abstract syntax trees in 
%Pivot~\cite{Pivot09} that we implemented using our pattern-matching library had 
%8 match statements with 5, 7, 8, 10, 15, 17, 30 and 63 case clauses, 
%respectively. With Pivot having the smallest number of node kinds among the 
%compiler frameworks we had a chance to work with, we expect a similar or larger 
%number of case clauses in other compiler applications.

When the class hierarchy is not flat, the above cascading-if can be replaced 
with a decision tree that tests base classes first and thus eliminates many of 
the derived classes from consideration -- an approach used by Emir to deal with 
type patterns in Scala~\cite[\textsection 4.2]{EmirThesis}. The intent is to 
replace a sequence of independent dynamic casts between classes that are far 
from each other in the hierarchy with nested dynamic casts between classes that 
are close to each other. Another advantage is the possibility to fail early. 
As can be seen from Figure~\ref{fig:DCastVis1} under ``Decision-Tree + 
dynamic\_cast'', when applicable, the optimization can be very useful. The class
hierarchy for this timing experiment formed a perfect binary tree with 
classes number 2*N and 2*N+1 derived from a class with number N. The hierarchy 
also explains the repetitive pattern of timings.

Several authors had noted the relationship between exception handling and type 
switching before~\cite{Glew99,ML2000}. Not surprisingly, the exception handling 
mechanism of C++ can be abused to implement the first-fit semantics of a type 
switch statement. The idea is to harness the fact that catch-handlers in C++ 
essentially use first-fit semantics to decide which one is going to handle a 
given exception. Unfortunately the approach is even slower than the use of 
\code{dynamic_cast} and we only list it here for comparison.

\subsection{Memoization Device}
\label{sec:memdev}

Let us look at a slightly more general problem than type switching. Consider a 
generalization of the switch statement that takes predicates on a subject as its 
clauses and executes the first statement $s_i$ whose predicate is enabled: 

\begin{lstlisting}[keepspaces]
switch (x) { case P1(x): s1; ... case Pn(x): sn; }
\end{lstlisting}

\noindent
Assuming that predicates depend only on $x$ and nothing else as well as that 
they do not involve any side effects, we can be sure that the next time we come 
to such a switch with the same value, the same predicate will be enabled 
first. Thus, we would like to avoid evaluating predicates and jump straight to 
the statement it guards. In a way we would like the switch to memoize which 
case is enabled for a given value of $x$.

The idea is to generate a simple cascading-if statement interleaved with jump 
targets and instructions that associate the original value with enabled target. 
The code before the statement looks up whether the association for a given value 
has already been established, and, if so, jumps directly to the target; otherwise 
the sequential execution of the cascading-if is started. To ensure 
that the actual code associated with the predicates remains unaware of this 
optimization, the code preceeding it after the target must re-establish any 
invariant guaranteed by sequential execution (\textsection\ref{sec:vtblmem}).

The above code can easily be produced in a compiler setting, but producing it in 
a library setting is a challenge. Inspired by Duff's Device~\cite{Duff}, 
we devised a construct that we call \emph{Memoization Device} that does just 
that in standard C++:

\begin{lstlisting}
typedef decltype(x) T;
static std::unordered_map<T,int> jump_targets;

switch (int& jump_to = jump_targets[x]) {
default: // entered when we have not seen x yet
    if (P1(x)) { jump_to = 1; case 1: s1; } else 
    if (P2(x)) { jump_to = 2; case 2: s2; } else
      ...
    if (Pn(x)) { jump_to = @$n$@; case @$n$@: sn; } else
                jump_to = @$n+1$@;
case @$n+1$@: // none of the predicates is true on x
}
\end{lstlisting}

\noindent
The static \code{jump_targets} hash table will be allocated upon first entry 
to the function. The map is initially empty and according to its logic, 
request for a key $x$ not yet in the map will allocate a 
new entry with its associated data default initialized (to 0 for int). Since 
there is no case label 0 in the switch, the default case will be taken, which, in 
turn, will initiate sequential execution of the interleaved cascading-if 
statement. Assignments to \code{jump_to} effectively establish association 
between value $x$ and corresponding predicate, since \code{jump_to} is just a 
reference to \code{jump_targets[x]}. The last assignment records absence of 
enabled predicates for the value.

To change the first-fit semantics of the above construct into \emph{sequential 
all-fit}, we remove the \code{else}s and rely on fall-through behavior of the 
switch. We also make the assignments conditional to make sure only the first one 
gets recorded:

\begin{lstlisting}
if (Pi(x)) { if (jump_to == 0) jump_to = @$i$@; case @$i$@: si; }
\end{lstlisting}

\noindent
Note that the protocol that has to be maintained by this structure does not 
depend on the actual values of case labels. We only require them to be 
different and include a predefined default value. The default clause can be 
replaced with a case clause for the predefined value, however keeping the default  
clause results in a faster code. A more important performance consideration is to 
keep the values close to each other. Not following this rule might result in a 
compiler choosing a decision tree over a jump table implementation of the 
switch, which in our experience significantly degrades the performance.

The first-fit semantics is not an inherent property of the memoization device. 
Assuming that the conditions are either mutually exclusive or imply one another, we 
can build a decision-tree-based memoization device that will effectively have 
\emph{most-specific} semantics -- an analog of best-fit semantics in predicate 
dispatching~\cite{ErnstKC98}.

Imagine that the predicates with the numbers $2i$ and $2i+1$ are mutually exclusive and 
each imply the value of the predicate with number $i$ i.e. $\forall x \in \mathsf{Domain}(P)$
\begin{eqnarray*}
P_{2i+1}(x)\rightarrow P_i(x) \wedge P_{2i}(x)\rightarrow P_i(x) \wedge \neg(P_{2i+1}(x) \wedge P_{2i}(x))
\end{eqnarray*}
\noindent
An example of predicates that satisfy this condition are class membership tests 
where the truth of testing membership in a derived class implies the truth of 
testing membership in its base class. 

The following decision-tree based memoization device will execute the statement 
$s_i$ associated with the \emph{most-specific} predicate $P_i$ (i.e. the 
predicate that implies all other predicates true on $x$) that evaluates to true 
or will skip the entire statement if none of the predicates is true on $x$.

\begin{lstlisting}
switch (int& jump_to = jump_targets[x]) {
default:
    if (P1(x)) {
        if (P2(x)) {
            if (P4(x)) { jump_to = 4; case 4: s4; } else
            if (P5(x)) { jump_to = 5; case 5: s5; } 
            jump_to = 2; case 2: s2;
        } else
        if (P3(x)) {
            if (P6(x)) { jump_to = 6; case 6: s6; } else
            if (P7(x)) { jump_to = 7; case 7: s7; } 
            jump_to = 3; case 3: s3;
        }
        jump_to = 1; case 1: s1;
    } else { jump_to = 0; case 0: ; }
}
\end{lstlisting}

\noindent
Our library solution prefers the simpler cascading-if approach only because the necessary 
structure of the code can be laid out directly with macros. A compiler solution 
will use the decision-tree approach whenever possible to lower the cost of the 
first match from linear in case's number to logarithmic as seen in Figure\ref{fig:DCastVis1}.

%When the predicates do not satisfy the implication or mutual exclusion properties 
%mentioned above, a compiler of a language based on predicate dispatching would 
%typically issue an ambiguity error. Some languages might choose to resolve it 
%according to lexical or some other ordering. In any case, the presence of 
%ambiguities or their resolution has nothing to do with memoization device 
%itself. The latter only helps optimize the execution once a particular choice of 
%semantics has been made and code implementing it has been laid out.

The main advantage of the memoization device is that it can be built around 
almost any code, providing that we can re-establish the invariants, guaranteed 
by sequential execution. Its main disadvantage is the size of the hash table 
that grows proportionally to the number of different values seen. Fortunately, 
the values can often be grouped into equivalence classes that do not change the 
outcome of the predicate. The map can then associate the equivalence class of a 
value with a target instead of associating the value with it. 

In application to type switching, the idea is to use the memoization device to 
learn the outcomes of type inclusion tests (with \code{dynamic_cast} used as a predicate), 
thus avoiding calls to it on subsequent runs. It is easy to see that objects can 
be grouped into equivalence classes based on their most-derived type without 
affecting the results of predicates -- the outcome of each type inclusion test will be the 
same on all the objects from the same equivalence class. We can use the 
address of class' \code{type_info} object obtained in constant time with 
\code{typeid()} operator as a unique identifier of each most-derived type. 
Presence of multiple \code{type_info} objects for the same class, as is often 
the case when dynamic linking is involved, is not a problem as we would 
effectively split a single equivalence class into multiple ones. This in fact 
would have been a solution if we were only interested in class membership. More 
often than not, however, we will be interesting in obtaining a reference to the  
target type of the subject and we saw in \textsection\ref{sec:casts} that proper 
this-pointer adjustments depend not only on the most-derived type, but also on 
target type and most importantly -- path to the subject's static type from the 
most-derived type in the inheritance graph. Ideally we would like to have 
different equivalence classes per different paths from object's most-derived 
type to its static types, but there seem to be no easy way of identifying them 
given just an object descriptor.

\subsection{Virtual Table Pointers}
\label{sec:vtp}

%In this section we show that under certain conditions the compiler cannot share 
%the same virtual tables between different classes or their subobjects. This 
%allows us to use virtual table pointers to \emph{uniquely} identify the 
%subobjects within the most-derived class.

A class that declares or inherits a virtual function is called a 
\emph{polymorphic class}. The C++ standard~\cite{C++11} does not prescribe any 
specific implementation technique for virtual function dispatch.
However, in practice, all C++ compilers use a strategy based on so-called
virtual function tables (or vtables for short) for efficient disptach. 
The vtable is part of the reification of a polymorphic class type.  
C++ compilers embed a pointer to a vtable (vtbl-pointer for short) in every object of
polymorphic class type. CFront, the first C++ compiler, puts the vtbl-pointer
at the end of an object. The so-called ``common vendor C++ ABI''\cite{C++ABI}, 
further referred to as \emph{C++ ABI} when not indicated otherwise, requires the 
vtbl-pointer to be at offset 0 of an object. The following compilers comply with 
the C++ ABI: GCC (3.x and up); Clang and llvm-g++; Linux versions of Intel and 
HP compilers, and compilers from ARM. % Phrase from http://morpher.com/documentation/articles/abi/ 
We do not have access to the unpublished Microsoft ABI, but we have
experimental evidence that Microsoft's C++ compiler also puts the vtbl-pointer 
at the start of an object. 

While the exact offset of the vtbl-pointer within the object is not important 
for our discussion, it is important to realize that every object of a static 
type \code{S*} or \code{S&} pointed to or referenced by a polymorphic class 
\code{S} will have a vtbl-pointer at a predefined offset. Such offset may be 
different for different static types \code{S}, in which case the compiler will 
know at which offset in type \code{S} the vtbl-pointer is located. 
For a library implementation we assume presence of a function 
\code{template <typename S> intptr_t vtbl(const S* s);}
that returns the address of the virtual table corresponding to the subobject 
referenced to by \code{s}. Such a function can be trivially implemented for the 
common C++ ABI, where the vtbl-pointer is always at offset 0:

\begin{lstlisting}
template <typename S> std::intptr_t vtbl(const S* s) {
    static_assert(std::is_polymorphic<S>::value, "error");
    return *reinterpret_cast<const std::intptr_t*>(s);
}
\end{lstlisting}

Consider a repeated multiple inheritance hierarchy from 
Figure~\ref{fig:objlayout}(1). Each of the \code{vtbl} fields shown in 
Figure~\ref{fig:objlayout}(2) will hold a vtbl-pointer referencing a group of 
virtual methods known in object's static type. Figure~\ref{fig:vtbl}(1) shows a 
typical layout of virtual function tables together with objects it points to for 
classes \code{B} and \code{D}.

\begin{figure}[htbp]
  \centering
    \includegraphics[width=0.49\textwidth]{v-table.pdf}
  \caption{VTable layout with and without RTTI}
  \label{fig:vtbl}
\end{figure}

Entries in the vtable to the right of the address pointed to by a vtbl-pointer 
represent pointers to functions, while entries to the left of it represent 
various additional fields like: pointer to class' type information, offset to 
top, offsets to virtual base classes etc. In many implementations, this-pointer 
adjustments required to properly dispatch the call were stored in the vtable 
along with function pointers. Today most of the implementations prefer to use 
\emph{thunks} or \emph{trampolines} -- additional entry points to a function, 
that adjust this-pointer before transferring the control to the function, -- 
which was shown to be more efficient~\cite{}. Thunks in general may only be 
needed when its virtual function gets overriden. In such case the overriden 
function may be called via pointer to base class or a pointer to derived class, 
which may not be at the same offset in the actual object.

The intuition behind our proposal is to use the values of vtbl-pointers stored 
inside the object to uniquely identify the subobject in it. There are several 
problems with the approach however. First of all the same vtbl-pointer is 
usually shared by multiple types, for example, the first vtbl-pointer in 
Figure~\ref{fig:objlayout}(2) will be shared by objects of static type 
\code{Z*}, \code{A*}, \code{B*} and \code{D*}. This is not a problem for our 
purpose, because the subobjects of these types will be at the same offset in the 
most-derived object. Secondly, and more importantly, however, there are legitimate 
optimizations that let the compiler share the same vtable among multiple 
subobjects of often unrelated types.

Generation of the \emph{Run-Time Type Information} (or RTTI for short) can 
typically be disabled with a compiler switch and the Figure~\ref{fig:vtbl}(2) 
shows the same vtable layouts once the RTTI has been disabled. Since neither 
\code{baz} nor \code{foo} were overriden, the prefix of the vtable for the 
\code{C} subobject in \code{D} is exactly the same as the vtable for its 
\code{B} subobject, the \code{A} subobject of \code{C} or the entire vtable of 
\code{A} and \code{B} classes. Such layout, for example, is produced by 
Microsoft Visual C++ 11 when the command-line option \code{/GR-} is specified. 
Visual C++ compiler has been known to unify code identical on binary level, 
which in some cases may result in sharing of the same vtable between unrelated 
classes (e.g. when virtual functions are empty).

%C++ supports multiple-inheritance of two kinds: repeated and virtual (shared). 
%\emph{Repeated inheritance} creates multiple independent subobjects of the same 
%type within the most-derived type. \emph{Virtual inheritance} creates only one 
%shared subobject, regardless of the inheritance paths. Consequently,
%it is not sufficient to talk only about the 
%static and dynamic types of an object -- one has to talk about a 
%\emph{subobject} of a certain static type accessible through a given inheritance 
%path within a dynamic type. 

We now would like to show more formally that in the presence of RTTI, a C++ ABI 
compliant implementation will always have all the vtbl-pointers different. To do 
so, we need look closer at the notion of subobject, which has been formalized 
before~\cite{RF95,WNST06,RDL11}. We follow here the presentation of Ramamanandro 
et al~\cite{RDL11}.

\subsection{Subobjects}
\label{sec:subobj}

In a given program $P$, a class $B$ is a \emph{direct repeated base class} of 
$D$ if $B$ is mentioned in the list of base classes of $D$ without the 
\code{virtual} keyword ($D \prec_R B$). Similarly, a class $B$ is a \emph{direct 
shared base class} of $D$ if $B$ is mentioned in the list of base classes of $D$ 
with the \code{virtual} keyword ($D \prec_S B$). A reflexive transitive closure 
of these relationships $\preceq^*=(\prec_R \cup \prec_S)^*$ defines the 
\emph{subtyping} relation on types of program $P$.
A base class \emph{subobject} of a given \emph{complete object} is represented by a pair 
$\sigma = (h,l)$ with $h \in \{\mathsf{Repeated},\mathsf{Shared}\}$ representing the 
kind of inheritance (single inheritance is $\mathsf{Repeated}$ with one base class) and $l$ 
representing the path in a non-virtual inheritance graph.
A predicate $C\leftY\sigma\rightY A$ states that $\sigma$ 
designates a subobject of static type $A$ within the most-derived object of 
type $C$. More formally:

\begin{mathpar}
\inferrule
{}
{C\leftY(\mathsf{Repeated},C::\epsilon)\rightY C}

\inferrule
{C \prec_R B \\ B\leftY(\mathsf{Repeated},l)\rightY A}
{C\leftY(\mathsf{Repeated},C::l)\rightY A}

\inferrule
{C \prec_S B \\ B\leftY(h,l)\rightY A}
{C\leftY(\mathsf{Shared},l)\rightY A}
\end{mathpar}

\noindent
$\epsilon$ indicates an empty path, but we will generally omit it in writing 
when understood from the context. In case of repeated inheritance in 
Figure~\ref{fig:inheritance}(1), an object of the most-derived class \code{D} 
will have the following $\mathsf{Repeated}$ subobjects:
\code{D::C::Y}, 
\code{D::B::A::Z}, 
\code{D::C::A::Z}, 
\code{D::B::A}, 
\code{D::C::A}, 
\code{D::B}, 
\code{D::C}, 
\code{D}.
Similarly, in case of virtual inheritance in the same expample, an object of the 
most-derived class \code{D} will have the following $\mathsf{Repeated}$ subobjects:
\code{D::C::Y}, 
\code{D::B}, 
\code{D::C}, 
\code{D}
as well as the following $\mathsf{Shared}$ subobjects: 
\code{D::A::Z}, 
\code{D::Z}, 
\code{D::A}.

It is easy to show by structural induction on the above definition, that 
$C\leftY\sigma\rightY A \implies \sigma=(h,C::l_1) \wedge \sigma=(h,l_2::A::\epsilon)$, 
which simply means that any path to a subobject of static type $A$ within the 
most-derived object of type $C$ starts with $C$ and ends with $A$. This 
objservation shows that $\sigma_\bot = (\mathsf{Shared},\epsilon)$ does not 
represent a valid subobject. If $\Sigma_P$ is the domain of all subobjects in 
the program $P$ extended with $\sigma_\bot$, then a \emph{cast} operation can be 
understood as a function $\delta : \Sigma_P \rightarrow \Sigma_P$. We use 
$\sigma_\bot$ to indicate an impossibility of a cast. The fact that $\delta$ is 
defined on subobjects as opposed to actual run-time values reflects the 
non-coercive nature of the operation -- i.e. the underlain value remains the 
same. Any implementation of such a function must thus satisfy the following 
condition:
\begin{eqnarray*}
\delta(\sigma_1) = \sigma_2 \wedge C \leftY\sigma_1\rightY A \implies C \leftY\sigma_2\rightY B
\end{eqnarray*}
\noindent
i.e. the most-derived type of the value does not change during casting, only the way 
we reference it does. We refer to $A$ as the \emph{source type} and $\sigma_1$ 
as the \emph{source subobject} of the cast, while to $B$ as the \emph{target 
type} and to $\sigma_2$ as the \emph{target subobject} of it. The type $C$ is 
the most-derived type of the value being casted.
The C++ semantics states more requirement to the implementation of $\delta$: 
e.g. $\delta(\sigma_\bot) = \sigma_\bot$ etc. but their precise modeling is out 
of the scope of this discussion. We would only like to point out here that since 
the result of the cast does not depend on the actual value and only on the 
source subobject and the target type, we can memoize the outcome of a cast on 
one instance in order to apply its results to another.

%Figure~\ref{fig:objlayout}(2)
%$Z\leftY(\mathsf{Repeated},      [Z])\rightY Z$,
%$A\leftY(\mathsf{Repeated},    [A,Z])\rightY Z$,
%$B\leftY(\mathsf{Repeated},  [B,A,Z])\rightY Z$,
%$D\leftY(\mathsf{Repeated},[D,B,A,Z])\rightY Z$,
%$C\leftY(\mathsf{Repeated},  [C,A,Z])\rightY Z$,
%$D\leftY(\mathsf{Repeated},[D,C,A,Z])\rightY Z$,
%$Y\leftY(\mathsf{Repeated},      [Y])\rightY Y$,  
%$C\leftY(\mathsf{Repeated},    [C,Y])\rightY Y$,
%$D\leftY(\mathsf{Repeated},  [D,C,Y])\rightY Y$,
%$A\leftY(\mathsf{Repeated},      [A])\rightY A$, 
%$B\leftY(\mathsf{Repeated},    [B,A])\rightY A$,
%$D\leftY(\mathsf{Repeated},  [D,B,A])\rightY A$,
%$C\leftY(\mathsf{Repeated},    [C,A])\rightY A$,
%$D\leftY(\mathsf{Repeated},  [D,C,A])\rightY A$,
%$B\leftY(\mathsf{Repeated},      [B])\rightY B$,
%$D\leftY(\mathsf{Repeated},    [D,B])\rightY B$,
%$C\leftY(\mathsf{Repeated},      [C])\rightY C$,
%$D\leftY(\mathsf{Repeated},    [D,C])\rightY C$,
%$D\leftY(\mathsf{Repeated},      [D])\rightY D$,
%
%Figure~\ref{fig:objlayout}(3)
%$Z\leftY(\mathsf{Repeated},      [Z])\rightY Z$,
%$A\leftY(\mathsf{Repeated},    [A,Z])\rightY Z$,
%$B\leftY(\mathsf{Shared},    [B,A,Z])\rightY Z$,
%$C\leftY(\mathsf{Shared},    [C,A,Z])\rightY Z$,
%$D\leftY(\mathsf{Shared},    [D,A,Z])\rightY Z$,
%$D\leftY(\mathsf{Shared},      [D,Z])\rightY Z$,
%$Y\leftY(\mathsf{Repeated},      [Y])\rightY Y$,  
%$C\leftY(\mathsf{Repeated},    [C,Y])\rightY Y$,
%$D\leftY(\mathsf{Repeated},  [D,C,Y])\rightY Y$,
%$A\leftY(\mathsf{Repeated},      [A])\rightY A$, 
%$B\leftY(\mathsf{Shared},      [B,A])\rightY A$,
%$C\leftY(\mathsf{Shared},      [C,A])\rightY A$,
%$D\leftY(\mathsf{Shared},      [D,A])\rightY A$,
%$B\leftY(\mathsf{Repeated},      [B])\rightY B$,
%$D\leftY(\mathsf{Repeated},    [D,B])\rightY B$,
%$C\leftY(\mathsf{Repeated},      [C])\rightY C$,
%$D\leftY(\mathsf{Repeated},    [D,C])\rightY C$,
%$D\leftY(\mathsf{Repeated},      [D])\rightY D$,

\subsection{Uniqueness of vtbl-pointers under the C++ ABI}
\label{sec:uniq}

%A class that declares or inherits a virtual function is called a 
%\emph{polymorphic class}~\cite[\textsection 10.3]{C++11}.  We say that
%a class is \emph{dynamic} \cite{C++ABI} if it requires a virtual table pointer 
%(because it or its bases have one or more virtual member functions or
%virtual base classes). 
%A \emph{virtual table pointer} (vtbl-pointer)
%is a data-member of an object pointing to the object's dynamic type vtable.
%In addition to dispatching virtual function calls, it is used to access
%virtual base class subobjects, and to 
%access \emph{RunTime Type Identification} (RTTI) data.
%An object of a class
%type with multiple inheritance may contain several vtbl-pointers
%(included in its subobjects). We assume that for every expression of
%static type \code{T} (a dynamic class type), a C++ compiler provides
%access to the vtbl-pointer of the (sub)object designated by that 
%expression (or at least documents the position of that
%pointer within an object). For the common vendor C++ ABI, we can state:
%\begin{lemma}
%  In an object layout that adheres to the ``common vendor C++ ABI'', 
%  an object of a polymorphic class always has a virtual table pointer
%  at offset 0.
%\label{lem:vtbl}
%\end{lemma}

%\noindent
%With no further assumption, we cannot use a vtable to uniquely identify
%its dynamic type or those of its subobjects. The reason is that a popular 
%compression technique is to share compiler-generated data, and not exclusively
%between subobjects in the class hierarchy.
%Use of such optimization will violate the 
%uniqueness of vtbl-pointers; however, we show below that in the presense of 
%runtime type identification information (RTTI), we have a form of injectivity
%that is sufficient for our needs.
Given a reference \code{a} to polymorphic type \code{A} that points to a subobject 
$\sigma$ of the most-derived type \code{C} (i.e. $C\leftY\sigma\rightY A$ is 
true), we will use the traditional field-access notion \code{a.vtbl} to refer to 
the virtual table of that subobject. The exact structure of the virtual table as 
mandated by the common vendor C++ ABI is immaterial for this discussion, but we 
mention a few fields that are important for the reasoning~\cite[\textsection 2.5.2]{C++ABI}:

\begin{itemize}
\setlength{\itemsep}{0pt}
\setlength{\parskip}{0pt}
\item \code{rtti(a.vtbl)}: the \emph{typeinfo pointer} points to the typeinfo 
      object used for RTTI. It is always present and is shown as the first field 
      to the left of any vtbl-pointer in Figure~\ref{fig:vtbl}(1).
\item \code{off2top(a.vtbl)}: the \emph{offset to top} holds the displacement to 
      the top of the object from the location within the object of the 
      vtbl-pointer that addresses this virtual table. It is always present and 
      is shown as the second field to the left of any vtbl-pointer in 
      Figure~\ref{fig:vtbl}(1). The numeric value shown indicates the actual 
      offset based on the object layout from Figure~\ref{fig:objlayout}(2).
\item \code{vbase(a.vtbl)}: \emph{Virtual Base (vbase) offsets} are used to access 
      the virtual bases of an object. Such an entry is required for each virtual 
      base class. None are shown in our example in Figure~\ref{fig:vtbl}(1) 
      since it discussed repeated inheritance, but they will occupy further 
      entries to the left of the vtbl-pointer, when present.
\end{itemize}

\noindent
We also use the notation $\mathit{offset}(\sigma)$ to refer to the offset of the 
given subobject $\sigma$ within $C$, known by the compiler.

\begin{theorem}
In an object layout that adheres to the common vendor C++ ABI with enabled RTTI, 
equality of vtbl-pointers of two objects of the same static type implies that 
they both belong to subobjects with the same inheritance path in the same most-derived class.

\noindent
$\forall a_1, a_2 : A\ |\ a_1\in C_1\leftY\sigma_1\rightY A \wedge a_2\in C_2\leftY\sigma_2\rightY A $ \\ 
$a_1.\textit{vtbl} = a_2.\textit{vtbl} \Rightarrow C_1 = C_2 \wedge \sigma_1 = \sigma_2$
\label{thm:vtbl}
\end{theorem}
\begin{proof}
Let us assume first $a_1.\textit{vtbl} = a_2.\textit{vtbl}$ but $C_1 \neq C_2$. In this case we 
have \code{rtti}$(a_1.\textit{vtbl}) = $\code{rtti}$(a_2.\textit{vtbl})$. By definition 
\code{rtti}$(a_1.\textit{vtbl}) = C_1$ while \code{rtti}$(a_2.\textit{vtbl}) = C_2$, which 
contradicts that $C_1 \neq C_2$. Thus $C_1 = C_2 = C$.

Let us assume now that $a_1.\textit{vtbl} = a_2.\textit{vtbl}$ but $\sigma_1 \neq \sigma_2$. 
Let $\sigma_1=(h_1,l_1),\sigma_2=(h_2,l_2)$ 

If $h_1 \neq h_2$ then one of them refers to a virtual base while the other to a 
repeated one. Assuming $h_1$ refers to a virtual base, \code{vbase}$(a_1.\textit{vtbl})$ 
has to be defined inside the vtable according to the ABI, while 
\code{vbase}$(a_2.\textit{vtbl})$ -- should not. This would contradict again that both 
$vtbl$ refer to the same virtual table.

We thus have $h_1 = h_2 = h$. If $h = \mathsf{Shared}$ then there is only one path to 
such $A$ in $C$, which would contradict $\sigma_1 \neq \sigma_2$. 
If $h = \mathsf{Repeated}$ then we must have that $l_1 \neq l_2$. In this case let $k$ be 
the first position in which they differ: 
$l_1^j=l_2^j \forall j<k \wedge l_1^k\neq l_2^k$. Since our class $A$ is a base 
class for classes $l_1^k$ and $l_2^k$, both of which are in turn base classes of 
$C$, the object identity requirement of C++ requires that the relevant subobjects 
of type $A$ have different offsets within class $C$: 
$\mathit{offset}(\sigma_1)\neq \mathit{offset}(\sigma_2)$ However 
$\mathit{offset}(\sigma_1)=$\code{off2top}$(a_1.\textit{vtbl})=$\code{off2top}$(a_2.\textit{vtbl})=\mathit{offset}(\sigma_2)$ 
since $a_1.\textit{vtbl} = a_2.\textit{vtbl}$, which contradicts that the offsets are different.
\end{proof}

\noindent
Conjecture in the other direction is not true in general as there may be 
duplicate vtables for the same type present at run-time. This happens in 
many C++ implementations in the presence of \emph{Dynamically Linked Libraries} 
(or DLLs for short) as the same class compiled into executable and DLL it loads 
may have identical vtables inside the executable's and DLL's binaries.

Note also that we require both static types to be the same. Dropping this 
requirement and saying that equality of vtbl-pointers also implies equality of 
the static types is not true in general because a derived class can share the 
vtbl-pointer with its primary base class. The theorem 
can be reformulated, however, stating that one static type will necessarily be a 
subtype of the other. The current formulation is sufficient for our purposes, 
while reformulation will require more elaborate discussion of the algebra 
of subobjects~\cite{RDL11}, which we touch only briefly.

%\begin{corollary}
%In an object layout that adheres to the common vendor C++ ABI with enabled RTTI, 
%the offset between two same subobjects of two different objects of the same 
%most-derived type is the same.
%$\forall c_1, c_2 : C\ |\ c_1,c_2 \in C\leftY\sigma_1\rightY C $ \\ 
%$a_1.\textit{vtbl} = a_2.\textit{vtbl} \Rightarrow C_1 = C_2 \wedge \sigma_1 = \sigma_2$
%
%
%Results of \code{dynamic_cast} can be reapplied to a different instance from 
%within the same subobject. 
%
%$\forall A,B \forall a_1, a_2 : A\ |\ a_1.\textit{vtbl} = a_2.\textit{vtbl} \Rightarrow$ \\
%\code{dynamic_cast<B>}$(a_1).\textit{vtbl}_j = $\code{dynamic_cast<B>}$(a_2).\textit{vtbl}_j \vee$ \\
%\code{dynamic_cast<B>}$(a_1)$ throws $\wedge$ \code{dynamic_cast<B>}$(a_2)$ throws.
%\label{crl:vtbl}
%\end{corollary}

%\noindent
During construction and deconstruction of 
an object, the value of a given vtbl-pointer may change. In particular, 
that value will reflect the fact that the most-derived type of the object is the type of its 
fully constructed part only. This, however, does not affect our reasoning, as during 
such transition we also treat the object to have the type of its fully 
constructed base only. Such interpretation is in line with the C++ semantics for 
virtual function calls and the use of RTTI during construction and destruction of an 
object. Once the complete object is fully constructed, the value of the 
vtbl-pointer will remain the same for the lifetime of the object.

\subsection{Vtable Pointer Memoization}
\label{sec:vtblmem}

%The memoization device can almost immediately be used for multi-way type testing by 
%using \code{dynamic_cast<Ti>} as a predicate $P_i$. This cannot be considered a 
%type switching solution, however, as one would expect to also have a reference 
%to the uncovered type. Using a \code{static_cast<Ti>} upon successful type test 
%would have been a solution if we did not have multiple inheritance. It certainly 
%can be used as such in languages with only single inheritance. For the fully 
%functional C++ solution, we combine the memoization device with the properties 
%of virtual table pointers into a \emph{Vtable Pointer Memoization} technique.

The C++ standard requires that information about types be available at run time 
for three distinct purposes:

\begin{itemize}
\setlength{\itemsep}{0pt}
\setlength{\parskip}{0pt}
\item to support the \code{typeid} operator,
\item to match an exception handler with a thrown object, and
\item to implement the \code{dynamic_cast} operator.
\end{itemize}

\noindent
and if any of these facilities are used in a program that was compiled with 
disabled RTTI, the compiler will emit an error or at least a warning. Some 
compilers (e.g. Visual C++) additionally let a library check presence of RTTI 
through a predefined macro, thus letting it report an error if its dependence on 
RTTI cannot be satisfied. Since our solution relies on \code{dynamic_cast} to 
perform casts at run-time, we implicitly rely on the presence of RTTI and thus 
fall into the setting that guarantees the preconditions of Theorem~\ref{thm:vtbl}.
Since all the objects that will be coming through a particular type switch will 
have the same static type, the theorem guarantees that different vtbl-pointers 
will correspond to different subobjects. The idea is thus to group them 
accordingly to the value of their vtbl-pointer and associate both jump target 
and the required offset with it through memoization device:

\begin{lstlisting}
typedef pair<ptrdiff_t,size_t> type_switch_info;
static unordered_map<intptr_t, type_switch_info> jump_targets;
type_switch_info& info = jump_targets[vtbl(x)];
const void*       tptr; 
switch (info.second) ...
\end{lstlisting}

\noindent
The code for the $i^{th}$ case now evaluates the required offset on the first 
entry and associates it and the target with the vtbl-pointer of the subject.
The call to \code{adjust_ptr<Ti>} re-establishes the invariant that 
\code{match} is a reference to type \code{Ti} of the subject \code{x}.
%The condition of the inner if-statement is only needed to implement the 
%sequential all-fit semantics and can be removed when fall-through behavior is 
%not required.

\begin{lstlisting}
    if (tptr = dynamic_cast<const Ti *>(x)) {
        if (info.second == 0) { // supports fall-through
            info.first  = intptr_t(tptr)-intptr_t(x); // offset
            info.second = @$i$@; // jump target
        }
case @$i$@: // @$i$@ is a constant here - clause's position in switch
        auto match = adjust_ptr<Ti>(x,info.first); 
        si;
    }
\end{lstlisting}

\noindent
%The use of dynamic cast makes a huge difference in comparison to the use of 
%static cast we dismissed above. First of all the C++ type system is much more 
%restrictive about the static cast and many cases where it is not allowed can 
%still be handled by dynamic cast. Examples of these include downcasting from an 
%ambiguous base class or cross-casting between unrelated base classes.
%
%An important benefit we get from this optimization is that the number of values 
%stored in the hash table is on the order $O(|A|)$, where $A$ represents the 
%static type of an object, while $|A|$ represents the number of classes directly 
%or indirectly derived from $A$. The linear coefficient of the big-o notation 
%reflects possibly multiple vtbl-pointers in derived classes due to multiple 
%inheritance.

Class \code{std::unordered_map} provides amortized constant time access on 
average and linear in the amount of elements in the worst case. We show in the 
next section that most of the time we will be bypassing traditional access to 
its elements. We need this extra optimization because, as-is, the type switch is 
still about 50\% slower than the visitor design pattern.

%Note that we can apply the reasoning of \textsection\ref{sec:memdev} and change 
%the first-fit semantics of the resulting match statement into a best-fit 
%semantics simply by changing the underlying cascading-if structure with decision 
%tree. A compiler implementation of a type switch based on Vtable Pointer 
%Memoization will certainly take advantage of this optimization to cut down the 
%cost of the first run on a given vtbl-pointer, when the actual memoization happens.

%\subsubsection{Structure of Virtual Table Pointers}
%\label{sec:sovtp}

\subsection{Minimization of Conflicts}
\label{sec:moc}

Virtual table pointers are not constant values and are not even guaranteed to be 
the same between different runs of the application, because techniques like 
\emph{address space layout randomization} or \emph{rebasing} of the module are 
likely to change them. The relative distance between them will remain the same 
as long as they come from the same module.

Knowing that vtbl-pointers point into an array of function pointers, we should 
expect them to be aligned accordingly and thus have a few lowest bits as zero. 
Moreover, since many derived classes do not introduce new virtual functions, 
the size of their virtual tables remains the same. When allocated sequentially 
in memory, we can expect a certain number of lowest bits in the vtbl-pointers 
pointing to them to be the same.
These assumptions, supported by actual observations, has made virtual table 
pointers of classes related by inheritance ideally suitable for hashing -- the 
values obtained by throwing away the common bits on the right were compactly 
distributed in small disjoint ranges. We use them to address a cache 
built on top of the hash table in order to eliminate a hash table lookup in most 
of the cases.

Let $\Xi$ be the domain of integral representations of pointers. Given a cache 
with $2^k$ entries, we use a family of hash functions $H_{kl} : \Xi \rightarrow [0..2^k-1]$ 
defined as $H_{kl}(v)=v/2^l \mod 2^k$ to index the cache, where $l \in [0..32]$ 
(assuming 32 bit addresses) is a parameter modeling the number of common bits on 
the right. Division and modulo operations are implemented with bit-shifts since 
denominator in each case is a power of 2, which in turn explains the choice of 
the cache size.

Given a hash function $H_{kl}$, pointers $v'$ and $v''$ are said to be in 
\emph{conflict} when $H_{kl}(v')=H_{kl}(v'')$. For a given set of pointers 
$V \in 2^{\Xi}$ we can always find such $k$ and $l$ that $H_{kl}$ will render no  
conclicts between its elements, however the required cache size $2^k$ can be too 
large to justify the use of memory. The value $K$ such that $2^{K-1} < |V| \leq 2^K$ 
is the smallest value of $k$ under which absence of conflicts is still possible. 
We thus allow $k$ to only vary in range $[K,K+1]$ to ensure that the cache size 
is never more than 4 times bigger than the minimum required cache size.

Given a set $V = \{v_1, ... , v_n\}$, we would like to find a pair of parameters 
$(k,l)$ such that $H_{kl}$ will render the least number of conflicts on the 
elements of $V$. Since for a fixed set $V$, parameters $k$ and $l$ vary in a 
finite range, we can always find the optimal $(k,l)$ by trying all the
combinations. Let $H_{kl}^V : V \rightarrow [0..2^k-1]$ be the hash function 
corresponding to such optimal $(k,l)$ for the set $V$. 

In our setting, set $V$ represents the set of vtbl-pointers coming through a 
particular type switch. While the exact values of these pointers are not known 
till run-time, their offset from the module's base address is. This can often be 
sufficient to at least estimate optimal $k$ and $l$ in a compiler setting. In 
the library setting we estimate them by recomputing them after a given amount of 
actual collisions happened in cache.

When $H_{kl}^V$ is injective (renders 0 conflicts on $V$), the frequency of any 
given vtbl-pointer $v_i$ coming through the type switch does not affect the 
overal performance of the switch. However when $H_{kl}^V$ is not injective, we 
would prefer the conflict to happen on less frequent vtbl-pointers.
Given a probability $p_i$ of each vtbl-pointer $v_i \in V$ we can compute the 
probability of conflict rendered by a given $H_{kl}$:

\begin{eqnarray*}
P_{kl}(V)=\sum\limits_{j=0}^{2^k-1}(\sum\limits_{v_{i} \in V^j_{kl}}p_{i})(1-\frac{\sum\limits_{v_i \in V^j_{kl}}p_i^2}{(\sum\limits_{v_{i} \in V^j_{kl}}p_{i})^2})
\end{eqnarray*}

\noindent 
where $V^j_{kl}=\{v \in V | H_{kl}(v)=j\}$. In this case, the optimal hash 
function $H_{kl}^V$ can similarly be defined as $H_{kl}$ that minimizes the 
above probability of conflict on $V$.

Probabilities $p_i$ can be estimated in a compiler settings through profiling, 
while in a library setting we let the user enable tracing of frequencies of 
each vtbl-pointer. This introduces an overhead of an increment into the critical 
path of execution, and according to our tests degrades the overal performance by 1-2\%. 
By default, we do not enable frequency tracing, however, because the significant 
drop in the number of actual collisions was not reflected in a noticeable 
decrease in execution time. This was because the total number of actual 
collisions, even in non-frequency based caching, was much smaller than the 
number of successful cache hits.

Assuming the uniform distribution of $v_i$ and substituting the probability 
$p_i=\frac{1}{n}$, where $n=|V|$, into the above formula we will get:

\begin{eqnarray*}
P_{kl}(V)=\sum\limits_{j=0}^{2^k-1}[|V^j_{kl}| \neq 0]\frac{|V^j_{kl}|-1}{n}
\end{eqnarray*}

\noindent
The value $|V^j_{kl}|-1$ represents the amount of ``extra'' pointers mapped into 
the entry $j$ in cache and thus $H_{kl}^V$ obtained by minimization of 
probability of conflict is the same as our original $H_{kl}^V$ minimizing the 
number of conflicts. An important observation here is that the exact location of 
these ``extra'' vtbl-pointers is not important, only the total number $m$ of 
them is. The probability of conflict under uniform distribution of $v_i$ is thus 
always going to have form $\frac{m}{n}$, where $0 \le m < n$.

%Depending on the number of actual collisions that happen in the cache, our 
%vtable pointer memoization technique can come close to, and even outperform, the 
%visitor design pattern. The numbers are, of course, averaged over many runs as 
%the first run on every vtbl-pointer will take an amount of time as shown in 
%Figure\ref{fig:DCastVis1}. We did however test our technique on real code and 
%can confirm that it does perform well in the real-world use cases.

%The information about jump targets and necessary offsets is just an example of 
%information we might want to be able to associate with, and access via, virtual 
%table pointers. Our implementation of \code{memoized_cast}~\cite[\textsection 9]{TR}, for example, 
%effectively reuses this general data structure with a different type of element 
%values. We thus created a generic reusable class \code{vtblmap<T>} that maps 
%vtbl-pointers to elements of type T. We will refer to the combined cache and 
%hash-table data structure, extended with the logic for minimizing conflicts 
%presented below, as a \emph{vtblmap} data structure.

%\subsubsection{Minimization of Conflicts}
%\label{sec:moc}

%The small number of cycles that the visitor design pattern needs to uncover a 
%type does not let us put too sophisticated cache indexing mechanisms into the 
%critical path of execution. This is why we limit our indexing function to shifts 
%and masking operations as well as choose the size of the cache to be a power of 2.
%
%As usual, by \emph{conflict} we mean a situation in which two or more keys 
%(vtbl-pointers here) are mapped to the same location in cache using a given 
%indexing function. The presence of conflicts means that accessing values of \code{vtblmap<T>} 
%associated with some vtbl-pointers may result in slower lookup of the element 
%inside the underlying hash table relative to a direct fetch from the cache.
%This `slower' lookup, as we mentioned, is constant on average and linear in the 
%size of the hash map in the worst case.
%
%Given $n$ vtbl-pointers we can always find a cache size that will render no 
%conflicts between them. The necessary size of such a cache, however, can be too 
%big to justify the use of memory. This is why, in our current implementation, we 
%always consider only 2 different cache sizes: $2^k$ and $2^{k+1}$ where 
%$2^{k-1} < n \leq 2^k$. This guarantees that the cache size is never more than 4 
%times bigger than the minimum required cache size.
%
%During our experiments, we noticed that often the change in the smallest 
%different bit happens only in a few vtbl-pointers, which was effectively 
%cutting the available cache space in half. To overcome this problem, we let the 
%number of bits by which we shift the vtbl-pointer vary further and compute it in 
%a way that minimizes the number of conflicts.
%
%To avoid doing any computations in the critical path, \code{vtblmap} only 
%recomputes the optimal shift and the size of the cache when an actual collision 
%happens. In order to avoid constant recomputations when conflicts are unavoidable, 
%we only reconfigure the optimal parameters if 
%the number of vtbl-pointers in the \code{vtblmap} has increased since the last 
%recomputation. Since the number of vtbl-pointers is of the order $O(|A|)$, where 
%$A$ is the static type of all vtbl-pointers coming through a \code{vtblmap}, the 
%restriction assures that reconfigurations will not happen infinitely often.
%
%To minimize the number of recomputations even further, our library communicates 
%to the \code{vtblmap}, through its constructor, the number of case clauses in 
%the underlying match statement. We use this number as an estimate of the expected 
%size of the \code{vtblmap} and pre-allocate the cache according to this estimated 
%number. The cache is still allowed to grow based on the actual number of 
%vtbl-pointers that comes through a \code{vtblmap}, but it never shrinks from the
%initial value. This improvement significantly minimizes the number of collisions 
%at early stages, as well as the number of possibilities we have to consider 
%during reconfiguration.
%
%The above logic always chooses the configuration that renders 
%no conflicts, when such a configuration is possible during recomputation of 
%optimal parameters. When this is not possible, it is natural to prefer collisions 
%to happen on less-frequent vtbl-pointers.

%We studied the frequency of vtbl-pointers that come through various match statements
%of a C++ pretty-printer that we implemented on top of the Pivot 
%framework~\cite{Pivot09} using our pattern-matching library. We ran the 
%pretty-printer on a set of C++ standard library headers and then ranked all the  
%classes from the most-frequent to the least-frequent ones, on average. The 
%resulting probability distribution resembled the power-law distribution, which means 
%that for that specific application, the probability of some vtbl-pointers was much 
%higher than the probability of many other vtbl-pointers taken altogether. In 
%our case, the two most frequent classes were representing the use of a variable in 
%a program, and their combined frequency was larger than the combined frequency 
%of all the other classes. Naturally, we would like to avoid conflicts on such 
%classes in the cache, when possible.
%
%To do this, our library provides a configuration flag that enables tracing the
%frequencies of each vtbl-pointer in a match statement and uses this information to 
%minimize the number of conflicts. Due to page limitations, we refer the reader 
%to the technical report accompanying this paper for more details on our 
%experiments with the use of vtbl-pointer frequencies~\cite[\textsection 5.3.2]{TR}. Here we will only 
%mention that, by default, we do not enable frequency tracing, because the 
%significant drop in the number of actual collisions was not reflected in a 
%noticeable decrease in execution time. This was because the total 
%number of actual collisions, even in non-frequency based caching, was much smaller 
%than the number of successful cache hits.


\subsection{An Attractive Non-Solution}
\label{sec:cotc}

%The memoization device outlined in \textsection\ref{sec:memdev} can, in principle, also be 
%applied to tagged classes. The dynamic cast will be replaced by a small 
%compile-time template meta-program that checks whether the class associated with 
%the given tag is derived from the target type of the case clause. If so, a static 
%cast can be used to obtain the offset.

%Despite its straightforwardness, we felt that it should be possible to do better 
%than the general solution, given that each class is already identified with a 
%dedicated constant known at compile time.

While Wirth' linked list encoding was considered slow for subtype testing, it can 
be adopted for quite efficient type switching on a class hierarchy with no 
repeated inheritance. The idea is to combine fast switching on closed 
algebraic datatypes with a loop that tries the tags of base classes when 
switching on derived tags fails.

%The nominal subtyping of \Cpp{} effectively gives every class multiple types. The 
%idea is thus to associate with the type not only its most-derived tag, but also 
%the list of tags of all its base classes. In a compiler implementation such a 
%list can be stored inside the virtual table of a class, while in our library 
%solution it is shared between all the instances with the same most-derived tag 
%in a less efficient global map, associating the tag to its tag list.

For simplicity of presentation we assume a pointer to an array of tags be available 
directly through the subject's \code{taglist} data member. The array is of 
variable size: its first element is always the tag of the subject's dynamic 
type, while its end is marked with a dedicated \code{end_of_list} marker, 
distinct from all the tags. The tags in between are topologically sorted 
according to the subtyping relation with incomparable siblings listed in 
\emph{local precedence order} -- the order of the direct base classes used in 
the class definition. The list resembles the \emph{class precedence list} of 
object-oriented descendants of Lisp (e.g. Dylan, Flavors, LOOPS, and CLOS) used 
there for \emph{linearization} of class hierarchies. 
We also assume the tag-constant associated with a class \code{Di} is accessible 
through a static member \code{Di::class_tag}. These simplifications are not 
essential and the library does not rely on any of them.
%Instead, the user can retroactively narrate to the library the specific tag 
%encoding used through a trait-like class.

A type switch, below, %, built on top of a hierarchy of tagged classes, 
proceeds as 
a regular switch on the subject's tag. If the jump succeeds, we found an exact 
match; otherwise, we get into a default clause that obtains the next tag in the
list and jumps back %to the beginning of the switch statement 
for a rematch:

\begin{lstlisting}[keepspaces]
    size_t attempt = 0; 
    size_t tag = subject->taglist[attempt];
ReMatch:
    switch (tag) {
    default:
        tag = subject->taglist[++attempt];
        goto ReMatch;
    case end_of_list: 
        break;
    case D1::class_tag: 
        D1& match = static_cast<D1&>(*subject); s1;
        break;
        ...
    case Dn::class_tag: 
        Dn& match = static_cast<Dn&>(*subject); sn;
        break;
    }
\end{lstlisting}

\noindent
The above structure, which we call a \emph{tag switch}, implements a variation of 
best-fit semantics based on local precedence order. It lets us dispatch to the case 
clause of the most-specialized class with an overhead of initializing two 
local variables, compared to an efficient switch used on algebraic data types. 
Dispatching to a case clause of a base class will take time roughly proportional 
to the distance between the matched base class and the derived class in the 
inheritance graph, thus the technique is not constant. When none of the base 
class tags was matched, we will necessarily reach the end\_of\_list marker %in the list 
and exit the loop. %As mentioned before, 
The default clause, %of the type switch 
again, can be implemented with a case clause on the subject type's tag: \code{case S::class_tag:}

The efficiency of the above code crucially depends on the set of tags 
being small and sequential to justify the use of a jump table instead of a
decision tree to implement the switch. This is usually not a problem in closed 
hierarchies based on tag encoding since the designer of the hierarchy handpicks 
the tags herself. The use of a static cast %to obtain proper reference once the most specialized derived class has been established, 
however, essentially limits the use of 
this mechanism to non-repeated inheritance only. This only refers to the way target 
classes inherit from the subject type -- they can freely inherit from other classes. 
%as long as they inherit the subject type through non-repeated inheritance only. 
Due to these restrictions, the technique is not open because it may  
violate independent extensibility. We discuss in \textsection\ref{sec:cmp} that 
making the technique more open will also eradicate its performance advantages.


%\section{(Ab)using Exceptions for Type Switching}
\label{sec:xpm}

Several authors had noted the relationship between exception handling and type 
switching before~\cite{Glew99,ML2000}. Not surprisingly, the exception handling 
mechanism of \Cpp{} can be abused to implement the first-fit semantics of a type 
switch statement. The idea is to harness the fact that catch-handlers in \Cpp{} 
essentially use first-fit semantics to decide which one is going to handle a 
given exception. The only problem is to raise an exception with a static type 
equal to the dynamic type of the subject.

To do this, we employ the \emph{polymorphic exception} idiom~\cite{PolyExcept} that 
introduces a virtual function \code{virtual void raise() const = 0;} into the 
base class, overridden by each derived class in syntactically the same way: 
\code{throw *this;}. The match statement then simply calls \code{raise} on its subject, 
while case clauses are turned into catch-handlers. The exact name of the 
function is not important, and is communicated to the library via 
\code{bindings}. The \code{raise} member function can be seen as an analog of 
the \code{accept} member function in the visitor design pattern, whose main purpose is 
to discover subject's most specific type. The analog of a call to \code{visit} 
to communicate that type is replaced, in this scheme, with exception unwinding 
mechanism.

Just because we can, it does not mean we should abuse the exception handling 
mechanism to give us the desired control flow. In the table-driven approach 
commonly used in high-performance implementations of exception handling, the 
speed of handling an exception is sacrificed to provide a zero execution-time 
overhead for when exceptions are not thrown~\cite{Schilling98}. Using exception 
handling to implement type switching will reverse the common and exceptional 
cases, significantly degrading performance. As can be seen in 
Figure\ref{fig:DCastVis1}, matching the type of the first case clause with 
polymorphic exception approach takes more than 7000 cycles and then grows 
linearly (with the position of case clause in the match statement), making it the 
slowest approach. The numbers illustrate why exception handling should only be 
used to deal with exceptional and not common cases.

Despite its total impracticality, the approach gave us a very practical idea of 
harnessing a \Cpp{} compiler to do \emph{redundancy checking} at compile time.

\subsection{Redundancy Checking}
\label{sec:redun}

Redundancy checking is only applicable to first-fit semantics of the match 
statement, and warns the user of any case clause that will never be entered 
because of a preceding one being more general.

We provide a library-configuration flag, which, when defined, effectively turns 
the entire match statement into a try-catch block with handlers accepting the 
target types of the case clauses. This forces the compiler to give warning when 
a more general catch handler preceds a more specific one effectively performing 
redundancy checking for us, e.g.:

\begin{lstlisting}
filename.cpp(55): warning C4286: 'ipr::Decl*' : is caught by 
                  base class ('ipr::Stmt*') on line 42
\end{lstlisting}

\noindent
Note that the message contains both the line number of the redundant case clause (55) 
and the line number of the case clause that makes it redundant (42).

Unfortunately, the flag cannot be always enabled, as the case labels of the underlying 
switch statement have to be eliminated in order to render a syntactically 
correct program. Nevertheless, we found the redundancy checking facility of the 
library extremely useful when rewriting visitor-based code: even though the 
order of overrides in a visitor's implementation does not matter, for some reason 
more general ones were inclined to happen before specific ones in the code we 
looked at. Perhaps programmers are inclined to follow the class declaration order when 
defining and implementing visitors.

A related \emph{completeness checking} -- test of whether a given match 
statement covers all possible cases -- needs to be reconsidered for extensible 
data types like classes, since one can always add a new variant to it. 
Completeness checking in this case may simply become equivalent to ensuring that 
there is either a default clause in the type switch or a clause with the static type 
of a subject as a target type. In fact, our library has an analog of a default 
clause called \code{Otherwise}-clause, which is implemented under the hood 
exactly as a regular case clause with the subject's static type as a target type.


%\section{Memoized Dynamic Cast}
\label{sec:memcast}

We saw in Corollary~\ref{crl:vtbl} that the results of \code{dynamic_cast} can 
be reapplied to a different instance from within the same subobject. This leads 
to a simple idea of memoizing the results of \code{dynamic_cast} and then using 
them on subsequent casts. In what follows we will only be dealing with the  
pointer version of the operator since the version on references that has a 
slight semantic difference can be easily implemented in terms of the pointer one.

The \code{dynamic_cast} operator in C++ involves two arguments: a value argument 
representing an object of a known static type as well as a type argument 
denoting the runtime type we are querying. Its behavior is twofold: on one side 
it should be able to determine when the object's most derived type is not a 
subtype of the queried type (or when the cast is ambiguous), while on the other 
it should be able to produce an offset to adjust the value argument by when it is.

We mimic the syntax of \code{dynamic_cast} by defining:

\begin{lstlisting}
template <typename T, typename S>
inline T memoized_cast(S* p);
\end{lstlisting}

\noindent
which lets the user replace all the uses of \code{dynamic_cast} in the program 
with \code{memoized_cast} with a simple:

\begin{lstlisting}
#define dynamic_cast memoized_cast
\end{lstlisting}

\noindent
It is important to stress that the offset is not a function of source and target 
types of \code{dynamic_cast} operator, which is why we cannot simply memoize the 
outcome inside the individual instantiations of \code{memoized_cast}.
The use of repeated multiple inheritance will result in classes having several 
different offsets associated with the same pair of source and target types 
depending on which subobject the cast is performed from. Accordingly to 
corollary~\ref{crl:vtbl}, however, it is a function of target type and the value 
of the vtbl-pointer stored in the object, because the vtbl-pointer uniquely 
determines the subobject within the most derived type. Our memoization of 
results of \code{dynamic_cast} should thus be specific to a vtbl-pointer and the 
target type. 

Our actual solution uses separate indexing of target types per each source type 
they are used with as well as allocates a different 
\code{vtblmap<std::vector<std::ptrdiff_t>>} for every source type. This lets us 
minimize unused entries within offset vectors by making sure only the plausible 
target types for a given source type are indexed. This solution should be 
suitable for most of the applications since we expect to have a fairly small 
amount of source types of \code{dynamic_cast} operator with a much larger amount 
of target types. For the unlikely case of small amount of target types and large 
amount of source types we let the user to revert the default behavior with a 
library configuration switch that allocates a single \code{vtblmap} per target type.

Use of \code{memoized_cast} to implement match statement potentially reuses the 
results of \code{dynamic_cast} computation across multiple independent match 
statements. This allows leveraging the cost of expensive first call with a 
given vtbl-pointer even further across all the match statements inside the 
program. The above define, with which a user can easily turn all dynamic casts 
into memoized casts can be used to speed-up existing code that uses dynamic 
casting without any refactoring overhead.


\section{Evaluation} %%%%%%%%%%%%%%%%%%%%%%%%%%%%%%%%%%%%%%%%%%%%%%%%%%%%%%%%%%%
\label{sec:eval}

\begin{figure*}
\begin{tabular}{@{}c@{ }l||@{ }r@{}@{ }r@{}|@{ }r@{}@{ }r@{}||@{ }r@{}@{ }r@{}|@{ }r@{}@{ }r@{}||@{ }r@{}@{ }r@{}|@{ }r@{}@{ }r@{}||@{ }r@{}@{ }r@{}|@{ }r@{}@{ }r@{}}
\hline % -----------------------------------------------------------------------------------------------------------------------------------------
\hline % -----------------------------------------------------------------------------------------------------------------------------------------
 &            & \multicolumn{4}{c||}{G++/32}  & \multicolumn{4}{c||}{G++/32}  & \multicolumn{4}{c||}{MS Visual C++/32} & \multicolumn{4}{c}{MS Visual C++/64} \\
\hline % -------------------------------------------------------------------------------------------------------------------------------------------------------------------------
 & Syntax     & \multicolumn{2}{c|}{Unified} & \multicolumn{2}{c||}{Special} & \multicolumn{2}{c|}{Unified} & \multicolumn{2}{c||}{Special} & \multicolumn{2}{c|}{Unified} & \multicolumn{2}{c||}{Special} & \multicolumn{2}{c|}{Unified} & \multicolumn{2}{c}{Special} \\
\hline % -------------------------------------------------------------------------------------------------------------------------------------------------------------------------
 & Encoding   & \Opn  & \Cls  & \Opn  & \Cls  & \Opn  & \Cls  & \Opn  & \Cls  & \Opn  & \Cls  & \Opn  & \Cls  & \Opn  & \Cls  & \Opn  & \Cls   \\
\hline % ----------------------------------------------------------------------------------------------------------------------------------
\hline % ----------------------------------------------------------------------------------------------------------------------------------
 & Repetitive &\glNGPp&\glNGKp&\glNSPp&\glNSKp&\gwNGPp&\gwNGKp&\gwNSPp&\gwNSKp&\VwNGPp&\VwNGKp&\VwNSPp&\VwNSKp&\VxNGPp&\VxNGKp&\VxNSPp&\VxNSKp \\
 & Sequential &\glNGPq&\glNGKq&\glNSPq&\glNSKq&\gwNGPq&\gwNGKq&\gwNSPq&\gwNSKq&\VwNGPq&\VwNGKq&\VwNSPq&\VwNSKq&\VxNGPq&\VxNGKq&\VxNSPq&\VxNSKq \\
 & Random     &\glNGPn&\glNGKn&\glNSPn&\glNSKn&\gwNGPn&\gwNGKn&\gwNSPn&\gwNSKn&\VwNGPn&\VwNGKn&\VwNSPn&\VwNSKn&\VxNGPn&\VxNGKn&\VxNSPn&\VxNSKn \\
\hline % ----------------------------------------------------------------------------------------------------------------------------------
\multirow{3}{*}{\begin{sideways}{\tiny Forward}\end{sideways}}
 & Repetitive &\glYGPp&\glYGKp&\glYSPp&\glYSKp&\gwYGPp&\gwYGKp&\gwYSPp&\gwYSKp&\VwYGPp&\VwYGKp&\VwYSPp&\VwYSKp&\VxYGPp&\VxYGKp&\VxYSPp&\VxYSKp \\
 & Sequential &\glYGPq&\glYGKq&\glYSPq&\glYSKq&\gwYGPq&\gwYGKq&\gwYSPq&\gwYSKq&\VwYGPq&\VwYGKq&\VwYSPq&\VwYSKq&\VxYGPq&\VxYGKq&\VxYSPq&\VxYSKq \\
 & Random     &\glYGPn&\glYGKn&\glYSPn&\glYSKn&\gwYGPn&\gwYGKn&\gwYSPn&\gwYSKn&\VwYGPn&\VwYGKn&\VwYSPn&\VwYSKn&\VxYGPn&\VxYGKn&\VxYSPn&\VxYSKn \\
\hline % ----------------------------------------------------------------------------------------------------------------------------------
\hline % ----------------------------------------------------------------------------------------------------------------------------------
 &            & \multicolumn{4}{c||}{Linux Desktop} & \multicolumn{12}{c}{Windows Laptop}                                                      \\
\hline % ----------------------------------------------------------------------------------------------------------------------------------
\end{tabular}
\caption{Relative performance of type switching versus visitors. Numbers 
in regular font (e.g. \f{67}), indicate that our type switching is faster than 
visitors by corresponding percentage. Numbers in bold font (e.g. \s{14}), 
indicate that visitors are faster by corresponding percentage.}
\label{relperf}
\end{figure*}

Our evaluation methodology consists of several benchmarks representing various 
uses of objects inspected with either visitors or type switching.

The \emph{repetitive} benchmark performs calls on different objects of the 
same most derived type. This scenario happens in object-oriented setting when a 
group of polymorphic objects is created and passed around (e.g. numerous 
particles of a given kind in a particle simulation system). We include it 
because double dispatch becomes twice faster (27 vs. 53 cycles) in this 
scenario compared to others due to cache and call target prediction mechanisms. 

The \emph{sequential} benchmark effectively uses an object of each derived type only 
once and then moves on to an object of a different type. The cache is typically 
reused the least in this scenario, which is typical of lookup tables, where each 
entry is implemented with a different derived class.

The \emph{random} benchmark is the most representative as it randomly makes calls on 
random objects, which will probably be the most common usage scenario in the 
real world.

The \emph{forwarding} benchmark is not a benchmark on its own, but rather a 
combinator that can be applied to any of the above scenarios. It refers to the 
common technique used by visitors where, for class hierarchies with multiple 
levels of inheritance, the \code{visit} method of a derived class will provide a 
default implementation of forwarding to its immediate base class, which, in turn, 
may forward it to its base class, etc. The use of forwarding in visitors is a 
way to achieve substitutability, which in type switches corresponds to the use 
of base classes in the case clauses.

The class hierarchy for non-forwarding test was a flat hierarchy with 100 
derived classes, encoding an algebraic data type. The class hierarchy for 
forwarding tests had two levels of inheritance with 5 intermediate base classes 
and 95 derived ones. 

Each benchmark was tested with either \emph{unified} or \emph{specialized} 
syntax, each including tests on polymorphic (\emph{Open}) and tagged 
(\emph{Tag}) encodings. Specialized syntax avoids generating unnecessary 
syntactic structure used to unify syntax, and thus produces faster code. We 
include it in results because a compiler implementation of type switching 
will only generate the best suitable code.

The benchmarks were executed in the following configurations refered to as 
\emph{Linux Desktop} and \emph{Windows Laptop} respectively:

\begin{itemize}
\setlength{\itemsep}{0pt}
\setlength{\parskip}{0pt}
\item Dell Dimension\textsuperscript{\textregistered} desktop with Intel\textsuperscript{\textregistered} Pentium\textsuperscript{\textregistered} 
      D (Dual Core) CPU at 2.80 GHz; 1GB of RAM; Fedora Core 13  
      \begin{itemize}
      \setlength{\itemsep}{0pt}
      \setlength{\parskip}{0pt}
      \item G++ 4.4.5 executed with -O2
      \end{itemize}
\item Sony VAIO\textsuperscript{\textregistered} laptop with Intel\textsuperscript{\textregistered} Core\texttrademark i5 460M 
      CPU at 2.53 GHz; 6GB of RAM; Windows 7 Professional
      \begin{itemize}
      \setlength{\itemsep}{0pt}
      \setlength{\parskip}{0pt}
      \item G++ 4.5.2 / MinGW executed with -O2; x86 binaries
      \item MS Visual C++ 2010 Professional x86/x64 binaries with profile-guided optimizations
      \end{itemize}
\end{itemize}

\noindent
The code on the critical path of our type switch implementation benefits 
significantly from branch hinting as some branches are much more likely than 
others. We use the branch hinting in GCC to guide the compiler, but, 
unfortunately, Visual C++ does not have similar facilities. Microsoft suggests 
to use \emph{Profile-Guided Optimization} to achieve the same, which is why the 
results for Visual C++ reported here have been obtained with profile-guided 
optimizations enabled. The results without profile-guided optimizations can be 
found in the accompanying technical report~\cite[\textsection 10]{TR}.
%The results of optimizing code created with Visual C++ by using profile 
%guided optimizations as currently Visual C++ does not have means for branch 
%hinting, which are supported by G++ and proven to be very effective in few 
%cruicial places. Profile guided optimization in Visual C++ lets compiler find 
%out experimentally what we would have otherwise hinted, even though this 
%includes other optimizations as well.

We compare the performance of our solution relative to the performance of visitors in 
Figure~\ref{relperf}. The values are given as percentages of performance increase 
against the slower technique. Numbers in regular font represent cases where type 
switching was faster, while numbers in bold indicate cases where visitors 
were faster.

We can see that type switching wins by a good margin on tag preceedence lists as 
well as in the presence of at least one level of forwarding. Note that the 
numbers are relative, and thus the ratio depends on both the performance of 
virtual calls and the performance of switch statements. Visual C++ was 
generating faster virtual function calls, while GCC was generating faster switch 
statements, which is why their relative performance seem to be much more 
favorable for us in the case of GCC.

Similarly the code for x64 is only slower relatively: the actual time spent for 
both visitors and type switching was smaller than that for x86, but it was much 
smaller for visitors than type switching, which resulted in worse relative 
performance.

\subsection{Vtable Pointer Memoization vs. TPL Dispatcher}
\label{sec:cmp}

With a few exceptions for x64, it can be seen from Figure~\ref{relperf} 
that the performance of the TPL dispatcher (the Tag column) dominates the 
performance of the vtable pointer memoization approach (the Open column). We believe 
that the difference, often significant, is the price one pays for the true 
openness of the vtable pointer memoization solution.

Unfortunately, the TPL dispatcher is not truly open. The use of tags, 
even if they would be allocated by compiler, may require integration efforts to 
ensure that different DLLs have not reused the same tags. Randomization of tags,
similar to a proposal of Garrigue~\cite{garrigue-98}, will not eliminate the 
problem and will surely replace jump tables in switches with decision trees. This 
will likely significantly degrade the numbers for the Tag column of 
Figure~\ref{relperf}, since the tags in our experiments were all sequential.

Besides, the TPL dispatcher approach relies on static cast to obtain the 
proper reference once the most specific case clause has been found. As we 
described in \textsection\ref{sec:vtblmem}, this has severe limitations in the 
presence of multiple inheritance, and thus is not as versatile as the other 
solution. Overcoming this problem will either require the use of 
\code{dynamic_cast} or techniques similar to those we used in vtable pointer 
memoization. This will likely degrade performance numbers for the Tag column even further.

Note also that the vtable pointer memoization approach can be used to implement both
first-fit and best-fit semantics, while the TPL dispatcher is only suitable 
for best-fit semantics. Their complexity guarantees also differ: vtable pointer 
memoization is constant on average, and slow on the first call. Tag list approach is 
logarithmic in the size of the class hierarchy on average (assuming a balanced 
hierarchy), including on the first call.

\subsection{Comparison with OCaml}
\label{sec:ocaml}

We now compare our solution to the built-in pattern-matching facility of OCaml~\cite{OPM01}. 
In this test, we timed a small OCaml application performing our sequential 
benchmark on an algebraic data type of 100 variants. Corresponding C++ 
applications were working with a flat class hierarchy of 100 derived classes. 
The difference between the C++ applications lies in the encoding (Open/Tag/Kind) 
and the syntax (Unified/Special) used. Kind encoding is the same as Tag encoding, but 
it does not require substitutability, and thus can be implemented with a direct 
switch on tags. It is only supported through specialized syntax in our library 
as it differs from the Tag encoding only semantically.

The optimized OCaml compiler \texttt{ocamlopt.opt} that we used to compile the code 
can be based on different toolsets on some platforms, e.g. Visual C++ or GCC 
on Windows. To make the comparison fair we had to make sure that the 
same toolset was used to compile the C++ code. We ran the tests 
on both of the machines described above using the following configurations: 

\begin{itemize}
\setlength{\itemsep}{0pt}
\setlength{\parskip}{0pt}
\item The tests on a Windows 7 laptop were all based on the \emph{Visual C++ toolset} 
      and used \texttt{ocamlopt.opt} version 3.11.0.
\item The tests on a Linux desktop were all based on the \emph{GCC toolset} and used 
      \texttt{ocamlopt.opt} version 3.11.2
\end{itemize}

\noindent
The timing results presented in Figure~\ref{fig:OCamlComparison} are averaged 
over 101 measurements and show the number of seconds it took to perform a 
million decompositions within our sequential benchmark.

\begin{figure}[htbp]
  \centering
    \includegraphics[width=0.47\textwidth]{OCamlComparison.png}
  \caption{Performance comparison of various encodings and syntax against OCaml code}
  \label{fig:OCamlComparison}
\end{figure}

We can see that the use of specialized syntax on a closed/sealed hierarchy can 
match the speed of, and even be four times faster than, the code generated by 
the native OCaml compiler. Once we go for an open solution, we become about 
30-50\% slower. 


\section{Related Work} %%%%%%%%%%%%%%%%%%%%%%%%%%%%%%%%%%%%%%%%%%%%%%%%%%%%%%%%%
\label{sec:rw}

\emph{Extensible Visitors with Default Cases}~\cite[\textsection 
4.2]{Zenger:2001} attempt to solve the extensibility problem of visitors; 
however, the solution has problems of its own. The visitation interface 
hierarchy can easily be grown linearly, but independent extensions by different  
authorities require developer's intervention. On top of the double dispatch the 
solution will incur two additional virtual calls and a dynamic cast for each 
level of visitor extension. The solution is simpler with virtual inheritance, 
which adds even more indirections.

L\"{o}h and Hinze proposed to extend Haskell's type system with open data types 
and open functions~\cite{LohHinze2006}. The solution allows top-level data types 
and functions to be marked as open with concrete variants and overloads defined 
anywhere in the program. The semantics of open extension is given by 
transformation into a single module, which assumes a whole-program view and thus 
is not an open solution unfortunately. Besides, open data types are extensible but not 
hierarchical, which avoids the problems discussed here.

Polymorphic variants in OCaml~\cite{garrigue-98} allow the addition of new 
variants as well as define subtyping on them. The subtyping, however, is not 
defined between the variants, but between combinations of them. 
This maintains disjointness between values from different variants and makes an 
important distinction between \emph{extensible sum types} like polymorphic 
variants and \emph{extensible hierarchical sum types} like classes. Our 
memoization device can be used to implement pattern matching on polymorphic 
variants.

\emph{Tom} is a pattern-matching compiler that can be used together with Java, C or 
Eiffel to bring a common pattern matching and term rewriting syntax into the 
languages~\cite{Moreau:2003}. In comparison to our approach, Tom has much bigger 
goals: the combination of pattern matching, term rewriting and strategies turns 
Tom into a fully fledged tree-transformation language. Its type patterns and \%match 
statement can be used as a type switch; however, Tom's handling of type 
switching is based on decision trees and an \code{instanceof}-like predicate, 
which are inefficient.

Pattern matching in Scala~\cite{Scala2nd} also support type switching through 
type patterns. The language supports extensible and hierarchical data types, but
their handling in a type switching constructs varies. Sealed classes are handled 
with an efficient switch over all tags, while extensible classes are similarly 
approached with a combination of an \code{InstanceOf} operator and a decision 
tree~\cite{EmirThesis}.

\section{Conclusions and Future Work} %%%%%%%%%%%%%%%%%%%%%%%%%%%%%%%%%%%%%%%%%%%%%%%%%%%%%%%%%%
\label{sec:cc}

Type switching is an open alternative to the visitor design pattern that overcomes 
the restrictions, inconveniences, and difficulties in teaching and using 
visitors. Our implementation significantly
outperforms the visitor design pattern in most cases and roughly equals it otherwise.
This is the case even though we use a library implementation and highly optimized
production-quality compilers. An important benefit of our solution is that it does not 
require any changes to the \Cpp{} object-model or require any computations at load 
time.

To provide a complete solution, we use the same syntax for closed sets of types, where our
performance roughly equals the equivalent built-in features in functional languages,
such as Haskell and OCaml.

We prove the uniqueness of vtbl-pointers in the presence of RTTI. This is 
potentially useful in other compiler optimizations that depend on the 
identity of subobjects. Our memoization device can also become valuable in 
optimizations that require mapping run-time values to execution paths, 
and is especially useful in library setting.

%We describe three techniques that can be used to implement type switching, type 
%testing, pattern matching, predicate dispatching, and other facilities that 
%depend on the run-time type of an argument as well as demonstrate their efficiency.
%
%The \emph{Memoization Device} is an optimization technique that maps run-time values 
%to execution paths, allowing to take shortcuts on subsequent runs with the same 
%value. The technique does not require code duplication and in typical cases adds 
%only a single indirect assignment to each of the execution paths. It can be 
%combined with other compiler optimizations and is particularly suitable for use 
%in a library setting.
%
%The \emph{Vtable Pointer Memoization} is a technique based on memoization device that 
%employs uniqueness of virtual table pointers to not only speed up execution, but 
%also properly uncover the dynamic type of an object. This technique is a 
%backbone of our fast type switch as well as memoized dynamic cast optimization.
%
%The \emph{TPL Dispatcher} is yet another technique that can be used to 
%implement best-fit type switching on tagged classes. The technique has its pros 
%and cons in comparison to vtable pointer memoization, which we discuss in the paper.
%
%These techniques can be used in a compiler and library setting, and support well 
%separate compilation and dynamic linking. They are open to class extensions and 
%interact well with other \Cpp{} facilities such as multiple inheritance and 
%templates. The techniques are not specific to \Cpp{} and can be adopted in other 
%languages for similar purposes.
%
%Using these techniques, we implemented a library for efficient type switching 
%in \Cpp{}. We used it to rewrite a code that relied heavily on 
%visitors, and discovered that the resulting code became much shorter, simpler, 
%and easier to maintain and comprehend.


%\section{Future Work} %%%%%%%%%%%%%%%%%%%%%%%%%%%%%%%%%%%%%%%%%%%%%%%%%%%%%%%%%%
%\label{sec:fw}

Using a library implementation was essential for experimentation and for being able to
test our ideas on multiple production-quality compiler systems.
However, now we hope to re-implement our ideas in a compiler.
This would allow us to improve further surface syntax, diagnostics, and performance.

%In the future we would like to provide an efficient multi-threaded 
%implementation of our library as currently it relies heavily on static variables 
%and global state, which will have problems in a multi-threaded environment. 
%
%Additionally, the match statement that we presented here deals with only one 
%subject at the moment, but we believe that the same technique can be used to 
%address multiple subjects.
%
%We would also like to experiment with other kinds of cache indexing functions in 
%order to decrease the frequency of conflicts, especially those coming from the use 
%of dynamically-linked libraries.
%
%Last but not least we would like to look at providing support for alternative 
%matching semantics.
%
%

\bibliographystyle{abbrvnat}
\bibliography{mlpatmat}
\end{document}
