\section{Conclusions and Future Work} %%%%%%%%%%%%%%%%%%%%%%%%%%%%%%%%%%%%%%%%%%
\label{sec:cc}

The \emph{Mach7} library provides functional-style 
pattern-matching facilities for \Cpp{}. The solution is open to new patterns, 
with the traditional patterns implemented as an example. It is
non-intrusive, so it can be applied retroactively. %to any polymorphic or tagged 
%class hierarchy. It also provides a uniform notation to these different 
%encodings of algebraic and extensible hierarchical data types in \Cpp{}.
The library provides efficient and expressive matching on multiple subjects and 
compares well to multiple dispatch alternatives in terms of both time and space.
We~also offer an alternative interpretation of the n+k patterns and show how some 
traditional generalizations of these patterns can be implemented in our library. 
%to arbitrary expressions by letting the user define   
%the exact semantics of such patterns. Our approach is more general than traditional approaches 
%as it does not require an
%equational view of such patterns. It also avoids hardcoding the 
%exact semantics of n+k patterns into the language. 
%
%We generalize n+k patterns to arbitrary expressions by letting the user define 
%the exact semantics of such patterns. Our approach is more general than traditional approaches 
%as it does not require an
%equational view of such patterns. It also avoids hardcoding the 
%exact semantics of n+k patterns into the language. 
%
%We used the library to rewrite existing code that relied heavily on the 
%visitor design pattern.
%Our pattern matching code was much shorter (both source and object code), 
%simpler, easier to maintain, comprehend, and faster. 
%This confirmed our view of the visitor pattern as a clever workaround,
%rather than a good solution to a fundamental problem.
%The library approach was essential 
%for experimentation in the context of real programs and for delivering 
%performance comparable with or superior to conventional techniques in the 
%context of industrial compilers.
%
\emph{Mach7} pattern matching code performs reasonably compared to
open multi-methods and visitors, demonstrating the effectiveness of the 
library-based approach.

The work presented here continues our research on pattern matching for 
\Cpp{}~\cite{TS12}. Due to page limit, we had to omit many interesting 
details that provide a better insight into our solution. We refer the reader to 
the first author's PhD thesis~\cite{SolodkyyThesis} for an in-depth discussion 
of open type switching, open pattern matching and open multi-methods in the 
context of \Cpp{}.

In the future, we would like to implement an actual language 
extension that will be capable of working with open patterns.  Given such an extension and its implementation,
we~would like to look into how code for such 
patterns can be optimized without hardcoding the knowledge of the semantics of the patterns
into the compiler. 
We would also like to experiment with other kinds of patterns (including those defined by the user),
look at the interaction of patterns with both the standard library and 
other facilities in the language, and make views less ad-hoc.

%Our generalization of n+k patterns depends on the properties of types involved 
%in the expression. This should let us experiment not only with generic 
%functions, but also with their generic inversions in the form of solvers. As 
%more \Cpp{}11 features become available in compilers it will also be interesting to 
%look at how use of these features affects the ease of use, performance, 
%readability, writability and debugging of the library and the user code that 
%uses it.

%In the nearest future, we would like to make our library to be safe and efficient 
%in a multi-threaded environment. We would also like to look into providing an 
%equally efficient and expressive support to multiple subjects, which in turn 
%would allow us to address asymmetric multiple dispatch.

%From Morten Rhiger:
%New languages are often constructed by piling new features on top of an existing language's de?nition and by integrating these features in the existing language's implementation. However, it is a sign of expressiveness if new features can be implemented within an
%existing language without changing its de?nition.
%Short of macros, functional languages such as Haskell and Standard ML require new
%features to be expressed in terms of typed higher-order functions. We have demonstrated
%how to extend - or, in Guy Steele's terminology, to "grow" (Steele Jr., 1999) - Haskell
%with our own statically typed implementation of pattern matching and we have shown how
%to extend this framework with patterns not currently supported by Haskell.
