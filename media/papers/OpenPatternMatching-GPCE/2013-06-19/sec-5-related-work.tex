\section{Related Work} %%%%%%%%%%%%%%%%%%%%%%%%%%%%%%%%%%%%%%%%%%%%%%%%%%%%%%%%%
\label{sec:rw}

Language support for pattern matching was first introduced for string manipulation 
in COMIT\cite{COMIT58}, which subsequently inspired similar primitives in 
SNOBOL\cite{SNOBOL64}. SNOBOL4 had string patterns as first-class data types 
providing operations of concatenation and alternation.
The first reference to a modern pattern-matching constructs seen in functional 
languages is usually attributed to Burstall's work on structural 
induction\cite{Burstall69provingproperties}. Pattern matching was further 
developed by the functional programming community, most notably 
%Hope\cite{BMS80}, Miranda\cite{Miranda85}, 
ML\cite{ML78} and Haskell\cite{haskell90}. In the context of object-oriented programming, 
pattern matching has been first explored in Pizza\cite{Odersky97pizzainto} and 
Scala\cite{Scala2nd,EmirThesis}. The idea of first class patterns dates back to 
at least Tullsen's proposal to add them to Haskell~\cite{Tullsen00}, calculus of 
such patterns has been studied in details by Jay~\cite{Jay09,PatCalc09}.

There are two main approaches to compiling pattern-matching code: the first is 
based on \emph{backtracking automata} and was introduced by Augustsson\cite{Augustsson85}, 
the second is based on \emph{decision trees} and was first described by 
Cardelli\cite{Cardelli84}. %, though he attributes the technique to Dave MacQueen 
%and Gilles Kahn in their implementation of the Hope compiler \cite{BMS80}.
Backtracking approach usually generates smaller code~\cite{OPM01}, while decision tree 
approach produces faster code by ensuring that each test is only 
performed once~\cite{Maranget08}. %With respect to matching a single expression our library 
%approach follows the naive backtracking approach, however our match statement is 
%based on a highly efficient type switching technique we developed\cite{TS12} 
%that outperforms similar solutions based on decision trees or visitor design pattern.

There have been several attempts to bring pattern matching into various 
languages in a form of a library. They differ on which abstractions of the host 
language were used to encode the patterns and the match statement. \emph{MatchO} was one of the first 
such attempts for Java~\cite{Visser06matchingobjects}. The approach follows 
\emph{patterns as objects} strategy.
%While the library 
%was essentially providing first class patterns into Java, the amount of 
%syntactic and run-time overhead was overwhelming. %The author offers to provide 
%customized syntax to patterns by invoking an inline parser, while to improve 
%performance the author looks into combining visitors and visitor combinators 
%with pattern matching. %This resembles our use of an efficient type switch to 
%uncover the dynamic type combined with slower but more flexible pattern 
%matching.
%
\emph{Functional C\#} was a similar attempt to bringing pattern matching into 
C\#\cite{FuncCSharp}. The approach uses lambda expressions and 
chaining of method calls to create a structure that is then evaluated at 
run-time for the first successful match. %The approach supports a form of 
%active patterns, simple n+k patterns, list and tuple patterns as well as type 
%patterns (without structural decomposition). 
%Unfortunately, the solution scales very poorly and is ill suited for tests 
%involving nesting of patterns.
%
%Both \emph{MatchO} and \emph{Functional C\#} were object-oriented solutions. 
In functional community, Rhiger explored introduction of pattern 
matching into Haskell as a library~\cite{Rhiger09}. He used functions to encode 
patterns and pattern combinators, which allowed him to detect pattern 
misapplication errors at compile time through the Haskell type system. 
%Unfortunately, the purity of a functional language prevents him from expressing 
%variable patterns (more specifically, the bindings they introduce) in a library 
%setting simply, which then affects the overall elegance of the solution. In 
%particular, it seems to be impossible in his approach to express equivalence 
%patterns or even equivalence combinator as the value of the bound variable is 
%not available in the left-hand side of the matching expression, only in the 
%right-hand side. The library is sufficiently expressive though to express basic 
%patterns (value, wildcard, predicate, etc.), boolean pattern combinators and set 
%patterns. The author reports that his solution requires 2-4 times more 
%reductions than Haskell's built-in pattern matching. He attributes the overhead 
%to currying and continuations passing and hypothesizes that it should be completely 
%eliminated by standard partial evaluation techniques.
%
\emph{Racket} has a powerful macro system that allows it to express open pattern 
matching in the language entirely as a library~\cite{Tobin-Hochstadt_2010}. 
%The work builds on earlier attempts to bring pattern matching into Scheme, 
%extending it and making it more efficient~\cite{Wright95}. 
The solution is remarkable in that unlike most of the library approaches to open pattern matching, 
it does not rely on naive backtracking and in fact encodes the optimized 
algorithm based on backtracking automata~\cite{Augustsson85,OPM01}. %The solution 
%is easy to extend with new kinds of patterns and its only disadvantage is the 
%fact that untyped nature of Racket makes any static checking of patterns hard if 
%possible.
%
\emph{Grace} is another programming language that provides a library solution to 
pattern matching through objects~\cite{Grace2012}. Similar to other control 
structures in the language, Grace encodes the match statement with partial 
functions and lambda expressions, while patterns are encoded as objects. 
%The language aims at  
%teaching various programming paradigms and has an interesting peculiarity: it 
%does not have pre-defined control structures, which instead are expressed 
%directly in the language with the help of partial functions and lambda 
%expressions. The \codeocaml{match}-statement, for example, takes form of a 
%clausal definition \codeocaml{match(e)case(c_1)...case(c_n)}, in which case 
%blocks \codeocaml{c_i} are single-argument blocks that take pattern objects as 
%an input. Patterns are objects in Grace. The compiler provides some syntactic 
%sugar to denote some common patterns, but otherwise user-defined patterns have 
%to be explicitly created. They are composed and applied at run-time using the 
%naive backtracking approach without any current provision for optimizations. 
%This is a design choice as Grace is not concerned with performance, but with the 
%expressiveness of the language and ease of teaching different programming 
%paradigms. Unlike many object-oriented languages, Grace does not use nominative 
%subtyping and thus their case analysis is never hierarchical; instead they use 
%structural subtyping, more common of functional languages.

%Newspeak and Grace treat their case statement as combination of partial 
%functions -- functions defined on a restricted domain of inputs. Functional C 
%sharp does ...

Multiple language extensions have been developed to provide pattern matching into 
a host language in a form of a compiler, preprocessor or tool.
%
\emph{Prop} %was a language extension that 
brought pattern matching and term 
rewriting into \Cpp{}~\cite{Prop96}. %The extension aimed at building 
%high-performance compiler and language transformation systems. 
It did not offer 
first-class patterns, but supported most of the functional-style patterns and 
provided an optimizing compiler for both pattern matching and garbage collected 
term rewriting. \emph{App} was another pattern-matching extension to \Cpp{}~\cite{App} 
that mainly concentrated on providing syntax for defining and pattern matching on algebraic data types. 
%them that was very close to Haskell. Unlike Prop, App concentrates on algebraic 
%data type decomposition and only supports wildcard, variable, constructor and a 
%zero pattern -- a value pattern for \code{nullptr}. App also does not provide 
%any optimizations and simply generates a cascade of type tests.
%
\emph{Tom} is a pattern-matching compiler that %took the ideas in Prop much 
%further, banking on the fact that pattern-matching primitives and optimizations 
%on them are language agnostic~\cite{Moreau:2003}. It 
brings a common 
pattern-matching and term-rewriting syntax into Java, C and Eiffel. %, but 
%otherwise can be implemented on top of many other programming languages. 
Thanks to its distinct syntax, it is transparent to the semantics of the host language 
and can be implemented as a preprocessor to many other languages. %Similar to Prop and App, 
Tom neither supports first-class patterns, nor is open to new patterns. 
%Tom is quite 
%transparent as to the concrete target language used and can potentially be 
%extended to other target languages besides the three supported now.
%Tom's goals differ from ours in aiming to be a
%tree-transformation language similar to Stratego/XT, XDuce and others. 
%Tom's approach is prone to general problems of any preprocessor based 
%solution\cite[\textsection 4.3]{SELL}. In particular, it is part of a dedicated tool chain.
%Our library approach avoids that and lets us employ the \Cpp{} semantics within 
%patterns: e.g. our patterns work directly on underlying user-defined data 
%structures, avoiding abstraction penalties. The tight integration with 
%the language semantics makes our patterns first-class citizens that can be 
%composed and passed to other functions. 
%
\emph{Matchete} is a language extension to Java that brings together different flavors 
of pattern matching: functional-style patterns, Perl-style regular expressions, 
XPath expressions, Erlang's bit-level patterns etc.~\cite{padl08}. The extension 
does not try to make patterns first class citizens, and instead concentrates on 
implementing existing best practices and their tight integration into Java. %This 
%also avoids the necessity to remain within boundaries of Java and allows them to 
%improve significantly the syntax. The extension offers a wide variety of 
%built-in patterns, however it does not try to optimize the matching and resorts 
%to naive backtracking. 
%
\emph{OOMatch} is another Java extension that brings pattern matching and multiple 
dispatch close together~\cite{OOMatch07thesis}. The approach generalizes 
multiple dispatch by offering to use patterns as multi-method arguments and then 
orders overriders based on the specificity of their arguments. %This allows 
%avoiding many ambiguities typical of symmetric multiple dispatch, however it 
%does not eliminate them entirely. 
Similar to others, the approach only deals with a limited set of built-in patterns.

\emph{Thorn} is dynamically typed scripting language that provides first-class 
patterns~\cite{Thorn2012}. %The pattern matching in Thorn, however, is not 
%brought through a library, instead it is tightly integrated into the language 
%and seamlessly interacts with other constructs. 
The language defines a handful 
of atomic patterns and pattern combinators to compose them, and, similar to 
Newspeak and Grace, uses the duality between partial functions and patterns to 
support user-defined patterns. %The duality is particularly suitable for 
%providing support to structural decomposition of classes: along with a regular 
%constructor, a class may provide destructuring constructor, which allow the use 
%of construction syntax in the pattern-matching context. 
%As with many other solutions to first-class patterns we looked at, the pattern matching is currently 
%performed in the naive backtracking way. Unlike the library solutions, however, 
%tighter integration into the language and presence of built-in patterns and 
%pattern combinators should enable at least some of the known optimization 
%techniques mentioned above~\cite{OPM01,Maranget08}, remember that Thorn is an 
%interpreter though.


When a class hierarchy is fixed, one can design a pattern language that involves 
semantic notions represented by the hierarchy. Pirkelbauer devised a pattern 
language for Pivot~\cite{Pivot09} capable of representing various entities in a 
\Cpp{} program using syntax very close to the \Cpp{} itself. The patterns were 
translated with a tool into a set of visitors implementing the underlying 
pattern-matching semantics efficiently~\cite{PirkelbauerThesis}. %Earlier, Cook 
%et al used expression templates to implement a query language for Pivot's class 
%hierarchy~\cite{iql04}. %Our current work is the result of a series 
%of experimental designs. The library approach was essential to provide 
%relatively quick turnaround for experiments and for maintaining and improving 
%performance for our applications.
