\documentclass[preprint]{sigplanconf}

\usepackage{amssymb}
\usepackage{amsthm}
\usepackage{breakurl}             % Not needed if you use pdflatex only.
\usepackage{color}
\usepackage{epsfig}
\usepackage{esvect}
\usepackage{listings}
\usepackage{mathpartir}
\usepackage{MnSymbol}
\usepackage{multirow}
\usepackage{rotating}
\usepackage{paralist}

\newtheorem{definition}{Definition}
\newtheorem{theorem}{Theorem}
\newtheorem{lemma}{Lemma}
\newtheorem{proposition}{Proposition}
\newtheorem{corollary}{Corollary}

\setlength{\parskip}{0cm}
%\setlength{\parindent}{1em}

\DeclareMathOperator*{\argmin}{arg\,min}
\DeclareRobustCommand{\Cpp}{C\texttt{++}}
\DeclareRobustCommand{\code}[1]{{\lstinline[breaklines=false,escapechar=@]{#1}}}
\DeclareRobustCommand{\codebr}[1]{{\lstinline[breaklines=true]{#1}}}
\DeclareRobustCommand{\codehaskell}[1]{{\lstinline[breaklines=false,language=Haskell]{#1}}}
\DeclareRobustCommand{\codeocaml}[1]{{\lstinline[breaklines=false,language=Caml]{#1}}}
\DeclareRobustCommand{\concept}[1]{{\small\textsc{#1}}}

\newcommand{\exclude}[1]{}
\newcommand{\halfline}{\vspace{-1.5ex}}

%%%%%%%%%%%%%%%%%%%%%%%%%%%%%%%%%%%%%%%%%%%%%%%%%%%%%%%%%%%%%%%%%%%%%%%%%%%%%%
%%%%%%%%%%%%%%%%%%%%%%%%%%%%%%%%%%%%%%%%%%%%%%%%%%%%%%%%%%%%%%%%%%%%%%%%%%%%%%

% listings settings

%%%%%%%%%%%%%%%%%%%%%%%%%%%%%%%%%%%%%%%%%%%%%%%%%%%%%%%%%%%%%%%%%%%%%%%%%%%%%%
%%%%%%%%%%%%%%%%%%%%%%%%%%%%%%%%%%%%%%%%%%%%%%%%%%%%%%%%%%%%%%%%%%%%%%%%%%%%%%

\lstdefinestyle{C++}{language=C++,%
showstringspaces=false,
  columns=fullflexible,
  escapechar=@,
  basicstyle=\sffamily,
%  commentstyle=\rmfamily\itshape,
  moredelim=**[is][\color{white}]{~}{~},
  morekeywords={concept,decltype,noexcept,nullptr,requires},
  literate={[<]}{{\textless}}1      {[>]}{{\textgreater}}1 %
           {<}{{$\langle$}}1        {>}{{$\rangle$}}1 %
           {<=}{{$\leq$}}1          {>=}{{$\geq$}}1          
           {==}{{$==$}}2            {!=}{{$\neq$}}1 %
           {=>}{{$\Rightarrow\;$}}1 {->}{{$\rightarrow{}$}}1 %
           {<:}{{$\subtype{}\ $}}1  {<-}{{$\leftarrow$}}1 %
           {s1;}{{$s_1$;}}3 {s2;}{{$s_2$;}}3 {s3;}{{$s_3$;}}3 {s4;}{{$s_4$;}}3 {s5;}{{$s_5$;}}3 {s6;}{{$s_6$;}}3 {s7;}{{$s_7$;}}3 {sn;}{{$s_n$;}}3 {si;}{{$s_i$;}}3%
           {P1}{{$P_1$}}2 {P2}{{$P_2$}}2 {P3}{{$P_3$}}2 {P4}{{$P_4$}}2 {P5}{{$P_5$}}2 {P6}{{$P_6$}}2 {P7}{{$P_7$}}2 {Pn}{{$P_n$}}2 {Pi}{{$P_i$}}2%
           {D1}{{$D_1$}}2 {D2}{{$D_2$}}2 {D3}{{$D_3$}}2 {D4}{{$D_4$}}2 {D5}{{$D_5$}}2 {D6}{{$D_6$}}2 {D7}{{$D_7$}}2 {Dn}{{$D_n$}}2 {Di}{{$D_i$}}2%
           {T1}{{$T_1$}}2 {T2}{{$T_2$}}2 {T3}{{$T_3$}}2 {T4}{{$T_4$}}2 {T5}{{$T_5$}}2 {T6}{{$T_6$}}2 {T7}{{$T_7$}}2 {Tn}{{$T_n$}}2 {Ti}{{$T_i$}}2 {Tm}{{$T_m$}}2%
           {e1}{{$e_1$}}2 {e2}{{$e_2$}}2 {e3}{{$e_3$}}2 {e4}{{$e_4$}}2%
           {E1}{{$E_1$}}2 {E2}{{$E_2$}}2 {E3}{{$E_3$}}2 {E4}{{$E_4$}}2 {Ei}{{$E_i$}}2%
           {m_e1}{{$m\_e_1$}}4 {m_e2}{{$m\_e_2$}}4 {m_e3}{{$m\_e_3$}}4 {m_e4}{{$m\_e_4$}}4%
           {Divide}{{Divide}}6 {Either}{Either}6 %
           {Times}{{Times}}5 %
           {Match}{{\emph{Match}}}5 %
           {Case}{{\emph{Case}}}4 %
           {Qua}{{\emph{Qua}}}3 %
           {When}{{\emph{When}}}4 %
           {Otherwise}{{\emph{Otherwise}}}9 %
           {EndMatch}{{\emph{EndMatch}}}8 %
           {CM}{{\emph{CM}}}2 {KS}{{\emph{KS}}}2 {KV}{{\emph{KV}}}2 %
           {EuclideanDomain}{\concept{EuclideanDomain}}{15}  %
           {LazyExpression}{\concept{LazyExpression}}{14}    %
           {Polymorphic}{\concept{Polymorphic}}{11}          %
           {Convertible}{\concept{Convertible}}{11}          %
           {Integral}{\concept{Integral}}8                   %
           {SameType}{\concept{SameType}}8                   %
           {Pattern}{\concept{Pattern}}7                     %
           {Regular}{\concept{Regular}}7                     %
           {Object}{\concept{Object}}6                       %
           {Field}{\concept{Field}}5                         %
}
\lstset{style=C++}

\lstdefinestyle{Haskell}{language=Haskell,%
  morekeywords={out,view,real}
  literate={=>}{{$\Rightarrow\;$}}1 {->}{{$\rightarrow{}$}}1 {<-}{{$\leftarrow$}}1 {\\}{{$\lambda$}}1,
  moredelim=**[is][\color{red}]{`}{`},
  moredelim=**[is][\color{white}]{~}{~}
}

\lstdefinestyle{GJ}{language=Java,
  moredelim=**[is][\color{red}]{`}{`},
  moredelim=**[is][\color{white}]{~}{~}
}
\lstdefinestyle{Eiffel}{language=Eiffel,%
  literate={->}{{$\rightarrow$}}1,
  moredelim=**[is][\color{red}]{`}{`},
  moredelim=**[is][\color{white}]{~}{~}
}
\lstdefinestyle{Csharp}{language=[Sharp]C,
  morekeywords={where,require,type},
  literate={->}{{$\rightarrow{}$}}1,
  moredelim=**[is][\color{red}]{`}{`},
  moredelim=**[is][\color{white}]{~}{~}
}

\lstdefinestyle{ML}{language=ML,%
  literate={->}{{$\rightarrow{}$}}1,
  moredelim=**[is][\color{red}]{`}{`},
  moredelim=**[is][\color{white}]{~}{~}
}

\lstdefinestyle{Caml}{language=Caml,%
  morekeywords={when}
  literate={->}{{$\rightarrow{}$}}1,
  moredelim=**[is][\color{red}]{`}{`},
  moredelim=**[is][\color{white}]{~}{~}
}

%% grammar commands
\newcommand{\Rule}[1]{{\rmfamily\itshape{#1}}}
\newcommand{\Alt}{\ensuremath{|}}
\newcommand{\is}{$::=$}
\newcommand{\subtype}{\ensuremath{\texttt{\raisebox{-0.1ex}{<}\raisebox{0.05ex}{:}}}}
\newcommand{\subtypeD}{\ensuremath{<:_d}}
\newcommand{\lazyevals}{\Downarrow}
\newcommand{\evals}{\Rightarrow}
\newcommand{\evalspp}{\Rightarrow^+}
\newcommand{\DynCast}[2]{\ensuremath{\mathsf{dyn\_cast}\langle{#1}\rangle({#2})}}
\newcommand{\nullptr}{\ensuremath{\bot}}
\newcommand{\True}{\ensuremath{\mathsf{true}}}
\newcommand{\False}{\ensuremath{\mathsf{false}}}

\newcommand{\CWildcard}{\ensuremath{\mathit{\bf wildcard}}}
\newcommand{\CValue}   {\ensuremath{\mathit{\bf value}}}
\newcommand{\CVariable}{\ensuremath{\mathit{\bf variable}}}
\newcommand{\CExpr}    {\ensuremath{\mathit{\bf expr}}}
\newcommand{\CGuard}   {\ensuremath{\mathit{\bf guard}}}
\newcommand{\CCnstr}   {\ensuremath{\mathit{\bf ctor}}}

\newcommand{\Wildcard}   {\ensuremath{\CWildcard}}
\newcommand{\Value}[1]   {\ensuremath{\CValue\langle{#1}\rangle}}
\newcommand{\Variable}[1]{\ensuremath{\CVariable\langle{#1}\rangle}}
\newcommand{\ExprU}[2]   {\ensuremath{\CExpr\langle{#1},{#2}\rangle}}
\newcommand{\ExprB}[3]   {\ensuremath{\CExpr\langle{#1},{#2},{#3}\rangle}}
\newcommand{\ExprK}[3]   {\ensuremath{\CExpr\langle{#1},{#2},\cdots,{#3}\rangle}}
\newcommand{\Guard}[2]   {\ensuremath{\CGuard\langle{#1},{#2}\rangle}}
\newcommand{\Cnstr}[3]   {\ensuremath{\CCnstr\langle{#1},{#2},\cdots,{#3}\rangle}}

\newcommand{\f}[1]{{ {{#1\%}}}}
\newcommand{\s}[1]{{ {\bf \underline{#1\%}}}}
\newcommand{\n}[1]{{ {\bf ~ ~ ~ ~ }}}
\newcommand{\Opn}{{\scriptsize {\bf Open}}}
\newcommand{\Cls}{{\scriptsize {\bf Tag}}}
\newcommand{\Unn}{{\scriptsize {\bf Union}}}

%\newcommand{\gwNGPp}{\n{}}
%\newcommand{\gwNGKp}{\n{}}
 \newcommand{\gwNGUp}{\n{}}
%\newcommand{\gwNSPp}{\n{}}
%\newcommand{\gwNSKp}{\n{}}
 \newcommand{\gwNSUp}{\n{}}
%\newcommand{\vwNGPp}{\n{}}
%\newcommand{\vwNGKp}{\n{}}
 \newcommand{\vwNGUp}{\n{}}
%\newcommand{\vwNSPp}{\n{}}
%\newcommand{\vwNSKp}{\n{}}
 \newcommand{\vwNSUp}{\n{}}
%\newcommand{\vxNGPp}{\n{}}
%\newcommand{\vxNGKp}{\n{}}
 \newcommand{\vxNGUp}{\n{}}
%\newcommand{\vxNSPp}{\n{}}
%\newcommand{\vxNSKp}{\n{}}
 \newcommand{\vxNSUp}{\n{}}

%\newcommand{\gwNGPq}{\n{}}
%\newcommand{\gwNGKq}{\n{}}
 \newcommand{\gwNGUq}{\n{}}
%\newcommand{\gwNSPq}{\n{}}
%\newcommand{\gwNSKq}{\n{}}
 \newcommand{\gwNSUq}{\n{}}
%\newcommand{\vwNGPq}{\n{}}
%\newcommand{\vwNGKq}{\n{}}
 \newcommand{\vwNGUq}{\n{}}
%\newcommand{\vwNSPq}{\n{}}
%\newcommand{\vwNSKq}{\n{}}
 \newcommand{\vwNSUq}{\n{}}
%\newcommand{\vxNGPq}{\n{}}
%\newcommand{\vxNGKq}{\n{}}
 \newcommand{\vxNGUq}{\n{}}
%\newcommand{\vxNSPq}{\n{}}
%\newcommand{\vxNSKq}{\n{}}
 \newcommand{\vxNSUq}{\n{}}

%\newcommand{\gwNGPn}{\n{}}
%\newcommand{\gwNGKn}{\n{}}
 \newcommand{\gwNGUn}{\n{}}
%\newcommand{\gwNSPn}{\n{}}
%\newcommand{\gwNSKn}{\n{}}
 \newcommand{\gwNSUn}{\n{}}
%\newcommand{\vwNGPn}{\n{}}
%\newcommand{\vwNGKn}{\n{}}
 \newcommand{\vwNGUn}{\n{}}
%\newcommand{\vwNSPn}{\n{}}
%\newcommand{\vwNSKn}{\n{}}
 \newcommand{\vwNSUn}{\n{}}
%\newcommand{\vxNGPn}{\n{}}
%\newcommand{\vxNGKn}{\n{}}
 \newcommand{\vxNGUn}{\n{}}
%\newcommand{\vxNSPn}{\n{}}
%\newcommand{\vxNSKn}{\n{}}
 \newcommand{\vxNSUn}{\n{}}


%\newcommand{\gwYGPp}{\n{}}
% \newcommand{\gwYGKp}{\n{}}
 \newcommand{\gwYGUp}{\n{}}
%\newcommand{\gwYSPp}{\n{}}
% \newcommand{\gwYSKp}{\n{}}
 \newcommand{\gwYSUp}{\n{}}
%\newcommand{\vwYGPp}{\n{}}
% \newcommand{\vwYGKp}{\n{}}
 \newcommand{\vwYGUp}{\n{}}
%\newcommand{\vwYSPp}{\n{}}
% \newcommand{\vwYSKp}{\n{}}
 \newcommand{\vwYSUp}{\n{}}
%\newcommand{\vxYGPp}{\n{}}
% \newcommand{\vxYGKp}{\n{}}
 \newcommand{\vxYGUp}{\n{}}
%\newcommand{\vxYSPp}{\n{}}
% \newcommand{\vxYSKp}{\n{}}
 \newcommand{\vxYSUp}{\n{}}

%\newcommand{\gwYGPq}{\n{}}
% \newcommand{\gwYGKq}{\n{}}
 \newcommand{\gwYGUq}{\n{}}
%\newcommand{\gwYSPq}{\n{}}
% \newcommand{\gwYSKq}{\n{}}
 \newcommand{\gwYSUq}{\n{}}
%\newcommand{\vwYGPq}{\n{}}
% \newcommand{\vwYGKq}{\n{}}
 \newcommand{\vwYGUq}{\n{}}
%\newcommand{\vwYSPq}{\n{}}
% \newcommand{\vwYSKq}{\n{}}
 \newcommand{\vwYSUq}{\n{}}
%\newcommand{\vxYGPq}{\n{}}
% \newcommand{\vxYGKq}{\n{}}
 \newcommand{\vxYGUq}{\n{}}
%\newcommand{\vxYSPq}{\n{}}
% \newcommand{\vxYSKq}{\n{}}
 \newcommand{\vxYSUq}{\n{}}

%\newcommand{\gwYGPn}{\n{}}
% \newcommand{\gwYGKn}{\n{}}
 \newcommand{\gwYGUn}{\n{}}
%\newcommand{\gwYSPn}{\n{}}
% \newcommand{\gwYSKn}{\n{}}
 \newcommand{\gwYSUn}{\n{}}
%\newcommand{\vwYGPn}{\n{}}
% \newcommand{\vwYGKn}{\n{}}
 \newcommand{\vwYGUn}{\n{}}
%\newcommand{\vwYSPn}{\n{}}
% \newcommand{\vwYSKn}{\n{}}
 \newcommand{\vwYSUn}{\n{}}
%\newcommand{\vxYGPn}{\n{}}
% \newcommand{\vxYGKn}{\n{}}
 \newcommand{\vxYGUn}{\n{}}
%\newcommand{\vxYSPn}{\n{}}
% \newcommand{\vxYSKn}{\n{}}
 \newcommand{\vxYSUn}{\n{}}

 \newcommand{\GwNGPp}{\n{}}
 \newcommand{\GwNGKp}{\n{}}
 \newcommand{\GwNGUp}{\n{}}
 \newcommand{\GwNSPp}{\n{}}
 \newcommand{\GwNSKp}{\n{}}
 \newcommand{\GwNSUp}{\n{}}
%\newcommand{\VwNGPp}{\n{}}
%\newcommand{\VwNGKp}{\n{}}
 \newcommand{\VwNGUp}{\n{}}
%\newcommand{\VwNSPp}{\n{}}
%\newcommand{\VwNSKp}{\n{}}
 \newcommand{\VwNSUp}{\n{}}
%\newcommand{\VxNGPp}{\n{}}
%\newcommand{\VxNGKp}{\n{}}
 \newcommand{\VxNGUp}{\n{}}
%\newcommand{\VxNSPp}{\n{}}
%\newcommand{\VxNSKp}{\n{}}
 \newcommand{\VxNSUp}{\n{}}

 \newcommand{\GwNGPq}{\n{}}
 \newcommand{\GwNGKq}{\n{}}
 \newcommand{\GwNGUq}{\n{}}
 \newcommand{\GwNSPq}{\n{}}
 \newcommand{\GwNSKq}{\n{}}
 \newcommand{\GwNSUq}{\n{}}
%\newcommand{\VwNGPq}{\n{}}
%\newcommand{\VwNGKq}{\n{}}
 \newcommand{\VwNGUq}{\n{}}
%\newcommand{\VwNSPq}{\n{}}
%\newcommand{\VwNSKq}{\n{}}
 \newcommand{\VwNSUq}{\n{}}
%\newcommand{\VxNGPq}{\n{}}
%\newcommand{\VxNGKq}{\n{}}
 \newcommand{\VxNGUq}{\n{}}
%\newcommand{\VxNSPq}{\n{}}
%\newcommand{\VxNSKq}{\n{}}
 \newcommand{\VxNSUq}{\n{}}

 \newcommand{\GwNGPn}{\n{}}
 \newcommand{\GwNGKn}{\n{}}
 \newcommand{\GwNGUn}{\n{}}
 \newcommand{\GwNSPn}{\n{}}
 \newcommand{\GwNSKn}{\n{}}
 \newcommand{\GwNSUn}{\n{}}
%\newcommand{\VwNGPn}{\n{}}
%\newcommand{\VwNGKn}{\n{}}
 \newcommand{\VwNGUn}{\n{}}
%\newcommand{\VwNSPn}{\n{}}
%\newcommand{\VwNSKn}{\n{}}
 \newcommand{\VwNSUn}{\n{}}
%\newcommand{\VxNGPn}{\n{}}
%\newcommand{\VxNGKn}{\n{}}
 \newcommand{\VxNGUn}{\n{}}
%\newcommand{\VxNSPn}{\n{}}
%\newcommand{\VxNSKn}{\n{}}
 \newcommand{\VxNSUn}{\n{}}


 \newcommand{\GwYGPp}{\n{}}
 \newcommand{\GwYGKp}{\n{}}
 \newcommand{\GwYGUp}{\n{}}
 \newcommand{\GwYSPp}{\n{}}
 \newcommand{\GwYSKp}{\n{}}
 \newcommand{\GwYSUp}{\n{}}
%\newcommand{\VwYGPp}{\n{}}
% \newcommand{\VwYGKp}{\n{}}
 \newcommand{\VwYGUp}{\n{}}
%\newcommand{\VwYSPp}{\n{}}
% \newcommand{\VwYSKp}{\n{}}
 \newcommand{\VwYSUp}{\n{}}
%\newcommand{\VxYGPp}{\n{}}
% \newcommand{\VxYGKp}{\n{}}
 \newcommand{\VxYGUp}{\n{}}
%\newcommand{\VxYSPp}{\n{}}
% \newcommand{\VxYSKp}{\n{}}
 \newcommand{\VxYSUp}{\n{}}

 \newcommand{\GwYGPq}{\n{}}
 \newcommand{\GwYGKq}{\n{}}
 \newcommand{\GwYGUq}{\n{}}
 \newcommand{\GwYSPq}{\n{}}
 \newcommand{\GwYSKq}{\n{}}
 \newcommand{\GwYSUq}{\n{}}
%\newcommand{\VwYGPq}{\n{}}
% \newcommand{\VwYGKq}{\n{}}
 \newcommand{\VwYGUq}{\n{}}
%\newcommand{\VwYSPq}{\n{}}
% \newcommand{\VwYSKq}{\n{}}
 \newcommand{\VwYSUq}{\n{}}
%\newcommand{\VxYGPq}{\n{}}
% \newcommand{\VxYGKq}{\n{}}
 \newcommand{\VxYGUq}{\n{}}
%\newcommand{\VxYSPq}{\n{}}
% \newcommand{\VxYSKq}{\n{}}
 \newcommand{\VxYSUq}{\n{}}

 \newcommand{\GwYGPn}{\n{}}
 \newcommand{\GwYGKn}{\n{}}
 \newcommand{\GwYGUn}{\n{}}
 \newcommand{\GwYSPn}{\n{}}
 \newcommand{\GwYSKn}{\n{}}
 \newcommand{\GwYSUn}{\n{}}
%\newcommand{\VwYGPn}{\n{}}
% \newcommand{\VwYGKn}{\n{}}
 \newcommand{\VwYGUn}{\n{}}
%\newcommand{\VwYSPn}{\n{}}
% \newcommand{\VwYSKn}{\n{}}
 \newcommand{\VwYSUn}{\n{}}
%\newcommand{\VxYGPn}{\n{}}
% \newcommand{\VxYGKn}{\n{}}
 \newcommand{\VxYGUn}{\n{}}
%\newcommand{\VxYSPn}{\n{}}
% \newcommand{\VxYSKn}{\n{}}
 \newcommand{\VxYSUn}{\n{}}

% This file defines variables with performance numbers for the table in Evaluation section
% Data from 2011-08-30 
\newcommand{\vwYGKp}{\s{3}}
\newcommand{\vwYGKn}{\s{8}}
\newcommand{\vwYGKq}{\s{11}}
\newcommand{\vwYGPp}{\f{10}}
\newcommand{\vwYGPn}{\f{14}}
\newcommand{\vwYGPq}{\s{0}}
\newcommand{\vwYSKp}{\s{7}}
\newcommand{\vwYSKn}{\s{7}}
\newcommand{\vwYSKq}{\s{10}}
\newcommand{\vwYSPp}{\f{10}}
\newcommand{\vwYSPn}{\f{14}}
\newcommand{\vwYSPq}{\s{0}}
\newcommand{\vwNGKp}{\f{35}}
\newcommand{\vwNGKn}{\s{6}}
\newcommand{\vwNGKq}{\s{5}}
\newcommand{\vwNGPp}{\f{1}}
\newcommand{\vwNGPn}{\s{1}}
\newcommand{\vwNGPq}{\s{10}}
\newcommand{\vwNSKp}{\f{133}}
\newcommand{\vwNSKn}{\f{25}}
\newcommand{\vwNSKq}{\f{59}}
\newcommand{\vwNSPp}{\f{1}}
\newcommand{\vwNSPn}{\s{1}}
\newcommand{\vwNSPq}{\s{8}}

\newcommand{\vxYGKp}{\s{61}}
\newcommand{\vxYGKn}{\s{24}}
\newcommand{\vxYGKq}{\s{25}}
\newcommand{\vxYGPp}{\s{24}}
\newcommand{\vxYGPn}{\s{24}}
\newcommand{\vxYGPq}{\s{36}}
\newcommand{\vxYSKp}{\s{79}}
\newcommand{\vxYSKn}{\s{25}}
\newcommand{\vxYSKq}{\s{35}}
\newcommand{\vxYSPp}{\s{9}}
\newcommand{\vxYSPn}{\s{23}}
\newcommand{\vxYSPq}{\f{133}}
\newcommand{\vxNGKp}{\s{8}}
\newcommand{\vxNGKn}{\s{5}}
\newcommand{\vxNGKq}{\s{0}}
\newcommand{\vxNGPp}{\s{33}}
\newcommand{\vxNGPn}{\s{47}}
\newcommand{\vxNGPq}{\s{43}}
\newcommand{\vxNSKp}{\f{38}}
\newcommand{\vxNSKn}{\f{12}}
\newcommand{\vxNSKq}{\f{3}}
\newcommand{\vxNSPp}{\s{27}}
\newcommand{\vxNSPn}{\s{44}}
\newcommand{\vxNSPq}{\s{45}}

\newcommand{\gwYGKp}{\f{88}}
\newcommand{\gwYGKn}{\f{32}}
\newcommand{\gwYGKq}{\f{250}}
\newcommand{\gwYGPp}{\f{67}}
\newcommand{\gwYGPn}{\f{28}}
\newcommand{\gwYGPq}{\f{87}}
\newcommand{\gwYSKp}{\f{79}}
\newcommand{\gwYSKn}{\f{31}}
\newcommand{\gwYSKq}{\f{259}}
\newcommand{\gwYSPp}{\f{67}}
\newcommand{\gwYSPn}{\f{27}}
\newcommand{\gwYSPq}{\f{90}}
\newcommand{\gwNGKp}{\f{116}}
\newcommand{\gwNGKn}{\f{29}}
\newcommand{\gwNGKq}{\f{43}}
\newcommand{\gwNGPp}{\f{55}}
\newcommand{\gwNGPn}{\s{0}}
\newcommand{\gwNGPq}{\f{1}}
\newcommand{\gwNSKp}{\f{216}}
\newcommand{\gwNSKn}{\f{542}}
\newcommand{\gwNSKq}{\f{520}}
\newcommand{\gwNSPp}{\f{55}}
\newcommand{\gwNSPn}{\f{1}}
\newcommand{\gwNSPq}{\f{3}}

\newcommand{\VwYGKp}{\f{16}}
\newcommand{\VwYGKn}{\f{11}}
\newcommand{\VwYGKq}{\f{168}}
\newcommand{\VwYGPp}{\f{10}}
\newcommand{\VwYGPn}{\f{19}}
\newcommand{\VwYGPq}{\f{153}}
\newcommand{\VwYSKp}{\f{31}}
\newcommand{\VwYSKn}{\f{24}}
\newcommand{\VwYSKq}{\f{185}}
\newcommand{\VwYSPp}{\f{10}}
\newcommand{\VwYSPn}{\f{18}}
\newcommand{\VwYSPq}{\f{153}}
\newcommand{\VwNGKp}{\f{61}}
\newcommand{\VwNGKn}{\f{18}}
\newcommand{\VwNGKq}{\f{13}}
\newcommand{\VwNGPp}{\f{4}}
\newcommand{\VwNGPn}{\s{17}}
\newcommand{\VwNGPq}{\s{9}}
\newcommand{\VwNSKp}{\f{124}}
\newcommand{\VwNSKn}{\f{43}}
\newcommand{\VwNSKq}{\f{34}}
\newcommand{\VwNSPp}{\f{4}}
\newcommand{\VwNSPn}{\s{18}}
\newcommand{\VwNSPq}{\f{3}}
               
\newcommand{\VxYGKp}{\s{5}}
\newcommand{\VxYGKn}{\s{2}}
\newcommand{\VxYGKq}{\f{132}}
\newcommand{\VxYGPp}{\s{5}}
\newcommand{\VxYGPn}{\s{5}}
\newcommand{\VxYGPq}{\f{130}}
\newcommand{\VxYSKp}{\s{9}}
\newcommand{\VxYSKn}{\s{10}}
\newcommand{\VxYSKq}{\f{118}}
\newcommand{\VxYSPp}{\s{6}}
\newcommand{\VxYSPn}{\s{6}}
\newcommand{\VxYSPq}{\f{145}}
\newcommand{\VxNGKp}{\f{20}}
\newcommand{\VxNGKn}{\f{7}}
\newcommand{\VxNGKq}{\f{34}}
\newcommand{\VxNGPp}{\s{14}}
\newcommand{\VxNGPn}{\s{27}}
\newcommand{\VxNGPq}{\f{2}}
\newcommand{\VxNSKp}{\f{47}}
\newcommand{\VxNSKn}{\f{16}}
\newcommand{\VxNSKq}{\f{14}}
\newcommand{\VxNSPp}{\s{0}}
\newcommand{\VxNSPn}{\s{27}}
\newcommand{\VxNSPq}{\f{1}}

% This file defines variables with performance numbers for the table in Evaluation section
% Data from 2011-11-04 collected on Sierra for Linux under g++ (GCC) 4.4.5 20101112 (Red Hat 4.4.5-2)

\newcommand{\glYGKp}{\f{45}}
\newcommand{\glYGKn}{\f{82}}
\newcommand{\glYGKq}{\f{77}}
\newcommand{\glYGPp}{\f{36}}
\newcommand{\glYGPn}{\f{76}}
\newcommand{\glYGPq}{\f{53}}
\newcommand{\glYSKp}{\f{53}}
\newcommand{\glYSKn}{\f{88}}
\newcommand{\glYSKq}{\f{86}}
\newcommand{\glYSPp}{\f{33}}
\newcommand{\glYSPn}{\f{78}}
\newcommand{\glYSPq}{\f{55}}
\newcommand{\glNGKp}{\f{54}}
\newcommand{\glNGKn}{\f{97}}
\newcommand{\glNGKq}{\f{109}}
\newcommand{\glNGPp}{\f{19}}
\newcommand{\glNGPn}{\f{57}}
\newcommand{\glNGPq}{\f{62}}
\newcommand{\glNSKp}{\f{124}}
\newcommand{\glNSKn}{\f{603}}
\newcommand{\glNSKq}{\f{640}}
\newcommand{\glNSPp}{\f{16}}
\newcommand{\glNSPn}{\f{56}}
\newcommand{\glNSPq}{\f{56}}


\newsavebox{\sembox}
\newlength{\semwidth}
\newlength{\boxwidth}

\newcommand{\Sem}[1]{%
\sbox{\sembox}{\ensuremath{#1}}%
\settowidth{\semwidth}{\usebox{\sembox}}%
\sbox{\sembox}{\ensuremath{\left[\usebox{\sembox}\right]}}%
\settowidth{\boxwidth}{\usebox{\sembox}}%
\addtolength{\boxwidth}{-\semwidth}%
\left[\hspace{-0.3\boxwidth}%
\usebox{\sembox}%
\hspace{-0.3\boxwidth}\right]%
}

\newcommand{\authormodification}[2]{{\color{#1}#2}}
\newcommand{\ys}[1]{\authormodification{blue}{#1}}
\newcommand{\bs}[1]{\authormodification{red}{#1}}
\newcommand{\gdr}[1]{\authormodification{magenta}{#1}}

\begin{document}

%\conferenceinfo{DSL 2011}{Bordeaux, France} 
%\copyrightyear{2011} 
%\copyrightdata{[to be supplied]} 

\titlebanner{Technical Report}        % These are ignored unless
\preprintfooter{Y.Solodkyy, G.Dos Reis, B.Stroustrup: An Elegant and Efficient Pattern Matching Library for C++}   % 'preprint' option specified.

\title{An Elegant and Efficient Pattern Matching Library for C++}
%\subtitle{your \code{visit}, Jim, is not \code{accept}able anymore}
\subtitle{\code{accepting} aint no \code{visit}ors}

\authorinfo{Yuriy Solodkyy\and Gabriel Dos Reis\and Bjarne Stroustrup}
           {Texas A\&M University\\ Texas, USA}
           {\{yuriys,gdr,bs\}@cse.tamu.edu}

\maketitle

\begin{abstract}
Pattern matching is an abstraction mechanism that greatly simplifies code. We 
present functional-programming-style pattern matching for C++ 
implemented as a library. The library provides a uniform notation for 
matching against open hierarchy of run-time polymorphic classes as well as closed 
set of classes (including classes tagged by user and discriminated unions)
for which compile-time polymorphism can be used. The
 library integrates well with programming styles supported by C++, in 
particular it supports virtual and repeated multiple inheritance and can
 be used in generic code.

Our library equals or outperforms the visitor design pattern, as commonly 
used for pattern-matching scenarios in C++, and for many use cases it 
equals or outperforms equivalent code in languages with built-in pattern
 matching. Our solution better addresses more problems than the visitor
 design pattern does: it is non-intrusive and does not have 
extensibility restrictions. It also avoids control 
inversion and can be used in pattern-matching scenarios that visitors are ill suited for.
Code using patterns is significantly more concise and easier to comprehend than alternative solutions in C++.

Implementing pattern matching as a library allows us to experiment with syntax, 
implementation algorithms, and use while preserving benefit from the 
performance and portability provided by industrial compilers and support
 tools. The solution approach can be reused in other object-oriented 
languages to implement \emph{type switching}, \emph{type testing}, 
\emph{pattern matching} and \emph{multiple dispatch} efficiently.

The library was motivated by and is used for applications involving large, typed, abstract syntax trees.
\end{abstract}

\category{D}{1}{5}
\category{D}{3}{3}

\terms
Languages, Design

\keywords
Pattern Matching, Type Switch, Typecase, Visitor Design Pattern, Expression Problem, Memoization, C++

\section{Introduction} %%%%%%%%%%%%%%%%%%%%%%%%%%%%%%%%%%%%%%%%%%%%%%%%%%%%%%%%%
\label{sec:intro}

%Motivate the problem
%Give a summary of the paper: what you did and how
%Explicitly state your contribution

Pattern matching is an abstraction supported by many programming languages.
It allows the user tersely to describe a (possibly infinite) set of 
values accepted by the pattern. A \emph{pattern} represents a predicate on 
values, and is usually  much more concise and readable than the 
equivalent predicate spelled out as imperative code.

Popularized by functional programming community, most notably Hope\cite{BMS80}, 
ML\cite{ML90}, Miranda\cite{Miranda85} and Haskell\cite{Haskell98Book}, for 
providing syntax very close to mathematical notations.
From there, it has 
found its way into many imperative programming languages e.g. 
Pizza\cite{Odersky97pizzainto}, Scala\cite{Scala2nd}, Fortress\cite{RPS10}, as 
well as dialects of Java\cite{Liu03jmatch:iterable,HydroJ2003}, C++\cite{Prop96}, 
Eiffel\cite{Moreau:2003} and others. It is relatively easy to provide a form of pattern 
matching when designing a new language, but to introduce it into a language in 
widespread use is a challenge. The obvious utility of the feature may be 
compromised by the need to fit into the language's syntax, semantics, and tool 
chains. A prototype implementation requires more effort than for an experimental 
language and is harder to get into use because mainstream users are unwilling 
to try non-portable, non-standard, and unoptimized features.

To balance the utility and effort we decided to take the Semantically 
Enhanced Library Language (SELL) approach\cite{SELL}. We provide the
general-purpose programming language with a library, extended with a tool 
support. This will typically (as in this case) not provide you 100\% of the functionality that a 
language extension would do, but it allows experimentation and special-purpose use
with existing compilers and tool chains. With pattern matching, we managed to avoid 
external tool support by relying on some pretty nasty macro hacking to provide a
conventional and convenient interface to an efficient library implementation.
By efficient, we mean about as fast as functional languages for closed cases and
much better than code generated for visitor patterns by commercial optimizers 
for open cases\cite{TypeSwitch}.

Our current solution is a proof of concept that sets a minimum threshold for 
performance, brevity, clarity and usefulness of a language solution for pattern 
matching in C++. It provides full functionality, so we can experiment with use 
of pattern matching in C++ and with language alternatives. To give an idea of 
what our library offers, consider an example from a domain where pattern matching 
is considered to provide terseness and clarity -- compiler construction. 
Consider for example a simple language of expressions:

\begin{lstlisting}
@$exp$ \is{} $val$ \Alt{} $exp+exp$ \Alt{} $exp-exp$ \Alt{} $exp*exp$ \Alt{} $exp/exp$@
\end{lstlisting}

\noindent
An OCaml data type describing this grammar as well as a simple evaluator of expressions 
in it, can be declared as following:

\begin{lstlisting}[language=Caml,keepspaces,columns=flexible]
type expr = Value of int 
          | Plus  of expr * expr | Minus  of expr * expr 
          | Times of expr * expr | Divide of expr * expr
          ;;

let rec eval e =
  match e with
            Value  v      -> v
          | Plus   (a, b) -> (eval a) + (eval b)
          | Minus  (a, b) -> (eval a) - (eval b)
          | Times  (a, b) -> (eval a) * (eval b)
          | Divide (a, b) -> (eval a) / (eval b)
          ;;
\end{lstlisting}

\noindent
The corresponding C++ data types would most likely be parameterized, but for
now we will just use simple classes:

\begin{lstlisting}[keepspaces,columns=flexible]
struct Expr { virtual @$\sim$@Expr() {} };
struct Value  : Expr { int value; };
struct Plus   : Expr { Expr* exp1; Expr* exp2; };
struct Minus  : Expr { Expr* exp1; Expr* exp2; };
struct Times  : Expr { Expr* exp1; Expr* exp2; };
struct Divide : Expr { Expr* exp1; Expr* exp2; };
\end{lstlisting}

\noindent
Using our library, we can express matching about as tersely as OCaml:

\begin{lstlisting}[keepspaces,columns=flexible]
int eval(const Expr& e)
{
    Match(e)
    {
      Case(Value,  n)    return n;
      Case(Plus,   a, b) return eval(a) + eval(b);
      Case(Minus,  a, b) return eval(a) - eval(b);
      Case(Times,  a, b) return eval(a) * eval(b);
      Case(Divide, a, b) return eval(a) / eval(b);
    }
    EndMatch
}
\end{lstlisting}

\noindent
To make the example fully functional we need to provide mappings of binding 
positions to corresponding class members:

\begin{lstlisting}[keepspaces,columns=flexible]
template <> struct bindings<Value>  { CM(0,Value::value); };
template <> struct bindings<Plus>   { CM(0,Plus::exp1); 
  ...                                 CM(1,Plus::exp2);   };
template <> struct bindings<Divide> { CM(0,Divide::exp1); 
                                      CM(1,Divide::exp2); };
\end{lstlisting}

\noindent
This binding code would be implicitly provided by the compiler had
we chosen that implementation strategy.

The syntax is provided without any external tool support. Instead we rely on a 
few C++11 features~\cite{C++11}, template meta-programming, and macros. It runs 
about as fast as OCaml and Haskell equivalents (\textsection\ref{sec:ocaml}), and, depending 
on the usage scenario, compiler and underlying hardware, comes close or 
outperforms the handcrafted C++ code based on the \emph{visitor design pattern} 
(\textsection\ref{sec:eval}).

\subsection{Motivation}

The ideas and the 
\emph{Mach7} library were motivated by our unsatisfactory experiences working 
with various \Cpp{} front-ends and program analysis frameworks~\cite{Pivot09,gdr-2012:liz,Phoenix,Clang}. 
The problem was not in the frameworks per se, but in the fact that we had to use
the \emph{visitor design pattern}~\cite{DesignPatterns1993} to inspect, traverse, and 
elaborate abstract syntax trees of target languages. We found visitors 
unsuitable to express application logic directly, surprisingly hard to teach 
students, and often slower than handcrafted workaround techniques. 
We found dynamic casts in many places, often nested, 
because users wanted to answer simple structural 
questions without having to resort to visitors. Users preferred shorter, cleaner, 
and more direct code to visitors, even at a high cost in performance (assuming 
that the programmer knew the cost). The usage of \code{dynamic\_cast} resembled 
the use of pattern matching in functional languages to unpack algebraic data 
types. Thus, our initial goal was to develop a domain-specific library for C++ 
to express various predicates on tree-like structures as elegantly as is done in functional 
languages. This grew into a general high-performance pattern-matching library.

The library is the latest in a series of 7 libraries. The earlier versions were 
superceded because they failed to meet our standards for notation, performance, 
or generality. Our standard is set by the principle that a fair comparison must 
be against the gold standard in a field. For example, if we work on a linear 
algebra library, we must compare to Fortran or one of the industrial C++ 
libraries, rather than Java or C. For pattern matching we chose optimized OCaml 
as our standard for closed (compile-time polymorphic) sets of classes and C++ 
for uses of the visitor pattern. For generality and simplicity of use, we deemed 
it essential to do both with a uniform syntax.

\subsection{The Expression Problem}
\label{sec:exp}

%Expression problem is a problem of supporting in a programming language modular 
%extensibility of both data and functions at the same time. Functional languages
%allow for easy addition of new functions at the expense of disallowing new data
%variants. Object-oriented languages allow for easy addition of new variants at 
%the expense of disallowing new functions. Many attempts have been made to 
%resolve this dilema in both camps, nevertheless no universally accepted solution 
%that is modular, open and efficient has been found.

%Visitor Design Pattern has became de-facto standard in dealing with expression 
%problem in many industry-strength object-oriented languages because of two 
%factors: its speed and being a library solution. It comes at the cost of 
%restricting extensibility of data, increased verbosity and being hard to teach 
%and understand, but nevertheless, remains the weapon of choice for interacting 
%with numerous object-oriented libraries and frameworks. 

Type switching is related to a more general problem manifesting the differences 
in functional and object-oriented programming styles.
Conventional algebraic datatypes, as found in most functional languages, allow 
for easy addition of new functions on existing data types. However, they fall short 
in extending data types themselves (e.g. with new constructors), which requires 
modifying the source code. Object-oriented languages make 
data type extension trivial through inheritance, but the addition of new 
functions operating on these classes typically requires changes to the class 
definition. This dilemma is known as the \emph{expression problem}~\cite{Cook90,exprproblem}.

Classes differ from algebraic data types in two important ways. Firstly, they
are \emph{extensible}, for new variants can be added later by inheriting from
the base class. Secondly, they are \emph{hierarchical} and thus typically 
\emph{non-disjoint} since variants can be inherited from other variants and form 
a subtyping relation between themselves~\cite{Glew99}. In contrast, variants in 
conventional algebraic data types are \emph{disjoint} and \emph{closed}.
Some functional languages e.g. ML2000~\cite{ML2000} and its predecessor, Moby, 
were experimenting with \emph{hierarchical extensible sum types}, which are 
closer to object-oriented classes then algebraic data types are, but, 
interestingly, they provided no %neither traditional nor efficient 
facilities for performing case analysis on them.

Functional languages allow for the easy addition of new functions on existing data 
types, but fall short in extending data types themselves (e.g. with new constructors), 
which requires modifying the source code. Object-oriented languages, on the 
other hand, make data type extension trivial through inheritance, but the addition 
of new functions that work on these classes typically requires changes to the class 
definition. This dilemma was first discussed by Cook~\cite{Cook90} and then 
accentuated by Wadler~\cite{exprproblem} under the name \emph{expression problem}. Quoting Wadler:

\emph{``The Expression Problem is a new name for an old problem. The goal is
to define a datatype by cases, where one can add new cases to the
datatype and new functions over the datatype, without recompiling
existing code, and while retaining static type safety (e.g., no
casts)''}.

To better understand the problem, note that classes differ from algebraic data 
types in two important ways: they are \emph{extensible} since new variants can 
be added by inheriting from the base class, as well as \emph{hierarchical} and 
thus \emph{non-disjoint} since variants can be inherited from other variants and 
form a subtyping relation between themselves~\cite{Glew99}. This is not the case 
with traditional algebraic data types in functional languages, where the set of 
variants is \emph{closed}, while the variants are \emph{disjoint}. Some 
functional languages e.g. ML2000~\cite{ML2000} and Moby~\cite{Moby} were 
experimenting with \emph{hierarchical extensible sum types}, which are closer to 
object-oriented classes then algebraic data types are, but, interestingly, they 
did not provide pattern matching facilities on them!

Zenger and Odersky refined the expression problem in the context of 
independently extensible solutions~\cite{fool12} as a challenge to find an 
implementation technique that satisfies the following requirements:
%
\begin{itemize}
\setlength{\itemsep}{0pt}
\setlength{\parskip}{0pt}
\item \emph{Extensibility in both dimensions}: It should be possible to add new 
      data variants, while adapting the existing operations accordingly. It 
      should also be possible to introduce new functions. 
\item \emph{Strong static type safety}: It should be impossible to apply a 
      function to a data variant, which it cannot handle. 
\item \emph{No modification or duplication}: Existing code should neither be 
      modified nor duplicated.
\item \emph{Separate compilation}: Neither datatype extensions nor addition of 
      new functions should require re-typechecking the original datatype or 
      existing functions. No safety checks should be deferred until link or 
      runtime.
\item \emph{Independent extensibility}: It should be possible to combine 
      independently developed extensions so that they can be used jointly.
\end{itemize}

%Paraphrasing, the expression problem can be summarized as a problem of 
%supporting modular extensibility of both data and functions at the same time in 
%one programming language.

\noindent
Object-oriented languages further complicate the matter with the fact that 
data variants are not necessarily disjoint and may form subtyping relationships  
between themselves. We thus introduced an additional requirement based on the
Liskov substitution principle~\cite{Lis87}:

\begin{itemize}
\setlength{\itemsep}{0pt}
\setlength{\parskip}{0pt}
\item \emph{Substitutability}: Operations expressed on more general data variants
      should be applicable to ones that are more specific
      (the latter being in a subtyping relation with the former).
\end{itemize}

%Depending on the semantics of the language's subtyping relation, 
%substitutability requirement may turn pattern matching into an expensive 
%operation. OCaml, for example, that uses structural subtyping on its object 
%types, does not offer pattern 

\noindent
We will refer to a solution that satisfies all of the above requirements as \emph{open}. 
Numerous solutions have been proposed to dealing with the expression problem in both 
functional~\cite{garrigue-98,LohHinze2006} and object-oriented 
camps~\cite{Palsberg98,Krishnamurthi98,Zenger:2001,runabout}, but very few have
made their way into one of the mainstream languages. We refer the reader to Zenger 
and Odersky's original manuscript for a discussion of the approaches~\cite{fool12}. 
Interestingly, most of the discussed object-oriented solutions focused on the visitor design pattern~\cite{DesignPatterns1993} and its extensions, 
which even today seem to be the most commonly used approach to dealing with the 
expression problem in object-oriented languages.

\subsection{Visitor Design Pattern}
\label{sec:vdp}

%Discuss visitor design pattern and its problems.
%\begin{itemize}
%\item Intrusive - requires changes to the hierarchy
%\item Not open  - addition of new classes changes visitor interface
%\item Does not provide by default relation between visitors of base and derived classes
%\item Control inversion
%\item Cannot be generically extended to handling n arguments
%\end{itemize}

The \emph{visitor design pattern}~\cite{DesignPatterns1993} was devised to solve the problem 
of extending existing classes with new functions in object-oriented languages. 
Consider the above Expr example and imagine that in addition to evaluation we would like to also provide a pretty 
printing of expressions. A typical object-oriented approach would be to 
introduce a virtual function \\ \code{virtual void print() const = 0;} inside 
the abstract base class \code{Expr}, which will be implemented correspondingly 
in all the derived classes. This works well as long as we know all the required  
operations on the abstract class in advance. Unfortunately, this is very 
difficult to achieve in reality as the code evolves, especially in a production 
environment. To put this in context, imagine that after the above interface with 
pretty-printing functionality has been deployed, we decided that we need 
similar functionality that saves the expression in XML format. Adding new 
virtual function implies modifying the base class and creating a versioning 
problem with the code that has been deployed already using the old interface.

To alleviate this problem, the Visitor Design Pattern separates the 
\emph{commonality} of all such future member-functions from their 
\emph{specifics}. The former deals with identifying the most-specific derived 
class of the receiver object known to the system at the time the base class was 
designed. The latter provides implementation of the required functionality once 
the most-specific derived class has been identified. The interaction between the 
two is encoded in the protocol that fixes a \emph{visitation interface} 
enumerating all known derived classes on one side and a dispatching mechanism 
that guarantees to select the most-specific case with respect to the dynamic 
type of the receiver in the visitation interface. An implementation of this 
protocol for our Expr example might look like the following:

\begin{lstlisting}
// Forward declaration of known derived classes
struct Value; struct Plus; ... struct Divide;
@\halfline@
// Visitation interface
struct ExprVisitor
{
    virtual void visit(const Value&)  = 0;
    virtual void visit(const Plus&)   = 0;
    ...  // One virtual function per each known derived class
    virtual void visit(const Divide&) = 0;
};
@\halfline@
// Abstract base and known derived classes
struct Expr { 
    virtual void accept(ExprVisitor&) const = 0; };
struct Value : Expr { ...
    void accept(ExprVisitor& v) const { v.visit(*this); } };
struct Plus  : Expr { ...
    void accept(ExprVisitor& v) const { v.visit(*this); } };
\end{lstlisting}

\noindent
Note that even though implementations of \code{accept} member-functions in all 
derived classes are syntactically identical, a different \code{visit} is called. 
We rely here on the overload resolution mechanism of C++ to pick the most 
specialized \code{visit} member-function applicable to the static type of 
\code{*this}.

%This mere code 
%maintenance convenience unfortunately, often confuses novices on what 
%is going on. We thus would like to point out that member-functions in the 
%visitation interface are not required to be called with the same name, -- we 
%could have equally well called them \code{visit_value}, \code{visit_plus} etc. 
%making the corresponding changes to calls inside \code{Value::accept}, 
%\code{Plus::accept} etc.

A user can now implement new functions by overriding \code{ExprVisitor}'s 
functions. For example:

\begin{lstlisting}
std::string to_str(const Expr* e) // Converts expressions to string
{
  struct ToStrVisitor : ExprVisitor
  {
    void visit(const Value& e) { result = std::to_string(e.value); }
    ...
    void visit(const Divide& e) { 
        result = to_str(e.exp1) + '/' + to_str(e.exp2); 
    }
    std::string result;
  } v;
  e->accept(v);
  return v.result;
}
\end{lstlisting}

\noindent
The function \code{eval} we presented above, as well as any new function that we 
would like to add to \code{Expr}, can now be implemented in much the same way, 
without the need to change the base interface. This flexibility does not come for free, 
though, and we would like to point out some pros and cons of this solution.

The most important advantage of the visitor design pattern is the {\bf possibility 
to add new operations} to the class hierarchy without the need to change 
the interface. Its second most-quoted advantage is {\bf speed} -- the 
overhead of two virtual function calls incurred by the double  
dispatch present in the visitor design pattern is often negligible on modern 
architectures. Yet another advantage that often remains unnoticed is that the 
above solution achieves extensibility of functions with {\bf library only means} 
by using facilities already present in the language. Nevertheless, there are 
quite a few disadvantages.

The solution is {\bf intrusive} since we had to inject syntactically the same 
definition of the \code{accept} method into every class participating in visitation. 
It is also {\bf specific to hierarchy}, as we had to declare a visitation 
interface specific to the base class. The amount of {\bf boilerplate code} 
required by visitor design pattern cannot go unnoticed. It also increases with 
every argument that has to be passed into the visitor to be available during the 
visitation. This aspect can be seen in the example from \textsection\ref{sec:xmpl} 
where we have to store both functors inside the visitor.

More importantly, visitors {\bf hinder extensibility} of the class hierarchy: 
new classes added to the hierarchy after the visitation interface has been 
fixed will be treated as their most derived base class present in the interface.
A solution to this problem has been proposed in the form of \emph{Extensible 
Visitors with Default Cases}~\cite[\textsection 4.2]{Zenger:2001}; however, the 
solution, after remapping it onto C++, has problems of its own, discussed in 
detail in related work in \textsection\ref{sec:rw}.

%The visitation interface 
%hierarchy can easily be grown linearly (adding new cases for the new classes in 
%the original hierarchy each time), but independent extensions by different  
%authorities require developer's intervention to unify them all, before they can 
%be used together. This may not be feasible in environments that use dynamic 
%linking. To avoid writing even more boilerplate code in new visitors, the 
%solution would require usage of virtual inheritance, which typically has 
%an overhead of extra memory dereferencing. On top of the double dispatch already 
%present in the visitor pattern, the solution will incur two additional virtual 
%calls and a dynamic cast for each level of visitor extension. Additional double 
%dispatch is incurred by forwarding of default handling from base visitor to a 
%derived one, while the dynamic cast is required for safety and can be replaced 
%with a static case when visitation interface is guaranteed to be grown linearly 
%(extended by one authority only). Yet another virtual call is required to be 
%able to forward computations to subcomponents on tree-like structures to the 
%most derived visitor. This last function lets one avoid the necessity of using 
%heap to allocate a temporary visitor through the \emph{Factory Design 
%Pattern}\cite{DesignPatterns1993} used in \emph{Extensible Visitor} solution 
%originally proposed by Krishnamurti, Felleisen and Friedman\cite{Krishnamurthi98}.

Once all the boilerplate related to visitors has been written and the visitation 
interface has been fixed we are still left with some annoyances incurred by the 
pattern. One of them is the necessity to work with the {\bf control inversion} 
that visitors put in place. Because of it we have to save any local state and 
any arguments that some of the \code{visit} callbacks might need from the 
calling environment. Similarly, we have to save the result of the visitation, 
as we cannot assume that all the visitors that will potentially be implemented 
on a given hierarchy will use the same result type. Using visitors in a generic 
algorithm requires even more precautions. We summarize these visitor-related 
issues in the following motivating example, followed by an illustration of a 
pattern-matching solution to the same problem enabled with our library.

\subsection{Motivating Example}
\label{sec:xmpl}

While comparing generic programming facilities available to functional and 
imperative languages (mainly Haskell and C++), Dos Reis and J\"arvi present the 
following example in Haskell describing a sum functor\cite{DRJ05}:

\begin{lstlisting}[language=Haskell,keepspaces]
data Either a b = Left a | Right b
@\halfline@
eitherLift :: (a -> c) -> (b -> d) -> Either a b -> Either c d
eitherLift f g (Left  x) = Left  (f x)
eitherLift f g (Right y) = Right (g y)
\end{lstlisting}

\noindent
In simple words, the function \codehaskell{eitherLift} above takes two functions and an 
object and depending on the actual type constructor the object was created with, 
calls first or second function on the embedded value, encoding the result 
correspondingly.

Its equivalent in C++ is not straightforward. Idiomatic, type-safe, handling of 
discriminated unions in C++ typically assumes use of the \emph{Visitor Design Pattern}\cite{DesignPatterns}, 
which in this case amounts to 25 lines of ``boiler plate code'' plus 14 lines 
of the specific functionality.

\begin{lstlisting}
template <class X, class Y> class Either;
template <class X, class Y> class Left;
template <class X, class Y> class Right;
@\halfline@
template <class X, class Y>
struct EitherVisitor {
    virtual void visit(const  Left<X,Y>&) = 0;
    virtual void visit(const Right<X,Y>&) = 0;
};
@\halfline@
template <class X, class Y>
struct Either {
    virtual @$\sim$@Either() {}
    virtual void accept(EitherVisitor<X,Y>& v) const = 0;
};
@\halfline@
template <class X, class Y>
struct Left : Either<X,Y> {
    const X& x;
    Left(const X& x) : x(x) {}
    void accept(EitherVisitor<X,Y>& v) const { v.visit(*this); }
};
@\halfline@
template <class X, class Y>
struct Right : Either<X,Y> {
    const Y& y;
    Right(const Y& y) : y(y) {}
    void accept(EitherVisitor<X,Y>& v) const { v.visit(*this); }
};
\end{lstlisting}

\noindent
The code above defines the necessary parameterized data structures as well as a 
correspondingly parameterized visitor class capable of introspecting it at 
run-time. The authors agree with us \emph{``The code has a fair amount of 
boilerplate to simulate pattern matching...''}\cite{DRJ05} The actual 
implementation of \codehaskell{lift} in C++ now amounts to declaring and 
invoking a visitor:

\begin{lstlisting}
template <class X, class Y, class S, class T>
const Either<S,T>& eitherLift(const Either<X,Y>& e, S f(X), T g(Y))
{
    typedef S (*F)(X);
    typedef T (*G)(Y);
    struct Impl : EitherVisitor<X,Y> {
        F f;
        G g;
        const Either<S,T>* value;
        Impl(F f, G g) : f(f), g(g), value() {}
        void visit(const Left<X,Y>& e) {
            value = left<S,T>(f(e.x));
        }
        void visit(const Right<X,Y>& e) {
            value = right<S,T>(g(e.y));
        }
    };
    Impl vis(f, g);
    e.accept(vis);
    return *vis.value;
}
\end{lstlisting}

\noindent
The same function expressed with our pattern-matching facility seems to be much 
closer to the original Haskell definition:

\begin{lstlisting}[keepspaces,columns=flexible]
template <class X, class Y, class S, class T>
const Either<S,T>* lift(const Either<X,Y>& e, S f(X), T g(Y))
{
    Match(e)
      Case(( Left<X,Y>), x) return  left<S,T>(f(x));
      Case((Right<X,Y>), y) return right<S,T>(g(y));
    EndMatch
}
\end{lstlisting}

\noindent
This is also as fast as the visitor solution, but unlike the visitors based 
approach it neither requires \code{EitherVisitor}, nor any of the injected 
\code{accept} member-functions. We do require binding definitions though to be 
able to bind variables \code{x} and \code{y}:

%@\footnote{We need to take the first argument in parentheses to avoid interpretation of comma in template argument list by the preprocessor}@
%\footnote{Definitions of obvious functions \code{left} and \code{right} have 
%been ommitted in both cases.}

\begin{lstlisting}[keepspaces,columns=flexible]
template <class X, class Y> 
    struct bindings<Left<X,Y>>  { CM(0, Left<X,Y>::x); };
template <class X, class Y> 
    struct bindings<Right<X,Y>> { CM(0,Right<X,Y>::y); };
\end{lstlisting}

\noindent
Note that these binding definitions are made once for all possible instantiations 
with the use of partial template specialization in C++ and would not be needed 
if we implemented pattern matching in a compiler rather than a library.

\subsection{Summary}

The contributions of the paper are twofold and can be summarized as following:

\begin{itemize}
\setlength{\itemsep}{0pt}
\setlength{\parskip}{0pt}
\item We present techniques based on memoization (\textsection\ref{sec:copc}) and 
class precedence list (\textsection\ref{sec:cotc}) that can be used to implement 
type switching efficiently based on the run-time type of the argument.

  \begin{itemize}
  \setlength{\itemsep}{0pt}
  \setlength{\parskip}{0pt}
  \item The techniques come close and often outperform its de facto contender -- 
        visitor design pattern -- without sacrificing extensibility (\textsection\ref{sec:eval}).
  \item They work in the presence of multiple inheritance, including repeated and 
        virtual inheritance, as well as in generic code (\textsection\ref{sec:vtblmem}).
  \item The solution is open by construction (\textsection\ref{sec:poets}), 
        non-intrusive, and avoids the control inversion typical for visitors.
  \item It applies to polymorphic (\textsection\ref{sec:vtp}-\ref{sec:vtblmem}) and 
        tagged (\textsection\ref{sec:cotc}) class hierarchies through a unified  
        syntax~\cite{AP}.
  \item Our memoization device (\textsection\ref{sec:memdev}) generalizes to 
        other languages and can be used to implement type switching 
        (\textsection\ref{sec:vtblmem}), type testing 
        (\textsection\ref{sec:poets},\cite[\textsection 4.7]{TR}), predicate dispatch 
        (\textsection\ref{sec:memdev}), and multiple dispatch 
        (\textsection\ref{sec:cc}) efficiently.
  \item We list conditions under which virtual table pointers, commonly used in 
        C++ implementations, uniquely identify the exact subobject within the 
        most derived type (\textsection\ref{sec:vtp}).
  \item We also build an efficient cache indexing function for virtual table 
        pointers that minimizes the amount of conflicts 
        (\textsection\ref{sec:sovtp},\ref{sec:moc},\cite[\textsection 4.3.5]{TR}).
  \end{itemize}
\item We present a functional style pattern matching for C++ built as a library 
      employing the above technique. Our solution:
  \begin{itemize}
  \item Is open, non-intrusive and avoids the control inversion typical for visitors.
  \item Can be applied retroactively to any polymorphic or tagged class hierarchy.
  \item Provides a unified syntax for various encodings of extensible 
        hierarchycal datatypes in C++.
  \item Generalizes the controversial n+k patterns by leaving semantic choices to the user.
  \item Supports a limited form of views.
  \item Is simpler to use than conventional object-oriented or union-based alternatives.
  \item Improves performance compared to alternatives in real applications.
  \end{itemize}
\end{itemize}

\noindent
Our technique can be used in a compiler and/or library setting to implement 
facilities that depend on dynamic type or run-time properties of objects: e.g. 
type switching, type testing, pattern matching, predicate dispatch, 
multi-methods etc. We also look at different approaches to endoding algebraic 
data types in C++ and present a unified pattern-matching syntax that works 
uniformly with all of them. 
We generalize Haskell's n+k patterns\cite{haskell98} to any invertible operations. 
Semantics issues that typically accompany n+k pattern are handled transparently 
by forwarding the problem into the concepts domain, thanks to the fact that we 
work in a library setting. We also provide support for views in a form that 
resembles extractors in Scala. 

A practical benefit of our solution is that it can be used right away with any 
compiler with a decent support of C++0x without requiring the installation of 
any additional tools or preprocessors. The solution is a proof of concept that 
sets a minimum threshold for the performance, brevity, clarity and usefulness of 
a language solution for open type switching in C++.

The rest of this paper is structured as following. In Section~\ref{sec:bg}, we 
present evolution of pattern matching in different languages, presenting 
informally through example commonly used terminology and semantics of various 
pattern-matching constructs. Section~\ref{sec:adt} presents various approaches 
that are taken in C++ to encoding algebraic data types.
Sections~\ref{sec:syn} and~\ref{sec:sem} describe the syntax and semantics of 
our pattern matching facilities. Sections~\ref{sec:slv} and~\ref{sec:view} discuss 
approach taken by our library in handling generalized n+k patterns and views. 
Section~\ref{sec:impl} discusses techniques that makes type switching, used as a 
back-bone of the match statement, efficient, while section~\ref{sec:eval} 
provides its performance evaluation against some common alternatives. 
Section~\ref{sec:rw} discusses related work, and section~\ref{sec:cc} concludes 
by discussing some future directions and possible improvements.

\section{Background} %%%%%%%%%%%%%%%%%%%%%%%%%%%%%%%%%%%%%%%%%%%%%%%%%%%%%%%%%%%
\label{sec:bg}

Pattern matching in the context of a programming language was first introduced 
in a string manipulation language SNOBOL\cite{SNOBOL64}. Its fourth 
reincarnation SNOBOL4 had patterns as first-class data types providing 
operations of concatenation and alternation on them\cite{SNOBOL71}. The first 
reference to a pattern-matching construct that resembles the one found in 
statically typed functional languages today is usually attributed to Burstall 
and his work on structural induction\cite{Burstall69provingproperties}.

In the context of object-oriented programming, pattern matching has been first 
explored in Pizza programming language\cite{Odersky97pizzainto}. These efforts 
have been continued in Scala\cite{Scala2nd} and together with notable work of 
Burak Emir on \emph{Object-Oriented Pattern Matching}\cite{EmirThesis} have 
resulted in incorporation of pattern matching into the language.

%The first tree based pattern matching methods were found in Fred McBride's 
%extension of LISP in 1970.

%ML and Haskell further popularized pattern matching ...

Pattern matching has been closely related to \emph{algebraic data types} and 
\emph{equational reasoning} since the early days of functional programming.
In languages like ML and Haskel an \emph{Algebraic Data Type} is a data type 
each of whose values is picked from a disjoint sum of (possibly recursive) data 
types, called \emph{variants}. Each of the variants is marked with a unique 
symbolic constant called \emph{constructor}, while the set of all constructors 
of a given type is called \emph{signature}. Constructors provide a convenient 
way of creating a value of its variant type as well as a way of discriminating 
its variant type from the algebraic data type through pattern matching.

Algebraic data type \codeocaml{expr} from Section~\ref{sec:intro} consists of 5 
variants, marked with constructors \codeocaml{Value}, \codeocaml{Plus}, 
\codeocaml{Minus}, \codeocaml{Times} and \codeocaml{Divide} respectively. 
Constructor \codeocaml{Value} expects a value of type \codeocaml{int} during 
construction, as well as any pattern that admits values of type \codeocaml{int} 
during decomposition through pattern matching. Similarly, the other four 
constructors expect a value of a cartesian product of two \codeocaml{expr} 
types during construction, as well as any pattern that would admit a value of 
such type during decomposition.

Algebraic data types can be parameterized and recursive, as demonstrated by the 
following Haskell code that defines a binary tree parameterized on type 
\codehaskell{k} of keys and type \codehaskell{d} of data stored in the nodes:

\begin{lstlisting}[language=Haskell]
data Tree k d = Node k d (Tree k d) (Tree k d) | Leaf
\end{lstlisting}

\noindent
Naturally, they can be decomposed in a generic algorithm like the function 
\code{find} below, defined through case analysis on the tree's structure:

\begin{lstlisting}[language=Haskell]
find :: (Ord k) => k -> Tree k d -> Maybe d
find i Leaf = Nothing
find i (Node key item left right) = 
    if i == key 
    then Just item 
    else 
        if i [<] key 
        then find i left 
        else find i right
\end{lstlisting}

\noindent
The set of values described by such an algebraic data type is defined 
inductively as the least set closed under constructor functions of its variants.
Algebraic data types draw their name from the practice of using case distinction 
in mathematical function definitions and proofs that involve \emph{algebraic 
terms}.

One of the main differences of algebraic data types from classes in 
object-oriented languages is that an algebraic data type definition is 
\emph{closed} because it fixes the structure of its instances once and for all. 
Once we have listed all the variants a given algebraic data type may have we 
cannot extend it with new variants without modifying its definition. This is not 
the case in object-oriented languages, where classes are \emph{open} to 
extension through subclassing. Notable exceptions to this restriction in 
functional community are \emph{polymorphic variants} in OCaml\cite{garrigue-98} 
and \emph{open data types} in Haskell\cite{LohHinze2006}, which allow addition 
of new variants later. These extensions, however, are simpler than object-oriented 
extensions as neither polymorphic variants nor open data types form subtyping 
relation between themselves: open data types do not introduce any subtyping 
relation, while the subtyping relation on polymorphic variants is a 
\emph{semantic subtyping} similar to that of XDuce\cite{HosoyaPierce2000}, which 
is based on the subset relation between values of the type. In either case they 
maintain the important property that each value of the underlying algebraic data 
type belongs to exactly one disjoint subset tagged with a constructor. The 
\emph{nominative subtyping} of object-oriented languages does not usually have 
this disjointness making classes effectively have multiple types. In particular, 
the case of disjoint constructors can be seen as a degenerated case of a flat 
class hierarchy among the multitude of possible class hierarchies.

Closedness of algebraic data types is particularly useful for reasoning about 
programs by case analysis and allows the compiler to perform an automatic 
\emph{incompleteness} check -- test of whether a given \emph{match statement} 
covers all possible cases. Similar reasoning about programs involving extensible 
data types is more involved as we are dealing with potentially open set of 
variants. \emph{Completeness} check in such scenario reduces to checking presence 
of a case that handles the static type of the subject. Absence of such a case,
however, does not necessarily imply incompleteness, only potential incompleteness, 
as the answer will depend on the actual set of variants available at run-time.

A related notion of \emph{redundancy} checking arises from the 
tradition of using \emph{first-fit} strategy in pattern matching. It warns the 
user of any \emph{case clause} inside a match statement that will 
never be entered because of a preceding one being more general. Object-oriented 
languages, especially C++, typically prefer \emph{best-fit} strategy (e.g. for 
overload resolution and class template specialization) because it is not prone 
to errors where semantics of a statement might change depending on the ordering 
of preceding definitions. The notable exception in C++ semantics that prefers 
the \emph{first-fit} strategy is ordering of \code{catch} handlers of a 
\code{try}-block. Similarly to functional languages the C++ compiler will perform 
\emph{redundancy} checking on catch handlers and issue a warning that lists the 
redundant cases. We use this property of the C++ type system to perform redundancy 
checking of our match statements in \textsection\ref{sec:redun}.

The patterns that work with algebraic data types we have seen so far are 
generally called \emph{tree patterns} or \emph{constructor patterns}. Their 
analog in object-oriented languages is often referred to as \emph{type pattern} 
since it may involve type testing and type casting. Special cases of these patterns 
are \emph{list patterns} and \emph{tuple patterns}. The former lets one split a 
list into a sequence of elements in its beginning and a tail with the help of 
list constructor \codehaskell{:} and an empty list constructor \codehaskell{[]} 
e.g. \codehaskell{[x:y:rest]}. The latter does the same with tuples using tuple
constructor \codehaskell{(,,...,)} e.g. \codehaskell{([x:xs],'b',(1,2.0),"hi",True)}.

Pattern matching is not used solely with algebraic data types and can equally 
well be applied to built-in types. The following Haskell code defines factorial 
function in the form of equations:

\begin{lstlisting}[language=Haskell]
factorial 0 = 1
factorial n = n * factorial (n-1)
\end{lstlisting}

\noindent
Here 0 in the left hand side of the first \emph{equation} is an example of a 
\emph{value pattern} (also known as \emph{constant pattern}) that will only 
match when the actual argument passed to the function factorial is 0. The 
\emph{variable pattern} \codehaskell{n} (also referred to as \emph{identifier 
pattern}) in the left hand side of the second equation will match any value, 
\emph{binding} variable \codehaskell{n} to that value in the right hand side of 
equation. Similarly to variable pattern, the \emph{wildcard pattern} \codehaskell{_} 
will match any value, neither binding it to a variable nor even obtaining it. 
Value patterns, variable patterns and wildcard patterns are  
generally called \emph{primitive patterns}. Patterns like variable and wildcard 
patterns that never fail to match are called \emph{irrefutable}, in contrast to 
\emph{refutable} patterns like value patterns, which may fail to match.

In Haskell 98\cite{Haskell98Book} the above definition of factorial could also 
be written as:

\begin{lstlisting}[language=Haskell]
factorial 0 = 1
factorial (n+1) = (n+1) * factorial n
\end{lstlisting}

\noindent
The \codehaskell{(n+1)} pattern in the left hand side of equation is an example of 
\emph{n+k pattern}. According to its informal semantics ``Matching an $n+k$ 
pattern (where $n$ is a variable and $k$ is a positive integer literal) against 
a value $v$ succeeds if $v \ge k$, resulting in the binding of $n$ to $v-k$, and 
fails otherwise''\cite{haskell98}. n+k patterns were introduced into Haskell to 
let users express inductive functions on natural numbers in much the same way as 
functions defined through case analysis on algebraic data types. Besides 
succinct notation, such language feature could facilitate automatic proof of 
termination of such functions by compiler. Peano numbers, used as an analogy to 
algebraic data type representation of natural numbers, is not always the best 
abstraction for representing other mathematical operations however. This,  
together with numerous ways of defining semantics of generalized n+k patterns 
were some of the reasons why the feature was never generalized in Haskell to 
other kinds of expressions, even though there were plenty of known applications. 
Moreover, numerous debates over semantics and usefulness of the feature 
resulted in n+k patterns being removed from the language altogether in Haskell 
2010 standard\cite{haskell2010}. Generalization of n+k patterns, called 
\emph{application patterns} has been studied by Nikolaas N. Oosterhof in his 
Master's thesis\cite{OosterhofThesis}. Application patterns essentially treat 
n+k patterns as equations, while matching against them attempts to solve or 
validate the equation.

While n+k patterns were something very few languages had, another common feature of 
many programming languages with pattern matching are guards. A \emph{guard} 
is a predicate attached to a pattern that may make use of the variables bound in 
it. The result of its evaluation will determine whether the case clause and the 
body associated with it will be \emph{accepted} or \emph{rejected}. The 
following OCaml code for $exp$ language from Section~\ref{sec:intro} defines the 
rules for factorizing expressions $e_1e_2+e_1e_3$ into $e_1(e_2+e_3)$ and 
$e_1e_2+e_3e_2$ into $(e_1+e_3)e_2$ with the help of guards spelled out after 
keyword \codeocaml{when}:

\begin{lstlisting}[language=Caml,keepspaces,columns=flexible]
let factorize e =
    match e with
      Plus(Times(e1,e2), Times(e3,e4)) when e1 = e3 
          -> Times(e1, Plus(e2,e4))
    | Plus(Times(e1,e2), Times(e3,e4)) when e2 = e4 
          -> Times(Plus(e1,e3), e4)
    |   e -> e
    ;;
\end{lstlisting}

\noindent
One may wonder why we could not simply write the above case clause as 
\codeocaml{Plus(Times(e,e2), Times(e,e4))} to avoid the guard? Patterns that 
permit use of the same variable in them multiple times are called 
\emph{equivalence patterns}, while the requirement of absence of such patterns 
in a language is called \emph{linearity}. Neither OCaml nor Haskell support such 
patterns, while Miranda\cite{Miranda85} as well as Tom's pattern matching 
extension to C, Java and Eiffel\cite{Moreau:2003} supports \emph{non-linear 
patterns}.

The example above illustrates yet another common pattern-matching facility -- 
\emph{nesting of patterns}. In general, a constructor pattern composed of a 
linear vector of (distinct) variables is called a \emph{simple pattern}. The 
same pattern composed not only of variables is called \emph{nested pattern}.
Using nested patterns, with a simple expression in the case clause we could
define a predicate that tests the top-level expression to be tagged with a
\codeocaml{Plus} constructor, while both of its arguments to be marked with 
\codeocaml{Times} constructor, binding their arguments (or potentially pattern 
matching further) respectively. Note that the visitor design pattern does not 
provide this level of flexibility and each of the nested tests might have 
required a new visitor to be written. Nesting of patterns like the one above is 
typically where users resort to \emph{type tests} and \emph{type casts} that in 
case of C++ can be combined into a single call to \code{dynamic_cast}.

Related to nested patterns are \emph{as-patterns} that help one take a value 
apart while still maintaining its integrity. The following rule could have been 
a part of a hypothetical rewriting system in OCaml similar to the one above. Its 
intention is to rewrite expressions of the form $\frac{e_1/e_2}{e_3/e_4}$ into 
$\frac{e_1}{e_2}\frac{e_4}{e_3} \wedge e_2\neq0 \wedge e_3\neq0 \wedge e_4\neq0$.

\begin{lstlisting}[language=Caml]
    | Divide(Divide(_,e2) as x, Divide(e3,e4))
          -> Times(x, Divide(e4, e3))
\end{lstlisting}

\noindent
We introduced a name ``x'' as a synonym of the result of matching the 
entire sub-expression \codeocaml{Divide(_,e2)} in order to refer it without 
recomposing in the right-hand side of the case clause. We omitted the 
conjunction of relevant non-zero checks for brevity, one can see that we will 
need access to \codeocaml{e2} in it however.

Decomposing algebraic data types through pattern matching has an important 
drawback that was originally spotted by Wadler\cite{Wadler87}: they expose 
concrete representation of an abstract data type, which conflicts with the 
principle of \emph{data abstraction}. To overcome the problem he proposed the 
notion of \emph{views} that represent conversions between different 
representations that are implicitly applied during pattern matching. As an 
example, imagine polar and cartesian representations of complex numbers. A user 
might choose polar representation as a concrete representation for the abstract 
data type \codeocaml{complex}, treating cartesian representation as view or vice 
versa:\footnote{We use the syntax from Wadler's original paper for this example}

\begin{lstlisting}[language=Haskell,columns=flexible]
complex ::= Pole real real
view complex ::= Cart real real
  in  (Pole r t) = Cart (r * cos t) (r * sin t)
  out (Cart x y) = Pole (sqrt(x^2 + y^2)) (atan2 x y)
\end{lstlisting}

\noindent
The operations then might be implemented in whatever representation is the most 
suitable, while the compiler will implicitly convert representation if needed:

\begin{lstlisting}[language=Haskell,columns=flexible]
  add  (Cart x1 y1) (Cart x2 y2) = Cart (x1 + x2) (y1 + y2)
  mult (Pole r1 t1) (Pole r2 t2) = Pole (r1 * r2) (t1 + t2)
\end{lstlisting}

\noindent
The idea of views were later adopted in various forms in several languages: 
Haskell\cite{views96}, Standard ML\cite{views98}, Scala (in the form of 
\emph{extractors}\cite{EmirThesis}) and F$\sharp$ (under the name of 
\emph{active patterns}\cite{Syme07}). We demonstrate our support of views in 
\textsection\ref{sec:view}.

%Views in functional programming languages [92, 71] are conversions from one data type to
%another that are implicitly applied in pattern matching. They play a role similar to extractors
%in Scala, in that they permit to abstract from the concrete data-type of the matched objects.
%However, unlike extractors, views are anonymous and are tied to a particular target data
%type.

Logic programming languages like Prolog take pattern matching to even greater 
level. The main difference between pattern matching in logic languages and 
functional languages is that functional pattern matching is a ``one-way'' 
matching where patterns are matched against values, possibly binding some 
variables in the pattern along the way. Pattern matching in logic programming is 
``two-way'' matching based on \emph{unification} where patterns can be matched 
against other patterns, possibly binding some variables in both patterns and 
potentially leaving some variables \emph{unbound} or partially bound -- i.e. 
bound to patterns. A hypothetical example of such functionality can be matching 
a pattern \codeocaml{Plus(x,Times(x,1))} against another pattern 
\codeocaml{Plus(Divide(y,2),z)}, which will result in binding \codeocaml{x} to a 
\codeocaml{Divide(y,2)} and \codeocaml{z} to \codeocaml{Times(Divide(y,2),1)} 
with \codeocaml{y} left unbound, leaving both \codeocaml{x} and \codeocaml{z} 
effectively a pattern.

\subsection{Algebraic Data Types in C++}
\label{sec:adt}

C++ does not have a direct support of algebraic data types, but they can usually 
be emulated in a number of ways. A pattern-matching solution that strives to be 
general will have to account for different encodings and be applicable to all of 
them.

Consider an ML data type of the form:

\begin{lstlisting}[language=ML,keepspaces,columns=flexible,escapechar=@]
datatype DT = @$C_1$@ of {@$L_{11}:T_{11},...,L_{1m}:T_{1m}$@} 
              | ...
              | @$C_k$@ of {@$L_{k1}:T_{k1},...,L_{kn}:T_{kn}$@}
\end{lstlisting}

\noindent There are at least 3 different ways to represent it in C++. Following 
Emir, we will refer to them as \emph{encodings}~\cite{EmirThesis}:

\begin{itemize}
\setlength{\itemsep}{0pt}
\setlength{\parskip}{0pt}
\item Polymorphic Base Class (or \emph{polymorphic encoding} for short)
\item Tagged Class (or \emph{tagged encoding} for short)
\item Discriminated Union (or \emph{union encoding} for short)
\end{itemize}

\noindent
In polymorphic and tagged encoding, base class \code{DT} represents algebraic 
data type, while derived classes represent variants. The only difference between 
the two is that in polymorphic encoding base class has virtual functions, while 
in tagged encoding it has a dedicated member of integral type that uniquely 
identifies the variant -- derived class. 

The first two encodings are inherently \emph{open} because the classes can be 
arbitrarily extended through subclassing. The last encoding is inherently 
\emph{closed} because we cannot add more members to the union without modifying 
its definition.

%In order to be able to provide a common syntax for these representations, we 
%need to understand better similarities and differences between them. Before we 
%look into them let's fix some terminology.

When we deal with pattern matching, the static type of the original expression 
we are matching may not necessarily be the same as the type of expression we 
match it with. We call the original expression a \emph{subject} and its static 
type -- \emph{subject type}. We call the type we are trying to match subject 
against -- a \emph{target type}.

In the simplest case, detecting that the target type is a given type or a type 
derived from it, is everything we want to know. We refer to such a use-case as 
\emph{type testing}. In the next simplest case, besides testing we might want to 
get a pointer or a reference to the target type of subject as casting it to such 
a type may involve a non-trivial computation only a compiler can safely 
generate. We refer to such a use-case as \emph{type identification}. Type 
identification of a given subject against multiple target types is typically 
referred to as \emph{type switching}.

Once we uncovered the target type, we may want to be able to decompose it 
\emph{structurally} (when the target type is a \emph{structured} data type like 
array, tuple or class) or \emph{algebraically} (when the target type is a scalar 
data type like \code{int} or \code{double}). Structural decomposition in our 
library can be performed with the help of \emph{tree patterns}, while algebraic 
decomposition can be done with the help of \emph{generalized n+k patterns}.

\subsubsection{Polymorphic Base Class}
\label{sec:pbc}

In this encoding user declares a polymorphic base class \code{DT} that will 
be extended by classes representing all the variants. Base class might declare 
several virtual functions that will be overridden by derived classes, for example 
\code{accept} used in a Visitor Design Pattern.

\begin{lstlisting}[keepspaces,columns=flexible]
class DT { virtual @$\sim$@DT{} };
class @$C_1$@ : public DT {@$T_{11} L_{11}; ... T_{1m} L_{1m};$@} 
...
class @$C_k$@ : public DT {@$T_{k1} L_{k1}; ... T_{kn} L_{kn};$@} 
\end{lstlisting}

The uncover the actual variant of such an algebraic data type, the user might 
use \code{dynamic_cast} to query one of the $k$ expected run-time types (an 
approach used by Rose\cite{SQ03}) or she might employ a visitor design pattern 
devised for this algebraic data type (an approach used by Pivot\cite{Pivot09} 
and Phoenix\cite{Phoenix}). The most attractive feature of this approach is that 
it is truly open as we can extend classes arbitrarily at will (leaving the 
orthogonal issues of visitors aside).

\subsubsection{Tagged Class}
\label{sec:tc}

This encoding is similar to the \emph{Polymorphic Base Class} in that we use 
derived classes to encode the variants. The main difference is that the user 
designates a member in the base class, whose value will uniquely 
determine the most derived class a given object is an instance of. Constructors 
of each variant $C_i$ are responsible for properly initializing the dedicated 
member with a unique value $c_i$ associated with that variant. Clang\cite{Clang} 
among others uses this approach.

\begin{lstlisting}[keepspaces,columns=flexible]
class DT { enum kinds {@$c_1, ..., c_k$@} m_kind; };
class @$C_1$@ : public DT {@$T_{11} L_{11}; ... T_{1m} L_{1m};$@} 
...
class @$C_k$@ : public DT {@$T_{k1} L_{k1}; ... T_{kn} L_{kn};$@} 
\end{lstlisting}

In such scenario the user might use a simple switch statement to uncover the 
type of the variant combined with a \code{static_cast} to properly cast the 
pointer or reference to an object. People might prefer this encoding to the one 
above for performance reasons as it is possible to avoid virtual dispatch with 
it altogether. Note, however, that once we allow for extensions and not limit 
ourselves with encoding algebraic data types only it also has a significant 
drawback in comparison to the previous approach: we can easily check that given 
object belongs to the most derived class, but we cannot say much about whether 
it belongs to one of its base classes. A visitor design pattern can be 
implemented to take care of this problem, but control inversion that comes along 
with it will certainly diminish the convenience of having just a switch 
statement. Besides, forwarding overhead might lose some of the performance 
benefits gained originally by putting a dedicated member into the base class.

\subsubsection{Discriminated Union}
\label{sec:du}

This encoding is popular in projects that are either implemented in C or 
originated from C before coming to C++. It involves a type that contains a union 
of its possible variants, discriminated with a dedicated value stored as a part 
of the structure. The approach is used by EDG front-end\cite{EDG} and many others.

\begin{lstlisting}[keepspaces,columns=flexible]
struct DT
{
    enum kinds {@$c_1, ..., c_k$@} m_kind;
    union {
        struct @$C_1$@ {@$T_{11} L_{11}; ... T_{1m} L_{1m};$@} @$C_1$@;
        ...
        struct @$C_k$@ {@$T_{k1} L_{k1}; ... T_{kn} L_{kn};$@} @$C_k$@; 
    };
};
\end{lstlisting}

As before, the user can use a switch statement to identify the variant $c_i$ and 
then access its members via $C_i$ union member. This approach is truly closed, as 
we cannot add new variants to the underlying union without modifying class 
definition. 

Note also that in this case both subject type and target types are the same and 
we use an integral constant to distinguish which member(s) of the underlying union 
is active now. In the other two cases the type of a subject is a base class of 
the target type and we use either run-time type information or the integral 
constant associated by the user with the target type to uncover the target type. 

\subsection{Expression Templates}

Interestingly enough C++ has a pure functional sublanguage in it that has a 
striking similarity to ML and Haskell. The sublanguage in question is template 
facilities of C++ that has been shown to be Turing 
complete\cite{veldhuizen:templates_turing_complete}. 

Haskell definition of \code{factorial} we saw earlier can be rewritten in 
template sublanguage of C++ as following:

\begin{lstlisting}
template <int N> 
    struct factorial { enum { result = N*factorial<N-1>::result }; };
template <>
    struct factorial<0> { enum { result = 1 }; };
\end{lstlisting}

\noindent
One can easily see similarity with equational definitions in Haskell, with the 
exception that more specific cases (specialization for 0) have to follow the 
general definition in C++. The main difference between Haskell definition and 
its C++ counterpart is that the former describes computations on \emph{run-time 
values}, while the latter can only work with \emph{compile-time values}.

Turns out we can even express our $exp$ language using this functional 
sublanguage:

\begin{lstlisting}
template <class T>
struct value {
    value(const T& t) : m_value(t) {}
    T m_value;
};

template <class T>
struct variable {
    variable() : m_value() {}
    T m_value;
};

template <typename E1, typename E2>
struct plus {
    plus(const E1& e1, const E2& e2) : m_e1(e1), m_e2(e2) {}
    const E1 m_e1; const E2 m_e2;
};

// ... definitions of other expressions
\end{lstlisting}

\noindent The idea is that expressions can be composed out of subexpressions, 
whose shape (type) is passed as arguments to above templates. Explicit 
description of such expressions is very tedious however and is thus never 
expressed directly, but as a result of corresponding operations: 

\begin{lstlisting}[keepspaces,columns=flexible]
template <typename T>
    value<T> val(const T& t) { return value<T>(t); }
template <typename E1, typename E2>
    plus<E1,E2> operator+(const E1& e1, const E2& e2)
    { return plus<E1,E2>(e1,e2); }
\end{lstlisting}

\noindent With this, one can now capture various expressions as following:

\begin{lstlisting}
variable<int> v;
auto x = v + val(3);
\end{lstlisting}

\noindent The type of variable \code{x} -- \code{plus<variable<int>,value<int>>}
 -- captures the structure of the expression, while the values inside of it 
represent various subexpressions the expression was created with. Such an 
expression can be arbitrarily, but finitely nested. Note that value 3 is not 
added to the value of variable \code{v} here, but the expression \code{v+3} is 
recorded, while the meaning to such expression can be given differently in 
different contexts. A general observation is that only the shape of the 
expression becomes fixed at compile time, while the values of variables involved 
in it can be changed arbitrarily at run time, allowing for \emph{lazy 
evaluation} of the expression. Polymorphic function \code{eval} below implements 
just that:

\begin{lstlisting}[keepspaces,columns=flexible]
template <typename T> 
    T eval(const value<T>& e) { return e.m_value; }
template <typename T> 
    T eval(const variable<T>& e) { return e.m_value; }
template <typename E1, typename E2> 
    auto eval(const plus<E1,E2>& e) 
         -> decltype(eval(e.m_e1) + eval(e.m_e2))
            { return eval(e.m_e1) + eval(e.m_e2); }
\end{lstlisting}

\noindent One can now modify value of the variable \code{v} and re-evaluate 
expression as following:

\begin{lstlisting}
v = 7;           // assumes overloading of assignment
int r = eval(x); // returns 10
\end{lstlisting}

\noindent The above technique for lazy evaluation of expressions was 
independently invented by Todd Veldhuizen and David Vandevoorde and is generally 
known in the C++ community by the name \emph{Expression Templates} that Todd 
coined\cite{Veldhuizen95expressiontemplates, vandevoorde2003c++}.  

Note again how implementation of \code{eval} resembles equations in Haskell that 
decompose an algebraic data type. The similarities are so striking that there 
were attempts to use Haskell as a pseudo code language for template 
metaprogramming in C++\cite{Milewski11}. A key observation in this analogy is 
that partial and explicit template specialization of C++ class templates are 
similar to defining equations for Haskell functions. Variables introduced via 
template clause of each equation serve as \emph{variable patterns}, while the 
names of actual templates describing arguments serve as \emph{variant 
constructors}. An important difference between the two is that Haskell's 
equations use \emph{first-fit} strategy making order of equations important, 
while C++ uses \emph{best-fit} strategy, thus making the order irrelevant.

Patterns expressed this way can be arbitrarily nested as long as they can be 
expressed in terms of the types involved and not the values they store. Using 
the above example, for instance, it is very easy to specialize \code{eval} for 
an expression of form $c_1*x+c_2$ where $c_i$ are some (not known) constant 
values and $x$ is any variable. Specializing for a concrete instance of that 
expression $2*x+3$ will be much harder, because in the representation we chose 
values 2 and 3 become run-time values and thus cannot participate in 
compile-time computations anymore. In this case we could have devised a template 
that allocates a dedicated type for each constant making such value part of the 
type:

\begin{lstlisting}[keepspaces,columns=flexible]
template <class T, T t> struct constant {};

template <typename T, T t>
    T eval(const constant<T,t>& e) { return t; }
template <typename E>
    auto eval(const times<constant<int,0>,E>& e) 
        -> decltype(eval(e.m_e2)) 
            { return (decltype(eval(e.m_e2)))(0); }
template <typename E>
    auto eval(const times<E,constant<int,0>>& e) 
        -> decltype(eval(e.m_e1)) 
            { return (decltype(eval(e.m_e1)))(0); }
\end{lstlisting}

\noindent Here the first equation for \code{eval} describes the necessary general 
case for handling expressions of type \code{constant<T,t>}, while the other two 
are redundant cases that can be seen as an optimization detecting expressions of 
the form $e*0$ and $0*e$ for any arbitrary expression $e$ and returning 0 
without actually computing $e$.

Unfortunately, a similar pattern to detect expressions of the form $x-x$ for any 
variable $x$ cannot be expressed because expression templates are blind to 
object identity and can only see their types. This means that expression 
templates of the form $x-y$ are indisthinguishable at compile time from 
expressions of the form $x-x$ because their types are identical.

Nevertheless, with all the limitations, expression templates provide an 
extremely powerful abstraction mechanism, which we use to express a 
pattern-language for our SELL. Coincidentally, we employ the compile-time 
pattern-matching facility already supported by C++ as a meta-language to 
implement its run-time counterpart.

\section{Pattern Matching SELL} %%%%%%%%%%%%%%%%%%%%%%%%%%%%%%%%%%%%%%%%%%%%%%%%
\label{sec:pm}

A Semantically Enhanced Library Language is not a language of its own, but 
rather a sub-language embedded into another language -- \Cpp{} in our case. The 
sub-language still has the facilities we typically associate with programming 
languages e.g. syntax, semantics, type system, but they are constrained by
the host language. A particular sub-language can often be implemented in 
different host languages, which is why it is important to describe it 
independently of its host. We thus shall abstract from describing the 
exact syntax and semantics of host-language features that are well understood and documented 
elsewhere~\cite{C++11}.

\subsection{Syntax}
\label{sec:syn}

\begin{figure}[h]
\centering
\begin{tabular}{rcll}
\Rule{Match Statement}     & $M$       & \is{}  & \code{Match(}$e$\code{)} $\{ \left[C s^*\right]^* \}$ \code{EndMatch} \\
\Rule{Case Clause}         & $C$       & \is{}  & \code{Qua(}$T\left[,\varpi\right]^*$\code{)}\Alt{}\code{When(}$\varpi^*$\code{)}\Alt{}\code{Case(}$T\left[,x\right]^*$\code{)}\Alt{}\code{Otherwise(}$x^*$\code{)} \\
\Rule{Target Expression}   & $T$       & \is{}  & $\tau$ \Alt{} $l$ \\
\Rule{Layout}              & $l$       & \is{}  & $c^{\mathsf{int}}$ \\
\Rule{Match Expression}    & $m$       & \is{}  & $\pi(e)$ \\
\Rule{Extended Pattern}    & $\varpi$  & \is{}  & $\pi$ \Alt{} $c$ \Alt{} $x$ \\
\Rule{Pattern}             & $\pi$     & \is{}  & $\mu$ \Alt{} $\varrho$ \Alt{} $\eta$ \Alt{} $\chi$ \Alt{} $\varsigma$ \Alt{} $\_$ \\
\Rule{Constructor Pattern} & $\mu$     & \is{}  & \code{match<}$\tau\left[,l\right]$\code{>(}$\varpi^*$\code{)} \\
\Rule{Guard Pattern}       & $\varrho$ & \is{}  & $\pi \models \xi$ \\
\Rule{n+k Pattern}         & $\eta$    & \is{}  & $\xi$ \\
\Rule{Variable Pattern}    & $\chi^{\mathsf{variable}\langle\tau\rangle}$   \\
\Rule{Value Pattern}       & $\varsigma^{\mathsf{value}\langle\tau\rangle}$ \\
\Rule{Wildcard Pattern}    & $\_^{\mathsf{wildcard}}$                       \\
\Rule{Lazy Expression}     & $\xi$     & \is{}  & $\varsigma$ \Alt{} $\chi$ \Alt{} $\xi \oplus c$ \Alt{} $c \oplus \xi$ \Alt{} $\ominus \xi$ \Alt{} $(\xi)$ \Alt{} $\xi \oplus \xi$ \Alt{} $\varphi(\xi^*)$ \\
\Rule{Lazy Function}       & $\varphi^{\xi^*\rightarrow \xi}$ \\
\Rule{Unary Operator}      & $\ominus$ & $\in$  & $\lbrace*,\&,+,-,!,\sim\rbrace$ \\
\Rule{Binary Operator}     & $\oplus$  & $\in$  & $\lbrace*,/,\%,+,-,\ll,\gg,\&,\wedge,|,<,\leq,>,\geq,=,\neq,\&\&,||\rbrace$ \\
\Rule{Type-Id}             & $\tau$    &        & \Cpp{}\cite[\textsection A.7]{C++11} \\
\Rule{Statement}           & $s$       &        & \Cpp{}\cite[\textsection A.5]{C++11} \\
\Rule{Expression}          & $e^\tau$  &        & \Cpp{}\cite[\textsection A.4]{C++11} \\
\Rule{Constant-Expression} & $c^\tau$  &        & \Cpp{}\cite[\textsection A.4]{C++11} \\
\Rule{Identifier}          & $x^\tau$  &        & \Cpp{}\cite[\textsection A.2]{C++11} \\
\end{tabular}
\caption{Abstract syntax of our pattern-matching SELL}
\label{syntax}
\end{figure}

Figure~\ref{syntax} presents the abstract syntax of our pattern-matching SELL. It presents 
the syntax embedded into the \Cpp{} without sacrificing 
the clarity of presentation. The idea is to show which interactions are possible 
within our SELL, while leaving the details of their implementation to 
\textsection\ref{sec:impl}. Where the specific technique we use to achieve such 
interactions crucially depends on the types of the entities involved,
we mention their type in the superscript. This dependence on the 
type system of the host language was also the reason why we chose abstract 
syntax over traditional EBNF. We make use 
of few non-terminals from the host language in order to put our constructs into 
context.

\emph{Match statement} is an analog of a switch statement with patterns as case 
clauses. Similar control structures exist in many programming languages and 
date back to at least Simula's Inspect statement~\cite{Simula67}.
In a library-based solution, we require it to be closed with a dedicated 
\code{EndMatch} macro to ensure proper nesting.

We support four kinds of \emph{case clauses}: \code{Qua}, \code{When}, 
\code{Case}, and \code{Otherwise}.
The distinction between them is only important for the library 
implementation and can trivially be inferred in a compiler solution.
A \code{Qua}-clause is the most general clause taking an  
expression that identifies the target type as well as a list of extended 
patterns.
A \code{Qua}-clause permits nested patterns, but requires all the 
variables used in the patterns to be explicitly pre-declared. \code{Case}-clause 
only accepts simple patterns, conveniently introducing all the variables into the 
clause's scope. 
A \code{When}-clause takes only patterns while its target type is 
the subject type.
An \code{Otherwise} clause is an irrefutable clause that is 
semantically equivalent to \code{Case}-clause with subject type used as a target 
type. When used it should be the last clause of the match statement.

A \emph{Target expression} is used by case clauses as either a target type or 
a constant value, representing \emph{layout}. \emph{Layout} is an enumerator 
that the user may use to define alternative bindings for the same class. They are 
discussed in \textsection\ref{sec:bnd}. The use of layout as target 
expression is only allowed for union encoding of algebraic data types 
(\textsection\ref{sec:unisyn}), in which case the library assumes the target 
type to be the subject type.

A \emph{Match expression} is an inline version of the match statement with 
a single \code{Qua}-clause. Applying a pattern to a subject checks whether the 
subject matches the pattern.

\emph{Pattern} summarizes all the patterns supported by the library. 
\emph{Extended pattern} indicates contexts in which our library implicitly 
permits the use constants as \emph{value patterns} and regular \Cpp{} variables as 
\emph{variable patterns}. The library recognizes them and transforms into 
$\varsigma$ and $\chi$ respectively.
A \emph{Constructor pattern} takes a target type, an optional layout and a list of 
nested sub-patterns.
\emph{Guard patterns} are composed from a pattern and a condition separated with 
\code{|=}.
We chose operator \code{|=} because of its low precedence 
allows most other operators to be used inside the 
condition without parenthesis. The right operand of \code{|=} is allowed to make use of any 
variables bound in the left operand. When used on arguments of a constructor 
pattern, it is also allowed to make use of any variables bound by preceding 
argument positions. 
\emph{n+k patterns} are a subset of \emph{lazy expressions} for which the user has 
provided \emph{solvers} -- overloaded functions defining semantics of matching a 
value against an expression(\textsection~\ref{sec:slv}).
\emph{Variable patterns} refers to variables whose \Cpp{} type is \code{variable<T>} for 
any given type \code{T}.
A \emph{Value pattern} is almost never declared explicitly, 
but is implicitly introduced by the library in the contexts where $c$ is 
accepted.
A \emph{Wildcard pattern} is represented with a constant of type 
\code{wildcard}.
A \emph{Lazy expression} refers to lazily evaluated expressions introduced by our SELL, 
as opposed to eagerly evaluated expressions, directly supported by \Cpp{}. The use 
of $c$ indicates contexts in which constants can be used as lazy expressions and 
is similarly replaced with $\varsigma$. \emph{Lazy function} represents 
functions that can participate in lazy evaluations. Such functions have to be 
declared in certain way and are discussed in \textsection\ref{sec:slv}. 

\emph{Binary operator} and \emph{unary operator} name a subset of \Cpp{} operators we 
make use of and provide support for in our pattern-matching library. 
The remaining syntactic categories refer to non-terminals in the \Cpp{} grammar 
bearing the same name.

\subsection{Typing Rules}

\begin{figure}[h]
\begin{mathpar}

\inferrule[T-Var]
{}
{\Gamma\vdash \chi : \Variable{T}}

\inferrule[T-Value]
{}
{\Gamma\vdash \varsigma : \Value{T}}

\inferrule[T-Wildcard]
{}
{\Gamma\vdash \_ : \Wildcard}

\inferrule[T-Unary]
{\Gamma\vdash \xi : E}
{\Gamma\vdash \ominus \xi : \ExprU{F_\ominus}{E} }

\inferrule[T-Binary]
{\Gamma\vdash \xi_1 : E_1 \\ \Gamma\vdash \xi_2 : E_2}
{\Gamma\vdash \xi_1 \oplus \xi_2 : \ExprB{F_\oplus}{E_1}{E_2} }

%\inferrule[T-Binary-Const-Left]
%{\Gamma\vdash \xi : E \\ \Gamma\vdash c : T}
%{\Gamma\vdash c \oplus \xi : \ExprB{F_\oplus}{\Value{T}}{E} }

%\inferrule[T-Binary-Const-Right]
%{\Gamma\vdash \xi : E \\ \Gamma\vdash c : T}
%{\Gamma\vdash \xi \oplus c : \ExprB{F_\oplus}{E}{\Value{T}} }

\inferrule[T-Function]
{\Gamma\vdash \xi_1 : E_1 \\ \cdots \\ \Gamma\vdash \xi_k : E_k}
{\Gamma\vdash \varphi(\xi_1,\cdots,\xi_k) : \ExprK{F_\varphi}{E_1}{E_k} }

\inferrule[T-Guard]
{\Gamma\vdash \pi : E_1 \\ \Gamma\vdash \xi : E_2}
{\Gamma\vdash \pi \models \xi : \Guard{E_1}{E_2} }

\inferrule[T-Constructor]
{\Gamma\vdash \varpi_1 : E_1 \\ \cdots \\ \Gamma\vdash \varpi_k : E_k}
{\Gamma\vdash \mathsf{match}\langle T\left[,l\right]\rangle(\varpi_1,\cdots,\varpi_k) : \Cnstr{T\left[,l\right]}{E_1}{E_k} }

%\inferrule[T-Extended-Pattern]
%{ \varpi = \pi \\ \Gamma\vdash \pi : E}
%{\Gamma\vdash \varpi : E}

%\inferrule[T-Extended-Value]
%{ \varpi = c \\ \Gamma\vdash c : T}
%{\Gamma\vdash \varpi : \Value{T}}

%\inferrule[T-Extended-Var]
%{ \varpi = x \\ \Gamma\vdash x : T}
%{\Gamma\vdash \varpi : \Variable{T}}

\end{mathpar}
\caption{Typing rules for our pattern-matching SELL}
\label{typing}
\end{figure}

Figure~\ref{typing} shows rules we use to type expressions in our SELL. The 
types presented are not necessarily the exact \Cpp{} types we use to encode them, 
but we keep the correspondence as close as possible to reflect the actual 
implementation. We use the following type constructors, indicated with their 
arity: $\CWildcard^0$, $\CValue^1$, $\CVariable^1$, $\CExpr^{1+n}$, $\CGuard^2$, 
$\CCnstr^{1+n}$. We assume that type variables $T_i$ range over any \Cpp{} types, 
while $E_i$ only range over types marked with these type constructors, to which 
we refer as \emph{SELL-types}.

The judgments are of the traditional form $\Gamma\vdash \varpi : E$ that can be 
interpreted as given a typing environment $\Gamma$, an extended pattern $\varpi$ is 
given a SELL-type $E$. $\Gamma$ represents the typing context of the \Cpp{} 
compiler with the allowance for our simplified representation of SELL-types.
Types $F_\oplus$, $F_\ominus$ and $F_\varphi$ are described in greater details 
in \textsection\ref{sec:sem}, while for the purpose of typing they can be 
interpreted as types that uniquely identify operations $\oplus$, $\ominus$ and 
$\varphi$ respectively.

To avoid confusion we would like to point out that syntactic categories $\chi$, 
$\varsigma$ and $\_$ are defined as objects of \Cpp{} types \code{value<T>}, 
\code{variable<T>} and \code{wildcard}, while here we type them with SELL-types 
$\Value{T}$, $\Variable{T}$ and $\Wildcard$. Internally these types are the same 
of course.

\subsection{Semantics}
\label{sec:sem}

We use natural semantics\cite{Kahn87} to describe the semantics of our 
pattern-matching extension. Because our SELL can be customized in a number of 
ways, we make use of several semantic functors that let the user define the 
semantics of the following operations:

\begin{compactitem}
\setlength{\itemsep}{0pt}
\setlength{\parskip}{0pt}
\item Type casting: $F_{dc}(\tau,v)$
\item Lazily evaluated functions: $F_\oplus,F_\ominus,F_\varphi$
\item Structural decomposition: $\Delta_i^{\tau,l}$
\item Algebraic decomposition: $\mathsf{solve}(\eta,v)$
\end{compactitem}

\noindent
The type of the subject used in pattern matching is not always the same as the 
type that a given pattern expects. The library in such a case may need to 
perform type casting of the subject, which may involve but is not limited to 
down-casting, up-casting, cross-casting or conversion. Depending on the types 
involved, such casting can be performed in different ways, which is why we 
abstract from a concrete semantics of such an operation with functor $F_{dc}$. 
We use the notation $F_{dc}(\tau,v)$ to refer to the result of casting value $v$ 
to target type $\tau$, which may result in a dedicated value $\nullptr$ that 
indicates impossibility of such a cast. We discuss various implementations of 
such a functor in \textsection\ref{sec:unisyn}.

Every function $\varphi$ that the user would like to be able to call lazily 
requires definition of a functor $F_\varphi$ that defines the semantics of such 
operation on any given argument types. The library defines such semantic objects 
$F_\oplus$ and $F_\ominus$ for every binary operation $\oplus$ and unary 
$\ominus$ it supports. The user is responsible for defining semantic functor 
$F_\varphi$ for every function $\varphi$ she would like to be able to evaluate 
lazily or use in a generalized n+k pattern. We show how to define such functors 
in \textsection\ref{sec:impl}, while here we use the notation $F(v_1,\cdots,v_k)$ 
to refer to the value representing the result of applying such a functor to 
values $v_1,\cdots,v_k$.

Each variant of an algebraic data type in a functional language has exactly one 
constructor, which makes it ideally suitable for structural decomposition of the 
type with pattern matching. Classes in \Cpp{} are allowed to have multiple 
constructors, which is why we need a mechanism that would let the user specify 
structural decomposition of a class. We do this with the help of bindings 
(\textsection\ref{sec:bnd}) represented here with a functor $\Delta_i^{\tau,l}$. 
We use the notation $\Delta_i^{\tau,l}(v)$ to refer to the value representing 
the $i^{th}$ component in layout $l$ of the structural decomposition of a value 
$v$ of type $\tau$.

Lastly, we let the user define the exact meaning of matching a value $v$ against 
an expression $\eta$ by case analysis on the structure of $\eta$. The exact 
details of defining such algebraic decomposition are given in 
\textsection\ref{sec:slv}, while here we use the notation $\mathsf{solve}(\eta,v)$ 
to refer to a boolean value indicating whether the generalized n+k pattern 
$\eta$ was accepted (true) or rejected (false).

We model the run-time environment of our SELL as a map $\Sigma: \chi\rightarrow T$ 
since our variables $\chi$ either hold a value of type \code{T} or refer to another 
variable of that type. In addition to meta-variables we have seen already, 
meta-variables $u,v$ and $b^{bool}$ range over values.

%\subsubsection{Evaluation Rules}
%\label{sec:eval}

\begin{figure}[h]
\begin{mathpar}

\inferrule[E-Value]
{\varsigma = \Value{\tau}(v)}
{\varsigma \lazyevals v}

\inferrule[E-Var]
{\chi = \Variable{\tau}(v)}
{\chi \lazyevals v}

\inferrule[E-Unary]
{\xi \lazyevals v}
{\ominus \xi \lazyevals F_\ominus(v)}

\inferrule[E-Binary]
{\xi_1 \lazyevals v_1 \\ \xi_2 \lazyevals v_2}
{\xi_1 \oplus \xi_2 \lazyevals F_\oplus(v_1,v_2)}

%\inferrule[E-Binary-Const-Left]
%{\xi \lazyevals v}
%{\xi \oplus c \lazyevals F_\oplus(v,c)}

%\inferrule[E-Binary-Const-Right]
%{\xi \lazyevals v}
%{c \oplus \xi \lazyevals F_\oplus(c,v)}

\inferrule[E-Function]
{\xi_1 \lazyevals v_1 \\ \cdots \\ \xi_k \lazyevals v_k}
{\varphi(\xi_1,\cdots,\xi_k) \lazyevals F_\varphi(v_1,\cdots,v_k)}

\end{mathpar}
\caption{Evaluation rules}
\label{evaluation}
\end{figure}

Figure~\ref{evaluation} shows the evaluation rules used to evaluate lazy expressions 
that our SELL introduces. The judgments are of the form $\Sigma\vdash \xi \lazyevals v$ 
stating that lazy expression $\xi$ evaluates to a value $v$ in a run-time 
environment $\Sigma$. We do not mention the run-time environment in the rules 
for brevity since the evaluation does not modify it.

%\subsubsection{Semantics of Matching Expressions}
%\label{sec:semme}

\begin{figure}[h]
\begin{mathpar}
\inferrule[P-Wildcard]
{}
{\Sigma\vdash \_(v_e) \evals \True,\Sigma}

\inferrule[P-Value]
{\varsigma \lazyevals u}
{\Sigma\vdash \varsigma^\tau(v_e) \evals (u=v),\Sigma}

\inferrule[P-Variable]
{u=F_{dc}(\tau,v_e)}
{\Sigma\vdash \chi^{\tau}(v_e) \evals (u \neq \nullptr{}),\Sigma[\chi\leftarrow u]}

\inferrule[P-n+k-Pattern]
{\Sigma\vdash \mathsf{solve}(\xi,v_e) \evals v,\Sigma'}
{\Sigma\vdash \xi(v_e) \evals v,\Sigma'}

\inferrule[P-Guard]
{\Sigma\vdash \pi(v_e) \evals b_\pi,\Sigma' \\ \Sigma'\vdash \xi \lazyevals b_\xi}
{\Sigma\vdash (\pi \models \xi)(v_e) \evals (b_\pi \wedge b_\xi),\Sigma'}

\inferrule[P-Constructor-Nullptr]
{F_{dc}(\tau,v_e)=\nullptr{}}
{\Sigma\vdash (\mathsf{match}\langle\tau\left[,l\right]\rangle(\varpi_1,...,\varpi_k))(v_e) \evals \False,\Sigma}

\inferrule[P-Constructor-Reject]
{ u=F_{dc}(\tau,v_e) \\
 \Sigma_1    \vdash \varpi_1(\Delta_1^{\tau,l}(u))         \evals \True, \Sigma_2 \\ \cdots \\
 \Sigma_{i-1}\vdash \varpi_{i-1}(\Delta_{i-1}^{\tau,l}(u)) \evals \True, \Sigma_i \\
 \Sigma_i    \vdash \varpi_i(\Delta_i^{\tau,l}(u))         \evals \False,\Sigma_{i+1}
}
{\Sigma\vdash (\mathsf{match}\langle\tau\left[,l\right]\rangle(\varpi_1,...,\varpi_k))(v_e) \evals \False,\Sigma_{i+1}}

\inferrule[P-Constructor-Accept]
{ u=F_{dc}(\tau,v_e) \\
 \Sigma_1    \vdash \varpi_1(\Delta_1^{\tau,l}(u)) \evals \True, \Sigma_2 \\ \cdots \\
 \Sigma_k    \vdash \varpi_k(\Delta_k^{\tau,l}(u)) \evals \True, \Sigma_{k+1}
}
{\Sigma\vdash (\mathsf{match}\langle\tau\left[,l\right]\rangle(\varpi_1,...,\varpi_k))(v_e) \evals \True,\Sigma_{k+1}}

\end{mathpar}
\caption{Semantics of matching expressions}
\label{exprsem}
\end{figure}

The rule set in Figure~\ref{exprsem} deals with pattern application $\pi(e)$, 
which essentially performs matching of a pattern $\pi$ against an expression 
$e$. To avoid dealing with the \Cpp{} semantics, we assume that the expression $e$ 
has already been evaluated to a value $v_e$. Our judgments are thus of the 
form $\Sigma\vdash \pi(v_e) \evals v,\Sigma'$ that can be interpreted as 
following: given an environment $\Sigma$ and a value $v_e$ representing the 
result of evaluating subject expression $e$ according to the \Cpp{} semantics, 
pattern application $\pi(v_e)$ results in value $v$ and environment $\Sigma'$. 

%\subsubsection{Semantics of Match Statement}
%\label{sec:semms}

\begin{figure}[h]
\begin{mathpar}
\inferrule[Match-True]
{ v_e \neq \nullptr \\
 \Sigma      \vdash_{v_e} C_1    \evals \False,\Sigma_1     \\ \cdots \\
 \Sigma_{i-2}\vdash_{v_e} C_{i-1}\evals \False,\Sigma_{i-1} \\
 \Sigma_{i-1}\vdash_{v_e} C_i    \evals \True, \Sigma_i
}
{\Sigma\vdash \mathsf{Match}(v_e) \{ \left[C_i \vec{s}_i\right]^*_{i=1..n} \} \mathsf{EndMatch} \evals i,\Sigma_i}

\inferrule[Match-False]
{ v_e \neq \nullptr \\
 \Sigma      \vdash_{v_e} C_1    \evals \False,\Sigma_1 \\ \cdots \\
 \Sigma_{n-1}\vdash_{v_e} C_n    \evals \False,\Sigma_n
}
{\Sigma\vdash \mathsf{Match}(e) \{ \left[C_i \vec{s}_i\right]^*_{i=1..n} \} \mathsf{EndMatch} \evals 0,\Sigma_n}

\inferrule[Qua]
{\Sigma \vdash \mathsf{match}\langle\tau,l\rangle(\vec{\varpi})(v_e) \evals b,\Sigma' }
{\Sigma \vdash_{v_e} \mathsf{Qua}(\tau,\vec{\varpi}) \evals b,\Sigma'[\mathsf{matched}^\tau\rightarrow F_{dc}(\tau,v_e)]}

\inferrule[When]
{\Sigma \vdash_{v^\tau_e} \mathsf{Qua}(\tau,\vec{\varpi}) \evals b,\Sigma'}
{\Sigma \vdash_{v^\tau_e}     \mathsf{When}(\vec{\varpi}) \evals b,\Sigma'}

\inferrule[Case]
{\Delta_i^\tau : \tau \rightarrow \tau_i, i=1..k \\
 \Sigma[x_i^{\tau_i}\rightarrow\nullptr]_{i=1..k} \vdash_{v_e} \mathsf{Qua}(\tau,x_1,...,x_k) \evals u,\Sigma' }
{\Sigma \vdash_{v_e} \mathsf{Case}(\tau,x_1,...,x_k) \evals u,\Sigma'}

\inferrule[Otherwise]
{\Sigma \vdash_{v^\tau_e} \mathsf{Case}(\tau,\vec{x}) \evals u,\Sigma'}
{\Sigma \vdash_{v^\tau_e} \mathsf{Otherwise}(\vec{x}) \evals u,\Sigma'}
\end{mathpar}
\caption{Semantics of match-statement}
\label{stmtsem}
\end{figure}

The rule set in Figure~\ref{stmtsem} describes the semantics of a \emph{match 
statement}. In order to avoid dealing with the semantics of the \Cpp{} statements, 
we define the semantics of a match-statement to be the index of the matching case 
clause and the run-time environment right before the clause's statement, or $0$ 
if none of the clauses matched. The judgments are thus of the form 
$\Sigma\vdash M \evals v,\Sigma'$ for match statement, and are slightly extended 
for case clauses $\Sigma\vdash_{v_e} C \evals b,\Sigma'$ with value $v_e$ of a 
subject that is passed along from the match statement onto the clauses.

The rules essentially describe the first-fit strategy for evaluating the clauses.
Evaluation of a \code{Qua}-clause is equivalent to evaluation of a corresponding 
match-expression on a constructor pattern. Successful match will introduce into 
the local scope of the clause a variable \code{matched} bound to the subject 
properly casted to the target type $\tau$. Evaluation of \code{When}-clause 
amounts to evaluation of a corresponding \code{Qua}-clause with target type 
being the subject type. Evaluation of \code{Case}-clauses amounts to evaluation 
of \code{Qua}-clauses in the environment extended with variables passed as 
arguments to the clause. Evaluation of default clause amounts to evaluating a 
corresponding \code{Case}-clause with target type being the subject type.


\section{Patterns Implementation} %%%%%%%%%%%%%%%%%%%%%%%%%%%%%%%%%%%%%%%%%%%%%%%%%%%%%%%
\label{sec:impl}

Our implementation of patterns and lazy 
evaluation of expressions is based on \emph{Expression Templates} 
\cite{Veldhuizen95expressiontemplates,vandevoorde2003c++}. It encodes 
expression trees with types, which are hidden from the user through the use of 
overloading.

There are 6 kinds of expression templates in our library: \code{wildcard}, 
\code{value<T>}, \code{variable<T>}, \code{expr<F,E...>}, \code{guard<E1,E2>}, 
\code{ctor<T,E...>}. Each models a \code{Pattern} concept.
Using the notation from C++0x\cite{C++0xConcepts}:

\begin{lstlisting}[keepspaces,columns=flexible]
concept Pattern<typename T> 
{
    typename R; // the return type of () isconvertible to bool
    Convertible<R,bool>;
    template <typename U> R operator()(const U& u) const; // application operator
};
\end{lstlisting}

Each of our six pattern kinds 
implements the application operator according to the semantics presented in 
Figure~\ref{exprsem}. The application operator's result has to be 
convertible to bool;
\code{true} indicates a successful match. A class might have several overloads of 
the above operator that distinguish cases of interest. We summarize the requirements on template parameters of each of our 
pattern in Figure~\ref{xt-reqs}.

\begin{figure}[h]
\centering
\begin{tabular}{llll}
{\bf Pattern}       & {\bf Parameters}          & {\bf Argument of application operator U}         \\ \hline
\code{wildcard}     & --                        & --                                               \\
\code{value<T>}     & \code{Regular<T>}         & \code{Convertible<U,T>}                          \\
\code{variable<T>}  & \code{Regular<T>}         & \code{Convertible<U,T>}                          \\
\code{expr<F,E...>} & \code{LazyExpression<E>}  & \code{Convertible<U,expr<F,E...>::result_type>}  \\
\code{guard<E1,E2>} & \code{LazyExpression<Ei>} & any type accepted by \code{E1::operator()}       \\
\code{ctor<T,E...>} & \code{Polymorphic<T>}     & \code{Polymorphic<U>} for open encoding          \\
                    & \code{Object<T>}          & \code{is_base_and_derived<U,T>} for tag encoding \\
\end{tabular}
\caption{Requirements on parameters and argument type of an application operator}
\label{xt-reqs}
\end{figure}

To support lazily evaluated expressions, classes \code{value<T>}, 
\code{variable<T>} and \code{expr<F,E...>} also model a \code{LazyExpression} 
concept:

\begin{lstlisting}[keepspaces,columns=flexible]
concept LazyExpression<typename T> 
{
    typename result_type;
    operator result_type(const T&); // conversion operator (to result_type)
};
\end{lstlisting}

\noindent
The \code{result_type} defines the type of the result of an argument expression.
It is \code{T} for \code{value<T>} and \code{variable<T>}. 
For \code{expr<F,E...>} it is defined to be \code{decltype(F()(E...))}, 
indicating the result of applying \code{F} to arguments of type 
\code{E...} .
The conversion operator invokes the evaluation of the 
lazy expression. The evaluation is performed according to the semantics 
presented in Figure~\ref{evaluation}.

Concepts were not included in C++11, so we emulate them using overloading and 
\code{enable_if}~\cite{jarvi:03:cuj_arbitrary_overloading}.

Each (possibly overloaded) operator and function that can be evaluated lazily 
and matched against in generalized n+k patterns are represented with a class whose 
only purpose is to transparently forward the call to an overloaded function that 
implements the semantics of the operation. We refer to such a class together with 
the overloaded function it forwards the calls to as a \emph{semantic functor}.
For example, this functor defines semantics of multiplication:

\begin{lstlisting}[keepspaces,columns=flexible]
struct mult 
{
  template <class A, class B> 
  auto operator()(A&& a, B&& b) const -> decltype(a*b) 
  { return std::forward<A>(a) * std::forward<B>(b); }   
};
\end{lstlisting}

\noindent
Unlike similar classes in STL, our representation of operations is not 
parameterized with argument types. This simplifies defining overloads of 
\code{solve} as shown below, reflecting the structure of the 
expression.

Variables of types \code{variable<T>} and \code{wildcard} as well as values 
wrapped into \code{value<T>} (implicitly by the library or explicitly with a 
call to function \code{val()}) constitute simple expressions in our pattern 
sub-language. Complex expressions are built by applying C++ operators 
listed in Figure~\ref{syntax} and functions overloaded on our lazy expressions. 
To support that, for every unary operator $\ominus$ and binary operator $\oplus$ 
and the semantic functors $F_\ominus$ and $F_\oplus$ we define for them, the 
library introduces:

\begin{lstlisting}[keepspaces]
template <LazyExpression E1>
    expr<@$F_\ominus$@,E1> operator@$\ominus$@(E1&& e1) noexcept 
    { return expr<@$F_\ominus$@,E1>(std::forward<E1>(e1)); }
template <LazyExpression E1, LazyExpression E2>
    expr<@$F_\oplus$@,E1,E2> operator@$\oplus$@(E1&& e1, E2&& e2) noexcept 
    { return expr<@$F_\oplus$@,E1,E2>(std::forward<E1>(e1),std::forward<E2>(e2)); }
template <Pattern P1, LazyExpression E2>
    guard<P1,E2> operator@$\models$@(P1&& p1, E2&& e2) noexcept 
    { return guard<P1,E2>(std::forward<P1>(p1),std::forward<E2>(e2)); }
template <typename T, LazyExpression... E>
    ctor<T,E...> match(E&&... e) noexcept 
    { return ctor<T,E...>(std::forward<E>(e)...); }
\end{lstlisting}

\subsection{Structural Decomposition}
\label{sec:bnd}

We use compile-time reflection to let the user specify information about 
a class hierarchy to the library. The information is provided as a specialization of a 
class-template \emph{bindings}. Among other things, bindings let the 
user define the semantic functor $\Delta_i^{\tau,l}$ we introduced in 
\textsection\ref{sec:sem}. The grammar in Figure~\ref{bind-syntax} defines the 
entities that may constitute a binding definition.

\begin{figure}[h]
\centering
\begin{tabular}{lp{1em}cl}
\Rule{bindings}                &           & \is{}  & $\delta^*$ \\
\Rule{binding definition}      & $\delta$  & \is{}  & \code{template <}$\left[\vec{p}\right]$\code{>} \\
                               &           &        & \code{struct bindings<} $\tau[$\code{<}$\vec{p}$\code{>}$]\left[,l\right]$\code{>} \\
                               &           &        & \code{\{} $\left[ks\right]\left[kv\right]\left[bc\right]\left[cm^*\right]$ \code{\};} \\
\Rule{class member}            & $cm$      & \is{}  & \code{CM(}$c^{size\_t},q$\code{);} \\
\Rule{kind selector}           & $ks$      & \is{}  & \code{KS(}$q$\code{);}    \\
\Rule{kind value}              & $kv$      & \is{}  & \code{KV(}$c$\code{);}    \\
\Rule{base class}              & $bc$      & \is{}  & \code{BC(}$\tau$\code{);} \\
\Rule{template-parameter-list} & $\vec{p}$ &        & C++\cite[\textsection A.12]{C++11} \\
\Rule{qualified-id}            & $q$       &        & C++\cite[\textsection A.4]{C++11} \\
\end{tabular}
\caption{Syntax for defining bindings}
\label{bind-syntax}
\end{figure}

\noindent
Any type $\tau$ may have arbitrary number of \emph{bindings} associated with it 
and distinguished through the \emph{layout} parameter $l$. The \emph{default 
binding} which omits layout parameter is implicitly associated with the layout whose
value is equal to predefined constant \code{default_layout = size_t(}$\sim$\code{0)}. 
A \emph{Binding definition} is a specialization of
\code{template <typename T, size_t l = default_layout> struct bindings;}
The body of the class consists of a sequence of specifiers, which generate the 
necessary definitions for querying bindings by the library code. Note that 
binding definitions made this way are \emph{non-intrusive} since the original 
class definition is not touched. They also respect \emph{encapsulation} since 
only the public members of the target type will be accessible from within 
\code{bindings} specialization.

A \emph{Class Member} specifier \code{CM(}$c,q$\code{)} takes a (zero-based) binding 
position $c$ and a member $q$, whose value will be matched against in $\tau$'s 
construction pattern. Qualified identifier is allowed to be of one of the 
following kinds:

\begin{compactitem}
\setlength{\itemsep}{0pt}
\setlength{\parskip}{0pt}
\item Data member of the target type
\item Nullary member-function of the target type
\item Unary external function taking the target type by pointer, reference or value.
\end{compactitem}

\noindent
Using \code{CM} specifier a user defines the semantic functor 
$\Delta_i^{\tau,l},i=1..k$ we introduced in \textsection\ref{sec:sem} as 
following:

\begin{lstlisting}[keepspaces]
template <typename... T> struct bindings<@$\tau$@<T...>> 
    { CM(0, @$\tau$@<T...>::member@$_0$@); ... CM(@$k$@, @$\tau$@<T...>::member@$_k$@); };
\end{lstlisting}

\noindent
A \emph{Kind Selector} specifier \code{KS(}$q$\code{)} is used to specify a member 
of the subject type that will uniquely identify the variant for \emph{tagged} 
and \emph{union} encodings. The member $q$ can be of any of the three categories 
listed for \code{CM}, but is required to return an \emph{integral type}.
A \emph{Kind Value} specifier \code{KV(}$c$\code{)} is used by \emph{tagged} and 
\emph{union} encodings to specify a constant $c$ that uniquely identifies given 
variant. 
A \emph{Base Class} specifier \code{BC(}$\tau$\code{)} is used by the \emph{tagged}
encoding to specify an immediate base class of the class whose bindings we 
define.

A \emph{Layout} parameter $l$ can be used to define multiple bindings for the same 
target type. This is particularly essential for \emph{union} encoding where the 
types of the variants are the same as the type of subject and thus layouts 
become the only way to associate variants with position bindings. For this 
reason, we require binding definitions for \emph{union} encoding always use the 
same constant $l$ as a kind value specified with \code{KV(l)} and the layout 
parameter $l$!

\subsection{Algebraic Decomposition}
\label{sec:slv}

Intuitively n+k patterns like $f(x,y)=v$ relate a known result of a given 
function application to its arguments. The case where multiple unknown arguments 
are matched against a single result should not be immediately discarded as there 
are known n-ary functions whose inverse is unique. An example of such function 
is Cantor pairing function that defines bijection between 
$\mathbb{N}\times\mathbb{N}$ and $\mathbb{N}$. Even when such mappings are not 
one-to-one, their restriction to a given argument often is. Most generalizations 
of n+k patterns seem to agree on the following rules:

\begin{itemize}
\setlength{\itemsep}{0pt}
\setlength{\parskip}{0pt}
\item Absence of solution that would result in a given value should be indicated 
      through rejection of the pattern.
\item Presence of a unique solution should be indicated with acceptance of the 
      pattern and binding of corresponding variables to the solution.
\end{itemize}

\noindent When multiple solutions are possible, returning a set or an enumerator 
is not usually considered due to differences in types: variable $x$ representing 
a solution to $f(x)=c$ is intuitively expected to have the same type as the 
argument of $f$, so that it can be applied to $x$. Rejecting the pattern might be 
a plausible approach in some applications where multiple solutions are treated as 
ambiguity. It is incapable of distinguishing absence of a solution from ambigous 
solution however, which is often desired.

Binding to an arbitrary solution in case of multiple ones might be sensitive as 
to which solution is chosen: some applications might prefer the 
smallest/largest one, some the smallest positive etc.
We believe that depending on application, different semantic choices can be 
valid, which is why we prefer not to make such choice for the user, but rather 
provide him with means to decide. In fact we go even further and do not require 
the values bound by our generalized n+k pattern to be a solution to the 
corresponding equation. We do this for several reasons:

\begin{itemize}
\item Due to numeric errors, truncation and integer overflow we will rarely 
      obtain the exact algebraic solution.
\item Curve fitting can be seen as pattern matching in some application domains. 
\item Sometimes we might be interested in matching against a projection of a 
      value onto some base and the obtained result may not necessarily yield a 
      solution to matching against the original value.
\end{itemize}

Consider matching an expression $x+1$ with variable $x$ of type \code{char} 
ranging over $[-128,127]$ against a value -128. Should the value 127 be 
considered a solution since 127+1 overflows in char resulting in -128? From the 
mathematical point of view it should not, but a particular application might 
accept such a solution for the sake of performance. Similarly, matching $3*y$ 
against 1.0 with a variable $y$ of type \code{double} will result in a number 
that is slightly different from $\frac{1}{3}$. Should such match be accepted the 
next logical request a user might have is to be able to match an expression of 
the form $n/m$ for integer variables $n$ and $m$ against a value 3.1415926 and 
get the closest fraction to it. Matching against such pattern does not have to 
be imprecise as one can match it against an object of a class representing 
rational numbers.

Taking the argument of precision even further one may want to be able to do 
curve fitting with generalized n+k patterns for the sake of expressive syntax. 
Consider an object that contains sampling of some random variable. A hypotetical 
match statement might be querying:

\begin{lstlisting}[keepspaces,columns=flexible]
match (random_variable) { 
    case Gaussian(@$\mu,\sigma^2$@): ... case Poisson(@$\lambda$@): ... case Bernoulli(@$p$@): ... 
}
\end{lstlisting} 

Fitting error threshold in such scenario can either be global or passed 
as a parameter into expressions we are matching against: e.g. 
\code{case Gaussian(}$0.01,\mu,\sigma^2$\code{):}.
Again the fitting does not have to be imprecise. Consider a library dealing with 
polynomials of arbitrary degree. Given a general polynomial we might want to be 
able to check a few special cases for which analytical solutions to some larger 
question exist:

\begin{lstlisting}[keepspaces,columns=flexible]
match (polynomial) { case a*X^1 + 1: ... case 2*X^2 + b*X^1 + c: ... }
\end{lstlisting} 

X in such scenario is not a variable but a placeholder value of a kind that lets 
us identify the degree, whose coefficient is sought. 

A simpler example of this kind is decomposition of a complex number with Euler's 
notation $a+b*i$ for scalar variables $a$ and $b$. With variables, such a 
generalized n+k pattern is irrefutable for all complex numbers, but when a more 
specific form is queried (e.g. $3+b*i$) a given complex number may fail to match 
such a pattern. While matching, we will project such a complex number along its 
real and imaginary components and will try matching the operands of addition 
using those projections. Solutions obtained along each projection may not 
necessarily combine into the final solution.

What all these examples have in common is not necessarily solving the equation 
that generalized n+k patterns represent, but the fact that we associate certain 
notations with certain mathematical entities they represent. Parameters of 
those expressions are typically associated with the parameterns of the 
underlying mathematical object and we perform decomposition of that object into 
parts. The structure of the expression tree is an analog of a constructor symbol in 
structural decomposition, while its leaves are placeholders for parameters to be 
matched against or inferred from the mathematical object in question.

Algebraic decomposition to mathematical entities is what views are to algebraic 
data types. Consider for example an object representing a 2D line. At different 
parts of the program we might need to decompose that line differently (hypothetical syntax):

\begin{lstlisting}[keepspaces,columns=flexible]
if (line matches m*X + c) ...                       // slope-intercept form
if (line matches a*X + b*Y = c) ...                 // linear equation form
if (line matches (Y-y0)*(x1-x0)=(y1-y0)*(X-x0)) ... // two-points form
\end{lstlisting}

As before, X and Y are not variables, but some syntactic entities that let us 
properly decompose parts. Matching against the slope-intercept notation will not 
be able to decompose a line of the form $y=c$, but otherwise still looks like 
solving an equation (even though quantified over all X). The other two notations 
include equality sign in their expression, which makes our argument that we 
decompose against a known notation (as opposed to solving some equation) 
stronger.

\subsubsection{Solvers}

The above class esssentially defines forward semantics of a family of 
operations. To define backward semantics of it for the use in n+k patterns, the 
user defines \emph{solvers} by overloading a function 

\begin{lstlisting}
template <LazyExpression E, typename S> bool solve(const E&, const S&);
\end{lstlisting}

The first argument of the function takes an expression template representing an 
expression we are matching against, while the second argument represents the 
expected result. The following example defines a generic solver for 
multiplication by a constant:

\begin{lstlisting}[keepspaces]
template <LazyExpression E, typename T> requires Field<E::result_type>
bool solve(const expr<mult,E,value<T>>& e, const E::result_type& r) {
    return solve(e.m_e1,r/eval(e.m_e2));
}
@\halfline@
template <LazyExpression E, typename T> requires Integral<E::result_type>
bool solve(const expr<mult,E,value<T>>& e, const E::result_type& r) {
    T t = eval(e.m_e2);
    return r%t == 0 && solve(e.m_e1,r/t);
}
\end{lstlisting}

\noindent
Note that we overload not only on the structure of the expression, but also on 
the properties of their result type (or any other type involved). In particular 
when the type of the result of the sub-expression models \code{Field} concept, 
we can rely on presence of unique inverse and simply call division without any 
additional checks. A similar overload for integral multiplication additionally 
checks that result is divisible by the constant, before generically forwarding 
the matching to the first argument of multiplication. This last overload 
combined with a similar solver for addition of integral types is everything the 
library needs to properly handle the definition of the \code{fib} function from 
\textsection\ref{sec:syn}.

A generic solver capable of decomposing a complex value using the Euler 
notation is very easy to define by fixing the structure of expression:

\begin{lstlisting}[keepspaces]
template <LazyExpression E1, LazyExpression E2> 
    requires SameType<E1::result_type,E2::result_type>
bool solve(
        const expr<plus,expr<mult,E1,value<complex<E1::result_type>>>,E2>& e, 
        const complex<E1::result_type>& r);
\end{lstlisting}

\noindent
Note that the template facilities of C++ resemble pattern-matching facilities of 
other languages. We essentially use these compile-time patterns to describe the 
structure of the expression this solver is applicable to: $e_1*c+e_2$ with types 
of $e_1$ and $e_2$ being the same as type on which a complex value $c$ is 
defined. The actual value of the complex constant $c$ will not be known until 
run-time, but assuming its imaginary part is not $0$, we will be able to 
generically obtain the values for sub-expressions.

Our approach is largely possible due to the fact that the library only serves as 
an interface between expressions and functions defining their semantics and 
algebraic decomposition. The fact that the user explicitly defines the variables 
she would like to use in patterns is also a key as it lets us specialize not 
only on the structure of the expression, but also on the types involved. 
Inference of such types in functional languages would be hard or impossible as the 
expression may have entirely different semantics depending on the types of 
arguments involved. Concept-based overloading simplifies significantly the case 
analysis on the properties of types, making the solvers generic and composable.
The approach is also viable as expressions are decomposed at compile-time and 
not at run-time, letting the compiler inline the entire composition of solvers. 

An obvious disadvantage of this approach is that the more complex expression 
becomes, the more overloads the user will have to provide to cover all 
expressions of interest. The set of overloads will also have to be made 
unambiguous for any given expression, which may be challenging for novices. An 
important restriction of this approach is its inability to detect multiple uses 
of the same variable in an expression at compile time. This happens because 
expression templates remember the form of an expression in a type, so use of two 
variables of the same type is indistinguishable from the use of the same 
variable twice. This can be worked around by giving different variables 
(slightly) different types or making additional checks as to the structure of 
expression at run-time, but that will make the library even more verbose or 
incur a significant run-time overhead.

\subsection{Views}
\label{sec:view}

Support of multiple bindings through layouts in our library effectively enables 
a facility similar to Wadler's \emph{views}\cite{Wadler87}. Reconsider example from 
\textsection\ref{sec:bg} that discusses cartesian and polar representations of 
complex numbers, demonstrating the notion of view. The same example recoded with 
our SELL looks as following:

\begin{lstlisting}[keepspaces,columns=flexible]
// Introduce layouts
enum { cartesian = default_layout, polar };
@\halfline@
// Define bindings with them
template <typename T> struct bindings<std::complex<T>>
  { CM(0,std::real<T>); CM(1,std::imag<T>); };
template <typename T> struct bindings<std::complex<T>, polar>
  { CM(0,std::abs<T>);  CM(1,std::arg<T>); };
@\halfline@
// Define views
template <typename T> 
  using Cartesian = view<std::complex<T>>;
template <typename T> 
  using Polar     = view<std::complex<T>, polar>;
@\halfline@
  std::complex<double> c;
  double a,b,r,f;
@\halfline@
  if (match<std::complex<double>>(a,b)(c)) // default
  if (match<   Cartesian<double>>(a,b)(c)) // same as above
  if (match<       Polar<double>>(r,f)(c)) // view
\end{lstlisting}

\noindent
The C++ standard effectively enforces the standard library to use cartesian 
representation\cite[\textsection26.4-4]{C++11}. Knowing that, we choose the 
\code{cartesian} layout to be default, with \code{polar} being an alternative 
layout for complex numbers. We then define bindings for each of these layouts as 
well as introduce template aliases (an analog of typedefs for parameterized 
classes) for each of the views. Template class \code{view<T,l>} defined by the 
library provides a way to bind together a target type with one of its layouts 
into a single type. This type can be used everywhere in the library where an 
original target type was expected, while the library will take care of decoding 
the type and layout from the view and passing them along where needed.

The first two match expressions are the same and incur no run-time overhead 
since they use default layout of the underlying type. The third match expression 
will implicitly convert cartesian representation into polar, thus incurring some 
overhead. This overhead would have been present in code that depends on polar 
coordinates anyways, since the user would have had to invoke the corresponding 
functions manually.

The important difference from Wadler's solution is that our views can only be 
used in a match expression and not as a constructor or arguments of a function 
etc.

\section{Match Statement} %%%%%%%%%%%%%%%%%%%%%%%%%%%%%%%%%%%%%%%%%%%%%%%%%%%%%%%
\label{sec:impl}

Our implementation of pattern matching expressions follows the naive way of 
essentially interpreting them through backtracking. On one hand, this was a 
consequence of working in a library setting, where code transformations are much 
harder to achive. On the other hand, from the very beginning we were trying to 
find an expressive alternative to object decomposition with either nested 
dynamic casts or visitor design pattern, and thus were not concerned with 
pattern matching on multiple arguments, where decision tree approach becomes 
more efficient. Dealing with single argument certainly leaves less choices for 
optimization, but does not eliminate them as repeated use of constructor-pattern 
with the same target type but different argument patterns essentially leads to 
the same inefficiencies. To tackle this issue in a library setting we rely on 
and give more control to the library user. For example, we fix the order of 
evaluation, but let guard-patterns be placed directly on the arguments of a 
constructor-pattern to let the user benefit from the consciesness of expression, 
while holding a grip on performance. Similarly, we added \code{Alt} sub-clauses 
to \code{Qua}-clause to syntactically separate fast type switching from slow 
sequential evaluation of pattern matching expressions. The fall-through behavior 
of the \code{Match}-statement allows the user to achieve the same effect 
directly with \code{Qua}-clauses, however the performance overhead involved 
justified the addition of otherwise syntactic sugar.

The interpretation of pattern matching expressions with expression templates follows 
very closely the composition of expressions described by abstract syntax 
in~\textsection\ref{sec:syn} as well as their application to subject expression 
described by evaluation rules in \textsection\ref{sec:semme}. This section thus
mainly concentrates on efficient implementation of a match statement as 
well as unification of its syntax to the three encodings of algebraic data types
outlined in \textsection\ref{sec:adt}. The discussion will largely focus on 
devising an efficient \emph{type-switch}, which is then used by our library as a 
backbone to the general match statement presented in~\textsection\ref{sec:semms}. 

By encoding algebraic data types with classes we alter their semantics in two 
important ways: we make them \emph{extensible} as new variants can be added by 
simply deriving from the base class, as well as \emph{hierarchical} as variants 
can be inherited from other variants and thus form a subtyping relation between 
themselves~\cite{Glew99}. This is not the case with traditional algebraic data 
types in functional languages, where the set of variants is \emph{closed}, while 
the variants are \emph{disjoint}. Some functional languages e.g. 
ML2000~\cite{ML2000} and Moby~\cite{Moby} were experimenting with 
\emph{hierarchical extensible sum types}, which are closer to object-oriented 
classes then algebraic data types are, but, interestingly, they did not provide 
pattern matching facilities on them. Working within a multi-paradigm  
programming language like C++, we will not be looking at algebraic data types in
the closed form they are present in functional languages, but rather in an 
open/extensible form discussed by Zenger~\cite{Zenger:2001}, Emir~\cite{EmirThesis}, 
L\"oh~\cite{LohHinze2006}, Glew~\cite{Glew99} and others. We will thus 
assume an object-oriented setting where new variants can be added later and form
subtyping relations between each other including those through multiple 
inheritance. We will look separately at polymorphic and tagged class encodings 
as our handling of these two encodings is significantly different. Before we 
look into these differences in greater details, however, we would like to look 
at the problem of type switching without specific implementation in mind as well 
as properties we would like to seek from such an implementation.

\input{sec-5-implementation-typeswitch-problem}
\input{sec-5-implementation-typeswitch-tagged}
\section{Type Switching}
\label{sec:copc}

%While \Cpp{} does not have direct support for algebraic data types, they can be 
%encoded with classes in a number of ways. One common such encoding is to 
%introduce an abstract base class representing an algebraic data type with 
%several derived classes representing variants. The variants can be 
%discriminated with either run-time type information (\emph{polymorphic 
%encoding}) or a unique tag inside a dedicated member of the common base class 
%(\emph{tagged encoding}).

\emph{Mach7} explicitly supports at least two encodings of algebraic
datatypes: runtime type information discriminant, and numerical tag
data member shared by all classes in a given hierarchy.  The library
handles them differently to let 
the user choose between openness and efficiency. The type switch for tagged 
encoding (\textsection\ref{sec:cotc}) is simpler and more efficient for many typical use cases, however, 
making it open eradicates its performance advantages (\textsection\ref{sec:cmp}). 
%The difference in 
%performance is the price we pay for keeping the solution open. We describe pros 
%and cons of each approach in \textsection\ref{sec:cmp}.

%The core of the proposal relies on two key aspects of \Cpp{} implementations:
%\begin{enumerate}
%\item a constant-time access to the virtual table pointer embedded in an object of
%  dynamic class type;
%\item injectivity of the relation between an object's inheritance path
%  and the virtual table pointer extracted from that object.
%\end{enumerate}

%\subsection{An Attractive Non-Solution}
\label{sec:cotc}

%The memoization device outlined in \textsection\ref{sec:memdev} can, in principle, also be 
%applied to tagged classes. The dynamic cast will be replaced by a small 
%compile-time template meta-program that checks whether the class associated with 
%the given tag is derived from the target type of the case clause. If so, a static 
%cast can be used to obtain the offset.

%Despite its straightforwardness, we felt that it should be possible to do better 
%than the general solution, given that each class is already identified with a 
%dedicated constant known at compile time.

While Wirth' linked list encoding was considered slow for subtype testing, it can 
be adopted for quite efficient type switching on a class hierarchy with no 
repeated inheritance. The idea is to combine fast switching on closed 
algebraic datatypes with a loop that tries the tags of base classes when 
switching on derived tags fails.

%The nominal subtyping of \Cpp{} effectively gives every class multiple types. The 
%idea is thus to associate with the type not only its most-derived tag, but also 
%the list of tags of all its base classes. In a compiler implementation such a 
%list can be stored inside the virtual table of a class, while in our library 
%solution it is shared between all the instances with the same most-derived tag 
%in a less efficient global map, associating the tag to its tag list.

For simplicity of presentation we assume a pointer to an array of tags be available 
directly through the subject's \code{taglist} data member. The array is of 
variable size: its first element is always the tag of the subject's dynamic 
type, while its end is marked with a dedicated \code{end_of_list} marker, 
distinct from all the tags. The tags in between are topologically sorted 
according to the subtyping relation with incomparable siblings listed in 
\emph{local precedence order} -- the order of the direct base classes used in 
the class definition. The list resembles the \emph{class precedence list} of 
object-oriented descendants of Lisp (e.g. Dylan, Flavors, LOOPS, and CLOS) used 
there for \emph{linearization} of class hierarchies. 
We also assume the tag-constant associated with a class \code{Di} is accessible 
through a static member \code{Di::class_tag}. These simplifications are not 
essential and the library does not rely on any of them.
%Instead, the user can retroactively narrate to the library the specific tag 
%encoding used through a trait-like class.

A type switch, below, %, built on top of a hierarchy of tagged classes, 
proceeds as 
a regular switch on the subject's tag. If the jump succeeds, we found an exact 
match; otherwise, we get into a default clause that obtains the next tag in the
list and jumps back %to the beginning of the switch statement 
for a rematch:

\begin{lstlisting}[keepspaces]
    size_t attempt = 0; 
    size_t tag = subject->taglist[attempt];
ReMatch:
    switch (tag) {
    default:
        tag = subject->taglist[++attempt];
        goto ReMatch;
    case end_of_list: 
        break;
    case D1::class_tag: 
        D1& match = static_cast<D1&>(*subject); s1;
        break;
        ...
    case Dn::class_tag: 
        Dn& match = static_cast<Dn&>(*subject); sn;
        break;
    }
\end{lstlisting}

\noindent
The above structure, which we call a \emph{tag switch}, implements a variation of 
best-fit semantics based on local precedence order. It lets us dispatch to the case 
clause of the most-specialized class with an overhead of initializing two 
local variables, compared to an efficient switch used on algebraic data types. 
Dispatching to a case clause of a base class will take time roughly proportional 
to the distance between the matched base class and the derived class in the 
inheritance graph, thus the technique is not constant. When none of the base 
class tags was matched, we will necessarily reach the end\_of\_list marker %in the list 
and exit the loop. %As mentioned before, 
The default clause, %of the type switch 
again, can be implemented with a case clause on the subject type's tag: \code{case S::class_tag:}

The efficiency of the above code crucially depends on the set of tags 
being small and sequential to justify the use of a jump table instead of a
decision tree to implement the switch. This is usually not a problem in closed 
hierarchies based on tag encoding since the designer of the hierarchy handpicks 
the tags herself. The use of a static cast %to obtain proper reference once the most specialized derived class has been established, 
however, essentially limits the use of 
this mechanism to non-repeated inheritance only. This only refers to the way target 
classes inherit from the subject type -- they can freely inherit from other classes. 
%as long as they inherit the subject type through non-repeated inheritance only. 
Due to these restrictions, the technique is not open because it may  
violate independent extensibility. We discuss in \textsection\ref{sec:cmp} that 
making the technique more open will also eradicate its performance advantages.


\subsection{An Open but Inefficient Solution}
\label{sec:poets}

Instead of starting with an efficient solution and trying to make it open, we 
start with an open solution and try to make it efficient. The following 
cascading-if statement implements the first-fit semantics for our type switch in 
a truly open fashion:

\begin{lstlisting}
if (T1* match=dynamic_cast<T1*>(subject)) {s1;} else
if (T2* match=dynamic_cast<T2*>(subject)) {s2;} else
...
if (Tn* match=dynamic_cast<Tn*>(subject)) {sn;}
\end{lstlisting}

\noindent
Despite the obvious simplicity, its main drawback is performance: a typical implementation of 
\code{dynamic_cast} takes time proportional to the distance between base and 
derived classes in the inheritance tree. What is worse is that due to the
sequential order of tests, the time to uncover the type in the $i^{th}$ case 
clause will be proportional to $i$, while failure to match will take the longest. 
This linear increase can be seen in the Figure~\ref{fig:DCastVis1}, where 
the above cascading-if was applied to a flat hierarchy encoding an algebraic 
data type with 100 variants. The same type-switching functionality implemented 
with the visitor design pattern took only 28 cycles regardless of the 
case.\footnote{Each case $i$ was timed multiple times, thus turning the experiment 
into a repetitive benchmark described in \textsection\ref{sec:eval}. In a more
realistic setting, represented by random and sequential benchmarks, the cost of 
double dispatch was varying between 52 and 55 cycles.}
This is more than 3 times faster than the 93 cycles it took to uncover even the 
first case with \code{dynamic_cast}, while it took 22760 cycles to uncover the 
last.
In a test involving a flat hierarchy of 100 variants, it took 93 cycles to 
discover the first type and 22760 to discover the last (with their linear combination 
for the types in between). A visitor design pattern could 
uncover any type in about 55 cycles, regardless of its position among the case 
clauses, while a switch based on sequential tags could achieve the same in less 
than 20 cycles. The idea is thus to combine the openness of the above structure 
with the efficiency of a jump table on small sequential values.

Relying on \code{dynamic_cast} also makes an implicit semantic choice where we 
are no longer looking for the first/best-fitting type that is in subtyping 
relation, but for the first/best-fitting type to which a cast is possible from 
the source subobject (\textsection\ref{sec:specifics}).

\begin{figure}[htbp]
  \centering
    \includegraphics[width=0.47\textwidth]{DCast-vs-Visitors1.png}
  \caption{Type switching based on na\"ive techniques}
  \label{fig:DCastVis1}
\end{figure}

%Seeing several solutions whose time increases with the position of the case 
%clause in the type switch, one may wonder how many such clauses a typical 
%program might have. A program dealing with abstract syntax trees in 
%Pivot~\cite{Pivot09} that we implemented using our pattern-matching library had 
%8 match statements with 5, 7, 8, 10, 15, 17, 30 and 63 case clauses, 
%respectively. With Pivot having the smallest number of node kinds among the 
%compiler frameworks we had a chance to work with, we expect a similar or larger 
%number of case clauses in other compiler applications.

When the class hierarchy is not flat and has several levels, the above cascading-if can be replaced 
with a decision tree that tests base classes first and thus eliminates many of 
the derived classes from consideration. This approach is used by Emir to deal with 
type patterns in Scala~\cite[\textsection 4.2]{EmirThesis}. The intent is to 
replace a sequence of independent dynamic casts between classes that are far 
from each other in the hierarchy with nested dynamic casts between classes that 
are close to each other. Another advantage is the possibility to fail early: 
if the type of the subject does not match any of the clauses, we will not have to try all the cases. 
A flat hierarchy, which will likely be formed by the leaves in even a multi-level 
hierarchy, will not be able to benefit from this optimization and 
will effectively degrade to the above cascading-if. Nevertheless, when 
applicable, the optimization can be very useful and its benefits can be seen in
Figure~\ref{fig:DCastVis1} under ``Decision-Tree + dynamic\_cast''. The class 
hierarchy for this timing experiment formed a perfect binary tree with 
classes number 2*N and 2*N+1 derived from a class with number N. The structure 
of the hierarchy also explains the repetitive pattern of timings.

The above solution either in a form of cascading-if or as a decision tree can be 
significantly improved by lowering the cost of a single \code{dynamic_cast}. 
We devised an asymptotically constant version of this operator that we call
\code{memoized_cast} in \textsection\ref{sec:memcast}. As can be seen 
from the graph titled ``Cascading-If + memoized\_cast'', it speeds up the 
above cascading-if solution by a factor of 18 on average, as well as outperforms 
the decision-tree based solution with dynamic\_cast for a number of case clauses 
way beyond those that can happen in a reasonable program. 
We leave the discussion of the technique until 
\textsection\ref{sec:memcast}, while we keep it in the chart to give perspective on 
an even faster solution to dynamic casting. The slowest implementation in the 
chart based on exception handling facilities of C++ is discussed in 
\textsection\ref{sec:xpm}.

The approach of Gibbs and Stroustrup~\cite{FastDynCast} employs divisibility of numbers to obtain a 
tag allocation scheme capable of performing type testing in constant time. 
Extended with a mechanism for storing offsets required for this-pointer 
adjustments, the technique can be used for extremely fast dynamic casting on 
quite large class hierarchies. The idea is to allocate tags 
for each class in such a way that tag of a class D is divisible by a tag of a 
class B if and only if class D is derived from class B. For comparison purposes 
we hand crafted this technique on the above flat and binary-tree hierarchies and 
then redid the timing experiments from Figure~\ref{fig:DCastVis1} using the fast 
dynamic cast. The results are presented in Figure~\ref{fig:DCastVis2}. For 
reference purposes we retained ``Visitor Design Pattern'' and ``Cascading-If + 
memoized\_cast'' timings from Figure~\ref{fig:DCastVis1} unchanged. Note that 
the Y-axis has been scaled-up 140 times, which is why the slope of 
``Cascading-If + memoized\_cast'' timings is so much steeper.

\begin{figure}[htbp]
  \centering
    \includegraphics[width=0.47\textwidth]{DCast-vs-Visitors2.png}
  \caption{Type switching based on the fast dynamic cast of Gibbs and Stroustrup~\cite{FastDynCast}}
  \label{fig:DCastVis2}
\end{figure}

As can be seen from the figure the use of our memoized\_cast implementation can 
get close in terms of performance to the fast dynamic cast, especially 
when combined with decision trees. An important difference that cannot be seen 
from the chart, however, is that the performance of memoized\_cast is 
asymptotic, while the performance of fast dynamic cast is guaranteed. This 
happens because the implementation of memoized\_cast will incur an overhead of 
a regular dynamic\_cast call on every first call with a given most derived type. 
Once that class is memoized, the performance will remain as shown. Averaged over 
all calls with a given type we can only claim we are asymptotically as good as 
fast dynamic cast.

Unfortunately fast dynamic casting is not truly open to fully satisfy our 
checklist. The structure of tags required by the scheme limits the number of 
classes it can handle. A 32-bit integer is estimated to be able to represent 7 
levels of a class hierarchy that forms a binary tree (255 classes), 6 levels of 
a similar ternary tree hierarchy (1093 classes) or just one level of a hierarchy 
with 9 base classes -- multiple inheritance is the worst case scenario of the 
scheme that quickly drains its allocation possibilities. Besides, similarly to 
other tag allocation schemes, presence of class extensions in \emph{Dynamically Linked Libraries} (DLLs) will likely 
require an integration effort to make sure different DLLs are not reusing prime 
numbers in a way that might result in an incorrect dynamic cast.

A number of other constant-time techniques for class-membership testing is 
surveyed by Gil and Zibin~\cite[\textsection 4]{PQEncoding}. They are intended 
for type testing, and thus will have to be combined with decision trees 
for type switching, resulting in similar to fast dynamic cast performance. 
They too assume access to the entire class hierarchy at compile time and thus 
are not open.

In view of the predictably-constant dispatching overhead of the visitor design pattern, 
it is clear that any open solution that will have a non-constant dispatching 
overhead will have a poor chance of being adopted. Multi-way switch on 
sequentially allocated tags~\cite{Spuler94} was one of the few techniques that 
could achieve constant overhead, and thus compete with and even outperform visitors. 
Unfortunately the scheme has problems of its own that make it unsuitable for 
truly open type-switching and here is why.

%To better understand the problem let us look at some existing solutions to type 
%switching that we found to be used in practice. 

%From our experience on this project we have noticed that we can only compete 
%with visitors when switch statements are implemented with a jump table. As soon 
%as compiler was putting even a single branch into the decision tree of cases, 
%the performance was degraded significantly. From this perspective we do not 
%regard solutions based on decision trees as efficient, since they do not let us 
%compete compete with the visitors solution.

The simple scheme of assigning a unique tag per variant (instantiatable class 
here) will not pass our first question because the tags of base and derived 
classes will have to be different if the base class can be instantiated on its 
own. In other words we will not be able to land on a case label of a base class, while 
having a derived tag only. The already mentioned partitioning of tags of derived 
classes based on the classes in case clauses also will not help as it assumes 
knowledge of all the classes and thus fails extensibility through DLLs.

In practical implementations hand crafted for a specific class hierarchy, tags 
often are not chosen arbitrarily, but to reflect the subtyping relation of the 
underlying hierarchy. Switching on base classes in such a setting will typically 
involve a call to some function $f$ that converts derived class' tag into a base 
class' tag. An example of such a scheme would be having a certain bit in the tag 
set for all the classes derived from a given base class. Unfortunately this 
solution creates more problems than it solves.

First of all the solution will not be able to recognize an exceptional case 
where most of the derived classes should be handled as a base class, while a few 
should be handled specifically. Applying the function $f$ puts several different 
types into an equivalence class with their base type, making them 
indistinguishable from each other.

Secondly, the assumed structure of tags is likely to make the set of tags 
sparse, effectively forcing the compiler to use a decision tree instead of a jump 
table to implement the switch. Even though conditional jump is reported to be 
faster than indirect jump on many computer architectures~\cite[\textsection 
4]{garrigue-98}, this did not seem to be the case in our experiments. Splitting 
of a jump table into two with a condition, that was sometimes happening because 
of our case label allocation scheme, was resulting in a noticeable degradation of 
performance in comparison to a single jump table.

Besides, as was seen in the scheme of Gibbs and Stroustrup, the assumed 
structure of tags can also significantly decrease the number of classes a given 
allocation scheme can handle. It is also interesting to note that even though 
their scheme can be easily adopted for type switching with decision trees, it is 
not easily adoptable for type switching with jump tables: in order to obtain 
tags of base classes we will have to decompose the derived tag into primes and 
then find all the dividers of the tag present in case clauses.

Several authors had noted the relationship between exception handling and type 
switching before~\cite{Glew99,ML2000}. Not surprisingly, the exception handling 
mechanism of \Cpp{} can be abused to implement the first-fit semantics of a type 
switch statement. The idea is to harness the fact that catch-handlers in \Cpp{} 
essentially use first-fit semantics to decide which one is going to handle a 
given exception. Unfortunately the approach is even slower than the use of 
\code{dynamic_cast} and we only list it here for comparison.

To summarize, truly open and efficient type switching is a non-trivial problem. 
The approaches we found in the literature were either open or efficient, 
but not both. Efficient implementation was typically achieved by sealing the 
class hierarchy and using a jump table on sequential tags. Open implementations 
were resorting to type testing and decision trees, which was not efficient. 
We are unaware of any efficient tag allocation scheme that can be used in a 
truly open scenario.

%%%%%%%%%%%%%%%%%%%%%%%%%%%%%%%%%%%%%%5555

%\noindent
%We chose to give it a first-fit semantics in our library as it was resembling 
%pattern matching facilities of other languages and was the most intuitive. The 
%following code can be generated to implement it:
%
%\begin{lstlisting}
%if (D1* derived = dynamic_cast<D1*>(base)) { s1; } else
%if (D2* derived = dynamic_cast<D2*>(base)) { s2; } else
%...
%if (Dn* derived = dynamic_cast<Dn*>(base)) { sn; }
%\end{lstlisting}

%\noindent
%Note that leaving \code{else} out will effectively turn it into an all-fit 
%statement with enabled statements executed in lexicographical order.
%
%The above code is easy to understand but is extremely inefficient as for an 
%object of dynamic type $D_i$ we will have to perform $i-1$ dynamic casts that 
%fail first. The diagram below compares the times spent by visitors and the above 
%type switch statement to uncover the $i^{th}$ case. We postpone the discussion 
%of \code{memoized_cast} until section \textsection\ref{}, here we would only 
%like to notice that even though faster than the actual dynamic cast it also bears 
%a linear coefficient, not present in visitors.

\section{Solution for Polymorphic Classes}
\label{sec:copc}

Our handling of type switches for polymorphic and tagged encodings differs 
with each having its pros and cons described in details in \textsection\ref{sec:cmp}.
In this section we will concentrate on the truly open type switch for 
polymorphic encoding. The type switch for tagged encoding (\textsection\ref{sec:cotc}) 
is simpler and more efficient, however, making it open will eradicate its 
performance advantages. The difference in performance is the price we pay for 
keeping the solution open.  The core of the proposal relies on two key
aspects of C++ implementations:
\begin{enumerate}
\item a constant-time access to the virtual table pointer embedded in an object of
  dynamic class type;
\item injectivity of the relation between an object's inheritance path
  and the virtual table pointer extracted from that object.
\end{enumerate}

\subsection{A Memoization Device}
\label{sec:memdev}

Let us look at a slightly more general problem than type switching. Consider a 
generalization of the switch statement that takes predicates on a subject as its 
clauses and executes the first statement $s_i$ whose predicate is enabled: 

\begin{lstlisting}[keepspaces]
switch (x)
{
    case P1(x): s1;
    ...
    case Pn(x): sn;
}
\end{lstlisting}

\noindent
Assuming that predicates are \emph{functional} (i.e. do not involve any side 
effects), the next time we execute the switch with the same value $x$, the same 
predicate will be enabled first. We thus would like to avoid evaluating 
preceding predicates and jump to the statement it guards. In a way, we 
would like the switch to memoize the case enabled for a given $x$. 

The idea is to generate a simple cascading-if statement interleaved with jump 
targets and instructions that associate the original value with enabled target. 
The code before the statement looks up whether the association for a given value 
has already been established, and, if so, jumps directly to the target; otherwise, 
the sequential execution of the cascading-if is started. To ensure 
that the actual code associated with the predicates remains unaware of this 
optimization, the code preceding it after the target must re-establish any 
invariant guaranteed by sequential execution (\textsection\ref{sec:vtblmem}).

Described code can be easily produced in a compiler setting, but generating it in 
a library is a challenge. Inspired by Duff's Device~\cite{Duff}, 
we devised a construct that we call \emph{Memoization Device} doing just 
that in standard \Cpp{}:

\begin{lstlisting}
typedef decltype(x) T; // T is the type of subject x
static std::unordered_map<T,size_t> jump_targets;

switch (size_t& jump_to = jump_targets[x]) {
default: // entered when we have not seen x yet
    if (P1(x)) { jump_to = 1; case 1: s1; } else 
    if (P2(x)) { jump_to = 2; case 2: s2; } else
 ...
    if (Pn(x)) { jump_to = @$n$@; case @$n$@: sn; } else
                jump_to = @$n+1$@;
case @$n+1$@: // none of the predicates is true on x
}
\end{lstlisting}

\noindent
The static \code{jump_targets} hash table will be allocated upon first entry 
to the function. The map is initially empty and according to its logic, 
request for a key $x$ not yet in the map will allocate a 
new entry with its associated data default initialized (to 0 for \code{size_t}). Since 
there is no case label 0 in the switch, the default case will be taken, which, in 
turn, will initiate sequential execution of the interleaved cascading-if 
statement. Assignments to \code{jump_to} effectively establish association 
between value $x$ and corresponding predicate, since \code{jump_to} is a 
reference to \code{jump_targets[x]}. The last assignment records absence of 
enabled predicates for $x$.

The sequential execution of the cascading-if statement will keep checking 
predicates $P_j(x)$ until the first predicate $P_i(x)$ that returns true. By 
assigning $i$ to \code{target} we will effectively associate $i$ with $x$ since 
\code{target} is just a reference to \code{jump_target_map[x]}. This association 
will make sure that the next time we are called with the value $x$ we will jump 
directly to the label $i$. When none of the predicates returns true, we will 
record it by associating $x$ with $N+1$, so that the next time we can jump 
directly to the end of the switch on $x$. 

The above construct effectively gives the entire statement first-fit semantics. 
In order to evaluate all the statements whose predicates are true, and thus 
give the construct all-fit semantics, we might want to be able to preserve the 
fall-through behavior of the switch. In this case we can still skip the initial 
predicates returning false and start from the first successful one. This can be 
easily achieved by removing all else statements and making if-statements 
independent as well as wrapping all assignments to \code{target} with a condition, 
to make sure only the first successful predicate executes it:

\begin{lstlisting}
if (Pi(x)) { if (jump_to == 0) jump_to = @$i$@; case @$i$@: si; }
\end{lstlisting}

\noindent
Note that the protocol that has to be maintained by this structure does not 
depend on the actual values of case labels. We only require them to be 
different and include a predefined default value. The default clause can be 
replaced with a case clause for the predefined value, but keeping the default  
clause generates faster code. A more important consideration is to 
keep the values close to each other. Not following this rule might result in a 
compiler choosing a decision tree over a jump table implementation of the 
switch, which in our experience significantly degrades the performance.

The first-fit semantics is not an inherent property of the memoization device. 
Assuming that the conditions are either mutually exclusive or imply one another, we 
can build a decision-tree-based memoization device that will effectively have 
\emph{most-specific} semantics -- an analog of best-fit semantics in predicate 
dispatching~\cite{ErnstKC98}.

Imagine that the predicates with the numbers $2i$ and $2i+1$ are mutually exclusive and 
each imply the value of the predicate with number $i$, i.e.
$\forall i\forall x\in\bigcap_j\mathsf{Domain}(P_j).P_{2i+1}(x)\rightarrow P_i(x)\wedge P_{2i}(x)\rightarrow P_i(x)\wedge\neg(P_{2i+1}(x)\wedge P_{2i}(x))$ holds. 
Examples of such predicates are class membership tests where the truth of 
testing membership in a derived class implies the truth of testing membership in 
its base class.

The following decision-tree-based memoization device will execute the statement 
$s_i$ associated with the \emph{most-specific} predicate $P_i$ (i.e. the 
predicate that implies all other predicates true on $x$) that evaluates to true 
or will skip the entire statement if none of the predicates is true on $x$.

\begin{lstlisting}
switch (size_t& jump_to = jump_targets[x]) {
default:
    if (P1(x)) {
        if (P2(x)) {
            if (P4(x)) { jump_to = 4; case 4: s4; } else
            if (P5(x)) { jump_to = 5; case 5: s5; } 
            jump_to = 2; case 2: s2;
        } else
        if (P3(x)) {
            if (P6(x)) { jump_to = 6; case 6: s6; } else
            if (P7(x)) { jump_to = 7; case 7: s7; } 
            jump_to = 3; case 3: s3;
        }
        jump_to = 1; case 1: s1;
    } else { jump_to = 0; case 0: ; }
}
\end{lstlisting}

\noindent
An example of predicates that satisfy this condition are class membership tests
where the truth of a predicate that tests membership in a derived class implies 
the truth of a predicate that tests membership in its base class. 
Our library solution prefers the simpler cascading-if approach only because the 
necessary code structure can be laid out with macros. A compiler solution 
will use the decision-tree approach whenever possible to lower the cost of the 
first match from linear in case's number to logarithmic as seen in Figure\ref{fig:DCastVis1}.

When the predicates do not satisfy the implication or mutual exclusion properties 
mentioned above, a compiler of a language based on predicate dispatching would 
typically issue an ambiguity error. Some languages might choose to resolve it 
according to lexical or some other ordering. In any case, the presence of 
ambiguities or their resolution has nothing to do with memoization device 
itself. The latter only helps optimize the execution once a particular choice of 
semantics has been made and code implementing it has been laid out.

The main advantage of the memoization device is that it can be built around 
almost any code, providing that we can re-establish the invariants guaranteed 
by sequential execution. Its main disadvantage is the size of the hash table 
that grows proportionally to the number of different values seen. Fortunately, 
the values can often be grouped into equivalence classes that do not change the 
outcome of the predicates. The map can then associate the equivalence class of a 
value with a target instead of associating the value with it. 

In application to type switching, the idea is to use the memoization device to 
learn the outcomes of type inclusion tests (with \code{dynamic_cast} used as a 
predicate). The objects can be grouped into equivalence classes based on their 
dynamic type: the outcome of each type inclusion test will be the same on 
all the objects of the same dynamic type. We can use the  
address of a class' \code{type_info} object obtained in constant time with the
\code{typeid()} operator as a unique identifier of each dynamic type. 
Presence of multiple \code{type_info} objects for the same class, as is often 
the case when dynamic linking is involved, is not a problem, as it would 
effectively split a single equivalence class into multiple ones. 

This could have been a solution if we were only interested in class membership. 
More often than not, however, we will be interested in obtaining a reference to 
the target type of the subject, and we saw in \textsection\ref{sec:specifics} 
that the cast between the source and target subobjects depends on 
the position of the source subobject in the dynamic type's subobject graph. 
We thus would like to have different equivalence classes for different 
subobjects, but there seems to be no easy way of identifying them given just an object descriptor.

\subsection{Virtual Table Pointers}
\label{sec:vtp}

%In this section we show that under certain conditions the compiler cannot share 
%the same virtual tables between different classes or their subobjects. This 
%allows us to use virtual table pointers to \emph{uniquely} identify the 
%subobjects within the most-derived class.

Before we discuss our solution we would like to talk about certain properties of 
the C++ run-time system that we rely on. In particular,
we show that under certain conditions the compiler cannot share 
the same virtual tables between different classes or subobjects of the same 
class. This allows us to use virtual table pointers to \emph{uniquely} identify 
the subobjects within the most derived class.

Strictly speaking, the C++ standard~\cite{C++0x} does not require implementations 
to use any specific technique (e.g. virtual tables) to implement virtual functions, 
however interoperability requirements have forced many compiler vendors to design a 
set of rules called Common Vendor Application Binary Interface (the C++ 
ABI)~\cite{C++ABI}. Most C++ compilers today follow these rules, with the 
notable exception of Microsoft Visual C++. The technique presented here will 
work with any C++ compiler that follows the C++ ABI. Microsoft's own ABI is not 
publically available and thus we cannot formally verify that it satisfies 
our requirements. Nevertheless, we did run numerous experiments with various 
class hierarchies and have sufficient confidence that our approach can be used 
in Visual C++. This is why we include experimental results for this compiler as 
well.

Besides single inheritance, which is supported by most object-oriented languages, 
C++ supports multiple-inheritance of two kinds: repeated and virtual (shared). 
\emph{Repeated inheritance} creates multiple independent subobjects of the same 
type within the most derived type. \emph{Virtual inheritance} creates only one 
shared subobject, regardless of the inheritance paths. Because of this 
peculiarity of the C++ type system it is not sufficient to talk only about the 
static and dynamic types of an object -- one has to talk about a 
\emph{subobject} of a certain static type accessible through a given inheritance 
path within a dynamic type.

\begin{figure}[tbp]
  \centering
    \includegraphics[width=0.47\textwidth]{Hierarchies.png}
  \caption{Single inheritance, repeated multiple inheritance and virtual multiple inheritance}
  \label{fig:hierarchy}
\end{figure}

\noindent
Note that the above picture portrais subobject relatedion, not the inheritance.

Figure~\ref{fig:objlayout} shows a typical object layout generated by a \Cpp{} 
compiler for class \code{D} from Figure~\ref{fig:inheritance}(1) under repeated 
(1) and virtual (2) inheritance of \code{A}. The layouts represent an encoding 
of the corresponding subobject graphs from Figures \ref{fig:inheritance}(2a) and 
\ref{fig:inheritance}(2b) respectively.

\begin{figure}[htbp]
  \centering
    \includegraphics[width=0.47\textwidth]{obj-layout.pdf}
  \caption{Object Layout under Multiple Inheritance}
  \label{fig:objlayout}
\end{figure}

Due to the extensibility of classes, the layout decisions for classes must be 
made independently of their derived classes -- a property of the \Cpp{} object 
model that we will refer to as \emph{layout independence}. In turn, the layout of derived   
classes must conform to the layout of their base classes relatively to the offset 
of the base class within the derived one. For example, the layout of \code{A} in 
\code{C} is exactly the same as the layout of \code{A} in \code{B} and is simply
the layout of \code{A}. Base classes inherited virtually do not contribute to 
the fixed layout because they are looked up indirectly at run-time; however, 
they are not exempt from layout independence, since their lookup rules are 
agnostic of the concrete dynamic type.
%Because of this indirection, the use of virtual inheritance incures slight 
%overhead at run-time. 

Under non-virtual inheritance, members of the base class are typically laid out 
before the members of derived class, resulting in the base class being at the 
same offset as the derived class itself. In our example, the offset of \code{A} 
in \code{B} under regular (non-virtual) inheritance of \code{A} is 0.
Under multiple inheritance, different base classes might be at different offsets 
in the derived class, which is why pointers of a given static type may be 
pointing only to certain subobjects in it. These positions are marked in the 
picture with vertical arrows decorated with the set of pointer types whose 
values may point into that position. Run-time conversions between such pointers 
represent casts between subobjects of the same dynamic type and may require 
adjustments to this-pointer (shown with dashed arrows) for type safety.

A class that declares or inherits a virtual function is called a 
\emph{polymorphic class}. The \Cpp{} standard~\cite{C++11} does not prescribe any 
specific implementation technique for virtual function dispatch.
However, in practice, all \Cpp{} compilers use a strategy based on so-called
virtual function tables (or vtables for short) for efficient dispatch. 
The vtable is part of the reification of a polymorphic class type.  
\Cpp{} compilers embed a pointer to a vtable (vtbl-pointer for short) in every object of
polymorphic class type (and thus every subobject of that type inside other 
classes due to layout independence). CFront, the first \Cpp{} compiler, puts the 
vtbl-pointer 
at the end of an object. The so-called ``common vendor \Cpp{} ABI''~\cite{C++ABI} requires the 
vtbl-pointer to be at offset 0 of an object. ~\footnote{The following compilers 
are known to comply with the \Cpp{} ABI: GCC (3.x and up); Clang and llvm-g++; 
Linux versions of Intel and HP compilers, and compilers from ARM. See 
http://morpher.com/documentation/articles/abi/ for details.}. 
We do not have 
access to the unpublished Microsoft ABI, but we have experimental evidence that 
their \Cpp{} compiler also puts the vtbl-pointer at the start of an object.

While the exact offset of the vtbl-pointer within the (sub)object is not important 
for this discussion, because of layout independence every (sub)object of a 
polymorphic type \code{S} will have a vtbl-pointer at a predefined offset. 
Such offset may be different for different static types \code{S}, in which case 
the compiler will know at which offset in type \code{S} the vtbl-pointer is 
located, but it will be the same within any subobject of a static type 
\code{S}. For a library implementation we assume the presence of a function 
\code{template <typename S> intptr_t vtbl(const S* s);} 
that returns the address of the virtual table corresponding to the subobject 
pointed to by \code{s}. Such a function can be trivially implemented for the 
common vendor \Cpp{} ABI, where the vtbl-pointer is always at offset 0:

\begin{lstlisting}
template <typename S> std::intptr_t vtbl(const S* s) {
    static_assert(std::is_polymorphic<S>::value, "error");
    return *reinterpret_cast<const std::intptr_t*>(s);
}
\end{lstlisting}

\noindent
Each of the \code{vtbl} fields shown in Figure~\ref{fig:objlayout} holds a 
vtbl-pointer referencing a group of virtual methods known in the object's static 
type. Figure~\ref{fig:vtbl}(1) shows a typical layout of virtual function tables 
together with objects it points to for classes \code{B} and \code{D}.

\noindent
\begin{figure}[htbp]
  \centering
    \includegraphics[width=0.49\textwidth]{v-table.pdf}
  \caption{VTable layout with and without RTTI}
  \label{fig:vtbl}
\end{figure}

Entries in the vtable to the right of the address pointed to by a vtbl-pointer 
represent pointers to functions, while entries to the left of it represent 
various additional fields like a pointer to a class' type information, offset to 
top, offsets to virtual base classes, etc. In many implementations, this-pointer 
adjustments required to dispatch properly the call were stored in the vtable 
along with function pointers. Today most implementations prefer to use 
\emph{thunks} or \emph{trampolines} -- additional entry points to a function, 
that adjust this-pointer before transferring the control to the function, -- 
which was shown to be more efficient~\cite{Driesen96}. Thunks in general may 
only be needed when virtual function is overridden. In such cases, the 
overridden function may be called via a pointer to a base class or a pointer to 
a derived class, which may not be at the same offset in the actual object.

The intuition behind our proposal is to use the values of vtbl-pointers stored 
inside the object to uniquely identify the subobject in it. There are several 
problems with the approach, however. First, the same vtbl-pointer is 
usually shared by multiple subobjects when one of them contains the other. For 
example, the first vtbl-pointer in Figure~\ref{fig:objlayout}(1) will be shared 
by objects of static type \code{Z*}, \code{A*}, \code{B*} and \code{D*}. This is 
not a problem for our purpose, because the subobjects of these types will be at 
the same offset in the object. Secondly, and more importantly, 
however, there are legitimate optimizations that let the compiler share the same 
vtable among multiple subobjects of often-unrelated types.

Generation of the \emph{Run-Time Type Information} (or RTTI for short) can 
typically be disabled with a compiler switch and the Figure~\ref{fig:vtbl}(2) 
shows the same vtable layouts once RTTI has been disabled. Since neither 
\code{baz} nor \code{foo} were overridden, the prefix of the vtable for the 
\code{C} subobject in \code{D} is exactly the same as the vtable for its 
\code{B} subobject, the \code{A} subobject of \code{C}, or the entire vtable of 
\code{A} and \code{B} classes. Such a layout, for example, is produced by 
Microsoft Visual \Cpp{} 11 when the command-line option \code{/GR-} is specified. 
The Visual \Cpp{} compiler has been known to unify code identical on binary level, 
which in some cases may result in sharing of the same vtable between unrelated 
classes (e.g. when virtual functions are empty).

%\Cpp{} supports multiple-inheritance of two kinds: repeated and virtual (shared). 
%\emph{Repeated inheritance} creates multiple independent subobjects of the same 
%type within the dynamic type. \emph{Virtual inheritance} creates only one 
%shared subobject, regardless of the inheritance paths. Consequently,
%it is not sufficient to talk only about the 
%static and dynamic types of an object -- one has to talk about a 
%\emph{subobject} of a certain static type accessible through a given inheritance 
%path within a dynamic type. 

We now would like to show more formally that in the presence of RTTI, a common vendor \Cpp{} ABI 
compliant implementation would always have all the vtbl-pointers different. To do 
so, we need a closer look at the notion of subobject, which has been formalized 
before~\cite{RF95,WNST06,RDL11}. We follow here the presentation of Ramamanandro 
et al~\cite{RDL11}.

\subsection{Subobjects}
\label{sec:subobj}

We assume a program $\mathfrak{P}$ is represented by its class table, which can be 
queried for inheritance relations between classes. All subsequent definitions 
are implicitly parameterized over a given program $\mathfrak{P}$. 
A class $B$ is a \emph{direct repeated base class} of  
$D$ if $B$ is mentioned in the list of base classes of $D$ without the 
\code{virtual} keyword ($D \prec_R B$). Similarly, a class $B$ is a \emph{direct 
shared base class} of $D$ if $B$ is mentioned in the list of base classes of $D$ 
with the \code{virtual} keyword ($D \prec_S B$). A reflexive transitive closure 
of these relationships $\preceq^*=(\prec_R \cup \prec_S)^*$ defines the 
\emph{subtyping} relation on types of program $\mathfrak{P}$.
A base class \emph{subobject} of a given \emph{complete object} is represented by a pair 
$\sigma = (h,l)$ with $h \in \{\mathsf{Repeated},\mathsf{Shared}\}$ representing the 
kind of inheritance (single inheritance is $\mathsf{Repeated}$ with one base class) and $l$ 
representing the path in a non-virtual inheritance graph.
A judgment of the form $\mathfrak{P}\vdash C\leftY\sigma\rightY A$ states that 
in a program $\mathfrak{P}$, $\sigma$ designates a subobject of static type $A$ 
within an object of type $C$. Omitting the context $\mathfrak{P}$: 

A predicate $C\leftY\sigma\rightY A$ they introduce means that $\sigma$ 
designates a subobject of static type $A$ within the most derived object of 
type $C$.

\begin{mathpar}
\inferrule
{C \prec_S B \\ B\leftY(h,l)\rightY A}
{C\leftY(\mathsf{Shared},l)\rightY A}

\inferrule
{}
{C\leftY(\mathsf{Repeated},C::\epsilon)\rightY C}

\inferrule
{C \prec_R B \\ B\leftY(\mathsf{Repeated},l)\rightY A}
{C\leftY(\mathsf{Repeated},C::l)\rightY A}
\end{mathpar}

\noindent
$\epsilon$ indicates an empty path, but we will generally omit it in writing 
when understood from the context. In the case of repeated inheritance in 
Figure~\ref{fig:inheritance}(1), an object of the dynamic class \code{D} 
will have the following $\mathsf{Repeated}$ subobjects:
\code{D::C::Y}, 
\code{D::B::A::Z}, 
\code{D::C::A::Z}, 
\code{D::B::A}, 
\code{D::C::A}, 
\code{D::B}, 
\code{D::C}, 
\code{D}.
Similarly, in case of virtual inheritance in the same example, an object of the 
dynamic class \code{D} will have the following $\mathsf{Repeated}$ subobjects:
\code{D::C::Y}, 
\code{D::B}, 
\code{D::C}, 
\code{D}
as well as the following $\mathsf{Shared}$ subobjects: 
\code{D::A::Z}, 
\code{D::Z}, 
\code{D::A}. See Figure~\ref{fig:inheritance} for illustration.

It is easy to show by structural induction on the above definition, that 
$C\leftY\sigma\rightY A \implies \sigma=(h,C::l_1) \wedge \sigma=(h,l_2::A::\epsilon)$, 
which simply means that any path to a subobject of static type $A$ within the 
object of dynamic type $C$ starts with $C$ and ends with $A$. This 
observation shows that $\sigma_\bot = (\mathsf{Shared},\epsilon)$ does not 
represent a valid subobject. If $\Sigma_\mathfrak{P}$ is the domain of all subobjects in 
the program $\mathfrak{P}$ extended with $\sigma_\bot$, then a \emph{cast} operation can be 
understood as a function $\delta : \Sigma_\mathfrak{P} \rightarrow \Sigma_\mathfrak{P}$. We use 
$\sigma_\bot$ to indicate an impossibility of a cast. The fact that $\delta$ is 
defined on subobjects as opposed to actual run-time values reflects the 
non-coercive nature of the operation, i.e. the underlying value remains the 
same. Any implementation of such a function must thus satisfy the following 
condition:
\begin{eqnarray*}
C \leftY\sigma_1\rightY A \wedge \delta(\sigma_1) = \sigma_2 \implies C \leftY\sigma_2\rightY B
\end{eqnarray*}
\noindent
i.e. the dynamic type of the value does not change during casting, only the way 
we reference it does. Following the definitions from 
\textsection\ref{sec:specifics}, $A$ is the \emph{source type} and $\sigma_1$ is 
the \emph{source subobject} of the cast, while $B$ is the \emph{target type} and 
$\sigma_2$ is the \emph{target subobject} of it. The type $C$ is the 
dynamic type of the value being casted. The \Cpp{} semantics states more 
requirements to the implementation of $\delta$: e.g. 
$\delta(\sigma_\bot) = \sigma_\bot$ etc. but their precise modeling is out of 
scope of this discussion. We would only like to point out here that since 
the result of the cast does not depend on the actual value and only on the 
source subobject and the target type, we can memoize the outcome of a cast on 
one instance in order to apply its results to another.

%Figure~\ref{fig:objlayout}(1)
%$Z\leftY(\mathsf{Repeated},      [Z])\rightY Z$,
%$A\leftY(\mathsf{Repeated},    [A,Z])\rightY Z$,
%$B\leftY(\mathsf{Repeated},  [B,A,Z])\rightY Z$,
%$D\leftY(\mathsf{Repeated},[D,B,A,Z])\rightY Z$,
%$C\leftY(\mathsf{Repeated},  [C,A,Z])\rightY Z$,
%$D\leftY(\mathsf{Repeated},[D,C,A,Z])\rightY Z$,
%$Y\leftY(\mathsf{Repeated},      [Y])\rightY Y$,  
%$C\leftY(\mathsf{Repeated},    [C,Y])\rightY Y$,
%$D\leftY(\mathsf{Repeated},  [D,C,Y])\rightY Y$,
%$A\leftY(\mathsf{Repeated},      [A])\rightY A$, 
%$B\leftY(\mathsf{Repeated},    [B,A])\rightY A$,
%$D\leftY(\mathsf{Repeated},  [D,B,A])\rightY A$,
%$C\leftY(\mathsf{Repeated},    [C,A])\rightY A$,
%$D\leftY(\mathsf{Repeated},  [D,C,A])\rightY A$,
%$B\leftY(\mathsf{Repeated},      [B])\rightY B$,
%$D\leftY(\mathsf{Repeated},    [D,B])\rightY B$,
%$C\leftY(\mathsf{Repeated},      [C])\rightY C$,
%$D\leftY(\mathsf{Repeated},    [D,C])\rightY C$,
%$D\leftY(\mathsf{Repeated},      [D])\rightY D$,
%
%Figure~\ref{fig:objlayout}(2)
%$Z\leftY(\mathsf{Repeated},      [Z])\rightY Z$,
%$A\leftY(\mathsf{Repeated},    [A,Z])\rightY Z$,
%$B\leftY(\mathsf{Shared},    [B,A,Z])\rightY Z$,
%$C\leftY(\mathsf{Shared},    [C,A,Z])\rightY Z$,
%$D\leftY(\mathsf{Shared},    [D,A,Z])\rightY Z$,
%$D\leftY(\mathsf{Shared},      [D,Z])\rightY Z$,
%$Y\leftY(\mathsf{Repeated},      [Y])\rightY Y$,  
%$C\leftY(\mathsf{Repeated},    [C,Y])\rightY Y$,
%$D\leftY(\mathsf{Repeated},  [D,C,Y])\rightY Y$,
%$A\leftY(\mathsf{Repeated},      [A])\rightY A$, 
%$B\leftY(\mathsf{Shared},      [B,A])\rightY A$,
%$C\leftY(\mathsf{Shared},      [C,A])\rightY A$,
%$D\leftY(\mathsf{Shared},      [D,A])\rightY A$,
%$B\leftY(\mathsf{Repeated},      [B])\rightY B$,
%$D\leftY(\mathsf{Repeated},    [D,B])\rightY B$,
%$C\leftY(\mathsf{Repeated},      [C])\rightY C$,
%$D\leftY(\mathsf{Repeated},    [D,C])\rightY C$,
%$D\leftY(\mathsf{Repeated},      [D])\rightY D$,

\subsection{Uniqueness of vtbl-pointers under common ABI}
\label{sec:uniq}

A class that declares or inherits a virtual function is called a 
\emph{polymorphic class}~\cite[\textsection 10.3]{C++0x}. The C++ ABI in turn defines 
\emph{dynamic class} to be a class requiring a virtual table pointer (because it 
or its bases have one or more virtual member functions or virtual base classes). 
A polymorphic class is thus a dynamic class by definition.

A \emph{virtual table pointer} (vtbl-pointer) is a member of object's layout 
pointing to a virtual table. A \emph{virtual table} is a table of information used 
to dispatch virtual functions, access virtual base class subobjects, and to 
access information for \emph{RunTime Type Identification} (RTTI). Because of repeated
inheritance, an object of given type may have several vtbl-pointers in it. Each 
such pointer corresponds to one of the polymorphic base classes. Given an object 
$a$ of static type $A$ that has $k$ vtbl-pointers in it, we will use the same 
notation we use for regular fields to refer them: $a.\textit{vtbl}_i$.

A \emph{primary base class} for a dynamic class is the unique base class (if any) 
with which it shares the virtual table pointer at offset 0. The data layout 
procedure for non-POD types described in \textsection2.4 of the C++ ABI~\cite{C++ABI} 
requires dynamic classes either to allocate vtable pointer at offset 0 or share 
the virtual table pointer from its primary base class, which is by definition at 
offset 0. For our purpose this means that we can rely on a virtual table pointer 
always being present at offset 0 for all dynamic classes, and thus for all polymorphic 
classes.

\begin{lemma}
In an object layout that adheres to the C++ ABI, a polymorphic class always has a 
virtual table pointer at offset 0.
\label{lem:vtbl}
\end{lemma}

\noindent
Knowing how to extract a vtbl-pointer as well as that all the objects of the 
same most derived type share the same vtbl-pointers, the idea is to use their 
values to uniquely identify the type and subobject within it. Unfortunately 
nothing in the C++ ABI states these pointers should be unique. A popular 
optimization technique lets the compiler share the virtual table of a derived 
class with its primary base class as long as the derived class that does not 
override any virtual methods. Use of such optimization will violate the 
uniqueness of vtbl-pointers; however, we show below that in the presense of 
RTTI, a C++ ABI-compliant implementation is guaranteed to have different values 
of vtbl-pointers in different subobjects.

%C++ standard requires an argument of \code{dynamic_cast} to be a pointer to or 
%an lvalue of a polymorphic type when performing \emph{downcast} -- a cast from 
%base to derived~\cite[\textsection 5.2.7-6]{C++0x}. We can thus always safely 
%extract virtual table pointer from offset 0 of any valid argument to 
%\code{dynamic_cast}.

%Similarly, each class that has virtual member functions or virtual bases has an 
%associated set of virtual tables. There may be multiple virtual tables for a 
%particular class, if it is used as a base class for other classes. However, the 
%virtual table pointers within all the objects (instances) of a particular 
%most-derived class point to the same set of virtual tables.

The exact content of the virtual table is not important for our discussion, but 
we would like to point out a few fields in it. The following definitions are 
copied verbatim from the C++ ABI~\cite[\textsection 2.5.2]{C++ABI}:

\begin{itemize}
\setlength{\itemsep}{0pt}
\setlength{\parskip}{0pt}
\item The \emph{typeinfo pointer} points to the typeinfo object used for RTTI. 
      It is always present.  
\item The \emph{offset to top} holds the displacement to the top of the object 
      from the location within the object of the virtual table pointer that 
      addresses this virtual table, as a \code{ptrdiff_t}. It is always present.
\item \emph{Virtual Base (vbase) offsets} are used to access the virtual bases 
      of an object. Such an entry is added to the derived class object address 
      (i.e. the address of its virtual table pointer) to get the address of a 
      virtual base class subobject. Such an entry is required for each virtual 
      base class.
\end{itemize}

\noindent
Given a virtual table pointer \code{vtbl}, we will refer to these fields as 
\code{rtti(vtbl)}, \code{off2top(vtbl)} and \code{vbase(vtbl)} respectively. 
We will also assume presence of a function $\mathit{offset}(\sigma)$ that defines the 
offset of the base class identified by the end of the path $\sigma$ within a 
class identified by its first element.

\begin{theorem}
In an object layout that adheres to the C++ ABI with present runtime type 
information, the equality of virtual table pointers of two objects of the same 
static type implies that they both belong to subobjects with the same 
inheritance path in the same most-derived type.
\begin{eqnarray*}
    \forall a_1, a_2 : A\ |\ a_1\in C_1\leftY\sigma_1\rightY A \wedge a_2\in C_2\leftY\sigma_2\rightY A \\
    a_1.\textit{vtbl}_i = a_2.\textit{vtbl}_i \Rightarrow C_1 = C_2 \wedge \sigma_1 = \sigma_2
\end{eqnarray*}
\label{thm:vtbl}
\end{theorem}
\begin{proof}
Let us assume first $a_1.\textit{vtbl}_i = a_2.\textit{vtbl}_i$ but $C_1 \neq C_2$. In this case we 
have \code{rtti}$(a_1.\textit{vtbl}_i) = $\code{rtti}$(a_2.\textit{vtbl}_i)$. By definition 
\code{rtti}$(a_1.\textit{vtbl}_i) = C_1$ while \code{rtti}$(a_2.\textit{vtbl}_i) = C_2$, which 
contradicts that $C_1 \neq C_2$. Thus $C_1 = C_2 = C$.

Let us assume now that $a_1.\textit{vtbl}_i = a_2.\textit{vtbl}_i$ but $\sigma_1 \neq \sigma_2$. 
Let $\sigma_i=\langle h_i,l_i\rangle,i=1,2$ 

If $h_1 \neq h_2$ then one of them refers to a virtual base while the other to a 
repeated one. Assuming $h_1$ refers to a virtual path, \code{vbase}$(a_1.\textit{vtbl}_i)$ 
has to be defined inside the vtable according to the ABI, while 
\code{vbase}$(a_2.\textit{vtbl}_i)$ -- should not. This would contradict again that both 
$vtbl_i$ refer to the same virtual table.

We thus have $h_1 = h_2 = h$. If $h = \mathrm{Shared}$ then there is only one path to 
such $A$ in $C$, which would contradict $\sigma_1 \neq \sigma_2$. 
If $h = \mathrm{Repeated}$ then we must have that $l_1 \neq l_2$. In this case let $k$ be 
the first position in which they differ: 
$l_1^j=l_2^j \forall j<k \wedge l_1^k\neq l_2^k$. Since our class $A$ is a base 
class for classes $l_1^k$ and $l_2^k$, both of which are in turn base classes of 
$C$, the object identity requirement of C++ requires that the relevant subobjects 
of type $A$ have different offsets within class $C$: 
$\mathit{offset}(\sigma_1)\neq \mathit{offset}(\sigma_2)$ However 
$\mathit{offset}(\sigma_1)=$\code{off2top}$(a_1.\textit{vtbl}_i)=$\code{off2top}$(a_2.\textit{vtbl}_i)=\mathit{offset}(\sigma_2)$ 
since $a_1.\textit{vtbl}_i = a_2.\textit{vtbl}_i$, which contradicts that the offsets are different.
\end{proof}

\noindent
Conjecture in the other direction is not true in general as there may be 
duplicate virtual tables for the same type present at run-time. This happens in 
many C++ implementations in the presence of DLLs as the same class compiled into 
executable and into a DLL it loads may have identical virtual tables inside the 
executable's and DLL's binaries.

Note also that we require both static types to be the same. Dropping this 
requirement and saying that equality of vtbl-pointers also implies equality of 
the static types is not true in general because a derived class will share the 
vtbl-pointer with its primary base class (see Lemma~\ref{lem:vtbl}). The theorem 
can be reformulated, however, stating that one static type will necessarily have 
to be a subtype of the other. The current formulation is sufficient for our 
purposes, while reformulation would have required more elaborate discussion of 
the algebra of subobjects~\cite{RDL11}, which we touch only briefly.

\begin{corollary}
Results of \code{dynamic_cast} can be reapplied to a different instance from 
within the same subobject. 

$\forall A,B \forall a_1, a_2 : A\ |\ a_1.\textit{vtbl}_i = a_2.\textit{vtbl}_i \Rightarrow$ \\
\code{dynamic_cast<B>}$(a_1).\textit{vtbl}_j = $\code{dynamic_cast<B>}$(a_2).\textit{vtbl}_j \vee$ \\
\code{dynamic_cast<B>}$(a_1)$ throws $\wedge$ \code{dynamic_cast<B>}$(a_2)$ throws.
\label{crl:vtbl}
\end{corollary}

\noindent
During construction and deconstruction of 
an object, the value of a given vtbl-pointer may change. In particular, 
that value will reflect the fact that the dynamic type of the object is the type of its 
fully constructed part only. This does not affect our reasoning, as during 
such transition we also treat the object to have the type of its fully 
constructed base only. Such interpretation is in line with the \Cpp{} semantics for 
virtual function calls and the use of RTTI during construction and destruction of an 
object. Once the complete object is fully constructed, the value of the 
vtbl-pointer will remain the same for the lifetime of the object.

\subsection{Vtable Pointer Memoization}
\label{sec:vtblmem}

The memoization device can almost immediately be used for multi-way type testing by 
using \code{dynamic_cast<Ti>} as a predicate $P_i$. This cannot be considered a 
type switching solution, however, as one would expect to also have a reference 
to the uncovered type. Using a \code{static_cast<Ti>} upon successful type test 
would have been a solution if we did not have multiple inheritance. It certainly 
can be used as such in languages with only single inheritance. For the fully 
functional \Cpp{} solution, we combine the memoization device with the properties 
of virtual table pointers into a \emph{Vtable Pointer Memoization} technique.

The \Cpp{} standard implies that information about types is available at run time 
for three distinct purposes~\cite[\textsection 2.9.1]{C++ABI}:

\begin{itemize}
\setlength{\itemsep}{0pt}
\setlength{\parskip}{0pt}
\item to support the \code{typeid} operator,
\item to match an exception handler with a thrown object, and
\item to implement the \code{dynamic_cast} operator.
\end{itemize}

\noindent
and if any of these facilities are used in a program that was compiled with 
RTTI disabled, the compiler shall emit a warning. Some 
compilers (e.g. Visual \Cpp{}) additionally let a library check presence of RTTI 
through a predefined macro, thus letting it report an error if its dependence on 
RTTI cannot be satisfied. Since our solution depends on \code{dynamic_cast}% to perform casts at run-time
, according to the third requirement we implicitly rely on the presence of RTTI and thus 
fall into the setting that guarantees the preconditions of Theorem~\ref{thm:vtbl}.
Besides, all the objects that will be coming through a particular type switch will 
have the same static type, and thus the theorem guarantees that different vtbl-pointers 
will correspond to different subobjects. The idea is thus to group them 
according to the value of their vtbl-pointer and associate both jump target 
and the required offset through the memoization device:

\begin{lstlisting}
typedef pair<ptrdiff_t,size_t> target_info; //(offset,target)
static unordered_map<intptr_t, target_info> jump_targets;
      auto*  sptr = &x; // name to access subject
const void*       tptr; 
target_info& info = jump_targets[vtbl(sptr)];
switch (info.second) {{ default: 
\end{lstlisting}

\noindent
We use the virtual table pointer extracted from a polymorphic object pointed to 
by \code{p} as a key for association. The value stored along the key in 
association now keeps both: the target for the switch as well as a memoized 
offset for dynamic cast. 

The code for the $i^{th}$ case now evaluates the required offset on the first 
entry and associates it and the target with the vtbl-pointer of the subject.
The call to \code{adjust_ptr<Ti>} re-establishes the invariant that 
\code{match} is a reference to type \code{Ti} of the subject \code{x}.
%The condition of the inner if-statement is only needed to implement the 
%sequential all-fit semantics and can be removed when fall-through behavior is 
%not required.

\begin{lstlisting}
  if (tptr = dynamic_cast<const Ti *>(sptr)) {
        if (info.second == 0) { // supports fall-through
      info.first  = intptr_t(tptr)-intptr_t(sptr); // offset
            info.second = @$i$@; // jump target
        }
  case @$i$@: // @$i$@ is a constant - clause's position in switch
    auto match = adjust_ptr<Ti>(sptr,info.first);
        si;
    }
\end{lstlisting}

\noindent
The main condition remains the same. We keep checking for the first initialization 
because we allow fall-through semantics here, letting the user break from the 
switch when needed. Upon first entry we compute the offset that the dynamic cast 
performed and save it together with target associated to the virtual table 
pointer. On the next iteration we will jump directly to the case label and 
restore the invariant of \code{matched} being a properly-casted reference to the 
derived object.

The use of dynamic cast makes a huge difference in comparison to the use of 
static cast we dismissed above. First of all the C++ type system is much more 
restrictive about the static cast and many cases where it is not allowed can 
still be handled by dynamic cast. Examples of these include downcasting from an 
ambiguous base class or cross-casting between unrelated base classes.

An important benefit we get from this optimization is that we do not store the 
actual values (pointers to objects) in the hash table anymore, but group them 
into equivalence classes based on their virtual table pointers. The number of 
such pointers in a program is always bound by $O(|A|)$, where $A$ represents the 
static type of an object, while $|A|$ represents the number of classes directly 
or indirectly derived from $A$. The linear coefficient hidden in big-o notation 
reflects possibly multiple vtbl-pointers in derived classes due to the use of 
multiple inheritance.

\begin{figure}[htbp]
  \centering
    \includegraphics[width=0.47\textwidth]{DCast-vs-Visitors3.png}
  \caption{Time to uncover i\textsuperscript{th} case. X-axis - case i; Y-axis - cycles per iteration}
  \label{fig:DCastVis3}
\end{figure}

The most important benefit of this optimization, however, is the constant time 
on average used to dispatch each of the case clauses, regardless of their 
position in the type switch. The net effect of this optimization can be seen in Figure~\ref{fig:DCastVis3}. 
We can see that the time does not increase with the position of the case we are 
handling. The spikes represent activities on computer during measurement and are 
present in both measurements. 
The constant time on average comes from the average complexity 
of accessing an element in an \code{unordered_map}, while its worst complexity can 
be proportional to the size of the map. We show in the next section, however, 
that most of the time we will be bypassing traditional access to elements of the 
map, because, as-is, the type switch is still about 50\% slower than the visitor 
design pattern.

\noindent
Class \code{std::unordered_map} provides amortized constant time access on 
average and linear in the number of elements in the worst case. We show in the 
next section that most of the time we will be bypassing traditional access to 
its elements. We need this extra optimization because, as-is, the type switch is 
still about 50\% slower than the visitor design pattern.
\footnote{We are using 
speedups throughout the paper when comparing performance, so ``X is 42\% faster 
than Y'' or equally ``Y is 42\% slower than X'' means that Y's execution time is 
1.42 times X's execution time.}

Note that we can apply the reasoning of \textsection\ref{sec:memdev} and change 
the first-fit semantics of the resulting match statement into a best-fit 
semantics simply by changing the underlying cascading-if structure with decision 
tree. A compiler implementation of a type switch based on Vtable Pointer 
Memoization will certainly take advantage of this optimization to cut down the 
cost of the first run on a given vtbl-pointer, when the actual memoization happens.

Looking back at the example from \textsection\ref{sec:intro} and allowing for a few 
unimportant omissions, the first code snippet corresponds to what the macro 
\code{Match(x)} is expanded to when given a subject expression \code{x}. In order to 
see what \code{Case(Ti)} is expanded to, the second snippet has to be split on 
the line containing \code{si;} (excluding \code{si;} itself, which comes from 
source) and the second part (i.e. \} here) moved in front of the first one. The 
macro thus closes the scope of the previous case clause before starting the new 
one. \code{Case}'s expansion only relies on names introduced by \code{Match(x)}, 
its argument \code{Ti}, and a constant $i$, which can be generated from the
\code{__LINE__} macro, or, better yet, the \code{__COUNTER__} macro when 
supported by the compiler. The \code{EndMatch} macro simply closes the scopes 
(i.e. \}\} here). We refer the reader to the library source code for 
further details.

\subsubsection{Structure of Virtual Table Pointers}
\label{sec:sovtp}

Virtual table pointers are not entirely random addresses in memory and have 
certain structure when we look at groups of those that are associated with 
classes related by inheritance. Let us first look at some vtbl pointers that 
were present in some of our tests. The 32-bit pointers are shown in binary form 
(lower bits on the right) and are sorted in ascending order:

\begin{verbatim}
00000001001111100000011001001000
00000001001111100000011001011100
00000001001111100000011001110000
 ...
00000001001111100000011111011000
00000001001111100000011111101100
\end{verbatim}

Virtual table pointers are not constant values and are not even guaranteed to be 
the same between different runs of the same application. Techniques like 
\emph{address space layout randomization} or simple \emph{rebasing} of the entire 
module are likely to change these values. The relative distance between them is 
likely to remain the same though as long as they come from the same module.

Comparing all the vtbl pointers that are coming through a given match statement 
we can trace ar run time the set of bits in which they do and do not differ. 
For the above example it may look as \texttt{00000001001111100000X11XXXXXXX00} 
where positions marked with X represent bits that are different in some vtbl 
pointers.

When a DLL is loaded it may have its own copy of vtables for classes also used 
in other modules as well as vtables for classes it introduces. Comparing 
similarly all vtbl pointers coming only from this DLL we can get a different 
pattern \\ \texttt{01110011100000010111XXXXXXXXX000} and when compared over all 
the loaded modules the pattern will likely becomes something like 
\texttt{0XXX00X1X0XXXXXX0XXXXXXXXXXXXX00}.

The common bits on the right come from the virtual table size and alignment 
requirements, and, depending on compiler, configuration, and class hierarchy could 
easily vary from 2 to 6 bits. Because the vtbl-pointer under the C++ ABI points into 
an array of function pointers, the alignment requirement of 4 bytes for those 
pointers on a 32-bit architecture is what makes at least the last 2 bits to be 0. 
For our purpose the exact number of bits on the right is not important as we 
evaluate this number at run time based on vtbl-pointers seen so far. Here we only 
would like to point out that there would be some number of common bits on the 
right.

Another observation we made during our experiments with the vtbl-pointers of various 
existing applications was that the values of the pointers where changing more 
frequently in the lower bits than in the higher ones. We believe that this was 
happening because programmers tend to group multiple derived classes in the same 
translation unit so the compiler was emitting virtual tables for them close to 
each other as well. 

Note that derived classes that do not introduce their own virtual functions 
(even if they override some existing ones) are likely to have virtual tables of 
the same size as their base class. Even when they do add new virtual functions, 
the size of their virtual tables can only increase relative to their base 
classes. This is why the difference between many consecutive vtbl-pointers that 
came through a given match statement was usually constant or very slightly 
different.

The changes in higher bits were typically due to separate compilation and 
especially due to dynamically loaded modules. When a DLL is loaded, it may have 
its own copies of vtables for classes that are also used in other modules, in addition to 
vtables for classes it introduces. Comparing all vtbl-pointers coming only from 
that DLL we can get a different pattern \texttt{01110011100000010111XXXXXXXXX000} 
and when compared over all the loaded modules the pattern will likely become 
something like \texttt{0XXX00X1X0XXXXXX0XXXXXXXXXXXXX00}. Overall they were not 
changing the general tendency we saw: smaller bits were changing more frequently 
than larger ones, with the exception of the lowest common bits, of course.

These observations made virtual table pointers of classes related by inheritance 
ideally suitable for indexing -- the values obtained by throwing away the common 
bits on the right were compactly distributed in small disjoint ranges. We use 
those values to address a cache built on top of the hash table in order to 
eliminate a hash table lookup in most of the cases.  The important 
guarantee about the validity of the cached hash table references comes from the 
C++0x standard, which states that ``insert and emplace members shall not affect 
the validity of references to container elements''~\cite[\textsection 
23.2.5(13)]{C++0x}. 

Depending on the number of actual collisions that happen in the cache, our 
vtable pointer memoization technique can come close to, and even outperform, the 
visitor design pattern. The numbers are, of course, averaged over many runs as 
the first run on every vtbl-pointer will take an amount of time as shown in 
Figure\ref{fig:DCastVis1}. We did however test our technique on real code and 
can confirm that it does perform well in the real-world use cases.

The information about jump targets and necessary offsets is just an example of 
information we might want to be able to associate with, and access via, virtual 
table pointers. Our implementation of \code{memoized_cast} from 
\textsection\ref{sec:memcast} effectively reuses this general data structure with 
a different type of element values. We thus created a generic reusable class 
\code{vtblmap<T>} that maps vtbl-pointers to elements of type T. We will refer 
to the combined cache and hash-table data structure, extended with the logic for 
minimizing conflicts presented below, as a \emph{vtblmap} data structure.

\subsection{Minimization of Conflicts}
\label{sec:moc}

Virtual table pointers are not constant values and are not even guaranteed to be 
the same between different runs of the application, because techniques like 
\emph{address space layout randomization} or \emph{rebasing} of the module are 
likely to change them. The relative distance between them will remain the same 
as long as they come from the same module.

Knowing that vtbl-pointers point into an array of function pointers, we should 
expect them to be aligned accordingly and thus have a few lowest bits as zero. 
Moreover, since many derived classes do not introduce new virtual functions, 
the size of their virtual tables remains the same. When allocated sequentially 
in memory, we can expect a certain number of lowest bits in the vtbl-pointers 
pointing to them to be the same.
These assumptions, supported by actual observations, made virtual table 
pointers of classes related by inheritance ideally suitable for hashing: the 
values obtained by throwing away the common bits on the right were compactly 
distributed in small disjoint ranges (\textsection\ref{sec:hierarchies}). We use 
them to address a cache built on top of the hash table in order to eliminate a 
hash table lookup in most of the cases.

Let $\Xi$ be the domain of integral representations of pointers. Given a cache 
with $2^k$ entries, we use a family of hash functions $H_{kl} : \Xi \rightarrow [0..2^k-1]$ 
defined as $H_{kl}(v)=v/2^l \mod 2^k$ to index the cache, where $l \in [0..32]$ 
(assuming 32 bit addresses) is a parameter modeling the number of common bits on 
the right. Division and modulo are implemented with bit operations since the
denominator in each case is a power of 2, which in turn explains the choice of 
the cache size.

Given a hash function $H_{kl}$, pointers $v'$ and $v''$ are said to be in 
\emph{conflict} when $H_{kl}(v')=H_{kl}(v'')$. For a given set of pointers 
$V \in 2^{\Xi}$, we can always find such $k$ and $l$ that $H_{kl}$ will render no  
conflicts between its elements, but the required cache size $2^k$ can be too 
large to justify the use of memory. The value $K$ such that $2^{K-1} < |V| \leq 2^K$ 
is the smallest value of $k$ under which absence of conflicts is still possible. 
We thus allow $k$ to vary only in the range $[K,K+1]$ to ensure that the cache size 
is never more than 4 times bigger than the minimum required cache size.

Given a set $V = \{v_1, ... , v_n\}$, we would like to find a pair of parameters 
$(k,l)$ such that $H_{kl}$ will render the least number of conflicts on the 
elements of $V$. Since for a fixed set $V$, parameters $k$ and $l$ vary in a 
finite range, we can always find the optimal $(k,l)$ by trying all the
combinations. Let $H_{kl}^V : V \rightarrow [0..2^k-1]$ be the hash function 
corresponding to such optimal $(k,l)$ for the set $V$. 

In our setting, the set $V$ represents the set of vtbl-pointers coming through a 
particular type switch. While the exact values of these pointers are not known 
until run-time, their offsets from the module's base address are. This is generally 
sufficient to estimate optimal $k$ and $l$ in a compiler setting. In the library 
setting, we recompute them after a given number of actual collisions in cache.

When $H_{kl}^V$ is injective (renders 0 conflicts on $V$), the frequency of any 
given vtbl-pointer $v_i$ coming through the type switch does not affect the 
overall performance of the switch. However when $H_{kl}^V$ is not injective, we 
would prefer the conflict to happen on less frequent vtbl-pointers.
Given a probability $p(v_i)$ of each vtbl-pointer $v_i \in V$ we can compute the 
probability of conflict rendered by a given $H_{kl}$:

\begin{eqnarray*}
p_{kl}(V)=\sum\limits_{j=0}^{2^k-1}\of{\sum\limits_{v_{i} \in V^j_{kl}}p(v_i)}\of{1-\frac{\sum\limits_{v_i \in V^j_{kl}}p(v_i)^2}{\of{\sum\limits_{v_{i} \in V^j_{kl}}p(v_i)}^2}}
\end{eqnarray*}

\noindent 
where $V^j_{kl}=\{v \in V | H_{kl}(v)=j\}$. In this case, the optimal hash 
function $H_{kl}^V$ can similarly be defined as $H_{kl}$ that minimizes the 
above probability of conflict on $V$.

The probabilities $p(v_i)$ can be estimated in a compiler setting through profiling, 
while in a library setting we let the user enable tracing of frequencies of 
each vtbl-pointer. This introduces an overhead of an increment into the critical 
path of execution, and according to our tests degrades the performance by 1-2\%. 
This should not be a problem as long as the overall performance gains from a
smaller probability of conflicts happening at run time. Unfortunately, in our 
tests the significant drop in the number of actual collisions was not reflected 
in a noticeable decrease in execution time, which is why we do not enable 
frequency tracing by default. As we will see in \textsection\ref{sec:hierarchies}, 
this was because the hash function $H_{kl}^V$ renders no conflicts on 
vtbl-pointers in most cases and the few collisions we were getting before 
inferring the optimal $k$ and $l$ even in non-frequency-based caching where 
incomparably smaller than the number of successful cache hits.

Assuming uniform distribution of $v_i$ in $V$ and substituting the probability 
$p(v_i)=\frac{1}{n}$, where $n=|V|$, into the above formula we get:

\begin{eqnarray*}
p_{kl}(V)=\sum\limits_{j=0}^{2^k-1}[|V^j_{kl}| \neq 0]\frac{|V^j_{kl}|-1}{n}
\end{eqnarray*}

\noindent
We use the Iverson bracket $[\pi]$ here to refer to the outcome of a predicate $\pi$ as numbers $0$ or $1$.
The value $|V^j_{kl}|$ represents the number of vtbl-pointers $v_i \in V$ that are mapped to the same location $j$ in cache with $H_{kl}^V$. Only 
one such vtbl-pointer will actually be present in that cache location at any given 
time, which is why the value $|V^j_{kl}|-1$ represents the number of ``extra'' 
pointers mapped into the entry $j$ on which a collision will happen. The overall 
probability of conflict thus only depends on the total number of these ``extra'' 
or conflicting vtbl-pointers. The $H_{kl}^V$ obtained by minimization of 
probability of conflict under uniform distribution of $v_i$ in $V$ is thus the 
same as the original $H_{kl}^V$ that was minimizing the number of conflicts. An 
important observation here is that since the exact location of these ``extra'' 
vtbl-pointers is not important and only the total number $m$ is, the probability 
of conflict under uniform distribution of $v_i$ in $V$ is always going to be of 
the discrete form $\frac{m}{n}$, where $0 \le m < n$.

%Depending on the number of actual collisions that happen in the cache, our 
%vtable pointer memoization technique can come close to, and even outperform, the 
%visitor design pattern. The numbers are, of course, averaged over many runs as 
%the first run on every vtbl-pointer will take an amount of time as shown in 
%Figure\ref{fig:DCastVis1}. We did however test our technique on real code and 
%can confirm that it does perform well in the real-world use cases.

%The information about jump targets and necessary offsets is just an example of 
%information we might want to be able to associate with, and access via, virtual 
%table pointers. Our implementation of \code{memoized_cast}~\cite[\textsection 9]{TR}, for example, 
%effectively reuses this general data structure with a different type of element 
%values. We thus created a generic reusable class \code{vtblmap<T>} that maps 
%vtbl-pointers to elements of type T. We will refer to the combined cache and 
%hash-table data structure, extended with the logic for minimizing conflicts 
%presented below, as a \emph{vtblmap} data structure.

%\subsubsection{Minimization of Conflicts}
%\label{sec:moc}

The small number of cycles that the visitor design pattern needs to uncover a 
type does not let us put too sophisticated cache indexing mechanisms into the 
critical path of execution. This is why we limit our indexing function to shifts 
and masking operations as well as choose the size of the cache to be a power of 2.

Throughout this section by \emph{collision} we will call a run-time condition in 
which the cache entry of an incoming vtbl pointer is occupied by another vtbl-pointer.
Collision requires vtblmap to fetch the data associated with the new 
vtbl-pointer from a slower hash-table and, under certain conditions, reconfigure 
cache for better performance. By \emph{conflict} we will call a different 
run-time condition under which given cache configuration maps two or more vtbl 
pointers to the same cache location. Presence of conflict does not necessarily 
imply presence of collisions, but collisions can only happen when there is a 
conflict. In the rest of this section we devise a mechanism that tries to 
minimize the amount of conflicts in a hope that it will also decrease the amount 
of actual collisions.

Given $n$ vtbl-pointers we can always find a cache size that will render no 
conflicts between them. The necessary size of such a cache, however, can be too 
big to justify the use of memory. This is why, in our current implementation, we 
always consider only 2 different cache sizes: $2^k$ and $2^{k+1}$ where 
$2^{k-1} < n \leq 2^k$. This guarantees that the cache size is never more than 4 
times bigger than the minimum required cache size.

During our experiments, we noticed that often the change in the smallest 
different bit happens only in a few vtbl-pointers, which was effectively 
cutting the available cache space in half. To overcome this problem, we let the 
number of bits by which we shift the vtbl-pointer vary further and compute it in 
a way that minimizes the number of conflicts.

To avoid doing any computations in the critical path, \code{vtblmap} only 
recomputes the optimal shift and the size of the cache when an actual collision 
happens. In order to avoid constant recomputations when conflicts are unavoidable, 
we add an additional restriction of only reconfiguring the optimal parameters if 
the number of vtbl-pointers in the \code{vtblmap} has increased since the last 
recomputation. Since the number of vtbl-pointers is of the order $O(|A|)$, where 
$A$ is the static type of all vtbl-pointers coming through a \code{vtblmap}, the 
restriction assures that reconfigurations will not happen infinitely often.

To minimize the number of recomputations even further, our library communicates 
to the \code{vtblmap}, through its constructor, the number of case clauses in 
the underlying match statement. We use this number as an estimate of the expected 
size of the \code{vtblmap} and pre-allocate the cache according to this estimated 
number. The cache is still allowed to grow based on the actual number of 
vtbl-pointers that comes through a \code{vtblmap}, but it never shrinks from the
initial value. This improvement significantly minimizes the number of collisions 
at early stages, as well as the number of possibilities we have to consider 
during reconfiguration.

The above logic of \code{vtblmap} always chooses the configuration that renders 
no conflicts, when such a configuration is possible during recomputation of 
optimal parameters. When this is not possible, it is natural to prefer collisions 
to happen on less-frequent vtbl-pointers.

We studied the frequency of vtbl-pointers that come through various match statements
of a C++ pretty-printer that we implemented on top of the Pivot 
framework~\cite{Pivot09} using our pattern-matching library. We ran the 
pretty-printer on a set of C++ standard library headers and then ranked all the  
classes from the most-frequent to the least-frequent ones, on average. The 
resulting probability distribution is shown with a thicker line in 
Figure\ref{fig:PowerLaw}.

\begin{figure}[htbp]
  \centering
    \includegraphics[width=0.47\textwidth]{std-lib-power-law-distributions.png}
  \caption{Probability distribution of various nodes in Pivot framework}
  \label{fig:PowerLaw}
\end{figure}

Note that Y-Axis is using logarithmic scale, suggesting that the resulting 
probability has power-law distribution. This is likely to be a specifics of our 
application, nevertheless, the above picture demonstrates that frequency of certain 
classes can be larger than the overall frequency of all the other classes. In 
our case, the two most frequent classes were representing the use of a variable in 
a program, and their combined frequency was larger than the frequency of all the 
other nodes. Naturally, we would like to avoid conflicts on such classes in the 
cache, when possible.

Let us assume that a given \code{vtblmap} contains a set of vtbl pointers 
$V = \{v_1, ... , v_n\}$ with known probabilities $p_i$ of occuring. For a cache 
of size $2^k$ and a shift by $l$ bits we get a cache-indexing function 
$f_{lk} : V \rightarrow [0..2^k-1]$ defined as $f_{lk}(v_i) = (v_i \gg l) \& (2^k-1)$.
To calculate the probability of conflict for a given $l$ and $k$ parameters, let 
us consider $j^{th}$ cache cell and a subset $V^j_{lk}=\{v \in V | f_{lk}(v)=j\}$. 
When the size of this subset $m=|V^j_{lk}|$ is greater than 1, we have a 
potential conflict as subsequent request for a vtbl pointer $v''$ might be 
different from the vtbl pointer $v'$ currenly stored in the cell $j$. Within the 
cell only the probability of not having a conflict is the probability of both 
values $v''$ and $v'$ be the same:
\begin{eqnarray*}
P(v''=v')=\sum\limits_{v_i \in V^j_{lk}}P(v''=v_i)P(v'=v_i)=\sum\limits_{v_i \in V^j_{lk}}P^2(v_i|V^j_{lk})=\\
=\sum\limits_{v_i \in V^j_{lk}}\frac{P^2(v_i)}{P^2(V^j_{lk})}=
\sum\limits_{v_i \in V^j_{lk}}\frac{p_i^2}{(\sum\limits_{v_{i'} \in V^j_{lk}}p_{i'})^2}=
\frac{\sum\limits_{v_i \in V^j_{lk}}p_i^2}{(\sum\limits_{v_{i} \in V^j_{lk}}p_{i})^2}
\end{eqnarray*}

The probability of having a conflict among the vtbl pointers of a given cell is 
thus one minus the above value:

\begin{eqnarray*}
P(v''\neq v')=1-\frac{\sum\limits_{v_i \in V^j_{lk}}p_i^2}{(\sum\limits_{v_{i} \in V^j_{lk}}p_{i})^2}
\end{eqnarray*}

To obtain probability of conflict given any vtbl pointer and not just the one 
from a given cell we need to sum up the above probabilities of conflict within a 
cell multiplied by the probability of vtbl pointer fall into that cell:

\begin{eqnarray*}
P_{lk}^{conflict}=\sum\limits_{j=0}^{2^k-1}P(V^j_{lk})(1-\frac{\sum\limits_{v_i \in V^j_{lk}}p_i^2}{(\sum\limits_{v_{i} \in V^j_{lk}}p_{i})^2})=\\
=\sum\limits_{j=0}^{2^k-1}(\sum\limits_{v_{i} \in V^j_{lk}}p_{i})(1-\frac{\sum\limits_{v_i \in V^j_{lk}}p_i^2}{(\sum\limits_{v_{i} \in V^j_{lk}}p_{i})^2})
\end{eqnarray*}

Our reconfiguration algorithm then iterates over possible values of $l$ and $k$ 
and chooses those that minimize the overal probability of conflict $P_{lk}^{conflict}$.
The only data still missing are the actual probabilities $p_i$ used by the above 
formula. They can be approximated in many different ways.

Besides probability distribution on all the tests, Figure~\ref{fig:PowerLaw} 
shows probabilities of a given node on each of the tests. The X-Axis in this 
case represents the ordering of all the nodes according to their overall rank 
of all the tests combined. As can be seen from the picture, the shape of each 
specific test's distribution still mimics the overal probability distribution. 
With this in mind we can simply let the user assign probabilities to each of the 
classes in the hierarchy and use these values during reconfiguration. The 
practical problem we came accross with this solution was that we wanted these 
probabilities be inheritable as Pivot separates interface and implementation 
classes and we prefered the user to define them on interfaces rather than on 
implementation classes. The easiest way to do so wast to write a dedicated 
function that would return the probabilities using a match statement. 
Unfortunately such a function will introduce a lot of overhead as it will 
ideally only be used very few times (since we try to minimize the amount of 
reconfiguration) and thus not be using memoized jumps but rather slow 
cascading-if.

A simpler and likely more precise way of estimating $p_i$ would be to count 
frequencies of each vtbl pointers directly inside the \code{vtblmap}. This 
introduces an overhead of an increment into the critical path of execution, but 
according to our tests was only degrading the overal performance by 1-2\%.
Instead, it was compensating with a smaller amount of conflicts and thus a 
potential gain of performance. We leave the choice of whether the library should 
count frequencies of each vtbl pointer to the user of the library as the 
concrete choice may be to advantage on some class hierarchies and to 
disadvantage on others.

Figure~\ref{fig:Collisions} compares the amount of collisions when frequency 
information is and is not used. The data was gathered from 312 tests on multiple 
match statements present in Pivot's C++ pretty printer when it was ran over 
standard library headers. In 122 of these test both schemes had 0 conflicts and 
these tests are thus not shown on the graph. The remaining tests where ranked by 
the amount of conflicts in the scheme that does not utilize frequency information.

\begin{figure}[htbp]
  \centering
    \includegraphics[width=0.47\textwidth]{CollisionsWithAndWithoutFrequencies.png}
  \caption{Decrease in number of collisions when probabilities of nodes are taken into account}
  \label{fig:Collisions}
\end{figure}

As can be seen from the graph, both schemes render quite low amount of 
collisions given that there was about 57000 calls in the rightmost test having 
the largest amount of conflicts. Taking into account that the Y-axis has 
logarithmic scale, the use of frequency information in many cases decreased the 
amount of conflicts by a factor of 2. The handfull of cases where the use of 
frequency increased the number of conflicts can be explained by the fact that 
the optimal values are not recomputed after each conflict, but after several 
conflicts and only if the amount of vtbl pointers in the vtblmap increased. These 
extra conditions sacrify optimality of parameters at any given time for the amount 
of times they are recomputed. By varying the number of conflicts we are willing 
to tolerate before reconfiguration we can decrease the number of conflicts by 
increasing the amount of recomputations and vise versa. From our experience, 
however, we saw that the drop in the number of conflicts was not translating 
into a proportional drop in execution time, while the amount of reconfigurations 
was proportional to the increase in execution time. This is why we choose to 
tolerate a relatively large amount of conflicts before recomputation just to 
keep the amount of recomputations low.

\input{sec-5-implementation-typeswitch-exceptions}
\section{Unified Syntax}
\label{sec:unisyn}

The discussion in this subsection will be irrelevant for a compiler 
implementation, nevertheless we include it because some of the challenges we 
came accross as well as techniques we used to overcome them might show up in 
other active libraries. The problem is that working in a library setting, the 
toolbox of properties we can automatically infer about user's class hierarchy, 
match statement, clauses in it, etc. is much more limited than the set of 
properties a compiler can infer. On one side such additional information may let 
us generate a better code, but on the other side we understand that it is 
important not to overburden the user's syntax with every bit of information she 
can possibly provide us with to generate a better code. Some examples of 
information we can use to generate a better code even in the library setting 
include:

\begin{itemize}
\setlength{\itemsep}{0pt}
\setlength{\parskip}{0pt}
\item Encoding we are dealing with (\textsection\ref{sec:adt})
\item Shape of the class hierarchy: flat/deep, single/multiple inheritance etc.
\item The amount of clauses in the match statement
\item Presense of Otherwise clause in the match statement
\item Presence of extensions in dynamically linked libraries
\end{itemize}

We try to infer the information when we can, but otherwise resort to a usually 
slower default that will work in all or most of the cases. The major source of 
inefficiency comes from the fact that macro resolution happens before any 
meta-programming techniques can be employed and thus the macros have to generate 
a syntactic structure that can essentially handle all the cases as opposed to 
the exact case. Each of the macros involved in rendering the syntactic structure 
of a match statement (e.g. \code{Match}, \code{Case}, \code{Otherwise}) have a 
version identified with a suffix that is specific to a combination of encoding 
and shape of the class hierarchy. By default the macros are resolved to a 
unified version that infers encoding with a template meta-program, but this 
resolution can be overriden with a configuration flag for a more specific 
version when all the match statements in user's program satisfy the requirements 
of that version. The user can also pin-point specific match statement with the 
most applicable version, but we discourage such use as performance differences 
are not big enough to justify the exposure of details.

To better understand what is going on, consider the following examples. Case 
labels for polymorphic base class encoding can be arbitrary, but preferably 
sequential numbers, while the case labels for tagged class and discriminated 
union encodings are the actual kind values associated with concrete variants.
Discriminated union and tagged class encodings can use both types (views in case
of unions) and kind values to identify the target variant, while polymorphic 
base class encoding can only use types for that. The latter encoding requires 
allocation of a static vtblmap in each match statement, not needed by any other 
encoding, while tagged class encoding on non-flat hierarchy requires the use of 
default label of the generated switch statement as well as a dedicated case 
label distinct from all kind values (\textsection\ref{sec:cotc}). 
When merging these and other requirements into a syntactic structure of a 
unified version capable of handling any encoding we essentially always have to 
reserve the use of default label (and thus not use it to generate 
\code{Otherwise}-clause), allocate an extra dedicated case label, introduce  
a loop over base classes used by tagged class encoding etc. This is a clear 
overhead for handling of a discriminated union encoding whose syntactic 
structure only involves a simple switch over kind values and default label to 
implement \code{Otherwise}. To minimize the effects of this overhead we rely on 
compiler's optimizer to inline calls specific to each encoding and either remove 
branching on conditions that will always be true after inlining or elminate dead 
code on conditions that will always be false after inlining. Luckily for us 
today's compilers do a great job in doing just that, rendering our unified 
version only slightly less efficient than the specialized ones. These 
differences can be best seen in Figure\ref{relperf} under corresponding entries 
of \emph{Unified} and \emph{Specialized} columns.

\section{Memoized Dynamic Cast}
\label{sec:memcast}

We saw in Corollary~\ref{crl:vtbl} that the results of \code{dynamic_cast} can 
be reapplied to a different instance from within the same subobject. This leads 
to simple idea of memoizing the results of \code{dynamic_cast} and then using 
them on subsequent casts. In what follows we will only be dealing with the 
pointer version of the operator since the version on references that has a 
slight semantic difference can be easily implemented in terms of the pointer one.

The \code{dynamic_cast} operator in \Cpp{} involves two arguments: a value argument 
representing an object of a known static type as well as a type argument 
denoting the runtime type we are querying. Its behavior is twofold: on one hand 
it should be able to determine when the object's dynamic type is not a 
subtype of the queried type (or when the cast is ambiguous), while on the other 
it should be able to produce an offset by which to adjust the value argument when it is.

We mimic the syntax of \code{dynamic_cast} by defining:

\begin{lstlisting}
template <typename T, typename S>
inline T memoized_cast(S*);
\end{lstlisting}

\noindent
which lets the user replace all the uses of \code{dynamic_cast} in the program 
with \code{memoized_cast} with a simple:

\begin{lstlisting}
#define dynamic_cast memoized_cast
\end{lstlisting}

\noindent
It is important to stress that the offset is not a function of the source and target 
types of the \code{dynamic_cast} operator, which is why we cannot simply memoize the 
outcome inside the individual instantiations of \code{memoized_cast}.
The use of repeated multiple inheritance will result in classes having several 
different offsets associated with the same pair of source and target types 
depending on which subobject the cast is performed from. According to 
corollary~\ref{crl:vtbl}, however, it is a function of target type and the value 
of the vtbl-pointer stored in the object, because the vtbl-pointer uniquely 
determines the subobject within the dynamic type. Our memoization of the 
results of \code{dynamic_cast} should thus be specific to a vtbl-pointer and the 
target type. 

The easiest way to achieve this would be to use a dedicated global
\code{vtblmap<std::ptrdiff_t>} (\textsection\ref{sec:sovtp}) per each 
instantiation of the \code{memoized_cast}. This, however, will create an 
unnecessarily large amount of vtblmap structures, many of which will be  
duplicating information and repeating the work already done. This will happen 
because instantiations of \code{memoized_cast} with same target but different 
source types can share their vtblmap structures since vtbl pointers of different 
source types are necessarily different according to Theorem~\ref{thm:vtbl}. 

Even though the above solution can be easily improved to allocates a single 
vtblmap per target type, an average application might have a lot of different 
target types. This is especially true for applications that will use our Match 
statement since we use \code{dynamic_cast} under the hood in each case clause. 
Indeed our C++ pretty printer was creating 160 vtblmaps of relatively small size 
each, which was increasing the executable size quite significantly because of 
numerous instantiations as well as noticably slowed down the compilation time.

To overcome the problem we turn each target type into a runtime instantiation 
index of the type and allocate a single \code{vtblmap<std::vector<std::ptrdiff_t>>} 
that associates vtbl pointers with a vector of offsets indexed by target type. 
The slight performance overhead that is brought by this improvement is specific 
to our library solution and would not be present in a compiler implementaion. 
Instead we get a much smaller memory footrpint, which can be made even smaller 
once we recognize the fact that global type indexing may effectively enumerate 
target classes that will never appear in the same Match statement. This will 
result in entries in the vector of offsets that are never used.

Our actual solution uses separate indexing of target types for each source type 
they are used with, and also allocates a different 
\code{vtblmap<std::vector<std::ptrdiff_t>>} for each source type. This lets us 
minimize unused entries within offset vectors by making sure only the plausible 
target types for a given source type are indexed. This solution should be 
suitable for most applications since we expect to have a fairly small 
number of source types for the \code{dynamic_cast} operator and a much larger number 
of target types. For the unlikely case of a small number of target types and large 
number of source types we allow the user to revert to the default behavior with a 
library configuration switch that allocates a single \code{vtblmap} per target type as 
we have already discussed above.

The use of \code{memoized_cast} to implement the \code{Match}-statement potentially reuses the 
results of \code{dynamic_cast} computations across multiple independent match 
statements. This allows leveraging the cost of the expensive first call with a 
given vtbl-pointer even further across all the match statements inside the 
program. The above define, with which a user can easily turn all dynamic casts 
into memoized casts, can be used to speed-up existing code that uses dynamic 
casting without any refactoring overhead.



%\subsection{Discussion}
%\label{sec:dsc}

%Let us look at both our techniques in the context of Zenger and Odersky 
%challenge to independently extensible solutions of extension problem discussed 
%in \textsection\ref{sec:exp}.

%\begin{itemize}
%\item Extensibility in both dimensions: \\
%      %It should be possible to add new data variants, while adapting the 
%      %existing operations accordingly. It should also be possible to introduce 
%      %new functions. 
%      Our techniques allow one to extend data with subclassing as well as 
%      introduce new functions through a match statement on corresponding 
%      encoding. The existing operations 
%\item Strong static type safety: \\
%      %It should be impossible to apply a function to a data variant, which it 
%      %cannot handle.
%\item No modification or duplication: \\
%      %Existing code should neither be modified nor duplicated.
%\item Separate compilation: \\
%      %Neither datatype extensions nor addition of new functions should require 
%      %re-typechecking the original datatype or existing functions. No safety 
%      %checks should be deferred until link or runtime.
%\item Independent extensibility: \\
%      %It should be possible to combine independently developed extensions so 
%      %that they can be used jointly.
%\end{itemize}
%
\section{Evaluation} %%%%%%%%%%%%%%%%%%%%%%%%%%%%%%%%%%%%%%%%%%%%%%%%%%%%%%%%%%%
\label{sec:eval}

We performed several independent studies of our approach to demonstrate its 
effectiveness. The first study compares our approach to the visitor design 
pattern and shows that the type switch is comparable or faster 
(\textsection\ref{sec:viscmp}). While we do not advocate for the closed solution 
of \textsection\ref{sec:cotc}, we included the comparison of type switching 
solutions made under open and closed world assumptions (\textsection\ref{sec:cmp}).
Our library supports both solutions with the same surface syntax, which is why 
we believe many users will try them both before settling on one.
The second study does a similar comparison with built-in facilities of Haskell 
and OCaml and shows that the open type switch for extensible and hierarchical 
data types can be almost as efficient as its equivalent for closed algebraic 
data types (\textsection\ref{sec:ocaml}). In the third study we looked at how 
well our caching mechanisms deal with some large real-world class hierarchies in 
order to demonstrate that our performance numbers were not established in overly 
idealistic conditions (\textsection\ref{sec:hierarchies}). In the last study we 
rewrote an existing visitor-based application using our approach in order to 
compare the ease of use, readability and maintainability of each approach, as 
well as to show the memory usage and the startup costs associated with our 
approach in a real application (\textsection\ref{sec:qualcmp}).

\subsection{Comparison with Visitor Design Pattern}
\label{sec:viscmp}

\begin{figure*}
\begin{tabular}{@{}c@{ }l||@{ }r@{}@{ }r@{}@{ }r@{}|@{ }r@{}@{ }r@{}@{ }r@{}||@{ }r@{}@{ }r@{}@{ }r@{}|@{ }r@{}@{ }r@{}@{ }r@{}||@{ }r@{}@{ }r@{}@{ }r@{}|@{ }r@{}@{ }r@{}@{ }r@{}}
\hline % -----------------------------------------------------------------------------------------------------------------------------------------
\hline % -----------------------------------------------------------------------------------------------------------------------------------------
 &            & \multicolumn{6}{c||}{G++/32 on Windows Laptop} & \multicolumn{6}{c||}{MS Visual C++/32}        & \multicolumn{6}{c}{MS Visual C++/64}           \\
\hline % -----------------------------------------------------------------------------------------------------------------------------------------
 & Syntax     & \multicolumn{3}{c|}{Unified} & \multicolumn{3}{c||}{Specialized} & \multicolumn{3}{c|}{Unified} & \multicolumn{3}{c||}{Specialized} & \multicolumn{3}{c|}{Unified} & \multicolumn{3}{c}{Specialized} \\
\hline % -----------------------------------------------------------------------------------------------------------------------------------------
 & Encoding   & \Opn  & \Cls  & \Unn  & \Opn  & \Cls  & \Unn  & \Opn  & \Cls  & \Unn  & \Opn  & \Cls  & \Unn  & \Opn  & \Cls  & \Unn  & \Opn  & \Cls  & \Unn   \\
\hline % -----------------------------------------------------------------------------------------------------------------------------------------
\hline % -----------------------------------------------------------------------------------------------------------------------------------------
 & Repetitive &\gwNGPp&\gwNGKp&\gwNGUp&\gwNSPp&\gwNSKp&\gwNSUp&\vwNGPp&\vwNGKp&\vwNGUp&\vwNSPp&\vwNSKp&\vwNSUp&\vxNGPp&\vxNGKp&\vxNGUp&\vxNSPp&\vxNSKp&\vxNSUp \\
 & Sequential &\gwNGPq&\gwNGKq&\gwNGUq&\gwNSPq&\gwNSKq&\gwNSUq&\vwNGPq&\vwNGKq&\vwNGUq&\vwNSPq&\vwNSKq&\vwNSUq&\vxNGPq&\vxNGKq&\vxNGUq&\vxNSPq&\vxNSKq&\vxNSUq \\
 & Random     &\gwNGPn&\gwNGKn&\gwNGUn&\gwNSPn&\gwNSKn&\gwNSUn&\vwNGPn&\vwNGKn&\vwNGUn&\vwNSPn&\vwNSKn&\vwNSUn&\vxNGPn&\vxNGKn&\vxNGUn&\vxNSPn&\vxNSKn&\vxNSUn \\
\hline % ------------------------------------------------------------------------------------------------------------------------------------------
\multirow{3}{*}{\begin{sideways}{\tiny Forward}\end{sideways}}
 & Repetitive &\gwYGPp&\gwYGKp&\gwYGUp&\gwYSPp&\gwYSKp&\gwYSUp&\vwYGPp&\vwYGKp&\vwYGUp&\vwYSPp&\vwYSKp&\vwYSUp&\vxYGPp&\vxYGKp&\vxYGUp&\vxYSPp&\vxYSKp&\vxYSUp \\
 & Sequential &\gwYGPq&\gwYGKq&\gwYGUq&\gwYSPq&\gwYSKq&\gwYSUq&\vwYGPq&\vwYGKq&\vwYGUq&\vwYSPq&\vwYSKq&\vwYSUq&\vxYGPq&\vxYGKq&\vxYGUq&\vxYSPq&\vxYSKq&\vxYSUq \\
 & Random     &\gwYGPn&\gwYGKn&\gwYGUn&\gwYSPn&\gwYSKn&\gwYSUn&\vwYGPn&\vwYGKn&\vwYGUn&\vwYSPn&\vwYSKn&\vwYSUn&\vxYGPn&\vxYGKn&\vxYGUn&\vxYSPn&\vxYSKn&\vxYSUn \\
\hline % -----------------------------------------------------------------------------------------------------------------------------------------
\hline % -----------------------------------------------------------------------------------------------------------------------------------------
 &            & \multicolumn{6}{c||}{G++/32 on Linux Desktop} & \multicolumn{6}{c||}{MS Visual C++/32 with PGO} & \multicolumn{6}{c}{MS Visual C++/64 with PGO} \\
\hline % -----------------------------------------------------------------------------------------------------------------------------------------
 & Syntax     & \multicolumn{3}{c|}{Unified} & \multicolumn{3}{c||}{Specialized} & \multicolumn{3}{c|}{Unified} & \multicolumn{3}{c||}{Specialized} & \multicolumn{3}{c|}{Unified} & \multicolumn{3}{c}{Specialized} \\
\hline % -----------------------------------------------------------------------------------------------------------------------------------------
 & Encoding   & \Opn  & \Cls  & \Unn  & \Opn  & \Cls  & \Unn  & \Opn  & \Cls  & \Unn  & \Opn  & \Cls  & \Unn  & \Opn  & \Cls  & \Unn  & \Opn  & \Cls  & \Unn   \\
\hline % -----------------------------------------------------------------------------------------------------------------------------------------
\hline % -----------------------------------------------------------------------------------------------------------------------------------------
 & Repetitive &\glNGPp&\glNGKp&\GwNGUp&\glNSPp&\glNSKp&\GwNSUp&\VwNGPp&\VwNGKp&\VwNGUp&\VwNSPp&\VwNSKp&\VwNSUp&\VxNGPp&\VxNGKp&\VxNGUp&\VxNSPp&\VxNSKp&\VxNSUp \\
 & Sequential &\glNGPq&\glNGKq&\GwNGUq&\glNSPq&\glNSKq&\GwNSUq&\VwNGPq&\VwNGKq&\VwNGUq&\VwNSPq&\VwNSKq&\VwNSUq&\VxNGPq&\VxNGKq&\VxNGUq&\VxNSPq&\VxNSKq&\VxNSUq \\
 & Random     &\glNGPn&\glNGKn&\GwNGUn&\glNSPn&\glNSKn&\GwNSUn&\VwNGPn&\VwNGKn&\VwNGUn&\VwNSPn&\VwNSKn&\VwNSUn&\VxNGPn&\VxNGKn&\VxNGUn&\VxNSPn&\VxNSKn&\VxNSUn \\
\hline % ------------------------------------------------------------------------------------------------------------------------------------------
\multirow{3}{*}{\begin{sideways}{\tiny Forward}\end{sideways}}
 & Repetitive &\glYGPp&\glYGKp&\GwYGUp&\glYSPp&\glYSKp&\GwYSUp&\VwYGPp&\VwYGKp&\VwYGUp&\VwYSPp&\VwYSKp&\VwYSUp&\VxYGPp&\VxYGKp&\VxYGUp&\VxYSPp&\VxYSKp&\VxYSUp \\
 & Sequential &\glYGPq&\glYGKq&\GwYGUq&\glYSPq&\glYSKq&\GwYSUq&\VwYGPq&\VwYGKq&\VwYGUq&\VwYSPq&\VwYSKq&\VwYSUq&\VxYGPq&\VxYGKq&\VxYGUq&\VxYSPq&\VxYSKq&\VxYSUq \\
 & Random     &\glYGPn&\glYGKn&\GwYGUn&\glYSPn&\glYSKn&\GwYSUn&\VwYGPn&\VwYGKn&\VwYGUn&\VwYSPn&\VwYSKn&\VwYSUn&\VxYGPn&\VxYGKn&\VxYGUn&\VxYSPn&\VxYSKn&\VxYSUn \\
\hline % -----------------------------------------------------------------------------------------------------------------------------------------
\hline % ----------------------------------------------------------------------------------------------------------------------------------
 &            & \multicolumn{6}{c||}{ } & \multicolumn{12}{c}{Windows Laptop}                                                      \\
\hline % ----------------------------------------------------------------------------------------------------------------------------------
\end{tabular}
\caption{Relative performance of type switching versus visitors. Numbers 
in regular font (e.g. \f{67}), indicate that our type switching is faster than 
visitors by corresponding percentage. Numbers in bold font (e.g. \s{14}), 
indicate that visitors are faster by corresponding percentage.}
\label{relperf}
\end{figure*}

Our comparison methodology involves several benchmarks representing various 
uses of objects inspected with either visitors or type switching.

The \emph{repetitive} benchmark (REP) performs calls on different objects of the 
same dynamic type. This scenario happens in object-oriented setting when a 
group of polymorphic objects is created and passed around (e.g. numerous 
particles of a given kind in a particle simulation system). We include it 
because double dispatch becomes twice faster (20 vs. 53 cycles) in this 
scenario compared to others due to hardware cache and call target prediction mechanisms. 

The \emph{sequential} benchmark (SEQ) effectively uses an object of each derived type only 
once and then moves on to an object of a different type. The cache is typically 
reused the least in this scenario, which is typical of lookup tables, where each 
entry is implemented with a different derived class.

The \emph{random} benchmark (RND) is the most representative as it randomly makes calls on 
different objects -- probably be the most common usage scenario in the real world.

Presence of \emph{forwarding} in any of these benchmarks refers to the 
common technique used by visitors where, for class hierarchies with multiple 
levels of inheritance, the \code{visit} method of a derived class will provide a 
default implementation of forwarding to its immediate base class, which, in turn, 
may forward it to its base class, etc. The use of forwarding in visitors is a 
way to achieve substitutability, which in type switch corresponds to the use 
of base classes in the case clauses.
This approach is used in Pivot, whose AST 
hierarchy consists of 154 node kinds, of which only 5 must be handled, while the 
rest will forward to them when visit for them was not overriden.

The class hierarchy for non-forwarding test was a flat hierarchy of 100 
derived classes, encoding an algebraic data type. The class hierarchy for 
forwarding tests had two levels of inheritance with 5 intermediate base classes 
and 95 derived ones. 

Each benchmark was tested with either \emph{unified} or \emph{specialized} 
syntax, each of which included tests on polymorphic (\emph{Open}) and tagged 
(\emph{Tag}) encodings. Specialized syntax avoids generating unnecessary 
syntactic structure used to unify syntax, and thus produces faster code. We 
include it in our results because a compiler implementation of type switching 
will only generate the best suitable code.

The benchmarks were executed in the following configurations referred to as 
\emph{Linux Desktop} and \emph{Windows Laptop} respectively:

\begin{itemize}
\setlength{\itemsep}{0pt}
\setlength{\parskip}{0pt}
\item \emph{Lnx}: Dell Dimension\textsuperscript{\textregistered} desktop with Intel\textsuperscript{\textregistered} Pentium\textsuperscript{\textregistered} 
      D (Dual Core) CPU at 2.80 GHz; 1GB of RAM; Fedora Core 13  
      \begin{itemize}
      \setlength{\itemsep}{0pt}
      \setlength{\parskip}{0pt}
      \item G++ 4.4.5 executed with -O2; x86 binaries
      \end{itemize}
\item \emph{Win}: Sony VAIO\textsuperscript{\textregistered} laptop with Intel\textsuperscript{\textregistered} Core\texttrademark i5 460M 
      CPU at 2.53 GHz; 6GB of RAM; Windows 7 Professional
      \begin{itemize}
      \setlength{\itemsep}{0pt}
      \setlength{\parskip}{0pt}
      \item G++ 4.6.1 / MinGW executed with -O2; x86 binaries
      \item MS Visual \Cpp{} 2010 Professional x86/x64 binaries with and without 
      Profile-Guided Optimizations
      \end{itemize}
\end{itemize}

\noindent
To improve accuracy, timing in all the configurations was performed with the 
help of \code{RDTSC} instruction available on x86 processors. For every number reported 
here we ran 101 experiments timing 1,000,000 dispatches each (all through either 
visitors or type switch). The first experiment was serving as a warm-up, during 
which the optimal caching parameters were inferred, and typically resulted in an 
outlier with the largest time. Averaged over 1,000,000 dispatches, the number of 
cycles per dispatch in each of the 101 experiments was sorted and the median was 
chosen. We preferred median to average to diminish the influence of other 
applications and OS interrupts as well as to improve reproducibility of timings 
between the runs of application. In particular, in the diagnostic boot of 
Windows, where the minimum of drivers and applications are loaded, we were 
getting the same number of cycles per iteration 70-80 out of 101 times. Timings 
in non-diagnostic boots had somewhat larger absolute values, however the 
relative performance of type switch against visitors remained unchanged and 
equally well reproducible.

\begin{figure}[htbp]
  \centering
    \includegraphics[width=0.47\textwidth]{VisitorsCompare.pdf}
  \caption{Absolute timings for different benchmarks}
  \label{fig:VisitorsComparison}
\end{figure}

To understand better the relative numbers of Figure~\ref{relperf}, we present 
in Figure~\ref{fig:VisitorsComparison} few absolute timings taken by visitors 
and open type switch to execute an iteration of a given benchmark. These absolute timings 
correspond to the relative numbers from column Open/G++/Win of Figure~\ref{relperf}.
The actual bars show the timings without forwarding, while the black lines 
indicate where the corresponding bar would be in the presence of forwarding. It 
is easy to see that visitors generally become slower in the presence of 
forwarding due to extra call, while type switch becomes faster due to smaller 
jump table. As discussed, both timings are much smaller for repetitive benchmark 
due to hardware cache.

Figure~\ref{relperf} provides a broader overview of how both techniques compare 
under different compiler/platform configurations. The values are given as 
percentages of performance increase against the slower technique. 
%Numbers in regular font represent cases where type switching was  
%faster, while underlined numbers in bold indicate cases where visitors were faster.

We can see that type switching wins by a good margin when implemented with tag switch (\textsection\ref{sec:cotc}) as  
well as in the presence of at least one level of forwarding. Note that the 
numbers are relative, and thus the ratio depends on both the performance of 
virtual function calls and the performance of switch statements. Visual \Cpp{} was 
generating faster virtual function calls, while GCC was generating faster switch 
statements, which is why their relative performance seem to be much more 
favorable for us in the case of GCC.
Similarly, the code for x86-64 is only slower relatively: the actual time spent for 
both visitors and type switching was smaller than that for x86-32, but it was much 
smaller for visitors than type switching, which resulted in worse relative 
performance.

The code on the critical path of our type switch implementation benefits 
significantly from branch hinting as some branches are much more likely than 
others. We use the branch hinting directives in GCC to guide the compiler, but, 
unfortunately, Visual \Cpp{} does not provide any similar facilities. Instead, 
Microsoft suggests using \emph{Profile-Guided Optimizations} (PGO) to achieve 
the same, which is why we list the results for Visual \Cpp{} both with and without 
profile-guided optimizations.
%The results without profile-guided optimizations can be 
%found in the accompanying technical report~\cite[\textsection 10]{TR}.
%The results of optimizing code created with Visual \Cpp{} by using profile 
%guided optimizations as currently Visual \Cpp{} does not have means for branch 
%hinting, which are supported by G++ and proven to be very effective in few 
%cruicial places. Profile guided optimization in Visual \Cpp{} lets compiler find 
%out experimentally what we would have otherwise hinted, even though this 
%includes other optimizations as well.

From the table it may seem that Visual C++ is generating not as good code as GCC 
does, but remember that these numbers are relative, and thus the ratio depends on  
both the performance of virtual calls and the performance of switch statements. Visual 
C++ was generating faster virtual function calls, while GCC was generating 
faster switch statements, which is why their relative performance seem to be much 
more favorable for us in the case of GCC.

Similarly the code for x64 is only slower relatively: the actual time spent for 
both visitors and type switching was smaller than that for x86, but it was much 
smaller for visitors than type switching, which resulted in worse relative 
performance.

\subsection{Open vs. Closed Type Switch}
\label{sec:cmp}

With only a few exceptions for x64, we saw in the Figure~\ref{relperf} that the 
performance of the closed tag switch (the Tag column) dominates the performance of the open type
swith (the Open column). We believe that the difference, often significant, is the price one pays 
for the true openness of the vtable pointer memoization solution.

As we mentioned in \textsection\ref{sec:cotc}, the use of tags, even when allocated 
by a compiler, may require integration efforts to ensure that different DLLs have 
not reused the same tags. Randomization of tags, similar to a proposal of 
Garrigue~\cite{garrigue-98}, will not eliminate the problem and will surely 
replace jump tables in switches with decision trees. This will likely 
significantly degrade the numbers for the part of Figure~\ref{relperf} 
representing closed tag switch, since the tags in our experiments were all 
sequential and small. 

The reliance of a tag switch on static cast, as described in \textsection\ref{sec:vtblmem}, this has severe limitations in the 
presence of multiple inheritance, and thus is not as versatile as open type 
switch. Overcoming this problem will either require the use of 
\code{dynamic_cast} or techniques similar to vtable pointer memoization, which 
will likely degrade tag switch's performance numbers even further.

Note also that the approach used to implement open type switch can be used to 
implement both first-fit and best-fit semantics, while the tag switch is only suitable 
for best-fit semantics. Their complexity guarantees also differ: open type 
switch is constant on average, but slow on the first call with given subobject. 
Tag switch is logarithmic in the size of the class hierarchy 
(assuming a balanced hierarchy), including the first call. This last point can 
very well be seen in Figure~\ref{relperf}, where the performance of a closed solution
degrades significantly in the presence of forwarding, while the performance of an
open solution improves.

\subsection{Comparison with OCaml and Haskell}
\label{sec:ocaml}

We now compare our solution to the built-in pattern-matching facility of 
OCaml~\cite{OPM01} and Haskell~\cite{Haskell98Book}.  
In this test, we timed small OCaml and Haskell applications performing our sequential 
benchmark on an algebraic data type of 100 variants. Corresponding \Cpp{} 
applications were working with a flat class hierarchy of 100 derived classes. 
The difference between the C++ applications lies in the encoding (Open/Tag/Kind) 
and the syntax (Unified/Special) used. Kind 
encoding is the same as Tag encoding, but it does not require substitutability, 
and thus can be implemented with a direct switch on tags without a ReMatch loop. 
It is only supported through specialized syntax in our library as it differs 
from the Tag encoding only semantically.

%The optimized OCaml compiler \texttt{ocamlopt.opt} that we used to compile the code 
%can be based on different toolsets on some platforms, e.g. Visual \Cpp{} or GCC 
%on Windows. To make the comparison fair we had to make sure that the 
%same toolset was used to compile the \Cpp{} code. We ran the tests 
%on both of the machines described above using the following configurations: 

%\begin{itemize}
%\setlength{\itemsep}{0pt}
%\setlength{\parskip}{0pt}
%\item The tests on a Windows 7 laptop were all based on the \emph{Visual \Cpp{} toolset} 
%      and used \texttt{ocamlopt.opt} version 3.11.0.
%\item The tests on a Linux desktop were all based on the \emph{GCC toolset} and used 
%      \texttt{ocamlopt.opt} version 3.11.2
%\end{itemize}

%\noindent
We used the optimizing OCaml compiler \texttt{ocamlopt.opt} version 3.11.0 working 
under the Visual \Cpp{} toolset as well as the Glasgow Haskell Compiler version 
7.0.3 (with -O switch) working under the MinGW toolset. The \Cpp{} applications 
were compiled with Visual \Cpp{} as well and all the tests were  
performed on the Windows 7 laptop. Similar to comparison with visitors,
the timing results presented in Figure~\ref{fig:OCamlComparison} are averaged 
over 101 measurements and show the number of seconds it took to perform a 
1,000,000 decompositions within our sequential benchmark. We compare here time 
and not cycles, as that was the only common measurement in all three 
environments.

\begin{figure}[htbp]
  \centering
    \includegraphics[width=0.47\textwidth]{OCamlComparison.pdf}
  \caption{Performance comparison with OCaml \& Haskell}
  \label{fig:OCamlComparison}
\end{figure}

We can see that the use of specialized syntax on a closed/sealed hierarchy can 
match the speed of, and even be four times faster than, the code generated by 
the native OCaml compiler. Once we go for an open solution, we become about 
30-50\% slower. 

\subsection{Dealing with real-world class hierarchies}
\label{sec:hierarchies}

For this experiment, we used a class hierarchy benchmark previously used in the 
literature to study efficiency of type inclusion testing and dispatching 
techniques~\cite{Vitek97,Krall97nearoptimal,PQEncoding,Ducournau08}.
We use the names of each benchmark from Vitek et al~\cite[Table 2]{Vitek97}, 
since the set of benchmarks we were working with was closest (though not exact) 
to that work.

While not all class hierarchies originated from \Cpp{}, for this experiment it 
was more important for us that the hierarchies were man-made. While converting 
the hierarchies into \Cpp{}, we had to prune inaccessible base classes (direct base  
class that is already an indirect base class) when used with repeated 
inheritance in order to satisfy semantic requirements of \Cpp{}. We maintained 
the same number of virtual functions present in each class as well as the number 
of data members; the benchmarks, however, did not preserve the exact types of those.
The data in Figure~\ref{fig:benchmarks} shows various parameters of the class 
hierarchies in each benchmark, after their adoption to \Cpp{}. 

\begin{figure}[htbp]
\footnotesize
\begin{tabular}{@{ }l@{ }||@{ }l@{ }|@{ }r@{ }|@{ }r@{ }|@{ }r@{ }|@{ }r@{ }|@{ }r@{ }|@{ }r@{ }|@{ }l@{ }|@{ }r@{ }|@{ }r@{ }|@{ }r@{ }}
\hline % --------------------------------------------------------------------------------------------------
\multicolumn{1}{@{}c@{}||}{\multirow{2}{*}{\tiny{\textsc{Library}}}} & 
\multicolumn{1}{@{ }c@{ }|}{\multirow{2}{*}{\tiny{\textsc{Language}}}} & 
\multicolumn{1}{@{ }c@{ }|}{\multirow{2}{*}{\tiny{\textsc{Classes}}}} &
\multicolumn{1}{@{ }c@{ }|}{\multirow{2}{*}{\tiny{\textsc{Paths}}}} & 
\multicolumn{1}{@{ }c@{ }|}{\multirow{2}{*}{\tiny{\textsc{Height}}}} & 
\multicolumn{1}{@{ }c@{ }|}{\multirow{2}{*}{\tiny{\textsc{Roots}}}} & 
\multicolumn{1}{@{ }c@{ }|}{\multirow{2}{*}{\tiny{\textsc{Leafs}}}} & 
\multicolumn{1}{@{ }c@{ }|}{\multirow{2}{*}{\tiny{\textsc{Both}}}} & 
\multicolumn{2}{@{}c@{}|}{\tiny{\textsc{Parents}}} & 
\multicolumn{2}{@{}c@{}}{\tiny{\textsc{Children}}} \\ \cline{9-12}
     &                             &      &       &    &     &      &     & \multicolumn{1}{@{}c@{}|}{\tiny{\textsc{avg}}} & \multicolumn{1}{@{}c@{}|}{\tiny{\textsc{max}}} & \multicolumn{1}{@{}c@{}|}{\tiny{\textsc{avg}}} & \multicolumn{1}{@{}c@{}}{\tiny{\textsc{max}}} \\
\hline % --------------------------------------------------------------------------------------------------
 DG2 & \tiny{\textsc{Smalltalk}}   &  534 &   534 & 11 &   2 &  381 &   1 & 1    &  1 & 3.48 &  59 \\ % digitalk2         
 DG3 & \tiny{\textsc{Smalltalk}}   & 1356 &  1356 & 13 &   2 &  923 &   1 & 1    &  1 & 3.13 & 142 \\ % digitalk3         
 ET+ & \tiny{\textsc{\Cpp{}}}      &  370 &   370 &  8 &  87 &  289 &  79 & 1    &  1 & 3.49 &  51 \\ % et++              
 GEO & \tiny{\textsc{Eiffel}}      & 1318 & 13798 & 14 &   1 &  732 &   0 & 1.89 & 16 & 4.75 & 323 \\ % geode             
 JAV & \tiny{\textsc{Java}}        &  604 &   792 & 10 &   1 &  445 &   0 & 1.08 &  3 & 4.64 & 210 \\ % java              
 LOV & \tiny{\textsc{Eiffel}}      &  436 &  1846 & 10 &   1 &  218 &   0 & 1.72 & 10 & 3.55 &  78 \\ % lov-object-editor 
 NXT & \tiny{\textsc{Objective-C}} &  310 &   310 &  7 &   2 &  246 &   1 & 1    &  1 & 4.81 & 142 \\ % nextstep          
 SLF & \tiny{\textsc{Self}}        & 1801 & 36420 & 17 &  51 & 1134 &   0 & 1.05 &  9 & 2.76 & 232 \\ % self              
 UNI & \tiny{\textsc{\Cpp{}}}      &  613 &   633 &  9 & 147 &  481 & 117 & 1.02 &  2 & 3.61 &  39 \\ % unidraw           
%    &                             &   51 &    51 &  7 &   1 &   29 &   0 & 1.00 &  1 & 2.27 &   5 \\ % v1-collection     
%    &                             &   18 &    18 &  5 &   1 &   11 &   0 & 1.00 &  1 & 2.43 &   5 \\ % v1-magnitude      
%    &                             &  383 &   383 &  9 &   1 &  244 &   0 & 1.00 &  1 & 2.75 &  86 \\ % v1-object-nometa  
%    &                             &    9 &     9 &  5 &   1 &    4 &   0 & 1.00 &  1 & 1.60 &   2 \\ % v1-set            
%    &                             &   16 &    16 &  7 &   1 &    7 &   0 & 1.00 &  1 & 1.67 &   2 \\ % v1-stream         
%    &                             &   53 &    53 &  8 &   1 &   31 &   0 & 1.00 &  1 & 2.36 &   7 \\ % v1-visualcomponent
 VA2$_a$& \tiny{\textsc{Smalltalk}}& 3241 &  3241 & 14 &   1 & 2582 &   0 & 1    &  1 & 4.92 & 249 \\ % visualage2.all    
 VA2$_k$& \tiny{\textsc{Smalltalk}}& 2320 &  2320 & 13 &   1 & 1868 &   0 & 1    &  1 & 5.13 & 240 \\ % visualage2.kern   
 VW1 & \tiny{\textsc{Smalltalk}}   &  387 &   387 &  9 &   1 &  246 &   0 & 1    &  1 & 2.74 &  87 \\ % visualworks1      
 VW2 & \tiny{\textsc{Smalltalk}}   & 1956 &  1956 & 15 &   1 & 1332 &   0 & 1    &  1 & 3.13 & 181 \\ % visualworks2      
%    &                             &    6 &     9 &  4 &   2 &    1 &   0 & 1.50 &  2 & 1.20 &   2 \\ % vtbl              
\hline % --------------------------------------------------------------------------------------------------
\multicolumn{2}{r|}{\tiny{\textsc{Overalls}}} &15246 & 63963 & 17 & 298 &10877 & 199 & 1.11 & 16 & 3.89 & 323 \\ % Overalls
\hline % --------------------------------------------------------------------------------------------------
\end{tabular}
\caption{Benchmark class hierarchies}
\label{fig:benchmarks}
\end{figure}

The number of paths represents the number of distinct inheritance paths from the 
classes in the hierarchy to the roots of the hierarchy. This number reflects the number of possible subobjects in the 
hierarchy. The roots listed in the table are classes with no base classes. We 
will subsequently use the term \emph{non-leaf} to refer to the possible root of 
a subhierarchy. Leafs are classes with no children, while \emph{both} refers to 
utility classes that are both roots and leafs and thus neither have base nor 
derived classes. The average for the number of parents and the number of 
children were computed only among the classes having at least one parent or at 
least one child correspondingly.

With few useful exceptions, it generally makes sense to apply type switch only 
to non-leaf nodes of the class hierarchy. 71\% of the classes in the entire 
benchmarks suite were leaf classes. Out of the 4369 non-leaf classes, 36\% were 
spawning a subhierarchy of only 2 classes (including the root), 15\% -- a 
subhierarchy of 3 classes, 10\% of 4, 7\% of 5 and so forth. 
Turning this into a cumulative distribution, $a\%$ of subhierarchies had more 
than $b$ classes in them:

%\noindent
\begin{tabular}
{l||@{ }c@{ }|@{ }c@{ }|@{ }c@{ }|@{ }c@{ }|@{ }c@{ }|@{ }c@{ }|@{ }c@{ }|@{ }c@{ }|@{ }c@{ }}
%{l||c|c|c|c|c|c|c|c|c}
$a$ & 1\% & 3\% & 5\% & 10\% & 20\% & 25\% & 50\% & 64\% & 100\% \\
\hline % --------------------------------------------------------------------------------------------------
$b$ & 700 & 110 & 50  & 20   & 10   & 7    & 3    & 2    & 1
\end{tabular}

%1\% of subhierarchies had more than 700 classes in them, 3\% of subhierarchies 
%had more than 110 classes, 5\% of subhierarchies had more than 50 classes, 10\% 
%of subhierarchies had more than 20 classes, 20\% of subhierarchies had more than 
%10 classes, 25\% of hierarchies had more than 7 classes, only 50\% of 
%hierarchies had more than 3 classes and 64\% -- more than 2.

\noindent
These numbers reflect the percentage of use cases one may expect in the real 
word that have a given number of case clauses in them.

For each non-leaf class $A$ we created a function performing a type switch on 
every possible derived class $D_i$ of it as well as itself. The function was 
then executed with every possible subobject $D_i\leftY\sigma_j\rightY A$ it can  
possibly be applied to, given the static type $A$ of the subject. It was 
executed multiple but the same number of times on each subobject to ensure 
uniformity on one side (since we do not have the data about the actual 
probabilities of each subobject in the benchmark hierarchies) as well as let the 
type switch infer the optimal parameters $k$ and $l$ of its cache indexing 
function $H_{kl}^V$. We then plotted a point in chart of Figure~\ref{fig:prob} 
relating 2 characteristics of each of the 4396 type switches tested: the optimal 
computed probability of conflict $p$ achieved by the type switch and the number 
of subobjects $n$ that came through that type switch. The actual frequencies of 
collisions were within one tenth of a percentage point of the computed 
probabilities, which is why we did not use them in the chart. To account for the 
fact that multiple experiments could have resulted in the same pair $(n,p)$, we 
use a shadow of each point to reflect somewhat the number of experiments 
yielding it.

\begin{figure}[htbp]
  \centering
    \includegraphics[width=0.49\textwidth]{ClassHierarchies.pdf}
  \caption{Probability of conflict in real hierarchies}
  \label{fig:prob}
\end{figure}

The curves on which the results of experiments line up correspond to the fact 
that under uniform distribution of $n$ subobjects, only a finite number of 
different values representing the probability of conflict $p$ is possible. In 
particular, all such values $p=\frac{m}{n}$, where $0 \le m < n$. The number $m$ 
reflects the number of subobjects an optimal cache indexing function $H_{kl}^V$ 
could not allocate their own entry for and we showed in \textsection\ref{sec:moc} 
that the probability of conflict under uniform distribution of $n$ subobjects 
depends only on $m$. The curves thus correspond to graphs of functions 
$y=\frac{m}{x}$ for different values of $m$. The points on the same curve (which 
becomes a line on a log-log plot) all share the same number $m$ of ``extra'' 
vtbl-pointers that optimal cache indexing function could not allocate individual 
entries for.

While it is hard to see from the chart, 87.5\% of all the points on the chart 
lay on the X-axis, which means that the optimal hash function for the 
corresponding type switches had no conflicts at all ($m=0$). In other words, only in 12.5\% 
of cases the optimal $H_{kl}^V$ for the set of vtbl-pointers $V$ coming through 
a given type switch had non-zero probability of conflict. Experiments laying on 
the first curve amount to 5.58\% of subhierarchies and represent the cases in 
which optimal $H_{kl}^V$ had only one ``extra'' vtbl-pointer ($m=1$). 2.63\% of 
experiments had $H_{kl}^V$ with 2 conflicts, 0.87\% with 3 and so forth as shown 
in Figure~\ref{fig:size}($K+1$).

\begin{figure}[htbp]
\small
\begin{tabular}
{@{}c@{}||@{}c@{ }|@{}c@{ }|@{}c@{ }|@{}c@{ }|@{}c@{ }|@{}c@{ }|@{}c@{ }|@{}c@{ }}
\hline % -------------------------------------------------------------------------
  $m$ &       0 &       1 &      2 &      3 &      4 &        5 &      6 & \textgreater 6 \\
\hline % -------------------------------------------------------------------------
$K+1$ & 87.50\% &  5.58\% & 2.63\% & 0.87\% & 0.69\% & 0.69\% & 0.30\% & 1.76\% \\
\hline % -------------------------------------------------------------------------
  $K$ & 72.55\% & 12.27\% & 4.87\% & 2.61\% & 1.42\% & 0.94\% & 0.80\% & 4.55\% 
\end{tabular}
\caption{Percentage of type switches with given number of conflicts ($m$) under different size constraints}
\label{fig:size}
\end{figure}

In cases when the user is willing to trade performance for better space 
efficiency she may restrict $k$ to $[K,K]$ instead of $[K,K+1]$ as discussed in 
\textsection\ref{sec:moc}. We redid all the 4396 experiments under this 
restriction and obtained a similar histogram shown in Figure~\ref{fig:size}($K$).
The average probability of conflict over the entire set increased from 0.011 to 
0.049, while the maximum probability of conflict increased from 0.333 to 0.375. 
The average load factor of the cache expectedly increased from 75.45\% to 82.47\%. 

It is important to understand that the high ratio of cases in which the hash 
function could deliver perfect indexing does not indicate that the hash function 
we used is better than other hash functions. It does indicate instead that the 
values representing vtbl-pointers in a given application are not random at all and 
are particularly suitable for such a hash function.

\subsection{Refactoring an existing visitors based application}
\label{sec:qualcmp}

For this experiment, we reimplemented a visitor based \Cpp{} pretty printer for 
Pivot\cite{Pivot09} using \emph{Mach7}. The Pivot's class hierarchy 
consists of 154 node kinds representing various entities in the \Cpp{} program. The 
original code had 8 visitor classes each handling 5, 7, 8, 10, 15, 17, 30 and 63 
cases, which we turned into 8 match statements with corresponding numbers of 
case clauses. Most of the rewrite was performed by sed-like replaces that 
converted visit methods into respective case-clauses. In several cases we had to 
manually reorder case-clauses to avoid redundancy as visit-methods for base classes 
were typically coming before the same for derived, while for type switching we 
needed them to come after due to first-fit semantics. Redundancy checking 
support provided by \emph{Mach7} and discussed in \textsection\ref{sec:redun} was invaluable in finding all such cases.

During this refactoring we have made several simplifications that became obvious 
in pattern-matching code, but were not in visitors code because of control 
inversion. Simplifications that were applicable to visitors code were eventually 
integrated into visitors code as well to make sure we do not compare 
algorithmically different code. In any case we were making sure that both 
approaches regardless of simplifications were producing byte-to-byte the same 
output as the original pretty printer we started from.

The size of executable for pattern-matching approach was smaller than that for 
visitors. So was also the source code. We extracted from both sources the 
functionality that was common to them and placed it in a separate translation 
unit to make sure it does not participate in the comparison. We kept all the 
comments however that were eqaully applicable to code in either approach.

Both pretty printers were executed on a set of header files from the \Cpp{} 
standard library and the produced output of both program was byte-to-byte the same. 
We timed execution of the pretty printing phase (not including loading and termination 
of the application or parsing of the source file) and observed that on small 
files (e.g. those from C run-time library and few small \Cpp{} files) 
visitors-based implementation was faster because the total number of nodes in 
AST and thus calls did not justify our set-up calls. In particular, 
visitor-based implementation of the pretty printer was faster on files of 44--588  
lines of code, with average 136 lines per those inputs, where visitors win. On 
these input files, visitors are faster by 1.17\%--21.42\% with an average speed-up of 
8.75\%. Open type switch based implementation of the pretty printer was faster on 
files of 144--9851 lines of code, with average 3497 lines per those input files, 
where open type switch wins. On these inputs, open type switch is faster by 0.18\% -- 32.99\% 
with an average speed-up of 5.53\%.

Figure~\ref{fig:mem} shows memory usage as well as cache hits and misses for 
the run of our pretty printer on \code{<queue>} standard library 
header, which had the largest LOC count after preprocessing in our test set.

\begin{figure}[htbp]
  \centering
    \includegraphics[width=0.49\textwidth]{Memory.pdf}
  \caption{Memory usage in real application}
  \label{fig:mem}
\end{figure}

The bars represent the total size of memory in bytes each of the 8 match 
statements (marked A-H) used. Information $[n/c]$ next to the letter indicates the 
actual number of different subobjects (i.e. vtbl-pointers) $n$ that came through 
that match statement, and the number of case clauses $c$ the match statement had 
(the library uses $c$ as an estimate of $n$). $n$ is also the number of cases the 
corresponding match statement had to be executed sequentially (instead of a 
direct jump).

The lower part of each bar (with respect to dividing line) corresponds to the memory used by cache, while the 
upper part -- to the memory used by the hash table. The ratio of the darker
section of each part to the entire part indicates the load factors 
of cache and hash-table respectively. The black box additionally indicates the 
proportion of cache entries that are allocated for only one vtbl-pointer and 
thus never result in a cache miss. %The non-transparent part without black box 
%represents the percentage of vtbl-pointers that have to share their cache entry 
%with at least one other vtbl-pointer and thus may result in collisions during 
%access.

The actual number of hits and misses for each of the match statements is 
indicated on top of the corresponding column. The sum of them is the total 
number of calls made. %Hits indicate situation when we found entry in cache and 
%didn't have to make roundtrip to the hash-table to get it. Misses indicate the 
%number of cases during actual run we had to pick the entry from the hash table 
%and update the cache with it. 
The number of misses is always larger than or equal to $n$ since we need to 
execute the switch sequentially on each of them once in order to memoize the 
outcome.

The library always preallocates memory for at least 8 subobjects to avoid 
unnecessary recomputations of optimal parameters $k$ and $l$ -- this is the case 
with the last 3 match statements. In all other cases it allocates the 
memory proportional to $2^{K+1}$ where $2^{K-1} < \max(n,c) \le 2^{K}$. We make 
$c$ a parameter, because in a library setting $n$ is not known up front and 
estimating it with $c$ allows us to avoid unnecessary recomputations of $l$ and 
$k$ even further. 

The table does not have to be hash table and can be implemented with 
any other container i.e. sorted vector, map etc. that let us find quickly by a given 
vtbl-pointer the data associated with it. In fact we provide a slightly less 
efficient caching container that avoids the table altogether, thus significantly 
reducing the memory requirements instead.

%During this refactoring we have made several simplifications that became obvious 
%in pattern-matching code, but were not in visitors code because of control 
%inversion. Simplifications that were applicable to visitors code were eventually 
%integrated into visitors code as well to make sure we do not compare 
%algorithmically different code. In any case we were making sure that both 
%approaches regardless of simplifications were producing byte-to-byte the same 
%output as the original pretty printer we started from.

%The size of executable for pattern-matching approach was smaller than that for 
%visitors. So was also the source code. We extracted from both sources the 
%functionality that was common to them and placed it in a separate translation 
%unit to make sure it does not participate in the comparison. We kept all the 
%comments however that were eqaully applicable to code in either approach.
%
Note that the visitors involved in the pretty printer above did not use 
forwarding: since all the \Cpp{} constructs were handled by the printer, every 
visit-method was overriden from those statically possible based on the static 
type of the argument.

Listing parameter for a case clause always causes access to member. Best hope is 
that compiler will eliminate it if it is not needed. At the moment we do not 
have means to detect empty macro arguments or \_.

In general from our rewriting experience we will not recommend rewriting 
existing visitor code with pattern matching for the simple reason that pattern 
matching code will likely follow the structure already set by the visitors. 
Pattern matching was most effective when writing new code, where we could design 
the structure of the code having the pattern-matching facility in our toolbox.

\subsection{Limitations}
\label{sec:lim}

Currently the definition of each class used in a case clause must be visible to 
the compiler because \code{dynamic_cast} operator used in the type switch does 
not allow incomplete types as a target type. For particularly large type 
switches (e.g. \textgreater 1000 case clauses) this may easily reach some 
compiler limitations. Both GCC and Visual \Cpp{}, for example, could not generate 
object files for such translation units simply because the sheer size of v-tables 
and other compiler data in it were exceeding the limits. The problem is not 
specific to our technique though and allowing \code{dynamic_cast} on classes 
that were declared but not defined yet would solve the problem.

While it might be reasonable to expect from linkers to layout v-tables close 
to each other -- the property that makes our hashing function efficient -- they 
are not required to do so. We believe, nevertheless, that should our approach 
become popular through the library implementation, its compiler implementation 
will encourage compiler vendors to enforce the property in order to keep the 
type switching fast.

\section{Related Work} %%%%%%%%%%%%%%%%%%%%%%%%%%%%%%%%%%%%%%%%%%%%%%%%%%%%%%%%%
\label{sec:rw}

There are two main approaches to compiling pattern matching code: the first is 
based on \emph{backtracking automata} and was introduced by Augustsson\cite{}, 
the second is based on \emph{decision trees} and is attributed in the literature 
to Dave MacQueen and Gilles Kahn in their implementation of Hope compiler \cite{}.
Backtracking approach usually generates smaller code, while decision tree 
approach produces faster code by ensuring that each primitive test is only 
performed once. Neither of the approaches addresses specifically type patterns 
or type switching and simply assumes presence of a primitive operation capable 
of performing type tests.

Memoization device we proposed is not specifically concerned with compiling 
pattern matching and can be used independently. In particular it can be combined 
with either backtracking or decision tree approaches to avoid subsequent 
decisions on datum that has already been seen.

%xxxxxxxxxxxxxxxxxxxxxxxxxxxxxxxxxxxxxxxxxxxxxxxxxxxxxxxxxxxxxxxxxxxxxxx

\emph{Extensible Visitors with Default Cases}~\cite[\textsection 
4.2]{Zenger:2001} attempt to solve the extensibility problem of visitors; 
however, the solution, after 
remapping it onto C++, has problems of its own. The visitation interface 
hierarchy can easily be grown linearly (adding new cases for the new classes in 
the original hierarchy each time), but independent extensions by different  
authorities require developer's intervention to unify them all, before they can 
be used together. This may not be feasible in environments that use dynamic 
linking. To avoid writing even more boilerplate code in new visitors, the 
solution would require usage of virtual inheritance, which typically has 
an overhead of extra memory dereferencing. On top of the double dispatch already 
present in the visitor pattern, the solution will incur two additional virtual 
calls and a dynamic cast for each level of visitor extension. Additional double 
dispatch is incurred by forwarding of default handling from a base visitor to a 
derived one, while the dynamic cast is required for safety and can be replaced 
with a static cast when the visitation interface is guaranteed to be grown linearly 
(extended by one authority only). Yet another virtual call is required to be 
able to forward computations to subcomponents on tree-like structures to the 
most derived visitor. This last function lets one avoid the necessity of using 
the heap to allocate a temporary visitor through the \emph{Factory Design 
Pattern}~\cite{DesignPatterns1993} used in the \emph{Extensible Visitor} solution 
originally proposed by Krishnamurti, Felleisen and Friedman~\cite{Krishnamurthi98}.

In order to address the expression problem in Haskell, L\"{o}h and Hinze proposed to 
extend its type system with open data types and open functions~\cite{LohHinze2006}.
Their solution allows the user to mark top-level data types and functions as 
open and then provide concrete variants and overloads anywhere in the program. 
Open data types are extensible but not hierarchical, which largely avoids the 
problems discussed here. The semantics of open extension is given by 
transformation into a single module, where all the definitions are seen in one 
place. This is a significant limitation of their approach that prevents it from 
being truly open, since it essentially assumes a whole-program view, which 
excludes any extension via DLLs. As is the case with many other implementations 
of open extensions, the authors rely on the closed world for efficient 
implementation: in their implementation, \emph{``data types can only be entirely 
abstract (not allowing pattern matching) or concrete with all constructors with 
the reason being that pattern matching can be compiled more efficiently if the 
layout of the data type is known completely''}. The authors also believe that 
\emph{there are no theoretical difficulties in lifting this restriction, but it 
might imply a small performance loss if closed functions pattern match on open 
data types}. Our work addresses exactly this problem, showing that it is not 
only theoretically possible but also practically efficient and in application to 
a broader problem.

Polymorphic variants in OCaml~\cite{garrigue-98} allow the addition of new variants 
later. They are simpler, however, than object-oriented extensions, as they do not 
form subtyping between variants themselves, but only between combinations of them. 
This makes an important distinction between \emph{extensible sum types} like 
polymorphic variants and \emph{extensible hierarchical sum types} like classes.
An important property of extensible sum types is that each value of the 
underlying algebraic data type belongs to exactly one disjoint subset, tagged with 
a constructor. The \emph{nominative subtyping} of object-oriented languages does 
not usually have this disjointness making classes effectively have multiple 
types. In particular, the case of disjoint constructors can be seen as a 
degenerated case of a flat class hierarchy among the multitude of possible class 
hierarchies.

\emph{Tom} is a pattern-matching compiler that can be used together with Java, C or 
Eiffel to bring a common pattern matching and term rewriting syntax into the 
languages\cite{Moreau:2003}. It works as a preprocessor that transforms 
syntactic extensions into imperative code in the target language. Tom is quite 
transparent as to the concrete target language used and can potentially be 
extended to other target languages besides the three supported now. In 
particular, it never uses any semantic information of the target language during 
the compilation process and it does not inspect nor modify the source language 
part (their preprocessor is only aware of parenthesis and block delimiters of 
the source language). Tom has a sublanguage called Gom that can be used to 
define algebraic data types in a uniform mannaer, which their preprocessor then 
transforms into conrete definitions in the target language. Alternatively, the 
user can provide mappings to his own data structures that the preprocessor will 
use to generate the code.

In comparison to our approach Tom has much bigger goals. The combination of 
pattern matching, term rewriting and strategies turns Tom into a 
tree-transformation language similar to Stratego/XT, XDuce and others. 
The main accent is made on expressivity and the speed of development, which 
makes one often wonder about the run-time complexity of the generated code.
Tom's approach is also prone to general problems of any preprocessor based 
solution\cite[\textsection 4.3]{SELL}. For example, when several preprocessors 
have to be used together, each independent extension may not be able to 
understand the other's syntax, making it impossible to form a toolchain.
A library approach we follow avoids most of these problems by relying only on a 
standard C++ compiler. It also lets us employ semantics of the language within 
patterns: e.g. our patterns work directly on underlying user-defined data 
structures, largely avoiding abstraction penalties. A tighter integration with 
the language semantics also makes our patterns first-class citizens that can be 
composed and passed to other functions. The approach we take to type switching 
can also be used by Tom's preprocessor to implement type patterns efficiently -- 
similarly to other object-oriented languages, Tom's handling of them is based on 
highly inefficient \code{instanceof} operator and its equivalents in other 
languages.

Pattern matching in Scala~\cite{Scala2nd} also allows type patterns and thus type 
switching. The language supports extensible and hierarchical data types, but 
their handling in a type switching constructs varies. Sealed classes are handled 
with an efficient switch over all tags, since sealed classes cannot be extended. 
Classes that are not sealed are similarly approached with a combination of an 
\code{InstanceOf} operator and a decision tree~\cite{EmirThesis}.

%An example would be our generalized n+k patterns where we 
%can turn any invertible function even user defined into a pattern.

There has been previous attempts to use pattern matching with the Pivot 
framework that we used to experiment with our library. In his dissertation, 
Pirkelbauer devised a pattern language capable of representing various entities 
in a C++ program. The patterns were then translated with a tool into a set of 
visitors implementing the underlying pattern matching 
semantics\cite{PirkelbauerThesis}. Earlier, Cook et al used expression templates 
to implement a query language for Pivot's Internal Program Representation 
\cite{iql04}. While their work was built around a concrete class hierarchy 
letting them put some semantic knowledge about concrete classes into the 
The principal difference of their work from this work is that 
authors were essentially creating a pattern matcher for a given class hierarchy 
and thus could take the semantics of the entities represented by classes in the 
hierarchy into account. Our approach is parametrized over class hierarchy and 
thus provides a rather lower level pattern-matching functionality that lets one 
simplify work with that hierarchy.  One can think of it as a generalized 
dynamic\_cast. To be continued...


\section{Future Work} %%%%%%%%%%%%%%%%%%%%%%%%%%%%%%%%%%%%%%%%%%%%%%%%%%%%%%%%%%
\label{sec:fw}

Using a library implementation was essential for experimentation and for being able to
test our ideas on multiple production-quality compiler systems.
However, now we hope to re-implement our ideas in a compiler.
This would allow us to improve further surface syntax, diagnostics, and performance.

In the nearest future, we would like to make our library to be safe and efficient 
in a multi-threaded environment. Currently it relies heavily on static variables 
and global state, which will have problems in a multi-threaded environment. 

The match statement that we presented here deals with only one subject at the 
moment, but we believe that our memoization device, along with the vtable pointer memoization 
technique we presented, can cope reasonably efficiently with multiple subjects. 
Their support will make our library more general by addressing asymmetric 
multiple dispatch.

We would also like to experiment with other kinds of cache indexing functions in 
order to decrease the frequency of conflicts, especially those coming from the use 
of dynamically-linked libraries.

Containers as described by the standard C++ library do not have the implicit 
recursive structure present in lists, sequences and other recursive data 
structures of functional languages. Viewing them as such with view will likely 
have a significant performance overhead, not usually affordable in the kind of 
applications C++ is used for. We therefore would like to experiment with some 
pattern matching alternatives that will let us work with STL containers 
efficiently yet expressively as in functional languages.
%
%Last but not least we would like to look at providing support for alternative 
%matching semantics.
%
%
\section{Conclusions} %%%%%%%%%%%%%%%%%%%%%%%%%%%%%%%%%%%%%%%%%%%%%%%%%%%%%%%%%%
\label{sec:cc}

We present a pattern-matching library for C++ provides fairly standard
pattern-matching facilities. Our solution is 
non-intrusive and can be retroactively applied to any polymorphic or tagged 
class hierarchy. It also provides a uniform notation to these different 
encodings of algebraic and extensible hierarchical data types in C++.

We generalize n+k patterns to arbitrary expressions by letting the user define 
the exact semantics of such patterns. Our approach is more general than traditional approaches 
as it does not require an
equational view of such patterns. It also avoids hardcoding the 
exact semantics of n+k patterns into the language. 

We used the library to rewrite existing code that relied heavily on the 
visitor design pattern.
Our pattern matching code was much shorter (both source and object code), 
simpler, easier to maintain, comprehend, and faster. 
This confirmed our view of the visitor pattern as a clever workaround,
rather than a good solution to a fundamental problem.
The library approach was essential 
for experimentation in the context of real programs and for delivering 
performance comparable with or superior to conventional techniques in the 
context of industrial compilers.

The work presented here is only the beginning of our research on pattern 
matching for C++. We would like to experiment with other kinds of patterns, 
including those defined by the user; look at the interaction of patterns with 
other facilities in the language and the standard library; make
views less ad hoc etc. For example, standard containers in C++ do not have the 
implicit recursive structure present in data types of functional languages and 
viewing them as such with views would incur significant overheads. We will
experiment with very general patterns as first-class citizens.

Our generalization of n+k patterns depends on the properties of types involved 
in the expression. This should let us experiment not only with generic 
functions, but also with their generic inversions in the form of solvers. As 
more C++11 features become available in compilers it will also be interesting to 
look at how use of these features affects the ease of use, performance, 
readability, writability and debugging of the library and the user code that 
uses it.

Type switching is an open alternative to the visitor design pattern that overcomes 
the restrictions, inconveniences, and difficulties in teaching and using 
visitors. Our implementation significantly
outperforms the visitor design pattern in most cases and roughly equals it otherwise.
This is the case even though we use a library implementation and highly optimized
production-quality compilers. An important benefit of our solution is that it does not 
require any changes to the \Cpp{} object-model or require any computations at load 
time.

To provide a complete solution, we use the same syntax for closed sets of types, where our
performance roughly equals the equivalent built-in features in functional languages,
such as Haskell and OCaml.

We prove the uniqueness of vtbl-pointers in the presence of RTTI. This is 
potentially useful in other compiler optimizations that depend on the 
identity of subobjects. Our memoization device can also become valuable in 
optimizations that require mapping run-time values to execution paths, 
and is especially useful in library setting.

We describe three techniques that can be used to implement type switching, type 
testing, pattern matching, predicate dispatching, and other facilities that 
depend on the run-time type of an argument as well as demonstrate their efficiency.

The \emph{Memoization Device} is an optimization technique that maps run-time values 
to execution paths, allowing to take shortcuts on subsequent runs with the same 
value. The technique does not require code duplication and in typical cases adds 
only a single indirect assignment to each of the execution paths. It can be 
combined with other compiler optimizations and is particularly suitable for use 
in a library setting.

The \emph{Vtable Pointer Memoization} is a technique based on memoization device that 
employs uniqueness of virtual table pointers to not only speed up execution, but 
also properly uncover the dynamic type of an object. This technique is a 
backbone of our fast type switch as well as memoized dynamic cast optimization.

The \emph{TPL Dispatcher} is yet another technique that can be used to 
implement best-fit type switching on tagged classes. The technique has its pros 
and cons in comparison to vtable pointer memoization, which we discuss in the paper.

These techniques can be used in a compiler and library setting, and support well 
separate compilation and dynamic linking. They are open to class extensions and 
interact well with other \Cpp{} facilities such as multiple inheritance and 
templates. The techniques are not specific to \Cpp{} and can be adopted in other 
languages for similar purposes.

Using the above techniques, we implemented a library for efficient type switching 
in \Cpp{}. We used the library to rewrite existing code that relied heavily on 
visitors, and discovered that the resulting code became much shorter, simpler, 
and easier to maintain and comprehend.

We used the library to rewrite existing code that relied heavily on 
visitors, and discovered that the resulting code became much shorter, simpler, and easier 
to maintain and comprehend.


% \acks

% We would like to thank Xavier Leroy and Luc Maranget for valuable feedback and 
% suggestions for improvements on the initial idea, Gregory Berkolaiko for ideas 
% related to minimization of conflicts, Jaakko Jarvi for assistance in comparison 
% to other languages, Andrew Sutton, Peter Pirkelbauer and Abe Skolnik for helpful 
% discussions and comments to numerous rewrites of this paper. 
% We also benefitted greatly from insightful comments by anonymous reviewers on 
% earlier revisions of this work. We would also like to thank Karel Driesen for 
% letting us use his class hierarchies benchmark for this work.

\bibliographystyle{abbrvnat}
\bibliography{mlpatmat}
\end{document}
