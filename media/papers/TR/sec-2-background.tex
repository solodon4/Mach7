\section{Background} %%%%%%%%%%%%%%%%%%%%%%%%%%%%%%%%%%%%%%%%%%%%%%%%%%%%%%%%%%%
\label{sec:bg}

Pattern matching in the context of a programming language was first introduced 
in a string manipulation language SNOBOL\cite{SNOBOL64}. Its fourth 
reincarnation SNOBOL4 had patterns as first-class data types providing 
operations of concatenation and alternation on them\cite{SNOBOL71}. The first 
reference to a pattern-matching construct that resembles the one found in 
statically typed functional languages today is usually attributed to Burstall 
and his work on structural induction\cite{Burstall69provingproperties}.

In the context of object-oriented programming, pattern matching has been first 
explored in Pizza programming language\cite{Odersky97pizzainto}. These efforts 
have been continued in Scala\cite{Scala2nd} and together with notable work of 
Burak Emir on \emph{Object-Oriented Pattern Matching}\cite{EmirThesis} have 
resulted in incorporation of pattern matching into the language.

%The first tree based pattern matching methods were found in Fred McBride's 
%extension of LISP in 1970.

%ML and Haskell further popularized pattern matching ...

Pattern matching has been closely related to \emph{algebraic data types} and 
\emph{equational reasoning} since the early days of functional programming.
In languages like ML and Haskel an \emph{Algebraic Data Type} is a data type 
each of whose values is picked from a disjoint sum of (possibly recursive) data 
types, called \emph{variants}. Each of the variants is marked with a unique 
symbolic constant called \emph{constructor}, while the set of all constructors 
of a given type is called \emph{signature}. Constructors provide a convenient 
way of creating a value of its variant type as well as a way of discriminating 
its variant type from the algebraic data type through pattern matching.

Algebraic data type \codeocaml{expr} from Section~\ref{sec:intro} consists of 5 
variants, marked with constructors \codeocaml{Value}, \codeocaml{Plus}, 
\codeocaml{Minus}, \codeocaml{Times} and \codeocaml{Divide} respectively. 
Constructor \codeocaml{Value} expects a value of type \codeocaml{int} during 
construction, as well as any pattern that admits values of type \codeocaml{int} 
during decomposition through pattern matching. Similarly, the other four 
constructors expect a value of a cartesian product of two \codeocaml{expr} 
types during construction, as well as any pattern that would admit a value of 
such type during decomposition.

Algebraic data types can be parameterized and recursive, as demonstrated by the 
following Haskell code that defines a binary tree parameterized on type 
\codehaskell{k} of keys and type \codehaskell{d} of data stored in the nodes:

\begin{lstlisting}[language=Haskell]
data Tree k d = Node k d (Tree k d) (Tree k d) | Leaf
\end{lstlisting}

\noindent
Naturally, they can be decomposed in a generic algorithm like the function 
\code{find} below, defined through case analysis on the tree's structure:

\begin{lstlisting}[language=Haskell]
find :: (Ord k) => k -> Tree k d -> Maybe d
find i Leaf = Nothing
find i (Node key item left right) = 
    if i == key 
    then Just item 
    else 
        if i [<] key 
        then find i left 
        else find i right
\end{lstlisting}

\noindent
The set of values described by such an algebraic data type is defined 
inductively as the least set closed under constructor functions of its variants.
Algebraic data types draw their name from the practice of using case distinction 
in mathematical function definitions and proofs that involve \emph{algebraic 
terms}.

One of the main differences of algebraic data types from classes in 
object-oriented languages is that an algebraic data type definition is 
\emph{closed} because it fixes the structure of its instances once and for all. 
Once we have listed all the variants a given algebraic data type may have we 
cannot extend it with new variants without modifying its definition. This is not 
the case in object-oriented languages, where classes are \emph{open} to 
extension through subclassing. Notable exceptions to this restriction in 
functional community are \emph{polymorphic variants} in OCaml\cite{garrigue-98} 
and \emph{open data types} in Haskell\cite{LohHinze2006}, which allow addition 
of new variants later. These extensions, however, are simpler than object-oriented 
extensions as neither polymorphic variants nor open data types form subtyping 
relation between themselves: open data types do not introduce any subtyping 
relation, while the subtyping relation on polymorphic variants is a 
\emph{semantic subtyping} similar to that of XDuce\cite{HosoyaPierce2000}, which 
is based on the subset relation between values of the type. In either case they 
maintain the important property that each value of the underlying algebraic data 
type belongs to exactly one disjoint subset tagged with a constructor. The 
\emph{nominative subtyping} of object-oriented languages does not usually have 
this disjointness making classes effectively have multiple types. In particular, 
the case of disjoint constructors can be seen as a degenerated case of a flat 
class hierarchy among the multitude of possible class hierarchies.

Closedness of algebraic data types is particularly useful for reasoning about 
programs by case analysis and allows the compiler to perform an automatic 
\emph{incompleteness} check -- test of whether a given \emph{match statement} 
covers all possible cases. Similar reasoning about programs involving extensible 
data types is more involved as we are dealing with potentially open set of 
variants. \emph{Completness} check in such scenario reduces to checking presence 
of a case that handles the static type of the subject. Absence of such a case,
however, does not necessarily imply incompletness, only potential incompletness, 
as the answer will depend on the actual set of variants available at run-time.

A related notion of \emph{redundancy} checking arises from the 
tradition of using \emph{first-fit} strategy in pattern matching. It warns the 
user of any \emph{case clause} inside a match statement that will 
never be entered because of a preceding one being more general. Object-oriented 
languages, especially C++, typically prefer \emph{best-fit} strategy (e.g. for 
overload resolution and class template specialization) because it is not prone 
to errors where semantics of a statement might change depending on the ordering 
of preceding definitions. The notable exception in C++ semantics that prefers 
the \emph{first-fit} strategy is ordering of \code{catch} handlers of a 
\code{try}-block. Similarly to functional languages the C++ compiler will perform 
\emph{redundancy} checking on catch handlers and issue a warning that lists the 
redundant cases. We use this property of the C++ type system to perform redundancy 
checking of our match statements in \textsection\ref{sec:redun}.

The patterns that work with algebraic data types we have seen so far are 
generally called \emph{tree patterns} or \emph{constructor patterns}. Their 
analog in object-oriented languages is often referred to as \emph{type pattern} 
since it may involve type testing and type casting. Special cases of these patterns 
are \emph{list patterns} and \emph{tuple patterns}. The former lets one split a 
list into a sequence of elements in its beginning and a tail with the help of 
list constructor \codehaskell{:} and an empty list constructor \codehaskell{[]} 
e.g. \codehaskell{[x:y:rest]}. The latter does the same with tuples using tuple
constructor \codehaskell{(,,...,)} e.g. \codehaskell{([x:xs],'b',(1,2.0),"hi",True)}.

Pattern matching is not used solely with algebraic data types and can equally 
well be applied to built-in types. The following Haskell code defines factorial 
function in the form of equations:

\begin{lstlisting}[language=Haskell]
factorial 0 = 1
factorial n = n * factorial (n-1)
\end{lstlisting}

\noindent
Here 0 in the left hand side of the first \emph{equation} is an example of a 
\emph{value pattern} (also known as \emph{constant pattern}) that will only 
match when the actual argument passed to the function factorial is 0. The 
\emph{variable pattern} \codehaskell{n} (also referred to as \emph{identifier 
pattern}) in the left hand side of the second equation will match any value, 
\emph{binding} variable \codehaskell{n} to that value in the right hand side of 
equation. Similarly to variable pattern, the \emph{wildcard pattern} \codehaskell{_} 
will match any value, neither binding it to a variable nor even obtaining it. 
Value patterns, variable patterns and wildcard patterns are  
generally called \emph{primitive patterns}. Patterns like variable and wildcard 
patterns that never fail to match are called \emph{irrefutable}, in contrast to 
\emph{refutable} patterns like value patterns, which may fail to match.

In Haskell 98\cite{Haskell98Book} the above definition of factorial could also 
be written as:

\begin{lstlisting}[language=Haskell]
factorial 0 = 1
factorial (n+1) = (n+1) * factorial n
\end{lstlisting}

\noindent
The \codehaskell{(n+1)} pattern in the left hand side of equation is an example of 
\emph{n+k pattern}. According to its informal semantics ``Matching an $n+k$ 
pattern (where $n$ is a variable and $k$ is a positive integer literal) against 
a value $v$ succeeds if $v \ge k$, resulting in the binding of $n$ to $v-k$, and 
fails otherwise''\cite{haskell98}. n+k patterns were introduced into Haskel to 
let users express inductive functions on natural numbers in much the same way as 
functions defined through case analysis on algebraic data types. Besides 
succinct notation, such language feature could facilitate automatic proof of 
termination of such functions by compiler. Peano numbers, used as an analogy to 
algebraic data type representation of natural numbers, is not always the best 
abstraction for representing other mathematical operations however. This,  
together with numerous ways of defining semantics of generalized n+k patterns 
were some of the reasons why the feature was never generalized in Haskell to 
other kinds of expressions, even though there were plenty of known applications. 
Moreover, numerous debates over semantics and usefulness of the feature 
resulted in n+k patterns being removed from the language altogether in Haskell 
2010 standard\cite{haskell2010}. Generalization of n+k patterns, called 
\emph{application patterns} has been studied by Nikolaas N. Oosterhof in his 
Master's thesis\cite{OosterhofThesis}. Application patterns essentially treat 
n+k patterns as equations, while matching against them attempts to solve or 
validate the equation.

While n+k patterns were something very few languages had, another common feature of 
many programming languages with pattern matching are guards. A \emph{guard} 
is a predicate attached to a pattern that may make use of the variables bound in 
it. The result of its evaluation will determine whether the case clause and the 
body associated with it will be \emph{accepted} or \emph{rejected}. The 
following OCaml code for $exp$ language from Section~\ref{sec:intro} defines the 
rules for factorizing expressions $e_1e_2+e_1e_3$ into $e_1(e_2+e_3)$ and 
$e_1e_2+e_3e_2$ into $(e_1+e_3)e_2$ with the help of guards spelled out after 
keyword \codeocaml{when}:

\begin{lstlisting}[language=Caml,keepspaces,columns=flexible]
let factorize e =
    match e with
      Plus(Times(e1,e2), Times(e3,e4)) when e1 = e3 
          -> Times(e1, Plus(e2,e4))
    | Plus(Times(e1,e2), Times(e3,e4)) when e2 = e4 
          -> Times(Plus(e1,e3), e4)
    |   e -> e
    ;;
\end{lstlisting}

\noindent
One may wonder why we could not simply write the above case clause as 
\codeocaml{Plus(Times(e,e2), Times(e,e4))} to avoid the guard? Patterns that 
permit use of the same variable in them multiple times are called 
\emph{equivalence patterns}, while the requirement of absence of such patterns 
in a language is called \emph{linearity}. Neither OCaml nor Haskell support such 
patterns, while Miranda\cite{Miranda85} as well as Tom's pattern matching 
extension to C, Java and Eiffel\cite{Moreau:2003} supports \emph{non-linear 
patterns}.

The example above illustrates yet another common pattern-matching facility -- 
\emph{nesting of patterns}. In general, a constructor pattern composed of a 
linear vector of (distinct) variables is called a \emph{simple pattern}. The 
same pattern composed not only of variables is called \emph{nested pattern}.
Using nested patterns, with a simple expression in the case clause we could
define a predicate that tests the top-level expression to be tagged with a
\codeocaml{Plus} constructor, while both of its arguments to be marked with 
\codeocaml{Times} constructor, binding their arguments (or potentially pattern 
matching further) respectively. Note that the visitor design pattern does not 
provide this level of flexibility and each of the nested tests might have 
required a new visitor to be written. Nesting of patterns like the one above is 
typically where users resort to \emph{type tests} and \emph{type casts} that in 
case of C++ can be combined into a single call to \code{dynamic_cast}.

Related to nested patterns are \emph{as-patterns} that help one take a value 
apart while still maintaining its integrity. The following rule could have been 
a part of a hypothetical rewriting system in OCaml similar to the one above. Its 
intention is to rewrite expressions of the form $\frac{e_1/e_2}{e_3/e_4}$ into 
$\frac{e_1}{e_2}\frac{e_4}{e_3} \wedge e_2\neq0 \wedge e_3\neq0 \wedge e_4\neq0$.

\begin{lstlisting}[language=Caml]
    | Divide(Divide(_,e2) as x, Divide(e3,e4))
          -> Times(x, Divide(e4, e3))
\end{lstlisting}

\noindent
We introduced a name ``x'' as a synonym of the result of matching the 
entire sub-expression \codeocaml{Divide(_,e2)} in order to refer it without 
recomposing in the right-hand side of the case clause. We omitted the 
conjunction of relevant non-zero checks for brevity, one can see that we will 
need access to \codeocaml{e2} in it however.

Decomposing algebraic data types through pattern matching has an important 
drawback that was originally spotted by Wadler\cite{Wadler87}: they expose 
concrete representation of an abstract data type, which conflicts with the 
principle of \emph{data abstraction}. To overcome the problem he proposed the 
notion of \emph{views} that represent conversions between different 
representations that are implicitly applied during pattern matching. As an 
example, imagine polar and cartesian representations of complex numbers. A user 
might choose polar representation as a concrete representation for the abstract 
data type \codeocaml{complex}, treating cartesian representation as view or vice 
versa:\footnote{We use syntax from Wadler's original paper for this example}

\begin{lstlisting}[language=Haskell,columns=flexible]
complex ::= Pole real real
view complex ::= Cart real real
  in  (Pole r t) = Cart (r * cos t) (r * sin t)
  out (Cart x y) = Pole (sqrt(x^2 + y^2)) (atan2 x y)
\end{lstlisting}

\noindent
The operations then might be implemented in whatever representation is the most 
suitable, while the compiler will implicitly convert representation if needed:

\begin{lstlisting}[language=Haskell,columns=flexible]
  add  (Cart x1 y1) (Cart x2 y2) = Cart (x1 + x2) (y1 + y2)
  mult (Pole r1 t1) (Pole r2 t2) = Pole (r1 * r2) (t1 + t2)
\end{lstlisting}

\noindent
The idea of views were later adopted in various forms in several languages: 
Haskell\cite{views96}, Standard ML\cite{views98}, Scala (in the form of 
\emph{extractors}\cite{EmirThesis}) and F$\sharp$ (under the name of 
\emph{active patterns}\cite{Syme07}).

%Views in functional programming languages [92, 71] are conversions from one data type to
%another that are implicitly applied in pattern matching. They play a role similar to extractors
%in Scala, in that they permit to abstract from the concrete data-type of the matched objects.
%However, unlike extractors, views are anonymous and are tied to a particular target data
%type.

Logic programming languages like Prolog take pattern matching to even greater 
level. The main difference between pattern matching in logic languages and 
functional languages is that functional pattern matching is a ``one-way'' 
matching where patterns are matched against values, possibly binding some 
variables in the pattern along the way. Pattern matching in logic programming is 
``two-way'' matching based on \emph{unification} where patterns can be matched 
against other patterns, possibly binding some variables in both patterns and 
potentially leaving some variables \emph{unbound} or partially bound -- i.e. 
bound to patterns. A hypothetical example of such functionality can be matching 
a pattern \codeocaml{Plus(x,Times(x,1))} against another pattern 
\codeocaml{Plus(Divide(y,2),z)}, which will result in binding \codeocaml{x} to a 
\codeocaml{Divide(y,2)} and \codeocaml{z} to \codeocaml{Times(Divide(y,2),1)} 
with \codeocaml{y} left unbound, leaving both \codeocaml{x} and \codeocaml{z} 
effectively a pattern.

\subsection{Expression Templates}

Interestingly enough C++ has a pure functional sublanguage in it that has a 
striking similarity to ML and Haskell. The sublanguage in question is template 
facilities of C++ that has been shown to be Turing 
complete\cite{veldhuizen:templates_turing_complete}. 

Haskell definition of \code{factorial} we saw earlier can be rewritten in 
template sublanguage of C++ as following:

\begin{lstlisting}
template <int N> 
    struct factorial { enum { result = N*factorial<N-1>::result }; };
template <>
    struct factorial<0> { enum { result = 1 }; };
\end{lstlisting}

\noindent
One can easily see similarity with equational definitions in Haskell, with the 
exception that more specific cases (specialization for 0) have to follow the 
general definition in C++. The main difference between Haskell definition and 
its C++ counterpart is that the former describes computations on \emph{run-time 
values}, while the latter can only work with \emph{compile-time values}.

Turns out we can even express our $exp$ language using this functional 
sublanguage:

\begin{lstlisting}
template <class T>
struct value {
    value(const T& t) : m_value(t) {}
    T m_value;
};

template <class T>
struct variable {
    variable() : m_value() {}
    T m_value;
};

template <typename E1, typename E2>
struct plus {
    plus(const E1& e1, const E2& e2) : m_e1(e1), m_e2(e2) {}
    const E1 m_e1; const E2 m_e2;
};

// ... definitions of other expressions
\end{lstlisting}

\noindent The idea is that expressions can be composed out of subexpressions, 
whose shape (type) is passed as arguments to above templates. Explicit 
description of such expressions is very tedious however and is thus never 
expressed directly, but as a result of corresponding operations: 

\begin{lstlisting}[keepspaces,columns=flexible]
template <typename T>
    value<T> val(const T& t) { return value<T>(t); }
template <typename E1, typename E2>
    plus<E1,E2> operator+(const E1& e1, const E2& e2)
    { return plus<E1,E2>(e1,e2); }
\end{lstlisting}

\noindent With this, one can now capture various expressions as following:

\begin{lstlisting}
variable<int> v;
auto x = v + val(3);
\end{lstlisting}

\noindent The type of variable \code{x} -- \code{plus<variable<int>,value<int>>}
 -- captures the structure of the expression, while the values inside of it 
represent various subexpressions the expression was created with. Such an 
expression can be arbitrarily, but finitely nested. Note that value 3 is not 
added to the value of variable \code{v} here, but the expression \code{v+3} is 
recorded, while the meaning to such expression can be given differently in 
different contexts. A general observation is that only the shape of the 
expression becomes fixed at compile time, while the values of variables involved 
in it can be changed arbitrarily at run time, allowing for \emph{lazy 
evaluation} of the expression. Polymorphic function \code{eval} below implements 
just that:

\begin{lstlisting}[keepspaces,columns=flexible]
template <typename T> 
    T eval(const value<T>& e) { return e.m_value; }
template <typename T> 
    T eval(const variable<T>& e) { return e.m_value; }
template <typename E1, typename E2> 
    auto eval(const plus<E1,E2>& e) 
         -> decltype(eval(e.m_e1) + eval(e.m_e2))
            { return eval(e.m_e1) + eval(e.m_e2); }
\end{lstlisting}

\noindent One can now modify value of the variable \code{v} and re-evaluate 
expression as following:

\begin{lstlisting}
v = 7;           // assumes overloading of assignment
int r = eval(x); // returns 10
\end{lstlisting}

\noindent The above technique for lazy evaluation of expressions was 
independently invented by Todd Veldhuizen and David Vandevoorde and is generally 
known in the C++ community by the name \emph{Expression Templates} that Todd 
coined\cite{Veldhuizen95expressiontemplates, vandevoorde2003c++}.  

Note again how implementation of \code{eval} resembles equations in Haskell that 
decompose an algebraic data type. The similarities are so striking that there 
were attempts to use Haskell as a pseudo code language for template 
metaprogramming in C++\cite{Milewski11}. A key observation in this analogy is 
that partial and explicit template specialization of C++ class templates are 
similar to defining equations for Haskell functions. Variables introduced via 
template clause of each equation serve as \emph{variable patterns}, while the 
names of actual templates describing arguments serve as \emph{variant 
constructors}. An important difference between the two is that Haskell's 
equations use \emph{first-fit} strategy making order of equations important, 
while C++ uses \emph{best-fit} strategy, thus making the order irrelevant.

Patterns expressed this way can be arbitrarily nested as long as they can be 
expressed in terms of the types involved and not the values they store. Using 
the above example, for instance, it is very easy to specialize \code{eval} for 
an expression of form $c_1*x+c_2$ where $c_i$ are some (not known) constant 
values and $x$ is any variable. Specializing for a concrete instance of that 
expression $2*x+3$ will be much harder, because in the representation we chose 
values 2 and 3 become run-time values and thus cannot participate in 
compile-time computations anymore. In this case we could have devised a template 
that allocates a dedicated type for each constant making such value part of the 
type:

\begin{lstlisting}[keepspaces,columns=flexible]
template <class T, T t> struct constant {};

template <typename T, T t>
    T eval(const constant<T,t>& e) { return t; }
template <typename E>
    auto eval(const times<constant<int,0>,E>& e) 
        -> decltype(eval(e.m_e2)) 
            { return (decltype(eval(e.m_e2)))(0); }
template <typename E>
    auto eval(const times<E,constant<int,0>>& e) 
        -> decltype(eval(e.m_e1)) 
            { return (decltype(eval(e.m_e1)))(0); }
\end{lstlisting}

\noindent Here the first equation for \code{eval} describes the necessary general 
case for handling expressions of type \code{constant<T,t>}, while the other two 
are redundant cases that can be seen as an optimization detecting expressions of 
the form $e*0$ and $0*e$ for any arbitrary expression $e$ and returning 0 
without actually computing $e$.

Unfortunately, a similar pattern to detect expressions of the form $x-x$ for any 
variable $x$ cannot be expressed because expression templates are blind to 
object identity and can only see their types. This means that expression 
templates of the form $x-y$ are indisthinguishable at compile time from 
expressions of the form $x-x$ because their types are identical.

Nevertheless, with all the limitations, expression templates provide an 
extremely powerful abstraction mechanism, which we use to express a 
pattern-language for our SELL. Coincidentally, we employ the compile-time 
pattern-matching facility already supported by C++ as a meta-language to 
implement its run-time counterpart.
