\section{Related Work} %%%%%%%%%%%%%%%%%%%%%%%%%%%%%%%%%%%%%%%%%%%%%%%%%%%%%%%%%
\label{sec:rw}

There are two main approaches to compiling pattern matching code: the first is 
based on \emph{backtracking automata} and was introduced by Augustsson\cite{}, 
the second is based on \emph{decision trees} and is attributed in the literature 
to Dave MacQueen and Gilles Kahn in their implementation of Hope compiler \cite{}.
Backtracking approach usually generates smaller code, while decision tree 
approach produces faster code by ensuring that each primitive test is only 
performed once. Neither of the approaches addresses specifically type patterns 
or type switching and simply assumes presence of a primitive operation capable 
of performing type tests.

Memoization device we proposed is not specifically concerned with compiling 
pattern matching and can be used independently. In particular it can be combined 
with either backtracking or decision tree approaches to avoid subsequent 
decisions on datum that has already been seen.

%xxxxxxxxxxxxxxxxxxxxxxxxxxxxxxxxxxxxxxxxxxxxxxxxxxxxxxxxxxxxxxxxxxxxxxx

\emph{Extensible Visitors with Default Cases}~\cite[\textsection 
4.2]{Zenger:2001} attempt to solve the extensibility problem of visitors; 
however, the solution, after 
remapping it onto C++, has problems of its own. The visitation interface 
hierarchy can easily be grown linearly (adding new cases for the new classes in 
the original hierarchy each time), but independent extensions by different  
authorities require developer's intervention to unify them all, before they can 
be used together. This may not be feasible in environments that use dynamic 
linking. To avoid writing even more boilerplate code in new visitors, the 
solution would require usage of virtual inheritance, which typically has 
an overhead of extra memory dereferencing. On top of the double dispatch already 
present in the visitor pattern, the solution will incur two additional virtual 
calls and a dynamic cast for each level of visitor extension. Additional double 
dispatch is incurred by forwarding of default handling from a base visitor to a 
derived one, while the dynamic cast is required for safety and can be replaced 
with a static cast when the visitation interface is guaranteed to be grown linearly 
(extended by one authority only). Yet another virtual call is required to be 
able to forward computations to subcomponents on tree-like structures to the 
most derived visitor. This last function lets one avoid the necessity of using 
the heap to allocate a temporary visitor through the \emph{Factory Design 
Pattern}~\cite{DesignPatterns1993} used in the \emph{Extensible Visitor} solution 
originally proposed by Krishnamurti, Felleisen and Friedman~\cite{Krishnamurthi98}.

In order to address the expression problem in Haskell, L\"{o}h and Hinze proposed to 
extend its type system with open data types and open functions~\cite{LohHinze2006}.
Their solution allows the user to mark top-level data types and functions as 
open and then provide concrete variants and overloads anywhere in the program. 
Open data types are extensible but not hierarchical, which largely avoids the 
problems discussed here. The semantics of open extension is given by 
transformation into a single module, where all the definitions are seen in one 
place. This is a significant limitation of their approach that prevents it from 
being truly open, since it essentially assumes a whole-program view, which 
excludes any extension via DLLs. As is the case with many other implementations 
of open extensions, the authors rely on the closed world for efficient 
implementation: in their implementation, \emph{``data types can only be entirely 
abstract (not allowing pattern matching) or concrete with all constructors with 
the reason being that pattern matching can be compiled more efficiently if the 
layout of the data type is known completely''}. The authors also believe that 
\emph{there are no theoretical difficulties in lifting this restriction, but it 
might imply a small performance loss if closed functions pattern match on open 
data types}. Our work addresses exactly this problem, showing that it is not 
only theoretically possible but also practically efficient and in application to 
a broader problem.

Polymorphic variants in OCaml~\cite{garrigue-98} allow the addition of new variants 
later. They are simpler, however, than object-oriented extensions, as they do not 
form subtyping between variants themselves, but only between combinations of them. 
This makes an important distinction between \emph{extensible sum types} like 
polymorphic variants and \emph{extensible hierarchical sum types} like classes.
An important property of extensible sum types is that each value of the 
underlying algebraic data type belongs to exactly one disjoint subset, tagged with 
a constructor. The \emph{nominative subtyping} of object-oriented languages does 
not usually have this disjointness making classes effectively have multiple 
types. In particular, the case of disjoint constructors can be seen as a 
degenerated case of a flat class hierarchy among the multitude of possible class 
hierarchies.

\emph{Tom} is a pattern-matching compiler that can be used together with Java, C or 
Eiffel to bring a common pattern matching and term rewriting syntax into the 
languages\cite{Moreau:2003}. It works as a preprocessor that transforms 
syntactic extensions into imperative code in the target language. Tom is quite 
transparent as to the concrete target language used and can potentially be 
extended to other target languages besides the three supported now. In 
particular, it never uses any semantic information of the target language during 
the compilation process and it does not inspect nor modify the source language 
part (their preprocessor is only aware of parenthesis and block delimiters of 
the source language). Tom has a sublanguage called Gom that can be used to 
define algebraic data types in a uniform mannaer, which their preprocessor then 
transforms into conrete definitions in the target language. Alternatively, the 
user can provide mappings to his own data structures that the preprocessor will 
use to generate the code.

In comparison to our approach Tom has much bigger goals. The combination of 
pattern matching, term rewriting and strategies turns Tom into a 
tree-transformation language similar to Stratego/XT, XDuce and others. 
The main accent is made on expressivity and the speed of development, which 
makes one often wonder about the run-time complexity of the generated code.
Tom's approach is also prone to general problems of any preprocessor based 
solution\cite[\textsection 4.3]{SELL}. For example, when several preprocessors 
have to be used together, each independent extension may not be able to 
understand the other's syntax, making it impossible to form a toolchain.
A library approach we follow avoids most of these problems by relying only on a 
standard C++ compiler. It also lets us employ semantics of the language within 
patterns: e.g. our patterns work directly on underlying user-defined data 
structures, largely avoiding abstraction penalties. A tighter integration with 
the language semantics also makes our patterns first-class citizens that can be 
composed and passed to other functions. The approach we take to type switching 
can also be used by Tom's preprocessor to implement type patterns efficiently -- 
similarly to other object-oriented languages, Tom's handling of them is based on 
highly inefficient \code{instanceof} operator and its equivalents in other 
languages.

Pattern matching in Scala~\cite{Scala2nd} also allows type patterns and thus type 
switching. The language supports extensible and hierarchical data types, but 
their handling in a type switching constructs varies. Sealed classes are handled 
with an efficient switch over all tags, since sealed classes cannot be extended. 
Classes that are not sealed are similarly approached with a combination of an 
\code{InstanceOf} operator and a decision tree~\cite{EmirThesis}.

%An example would be our generalized n+k patterns where we 
%can turn any invertible function even user defined into a pattern.

There has been previous attempts to use pattern matching with the Pivot 
framework that we used to experiment with our library. In his dissertation, 
Pirkelbauer devised a pattern language capable of representing various entities 
in a C++ program. The patterns were then translated with a tool into a set of 
visitors implementing the underlying pattern matching 
semantics\cite{PirkelbauerThesis}. Earlier, Cook et al used expression templates 
to implement a query language for Pivot's Internal Program Representation 
\cite{iql04}. While their work was built around a concrete class hierarchy 
letting them put some semantic knowledge about concrete classes into the 
The principal difference of their work from this work is that 
authors were essentially creating a pattern matcher for a given class hierarchy 
and thus could take the semantics of the entities represented by classes in the 
hierarchy into account. Our approach is parametrized over class hierarchy and 
thus provides a rather lower level pattern-matching functionality that lets one 
simplify work with that hierarchy.  One can think of it as a generalized 
dynamic\_cast. To be continued...

