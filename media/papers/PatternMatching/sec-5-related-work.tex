\section{Related Work} %%%%%%%%%%%%%%%%%%%%%%%%%%%%%%%%%%%%%%%%%%%%%%%%%%%%%%%%%
\label{sec:rw}

Language support for pattern matching was first introduced for string 
manipulation in SNOBOL\cite{SNOBOL64}. SNOBOL4 had patterns as first-class data 
types providing operations of concatenation and alternation\cite{SNOBOL71}. The 
first reference to a modern pattern-matching constructs seen in functional 
languages is usually attributed to Burstall's work on structural 
induction\cite{Burstall69provingproperties}. Pattern matching was further 
developed by the functional programming community, most notably 
Hope\cite{BMS80}, ML\cite{ML90}, Miranda\cite{Miranda85} and 
Haskell\cite{Haskell98Book}. In the context of object-oriented programming, 
pattern matching has been first explored in Pizza\cite{Odersky97pizzainto} and 
Scala\cite{Scala2nd,EmirThesis}.

There are two main approaches to compiling pattern-matching code: the first is 
based on \emph{backtracking automata} and was introduced by Augustsson\cite{Augustsson85}, 
the second is based on \emph{decision trees} and was first described by 
Cardelli\cite{Cardelli84}, though he attributes the technique to Dave MacQueen 
and Gilles Kahn in their implementation of the Hope compiler \cite{BMS80}.
Backtracking approach usually generates smaller code, while decision tree 
approach produces faster code by ensuring that each primitive test is only 
performed once. With respect to matching a single expression our library 
approach follows the naive backtracking approach, however our match statement is 
based on a highly efficient type switching technique we developed\cite{TypeSwitch} 
that outperforms similar solutions based on decision trees or visitor design pattern.

Tom is a pattern-matching compiler that can be used together with Java, C or 
Eiffel to bring a common pattern matching and term rewriting syntax into the 
languages\cite{Moreau:2003}. It works as a preprocessor that transforms 
syntactic extensions into imperative code in the target language. Tom is quite 
transparent as to the concrete target language used and can potentially be 
extended to other target languages besides the three supported now.
Tom's  goals differ from ours in aiming to be a
tree-transformation language similar to Stratego/XT, XDuce and others. 
Tom's approach is prone to general problems of any preprocessor based 
solution\cite[\textsection 4.3]{SELL}. In particular, it is part of a dedicated toolchain.
Our library approach avoids that and lets us employ the C++ semantics within 
patterns: e.g. our patterns work directly on underlying user-defined data 
structures, avoiding abstraction penalties. The tight integration with 
the language semantics makes our patterns first-class citizens that can be 
composed and passed to other functions. 

Prop is another language extension that brings pattern matching into 
C++~\cite{Prop96}. This extension is not focused on pattern 
matching, but is intended for building high performance 
compiler and language transformation systems. It supports value-, variable-, 
wildcard-, constructor-, nested-, as-, type- and numerous sequence patterns.

Functional C\# is similar to our approach in trying to bring pattern matching 
into the C\# as a library\cite{FuncCSharp}. The approach uses lambda functions 
and chaining of method calls to create a structure that is then interpreted at 
run-time for the first successful predicate. The approach supports a form of 
active patterns, simple n+k patterns, list and tuple patterns as well as type 
patterns (without structural decomposition). 
However, an approach based on sequential type tests 
scales very poorly for match statements with more than two case clauses, making 
it unreasonably slower than the visitor design pattern~\cite{TypeSwitch}. Besides, the approach 
seems to be ill suited for tests involving nesting of patterns.

When the class hierarchy is fixed, one can design a pattern language that involves 
semantic notions represented by the hierarchy. Pirkelbauer devised a pattern 
language for Pivot capable of representing various entities in a C++ program using syntax very close to the C++ itself. 
Interestingly, the patterns were translated with a tool into a set of visitors 
implementing the underlying pattern-matching semantics\cite{PirkelbauerThesis}. 
Earlier, Cook et al used expression templates to implement a query language for 
Pivot's class hierarchy~\cite{iql04}. Our current work is the result of a series 
of experimental designs. The library approach was essential to provide 
relatively quick turnaround for experiments and for maintaining and improving 
performance for our applications.
