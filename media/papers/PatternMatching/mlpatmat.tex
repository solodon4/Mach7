\documentclass{llncs}

\usepackage{amssymb}
%\usepackage{amsthm}
%\usepackage{breakurl}             % Not needed if you use pdflatex only.
%\usepackage{color}
%\usepackage{epsfig}
%\usepackage{esvect}
\usepackage{listings}
\usepackage{mathpartir}
%\usepackage{MnSymbol}
%\usepackage{multirow}
%\usepackage{rotating}
\usepackage{paralist}

\setlength{\parskip}{0cm}
%\setlength{\parindent}{1em}

\lstdefinestyle{C++}{language=C++,%
showstringspaces=false,
  columns=fullflexible,
  escapechar=@,
  basicstyle=\sffamily,
%  commentstyle=\rmfamily\itshape,
  moredelim=**[is][\color{white}]{~}{~},
  morekeywords={axiom,concept,decltype,noexcept,nullptr,requires},
  literate={[<]}{{\textless}}1      {[>]}{{\textgreater}}1 %
           {*}{{$*$}}1 % Without this, star (*) is rendered in Type 3 font
           {<}{{$\langle$}}1        {>}{{$\rangle$}}1 %
           {<=}{{$\leq$}}1          {>=}{{$\geq$}}1 %
           {==}{{$==$}}2            {!=}{{$\neq$}}1 %
           {=>}{{$\Rightarrow\;$}}1 {->}{{$\rightarrow{}$}}1 %
           {<:}{{$\subtype{}\ $}}1  {<-}{{$\leftarrow$}}1 %
           {s1;}{{$s_1$;}}3 {s2;}{{$s_2$;}}3 {s3;}{{$s_3$;}}3 {s4;}{{$s_4$;}}3 {s5;}{{$s_5$;}}3 {s6;}{{$s_6$;}}3 {s7;}{{$s_7$;}}3 {sn;}{{$s_n$;}}3 {si;}{{$s_i$;}}3%
           {P1}{{$P_1$}}2 {P2}{{$P_2$}}2 {P3}{{$P_3$}}2 {P4}{{$P_4$}}2 {P5}{{$P_5$}}2 {P6}{{$P_6$}}2 {P7}{{$P_7$}}2 {Pn}{{$P_n$}}2 {Pi}{{$P_i$}}2%
           {D1}{{$D_1$}}2 {D2}{{$D_2$}}2 {D3}{{$D_3$}}2 {D4}{{$D_4$}}2 {D5}{{$D_5$}}2 {D6}{{$D_6$}}2 {D7}{{$D_7$}}2 {Dn}{{$D_n$}}2 {Di}{{$D_i$}}2%
           {T1}{{$T_1$}}2 {T2}{{$T_2$}}2 {T3}{{$T_3$}}2 {T4}{{$T_4$}}2 {T5}{{$T_5$}}2 {T6}{{$T_6$}}2 {T7}{{$T_7$}}2 {Tn}{{$T_n$}}2 {Ti}{{$T_i$}}2 {Tm}{{$T_m$}}2%
           {e1}{{$e_1$}}2 {e2}{{$e_2$}}2 {e3}{{$e_3$}}2 {e4}{{$e_4$}}2%
           {E1}{{$E_1$}}2 {E2}{{$E_2$}}2 {E3}{{$E_3$}}2 {E4}{{$E_4$}}2 {Ei}{{$E_i$}}2%
           {m_e1}{{$m\_e_1$}}4 {m_e2}{{$m\_e_2$}}4 {m_e3}{{$m\_e_3$}}4 {m_e4}{{$m\_e_4$}}4%
           {Divide}{{Divide}}6 {Either}{Either}6 %
           {Times}{{Times}}5 %
           {Match}{{\emph{Match}}}5 %
           {Case}{{\emph{Case}}}4 %
           {Qua}{{\emph{Qua}}}3 %
           {When}{{\emph{When}}}4 %
           {Otherwise}{{\emph{Otherwise}}}9 %
           {EndMatch}{{\emph{EndMatch}}}8 %
           {CM}{{\emph{CM}}}2 {KS}{{\emph{KS}}}2 {KV}{{\emph{KV}}}2 %
           {EuclideanDomain}{\concept{EuclideanDomain}}{15}  %
           {LazyExpression}{\concept{LazyExpression}}{14}    %
           {Polymorphic}{\concept{Polymorphic}}{11}          %
           {Convertible}{\concept{Convertible}}{11}          %
           {Integral}{\concept{Integral}}8                   %
           {SameType}{\concept{SameType}}8                   %
           {Pattern}{\concept{Pattern}}7                     %
           {Regular}{\concept{Regular}}7                     %
           {Object}{\concept{Object}}6                       %
           {Field}{\concept{Field}}5                         %
}
\lstset{style=C++}

\lstdefinestyle{Haskell}{language=Haskell,%
  morekeywords={out,view,real}
  literate={=>}{{$\Rightarrow\;$}}1 {->}{{$\rightarrow{}$}}1 {<-}{{$\leftarrow$}}1 {\\}{{$\lambda$}}1,
  moredelim=**[is][\color{red}]{`}{`},
  moredelim=**[is][\color{white}]{~}{~}
}

\lstdefinestyle{Caml}{language=Caml,%
  morekeywords={when}
  literate={->}{{$\rightarrow{}$}}1,
  moredelim=**[is][\color{red}]{`}{`},
  moredelim=**[is][\color{white}]{~}{~}
}

\DeclareRobustCommand{\Cpp}{C\texttt{++}}
\DeclareRobustCommand{\code}[1]{{\lstinline[keepspaces,breaklines=false,escapechar=@]{#1}}}
\DeclareRobustCommand{\codebr}[1]{{\lstinline[breaklines=true]{#1}}}
\DeclareRobustCommand{\codehaskell}[1]{{\lstinline[breaklines=false,language=Haskell]{#1}}}
\DeclareRobustCommand{\codeocaml}[1]{{\lstinline[breaklines=false,language=Caml]{#1}}}
\DeclareRobustCommand{\concept}[1]{{\small\textsc{#1}}}
\newcommand{\exclude}[1]{}
\newcommand{\halfline}{\vspace{-1.5ex}}

%\newtheorem{lemma}{Lemma}
%\newtheorem{theorem}{Theorem}
%\newtheorem{corollary}{Corollary}

%% grammar commands
\newcommand{\Rule}[1]{{\rmfamily\itshape{#1}}}
\newcommand{\Alt}{\ensuremath{\mid}}
\newcommand{\SynCat}[1]{\ensuremath{\mathit{#1}}}
\newcommand{\is}{\ensuremath{::=}}
\newcommand{\subtype}{\ensuremath{\texttt{\raisebox{-0.1ex}{<}\raisebox{0.05ex}{:}}}}
\newcommand{\subtypeD}{\ensuremath{<:_d}}
\newcommand{\lazyevals}{\Downarrow}
\newcommand{\evals}{\Rightarrow}
\newcommand{\evalspp}{\Rightarrow^+}
\newcommand{\DynCast}[2]{\ensuremath{\mathsf{dyn\_cast}\langle{#1}\rangle({#2})}}
\newcommand{\nullptr}{\ensuremath{\bot}}
\newcommand{\of}[1]{\left(#1\right)}
\newcommand{\True}{\ensuremath{\mathsf{true}}}
\newcommand{\False}{\ensuremath{\mathsf{false}}}

\newcommand{\CWildcard}{\ensuremath{\mathit{\bf wildcard}}}
\newcommand{\CValue}   {\ensuremath{\mathit{\bf value}}}
\newcommand{\CVariable}{\ensuremath{\mathit{\bf variable}}}
\newcommand{\CExpr}    {\ensuremath{\mathit{\bf expr}}}
\newcommand{\CGuard}   {\ensuremath{\mathit{\bf guard}}}
\newcommand{\CCnstr}   {\ensuremath{\mathit{\bf ctor}}}

\newcommand{\Wildcard}   {\ensuremath{\CWildcard}}
\newcommand{\Value}[1]   {\ensuremath{\CValue\langle{#1}\rangle}}
\newcommand{\Variable}[1]{\ensuremath{\CVariable\langle{#1}\rangle}}
\newcommand{\ExprU}[2]   {\ensuremath{\CExpr\langle{#1},{#2}\rangle}}
\newcommand{\ExprB}[3]   {\ensuremath{\CExpr\langle{#1},{#2},{#3}\rangle}}
\newcommand{\ExprK}[3]   {\ensuremath{\CExpr\langle{#1},{#2},\cdots,{#3}\rangle}}
\newcommand{\Guard}[2]   {\ensuremath{\CGuard\langle{#1},{#2}\rangle}}
\newcommand{\Cnstr}[3]   {\ensuremath{\CCnstr\langle{#1},{#2},\cdots,{#3}\rangle}}

\newcommand{\f}[1]{{ {{#1\%}}}}
\newcommand{\s}[1]{{ {\bf \underline{#1\%}}}}
\newcommand{\n}[1]{{ {\bf ~ ~ ~ ~ }}}
\newcommand{\Opn}{{\scriptsize {\bf Open}}}
\newcommand{\Cls}{{\scriptsize {\bf Tag}}}
\newcommand{\Unn}{{\scriptsize {\bf Union}}}

%%\newcommand{\gwNGPp}{\n{}}
%\newcommand{\gwNGKp}{\n{}}
 \newcommand{\gwNGUp}{\n{}}
%\newcommand{\gwNSPp}{\n{}}
%\newcommand{\gwNSKp}{\n{}}
 \newcommand{\gwNSUp}{\n{}}
%\newcommand{\vwNGPp}{\n{}}
%\newcommand{\vwNGKp}{\n{}}
 \newcommand{\vwNGUp}{\n{}}
%\newcommand{\vwNSPp}{\n{}}
%\newcommand{\vwNSKp}{\n{}}
 \newcommand{\vwNSUp}{\n{}}
%\newcommand{\vxNGPp}{\n{}}
%\newcommand{\vxNGKp}{\n{}}
 \newcommand{\vxNGUp}{\n{}}
%\newcommand{\vxNSPp}{\n{}}
%\newcommand{\vxNSKp}{\n{}}
 \newcommand{\vxNSUp}{\n{}}

%\newcommand{\gwNGPq}{\n{}}
%\newcommand{\gwNGKq}{\n{}}
 \newcommand{\gwNGUq}{\n{}}
%\newcommand{\gwNSPq}{\n{}}
%\newcommand{\gwNSKq}{\n{}}
 \newcommand{\gwNSUq}{\n{}}
%\newcommand{\vwNGPq}{\n{}}
%\newcommand{\vwNGKq}{\n{}}
 \newcommand{\vwNGUq}{\n{}}
%\newcommand{\vwNSPq}{\n{}}
%\newcommand{\vwNSKq}{\n{}}
 \newcommand{\vwNSUq}{\n{}}
%\newcommand{\vxNGPq}{\n{}}
%\newcommand{\vxNGKq}{\n{}}
 \newcommand{\vxNGUq}{\n{}}
%\newcommand{\vxNSPq}{\n{}}
%\newcommand{\vxNSKq}{\n{}}
 \newcommand{\vxNSUq}{\n{}}

%\newcommand{\gwNGPn}{\n{}}
%\newcommand{\gwNGKn}{\n{}}
 \newcommand{\gwNGUn}{\n{}}
%\newcommand{\gwNSPn}{\n{}}
%\newcommand{\gwNSKn}{\n{}}
 \newcommand{\gwNSUn}{\n{}}
%\newcommand{\vwNGPn}{\n{}}
%\newcommand{\vwNGKn}{\n{}}
 \newcommand{\vwNGUn}{\n{}}
%\newcommand{\vwNSPn}{\n{}}
%\newcommand{\vwNSKn}{\n{}}
 \newcommand{\vwNSUn}{\n{}}
%\newcommand{\vxNGPn}{\n{}}
%\newcommand{\vxNGKn}{\n{}}
 \newcommand{\vxNGUn}{\n{}}
%\newcommand{\vxNSPn}{\n{}}
%\newcommand{\vxNSKn}{\n{}}
 \newcommand{\vxNSUn}{\n{}}


%\newcommand{\gwYGPp}{\n{}}
% \newcommand{\gwYGKp}{\n{}}
 \newcommand{\gwYGUp}{\n{}}
%\newcommand{\gwYSPp}{\n{}}
% \newcommand{\gwYSKp}{\n{}}
 \newcommand{\gwYSUp}{\n{}}
%\newcommand{\vwYGPp}{\n{}}
% \newcommand{\vwYGKp}{\n{}}
 \newcommand{\vwYGUp}{\n{}}
%\newcommand{\vwYSPp}{\n{}}
% \newcommand{\vwYSKp}{\n{}}
 \newcommand{\vwYSUp}{\n{}}
%\newcommand{\vxYGPp}{\n{}}
% \newcommand{\vxYGKp}{\n{}}
 \newcommand{\vxYGUp}{\n{}}
%\newcommand{\vxYSPp}{\n{}}
% \newcommand{\vxYSKp}{\n{}}
 \newcommand{\vxYSUp}{\n{}}

%\newcommand{\gwYGPq}{\n{}}
% \newcommand{\gwYGKq}{\n{}}
 \newcommand{\gwYGUq}{\n{}}
%\newcommand{\gwYSPq}{\n{}}
% \newcommand{\gwYSKq}{\n{}}
 \newcommand{\gwYSUq}{\n{}}
%\newcommand{\vwYGPq}{\n{}}
% \newcommand{\vwYGKq}{\n{}}
 \newcommand{\vwYGUq}{\n{}}
%\newcommand{\vwYSPq}{\n{}}
% \newcommand{\vwYSKq}{\n{}}
 \newcommand{\vwYSUq}{\n{}}
%\newcommand{\vxYGPq}{\n{}}
% \newcommand{\vxYGKq}{\n{}}
 \newcommand{\vxYGUq}{\n{}}
%\newcommand{\vxYSPq}{\n{}}
% \newcommand{\vxYSKq}{\n{}}
 \newcommand{\vxYSUq}{\n{}}

%\newcommand{\gwYGPn}{\n{}}
% \newcommand{\gwYGKn}{\n{}}
 \newcommand{\gwYGUn}{\n{}}
%\newcommand{\gwYSPn}{\n{}}
% \newcommand{\gwYSKn}{\n{}}
 \newcommand{\gwYSUn}{\n{}}
%\newcommand{\vwYGPn}{\n{}}
% \newcommand{\vwYGKn}{\n{}}
 \newcommand{\vwYGUn}{\n{}}
%\newcommand{\vwYSPn}{\n{}}
% \newcommand{\vwYSKn}{\n{}}
 \newcommand{\vwYSUn}{\n{}}
%\newcommand{\vxYGPn}{\n{}}
% \newcommand{\vxYGKn}{\n{}}
 \newcommand{\vxYGUn}{\n{}}
%\newcommand{\vxYSPn}{\n{}}
% \newcommand{\vxYSKn}{\n{}}
 \newcommand{\vxYSUn}{\n{}}

 \newcommand{\GwNGPp}{\n{}}
 \newcommand{\GwNGKp}{\n{}}
 \newcommand{\GwNGUp}{\n{}}
 \newcommand{\GwNSPp}{\n{}}
 \newcommand{\GwNSKp}{\n{}}
 \newcommand{\GwNSUp}{\n{}}
%\newcommand{\VwNGPp}{\n{}}
%\newcommand{\VwNGKp}{\n{}}
 \newcommand{\VwNGUp}{\n{}}
%\newcommand{\VwNSPp}{\n{}}
%\newcommand{\VwNSKp}{\n{}}
 \newcommand{\VwNSUp}{\n{}}
%\newcommand{\VxNGPp}{\n{}}
%\newcommand{\VxNGKp}{\n{}}
 \newcommand{\VxNGUp}{\n{}}
%\newcommand{\VxNSPp}{\n{}}
%\newcommand{\VxNSKp}{\n{}}
 \newcommand{\VxNSUp}{\n{}}

 \newcommand{\GwNGPq}{\n{}}
 \newcommand{\GwNGKq}{\n{}}
 \newcommand{\GwNGUq}{\n{}}
 \newcommand{\GwNSPq}{\n{}}
 \newcommand{\GwNSKq}{\n{}}
 \newcommand{\GwNSUq}{\n{}}
%\newcommand{\VwNGPq}{\n{}}
%\newcommand{\VwNGKq}{\n{}}
 \newcommand{\VwNGUq}{\n{}}
%\newcommand{\VwNSPq}{\n{}}
%\newcommand{\VwNSKq}{\n{}}
 \newcommand{\VwNSUq}{\n{}}
%\newcommand{\VxNGPq}{\n{}}
%\newcommand{\VxNGKq}{\n{}}
 \newcommand{\VxNGUq}{\n{}}
%\newcommand{\VxNSPq}{\n{}}
%\newcommand{\VxNSKq}{\n{}}
 \newcommand{\VxNSUq}{\n{}}

 \newcommand{\GwNGPn}{\n{}}
 \newcommand{\GwNGKn}{\n{}}
 \newcommand{\GwNGUn}{\n{}}
 \newcommand{\GwNSPn}{\n{}}
 \newcommand{\GwNSKn}{\n{}}
 \newcommand{\GwNSUn}{\n{}}
%\newcommand{\VwNGPn}{\n{}}
%\newcommand{\VwNGKn}{\n{}}
 \newcommand{\VwNGUn}{\n{}}
%\newcommand{\VwNSPn}{\n{}}
%\newcommand{\VwNSKn}{\n{}}
 \newcommand{\VwNSUn}{\n{}}
%\newcommand{\VxNGPn}{\n{}}
%\newcommand{\VxNGKn}{\n{}}
 \newcommand{\VxNGUn}{\n{}}
%\newcommand{\VxNSPn}{\n{}}
%\newcommand{\VxNSKn}{\n{}}
 \newcommand{\VxNSUn}{\n{}}


 \newcommand{\GwYGPp}{\n{}}
 \newcommand{\GwYGKp}{\n{}}
 \newcommand{\GwYGUp}{\n{}}
 \newcommand{\GwYSPp}{\n{}}
 \newcommand{\GwYSKp}{\n{}}
 \newcommand{\GwYSUp}{\n{}}
%\newcommand{\VwYGPp}{\n{}}
% \newcommand{\VwYGKp}{\n{}}
 \newcommand{\VwYGUp}{\n{}}
%\newcommand{\VwYSPp}{\n{}}
% \newcommand{\VwYSKp}{\n{}}
 \newcommand{\VwYSUp}{\n{}}
%\newcommand{\VxYGPp}{\n{}}
% \newcommand{\VxYGKp}{\n{}}
 \newcommand{\VxYGUp}{\n{}}
%\newcommand{\VxYSPp}{\n{}}
% \newcommand{\VxYSKp}{\n{}}
 \newcommand{\VxYSUp}{\n{}}

 \newcommand{\GwYGPq}{\n{}}
 \newcommand{\GwYGKq}{\n{}}
 \newcommand{\GwYGUq}{\n{}}
 \newcommand{\GwYSPq}{\n{}}
 \newcommand{\GwYSKq}{\n{}}
 \newcommand{\GwYSUq}{\n{}}
%\newcommand{\VwYGPq}{\n{}}
% \newcommand{\VwYGKq}{\n{}}
 \newcommand{\VwYGUq}{\n{}}
%\newcommand{\VwYSPq}{\n{}}
% \newcommand{\VwYSKq}{\n{}}
 \newcommand{\VwYSUq}{\n{}}
%\newcommand{\VxYGPq}{\n{}}
% \newcommand{\VxYGKq}{\n{}}
 \newcommand{\VxYGUq}{\n{}}
%\newcommand{\VxYSPq}{\n{}}
% \newcommand{\VxYSKq}{\n{}}
 \newcommand{\VxYSUq}{\n{}}

 \newcommand{\GwYGPn}{\n{}}
 \newcommand{\GwYGKn}{\n{}}
 \newcommand{\GwYGUn}{\n{}}
 \newcommand{\GwYSPn}{\n{}}
 \newcommand{\GwYSKn}{\n{}}
 \newcommand{\GwYSUn}{\n{}}
%\newcommand{\VwYGPn}{\n{}}
% \newcommand{\VwYGKn}{\n{}}
 \newcommand{\VwYGUn}{\n{}}
%\newcommand{\VwYSPn}{\n{}}
% \newcommand{\VwYSKn}{\n{}}
 \newcommand{\VwYSUn}{\n{}}
%\newcommand{\VxYGPn}{\n{}}
% \newcommand{\VxYGKn}{\n{}}
 \newcommand{\VxYGUn}{\n{}}
%\newcommand{\VxYSPn}{\n{}}
% \newcommand{\VxYSKn}{\n{}}
 \newcommand{\VxYSUn}{\n{}}

% This file defines variables with performance numbers for the table in Evaluation section
% Data from 2011-08-30 
\newcommand{\vwYGKp}{\s{3}}
\newcommand{\vwYGKn}{\s{8}}
\newcommand{\vwYGKq}{\s{11}}
\newcommand{\vwYGPp}{\f{10}}
\newcommand{\vwYGPn}{\f{14}}
\newcommand{\vwYGPq}{\s{0}}
\newcommand{\vwYSKp}{\s{7}}
\newcommand{\vwYSKn}{\s{7}}
\newcommand{\vwYSKq}{\s{10}}
\newcommand{\vwYSPp}{\f{10}}
\newcommand{\vwYSPn}{\f{14}}
\newcommand{\vwYSPq}{\s{0}}
\newcommand{\vwNGKp}{\f{35}}
\newcommand{\vwNGKn}{\s{6}}
\newcommand{\vwNGKq}{\s{5}}
\newcommand{\vwNGPp}{\f{1}}
\newcommand{\vwNGPn}{\s{1}}
\newcommand{\vwNGPq}{\s{10}}
\newcommand{\vwNSKp}{\f{133}}
\newcommand{\vwNSKn}{\f{25}}
\newcommand{\vwNSKq}{\f{59}}
\newcommand{\vwNSPp}{\f{1}}
\newcommand{\vwNSPn}{\s{1}}
\newcommand{\vwNSPq}{\s{8}}

\newcommand{\vxYGKp}{\s{61}}
\newcommand{\vxYGKn}{\s{24}}
\newcommand{\vxYGKq}{\s{25}}
\newcommand{\vxYGPp}{\s{24}}
\newcommand{\vxYGPn}{\s{24}}
\newcommand{\vxYGPq}{\s{36}}
\newcommand{\vxYSKp}{\s{79}}
\newcommand{\vxYSKn}{\s{25}}
\newcommand{\vxYSKq}{\s{35}}
\newcommand{\vxYSPp}{\s{9}}
\newcommand{\vxYSPn}{\s{23}}
\newcommand{\vxYSPq}{\f{133}}
\newcommand{\vxNGKp}{\s{8}}
\newcommand{\vxNGKn}{\s{5}}
\newcommand{\vxNGKq}{\s{0}}
\newcommand{\vxNGPp}{\s{33}}
\newcommand{\vxNGPn}{\s{47}}
\newcommand{\vxNGPq}{\s{43}}
\newcommand{\vxNSKp}{\f{38}}
\newcommand{\vxNSKn}{\f{12}}
\newcommand{\vxNSKq}{\f{3}}
\newcommand{\vxNSPp}{\s{27}}
\newcommand{\vxNSPn}{\s{44}}
\newcommand{\vxNSPq}{\s{45}}

\newcommand{\gwYGKp}{\f{88}}
\newcommand{\gwYGKn}{\f{32}}
\newcommand{\gwYGKq}{\f{250}}
\newcommand{\gwYGPp}{\f{67}}
\newcommand{\gwYGPn}{\f{28}}
\newcommand{\gwYGPq}{\f{87}}
\newcommand{\gwYSKp}{\f{79}}
\newcommand{\gwYSKn}{\f{31}}
\newcommand{\gwYSKq}{\f{259}}
\newcommand{\gwYSPp}{\f{67}}
\newcommand{\gwYSPn}{\f{27}}
\newcommand{\gwYSPq}{\f{90}}
\newcommand{\gwNGKp}{\f{116}}
\newcommand{\gwNGKn}{\f{29}}
\newcommand{\gwNGKq}{\f{43}}
\newcommand{\gwNGPp}{\f{55}}
\newcommand{\gwNGPn}{\s{0}}
\newcommand{\gwNGPq}{\f{1}}
\newcommand{\gwNSKp}{\f{216}}
\newcommand{\gwNSKn}{\f{542}}
\newcommand{\gwNSKq}{\f{520}}
\newcommand{\gwNSPp}{\f{55}}
\newcommand{\gwNSPn}{\f{1}}
\newcommand{\gwNSPq}{\f{3}}

\newcommand{\VwYGKp}{\f{16}}
\newcommand{\VwYGKn}{\f{11}}
\newcommand{\VwYGKq}{\f{168}}
\newcommand{\VwYGPp}{\f{10}}
\newcommand{\VwYGPn}{\f{19}}
\newcommand{\VwYGPq}{\f{153}}
\newcommand{\VwYSKp}{\f{31}}
\newcommand{\VwYSKn}{\f{24}}
\newcommand{\VwYSKq}{\f{185}}
\newcommand{\VwYSPp}{\f{10}}
\newcommand{\VwYSPn}{\f{18}}
\newcommand{\VwYSPq}{\f{153}}
\newcommand{\VwNGKp}{\f{61}}
\newcommand{\VwNGKn}{\f{18}}
\newcommand{\VwNGKq}{\f{13}}
\newcommand{\VwNGPp}{\f{4}}
\newcommand{\VwNGPn}{\s{17}}
\newcommand{\VwNGPq}{\s{9}}
\newcommand{\VwNSKp}{\f{124}}
\newcommand{\VwNSKn}{\f{43}}
\newcommand{\VwNSKq}{\f{34}}
\newcommand{\VwNSPp}{\f{4}}
\newcommand{\VwNSPn}{\s{18}}
\newcommand{\VwNSPq}{\f{3}}
               
\newcommand{\VxYGKp}{\s{5}}
\newcommand{\VxYGKn}{\s{2}}
\newcommand{\VxYGKq}{\f{132}}
\newcommand{\VxYGPp}{\s{5}}
\newcommand{\VxYGPn}{\s{5}}
\newcommand{\VxYGPq}{\f{130}}
\newcommand{\VxYSKp}{\s{9}}
\newcommand{\VxYSKn}{\s{10}}
\newcommand{\VxYSKq}{\f{118}}
\newcommand{\VxYSPp}{\s{6}}
\newcommand{\VxYSPn}{\s{6}}
\newcommand{\VxYSPq}{\f{145}}
\newcommand{\VxNGKp}{\f{20}}
\newcommand{\VxNGKn}{\f{7}}
\newcommand{\VxNGKq}{\f{34}}
\newcommand{\VxNGPp}{\s{14}}
\newcommand{\VxNGPn}{\s{27}}
\newcommand{\VxNGPq}{\f{2}}
\newcommand{\VxNSKp}{\f{47}}
\newcommand{\VxNSKn}{\f{16}}
\newcommand{\VxNSKq}{\f{14}}
\newcommand{\VxNSPp}{\s{0}}
\newcommand{\VxNSPn}{\s{27}}
\newcommand{\VxNSPq}{\f{1}}

% This file defines variables with performance numbers for the table in Evaluation section
% Data from 2011-11-04 collected on Sierra for Linux under g++ (GCC) 4.4.5 20101112 (Red Hat 4.4.5-2)

\newcommand{\glYGKp}{\f{45}}
\newcommand{\glYGKn}{\f{82}}
\newcommand{\glYGKq}{\f{77}}
\newcommand{\glYGPp}{\f{36}}
\newcommand{\glYGPn}{\f{76}}
\newcommand{\glYGPq}{\f{53}}
\newcommand{\glYSKp}{\f{53}}
\newcommand{\glYSKn}{\f{88}}
\newcommand{\glYSKq}{\f{86}}
\newcommand{\glYSPp}{\f{33}}
\newcommand{\glYSPn}{\f{78}}
\newcommand{\glYSPq}{\f{55}}
\newcommand{\glNGKp}{\f{54}}
\newcommand{\glNGKn}{\f{97}}
\newcommand{\glNGKq}{\f{109}}
\newcommand{\glNGPp}{\f{19}}
\newcommand{\glNGPn}{\f{57}}
\newcommand{\glNGPq}{\f{62}}
\newcommand{\glNSKp}{\f{124}}
\newcommand{\glNSKn}{\f{603}}
\newcommand{\glNSKq}{\f{640}}
\newcommand{\glNSPp}{\f{16}}
\newcommand{\glNSPn}{\f{56}}
\newcommand{\glNSPq}{\f{56}}


\newsavebox{\sembox}
\newlength{\semwidth}
\newlength{\boxwidth}

\newcommand{\Sem}[1]{%
\sbox{\sembox}{\ensuremath{#1}}%
\settowidth{\semwidth}{\usebox{\sembox}}%
\sbox{\sembox}{\ensuremath{\left[\usebox{\sembox}\right]}}%
\settowidth{\boxwidth}{\usebox{\sembox}}%
\addtolength{\boxwidth}{-\semwidth}%
\left[\hspace{-0.3\boxwidth}%
\usebox{\sembox}%
\hspace{-0.3\boxwidth}\right]%
}

\newcommand{\authormodification}[2]{{\color{#1}#2}}
\newcommand{\ys}[1]{\authormodification{blue}{#1}}
\newcommand{\bs}[1]{\authormodification{red}{#1}}
\newcommand{\gdr}[1]{\authormodification{magenta}{#1}}

\begin{document}
%
\frontmatter          % for the preliminaries
%
\pagestyle{headings}  % switches on printing of running heads
%\addtocmark{Hamiltonian Mechanics} % additional mark in the TOC

%\conferenceinfo{PLDI 2012}{Beijing, China} 
%\copyrightyear{2012} 
%\copyrightdata{[to be supplied]} 

%\titlebanner{Draft}        % These are ignored unless
%\preprintfooter{Y.Solodkyy, G.Dos Reis, B.Stroustrup: An Elegant and Efficient Pattern Matching Library for C++}   % 'preprint' option specified.

\title{An Elegant and Efficient Pattern Matching Library for C++}
%\subtitle{your \code{visit}, Jim, is not \code{accept}able anymore}
%\subtitle{\code{accepting} aint no \code{visit}ors}

\author{Yuriy Solodkyy\and Gabriel Dos Reis\and Bjarne Stroustrup}
%
\authorrunning{Solodkyy et al.} % abbreviated author list (for running head)
%
%%%% list of authors for the TOC (use if author list has to be modified)
\tocauthor{Yuriy Solodkyy, Gabriel Dos Reis, and Bjarne Stroustrup}
%
\institute{Texas A\&M University, College Station TX 77843, USA,\\
\email{\{yuriys,gdr,bs\}@cse.tamu.edu}
}

\maketitle              % typeset the title of the contribution

\begin{abstract}
Pattern matching is an abstraction mechanism that greatly simplifies code. We 
present functional-programming-style pattern matching for C++ implemented as a 
library. The library provides a uniform notation for matching against built-in 
and user-defined types, including the numerous encodings of algebraic as well as 
extensible hierarchical data types used in C++. The library integrates well with 
programming styles supported by ISO C++ and can be used in the presence of 
multiple inheritance as well as in generic code.

Our patterns are first-class citizens, and the set of patterns supported by 
the library is open. The traditional patterns seen in functional languages are 
expressed as user-defined entities, including the typical constructor patterns, 
controversial n+k patterns and expressive views. Implementing pattern matching 
as a library allows us to experiment with the expressiveness of syntax, 
under-the-hood tweaks, and styles of use while preserving the performance and 
portability provided by industrial compilers and support tools.

The library was originally motivated by and is used for applications involving 
large, typed, abstract syntax trees. Code written using our patterns is 
significantly more concise and easier to comprehend than alternative solutions 
in C++. Compared to the visitor design pattern, pattern matching is non-intrusive, 
does not have extensibility restrictions, avoids control inversion, is 
faster, and can be used in scenarios for which visitors are ill suited.
\keywords{Pattern Matching, Visitor Design Pattern, C++}
\end{abstract}

\section{Introduction} %%%%%%%%%%%%%%%%%%%%%%%%%%%%%%%%%%%%%%%%%%%%%%%%%%%%%%%%%
\label{sec:intro}

%Motivate the problem
%Give a summary of the paper: what you did and how
%Explicitly state your contribution

Pattern matching is an abstraction supported by many programming languages.
It allows the user tersely to describe a (possibly infinite) set of 
values accepted by the pattern. A \emph{pattern} represents a predicate on 
values, and is usually  much more concise and readable than the 
equivalent predicate spelled out as imperative code.

Popularized by functional programming community, most notably Hope\cite{BMS80}, 
ML\cite{ML90}, Miranda\cite{Miranda85} and Haskell\cite{Haskell98Book}, for 
providing syntax very close to mathematical notations.
From there, it has 
found its way into many imperative programming languages e.g. 
Pizza\cite{Odersky97pizzainto}, Scala\cite{Scala2nd}, Fortress\cite{RPS10}, as 
well as dialects of Java\cite{Liu03jmatch:iterable,HydroJ2003}, C++\cite{Prop96}, 
Eiffel\cite{Moreau:2003} and others. It is relatively easy to provide a form of pattern 
matching when designing a new language, but to introduce it into a language in 
widespread use is a challenge. The obvious utility of the feature may be 
compromised by the need to fit into the language's syntax, semantics, and tool 
chains. A prototype implementation requires more effort than for an experimental 
language and is harder to get into use because mainstream users are unwilling 
to try non-portable, non-standard, and unoptimized features.

To balance the utility and effort we decided to take the Semantically 
Enhanced Library Language (SELL) approach\cite{SELL}. We provide the
general-purpose programming language with a library, extended with a tool 
support. This will typically (as in this case) not provide you 100\% of the functionality that a 
language extension would do, but it allows experimentation and special-purpose use
with existing compilers and tool chains. With pattern matching, we managed to avoid 
external tool support by relying on some pretty nasty macro hacking to provide a
conventional and convenient interface to an efficient library implementation.
By efficient, we mean about as fast as functional languages for closed cases and
much better than code generated for visitor patterns by commercial optimizers 
for open cases\cite{TypeSwitch}.

Our current solution is a proof of concept that sets a minimum threshold for 
performance, brevity, clarity and usefulness of a language solution for pattern 
matching in C++. It provides full functionality, so we can experiment with use 
of pattern matching in C++ and with language alternatives. To give an idea of 
what our library offers, consider an example from a domain where pattern matching 
is considered to provide terseness and clarity -- compiler construction. 
Consider for example a simple language of expressions:

\begin{lstlisting}
@$exp$ \is{} $val$ \Alt{} $exp+exp$ \Alt{} $exp-exp$ \Alt{} $exp*exp$ \Alt{} $exp/exp$@
\end{lstlisting}

\noindent
An OCaml data type describing this grammar as well as a simple evaluator of expressions 
in it, can be declared as following:

\begin{lstlisting}[language=Caml,keepspaces,columns=flexible]
type expr = Value of int 
          | Plus  of expr * expr | Minus  of expr * expr 
          | Times of expr * expr | Divide of expr * expr
          ;;

let rec eval e =
  match e with
            Value  v      -> v
          | Plus   (a, b) -> (eval a) + (eval b)
          | Minus  (a, b) -> (eval a) - (eval b)
          | Times  (a, b) -> (eval a) * (eval b)
          | Divide (a, b) -> (eval a) / (eval b)
          ;;
\end{lstlisting}

\noindent
The corresponding C++ data types would most likely be parameterized, but for
now we will just use simple classes:

\begin{lstlisting}[keepspaces,columns=flexible]
struct Expr { virtual @$\sim$@Expr() {} };
struct Value  : Expr { int value; };
struct Plus   : Expr { Expr* exp1; Expr* exp2; };
struct Minus  : Expr { Expr* exp1; Expr* exp2; };
struct Times  : Expr { Expr* exp1; Expr* exp2; };
struct Divide : Expr { Expr* exp1; Expr* exp2; };
\end{lstlisting}

\noindent
Using our library, we can express matching about as tersely as OCaml:

\begin{lstlisting}[keepspaces,columns=flexible]
int eval(const Expr& e)
{
    Match(e)
    {
      Case(Value,  n)    return n;
      Case(Plus,   a, b) return eval(a) + eval(b);
      Case(Minus,  a, b) return eval(a) - eval(b);
      Case(Times,  a, b) return eval(a) * eval(b);
      Case(Divide, a, b) return eval(a) / eval(b);
    }
    EndMatch
}
\end{lstlisting}

\noindent
To make the example fully functional we need to provide mappings of binding 
positions to corresponding class members:

\begin{lstlisting}[keepspaces,columns=flexible]
template <> struct bindings<Value>  { CM(0,Value::value); };
template <> struct bindings<Plus>   { CM(0,Plus::exp1); 
  ...                                 CM(1,Plus::exp2);   };
template <> struct bindings<Divide> { CM(0,Divide::exp1); 
                                      CM(1,Divide::exp2); };
\end{lstlisting}

\noindent
This binding code would be implicitly provided by the compiler had
we chosen that implementation strategy.

The syntax is provided without any external tool support. Instead we rely on a 
few C++11 features~\cite{C++11}, template meta-programming, and macros. It runs 
about as fast as OCaml and Haskell equivalents (\textsection\ref{sec:ocaml}), and, depending 
on the usage scenario, compiler and underlying hardware, comes close or 
outperforms the handcrafted C++ code based on the \emph{visitor design pattern} 
(\textsection\ref{sec:eval}).

\subsection{Motivation}

The ideas and the 
\emph{Mach7} library were motivated by our unsatisfactory experiences working 
with various \Cpp{} front-ends and program analysis frameworks~\cite{Pivot09,gdr-2012:liz,Phoenix,Clang}. 
The problem was not in the frameworks per se, but in the fact that we had to use
the \emph{visitor design pattern}~\cite{DesignPatterns1993} to inspect, traverse, and 
elaborate abstract syntax trees of target languages. We found visitors 
unsuitable to express application logic directly, surprisingly hard to teach 
students, and often slower than handcrafted workaround techniques. 
We found dynamic casts in many places, often nested, 
because users wanted to answer simple structural 
questions without having to resort to visitors. Users preferred shorter, cleaner, 
and more direct code to visitors, even at a high cost in performance (assuming 
that the programmer knew the cost). The usage of \code{dynamic\_cast} resembled 
the use of pattern matching in functional languages to unpack algebraic data 
types. Thus, our initial goal was to develop a domain-specific library for C++ 
to express various predicates on tree-like structures as elegantly as is done in functional 
languages. This grew into a general high-performance pattern-matching library.

The library is the latest in a series of 7 libraries. The earlier versions were 
superceded because they failed to meet our standards for notation, performance, 
or generality. Our standard is set by the principle that a fair comparison must 
be against the gold standard in a field. For example, if we work on a linear 
algebra library, we must compare to Fortran or one of the industrial C++ 
libraries, rather than Java or C. For pattern matching we chose optimized OCaml 
as our standard for closed (compile-time polymorphic) sets of classes and C++ 
for uses of the visitor pattern. For generality and simplicity of use, we deemed 
it essential to do both with a uniform syntax.

\subsection{The Expression Problem}
\label{sec:exp}

%Expression problem is a problem of supporting in a programming language modular 
%extensibility of both data and functions at the same time. Functional languages
%allow for easy addition of new functions at the expense of disallowing new data
%variants. Object-oriented languages allow for easy addition of new variants at 
%the expense of disallowing new functions. Many attempts have been made to 
%resolve this dilema in both camps, nevertheless no universally accepted solution 
%that is modular, open and efficient has been found.

%Visitor Design Pattern has became de-facto standard in dealing with expression 
%problem in many industry-strength object-oriented languages because of two 
%factors: its speed and being a library solution. It comes at the cost of 
%restricting extensibility of data, increased verbosity and being hard to teach 
%and understand, but nevertheless, remains the weapon of choice for interacting 
%with numerous object-oriented libraries and frameworks. 

Type switching is related to a more general problem manifesting the differences 
in functional and object-oriented programming styles.
Conventional algebraic datatypes, as found in most functional languages, allow 
for easy addition of new functions on existing data types. However, they fall short 
in extending data types themselves (e.g. with new constructors), which requires 
modifying the source code. Object-oriented languages make 
data type extension trivial through inheritance, but the addition of new 
functions operating on these classes typically requires changes to the class 
definition. This dilemma is known as the \emph{expression problem}~\cite{Cook90,exprproblem}.

Classes differ from algebraic data types in two important ways. Firstly, they
are \emph{extensible}, for new variants can be added later by inheriting from
the base class. Secondly, they are \emph{hierarchical} and thus typically 
\emph{non-disjoint} since variants can be inherited from other variants and form 
a subtyping relation between themselves~\cite{Glew99}. In contrast, variants in 
conventional algebraic data types are \emph{disjoint} and \emph{closed}.
Some functional languages e.g. ML2000~\cite{ML2000} and its predecessor, Moby, 
were experimenting with \emph{hierarchical extensible sum types}, which are 
closer to object-oriented classes then algebraic data types are, but, 
interestingly, they provided no %neither traditional nor efficient 
facilities for performing case analysis on them.

Functional languages allow for the easy addition of new functions on existing data 
types, but fall short in extending data types themselves (e.g. with new constructors), 
which requires modifying the source code. Object-oriented languages, on the 
other hand, make data type extension trivial through inheritance, but the addition 
of new functions that work on these classes typically requires changes to the class 
definition. This dilemma was first discussed by Cook~\cite{Cook90} and then 
accentuated by Wadler~\cite{exprproblem} under the name \emph{expression problem}. Quoting Wadler:

\emph{``The Expression Problem is a new name for an old problem. The goal is
to define a datatype by cases, where one can add new cases to the
datatype and new functions over the datatype, without recompiling
existing code, and while retaining static type safety (e.g., no
casts)''}.

To better understand the problem, note that classes differ from algebraic data 
types in two important ways: they are \emph{extensible} since new variants can 
be added by inheriting from the base class, as well as \emph{hierarchical} and 
thus \emph{non-disjoint} since variants can be inherited from other variants and 
form a subtyping relation between themselves~\cite{Glew99}. This is not the case 
with traditional algebraic data types in functional languages, where the set of 
variants is \emph{closed}, while the variants are \emph{disjoint}. Some 
functional languages e.g. ML2000~\cite{ML2000} and Moby~\cite{Moby} were 
experimenting with \emph{hierarchical extensible sum types}, which are closer to 
object-oriented classes then algebraic data types are, but, interestingly, they 
did not provide pattern matching facilities on them!

Zenger and Odersky refined the expression problem in the context of 
independently extensible solutions~\cite{fool12} as a challenge to find an 
implementation technique that satisfies the following requirements:
%
\begin{itemize}
\setlength{\itemsep}{0pt}
\setlength{\parskip}{0pt}
\item \emph{Extensibility in both dimensions}: It should be possible to add new 
      data variants, while adapting the existing operations accordingly. It 
      should also be possible to introduce new functions. 
\item \emph{Strong static type safety}: It should be impossible to apply a 
      function to a data variant, which it cannot handle. 
\item \emph{No modification or duplication}: Existing code should neither be 
      modified nor duplicated.
\item \emph{Separate compilation}: Neither datatype extensions nor addition of 
      new functions should require re-typechecking the original datatype or 
      existing functions. No safety checks should be deferred until link or 
      runtime.
\item \emph{Independent extensibility}: It should be possible to combine 
      independently developed extensions so that they can be used jointly.
\end{itemize}

%Paraphrasing, the expression problem can be summarized as a problem of 
%supporting modular extensibility of both data and functions at the same time in 
%one programming language.

\noindent
Object-oriented languages further complicate the matter with the fact that 
data variants are not necessarily disjoint and may form subtyping relationships  
between themselves. We thus introduced an additional requirement based on the
Liskov substitution principle~\cite{Lis87}:

\begin{itemize}
\setlength{\itemsep}{0pt}
\setlength{\parskip}{0pt}
\item \emph{Substitutability}: Operations expressed on more general data variants
      should be applicable to ones that are more specific
      (the latter being in a subtyping relation with the former).
\end{itemize}

%Depending on the semantics of the language's subtyping relation, 
%substitutability requirement may turn pattern matching into an expensive 
%operation. OCaml, for example, that uses structural subtyping on its object 
%types, does not offer pattern 

\noindent
We will refer to a solution that satisfies all of the above requirements as \emph{open}. 
Numerous solutions have been proposed to dealing with the expression problem in both 
functional~\cite{garrigue-98,LohHinze2006} and object-oriented 
camps~\cite{Palsberg98,Krishnamurthi98,Zenger:2001,runabout}, but very few have
made their way into one of the mainstream languages. We refer the reader to Zenger 
and Odersky's original manuscript for a discussion of the approaches~\cite{fool12}. 
Interestingly, most of the discussed object-oriented solutions focused on the visitor design pattern~\cite{DesignPatterns1993} and its extensions, 
which even today seem to be the most commonly used approach to dealing with the 
expression problem in object-oriented languages.

\subsection{Visitor Design Pattern}
\label{sec:vdp}

%Discuss visitor design pattern and its problems.
%\begin{itemize}
%\item Intrusive - requires changes to the hierarchy
%\item Not open  - addition of new classes changes visitor interface
%\item Does not provide by default relation between visitors of base and derived classes
%\item Control inversion
%\item Cannot be generically extended to handling n arguments
%\end{itemize}

The \emph{visitor design pattern}~\cite{DesignPatterns1993} was devised to solve the problem 
of extending existing classes with new functions in object-oriented languages. 
Consider the above Expr example and imagine that in addition to evaluation we would like to also provide a pretty 
printing of expressions. A typical object-oriented approach would be to 
introduce a virtual function \\ \code{virtual void print() const = 0;} inside 
the abstract base class \code{Expr}, which will be implemented correspondingly 
in all the derived classes. This works well as long as we know all the required  
operations on the abstract class in advance. Unfortunately, this is very 
difficult to achieve in reality as the code evolves, especially in a production 
environment. To put this in context, imagine that after the above interface with 
pretty-printing functionality has been deployed, we decided that we need 
similar functionality that saves the expression in XML format. Adding new 
virtual function implies modifying the base class and creating a versioning 
problem with the code that has been deployed already using the old interface.

To alleviate this problem, the Visitor Design Pattern separates the 
\emph{commonality} of all such future member-functions from their 
\emph{specifics}. The former deals with identifying the most-specific derived 
class of the receiver object known to the system at the time the base class was 
designed. The latter provides implementation of the required functionality once 
the most-specific derived class has been identified. The interaction between the 
two is encoded in the protocol that fixes a \emph{visitation interface} 
enumerating all known derived classes on one side and a dispatching mechanism 
that guarantees to select the most-specific case with respect to the dynamic 
type of the receiver in the visitation interface. An implementation of this 
protocol for our Expr example might look like the following:

\begin{lstlisting}
// Forward declaration of known derived classes
struct Value; struct Plus; ... struct Divide;
@\halfline@
// Visitation interface
struct ExprVisitor
{
    virtual void visit(const Value&)  = 0;
    virtual void visit(const Plus&)   = 0;
    ...  // One virtual function per each known derived class
    virtual void visit(const Divide&) = 0;
};
@\halfline@
// Abstract base and known derived classes
struct Expr { 
    virtual void accept(ExprVisitor&) const = 0; };
struct Value : Expr { ...
    void accept(ExprVisitor& v) const { v.visit(*this); } };
struct Plus  : Expr { ...
    void accept(ExprVisitor& v) const { v.visit(*this); } };
\end{lstlisting}

\noindent
Note that even though implementations of \code{accept} member-functions in all 
derived classes are syntactically identical, a different \code{visit} is called. 
We rely here on the overload resolution mechanism of C++ to pick the most 
specialized \code{visit} member-function applicable to the static type of 
\code{*this}.

%This mere code 
%maintenance convenience unfortunately, often confuses novices on what 
%is going on. We thus would like to point out that member-functions in the 
%visitation interface are not required to be called with the same name, -- we 
%could have equally well called them \code{visit_value}, \code{visit_plus} etc. 
%making the corresponding changes to calls inside \code{Value::accept}, 
%\code{Plus::accept} etc.

A user can now implement new functions by overriding \code{ExprVisitor}'s 
functions. For example:

\begin{lstlisting}
std::string to_str(const Expr* e) // Converts expressions to string
{
  struct ToStrVisitor : ExprVisitor
  {
    void visit(const Value& e) { result = std::to_string(e.value); }
    ...
    void visit(const Divide& e) { 
        result = to_str(e.exp1) + '/' + to_str(e.exp2); 
    }
    std::string result;
  } v;
  e->accept(v);
  return v.result;
}
\end{lstlisting}

\noindent
The function \code{eval} we presented above, as well as any new function that we 
would like to add to \code{Expr}, can now be implemented in much the same way, 
without the need to change the base interface. This flexibility does not come for free, 
though, and we would like to point out some pros and cons of this solution.

The most important advantage of the visitor design pattern is the {\bf possibility 
to add new operations} to the class hierarchy without the need to change 
the interface. Its second most-quoted advantage is {\bf speed} -- the 
overhead of two virtual function calls incurred by the double  
dispatch present in the visitor design pattern is often negligible on modern 
architectures. Yet another advantage that often remains unnoticed is that the 
above solution achieves extensibility of functions with {\bf library only means} 
by using facilities already present in the language. Nevertheless, there are 
quite a few disadvantages.

The solution is {\bf intrusive} since we had to inject syntactically the same 
definition of the \code{accept} method into every class participating in visitation. 
It is also {\bf specific to hierarchy}, as we had to declare a visitation 
interface specific to the base class. The amount of {\bf boilerplate code} 
required by visitor design pattern cannot go unnoticed. It also increases with 
every argument that has to be passed into the visitor to be available during the 
visitation. This aspect can be seen in the example from \textsection\ref{sec:xmpl} 
where we have to store both functors inside the visitor.

More importantly, visitors {\bf hinder extensibility} of the class hierarchy: 
new classes added to the hierarchy after the visitation interface has been 
fixed will be treated as their most derived base class present in the interface.
A solution to this problem has been proposed in the form of \emph{Extensible 
Visitors with Default Cases}~\cite[\textsection 4.2]{Zenger:2001}; however, the 
solution, after remapping it onto C++, has problems of its own, discussed in 
detail in related work in \textsection\ref{sec:rw}.

%The visitation interface 
%hierarchy can easily be grown linearly (adding new cases for the new classes in 
%the original hierarchy each time), but independent extensions by different  
%authorities require developer's intervention to unify them all, before they can 
%be used together. This may not be feasible in environments that use dynamic 
%linking. To avoid writing even more boilerplate code in new visitors, the 
%solution would require usage of virtual inheritance, which typically has 
%an overhead of extra memory dereferencing. On top of the double dispatch already 
%present in the visitor pattern, the solution will incur two additional virtual 
%calls and a dynamic cast for each level of visitor extension. Additional double 
%dispatch is incurred by forwarding of default handling from base visitor to a 
%derived one, while the dynamic cast is required for safety and can be replaced 
%with a static case when visitation interface is guaranteed to be grown linearly 
%(extended by one authority only). Yet another virtual call is required to be 
%able to forward computations to subcomponents on tree-like structures to the 
%most derived visitor. This last function lets one avoid the necessity of using 
%heap to allocate a temporary visitor through the \emph{Factory Design 
%Pattern}\cite{DesignPatterns1993} used in \emph{Extensible Visitor} solution 
%originally proposed by Krishnamurti, Felleisen and Friedman\cite{Krishnamurthi98}.

Once all the boilerplate related to visitors has been written and the visitation 
interface has been fixed we are still left with some annoyances incurred by the 
pattern. One of them is the necessity to work with the {\bf control inversion} 
that visitors put in place. Because of it we have to save any local state and 
any arguments that some of the \code{visit} callbacks might need from the 
calling environment. Similarly, we have to save the result of the visitation, 
as we cannot assume that all the visitors that will potentially be implemented 
on a given hierarchy will use the same result type. Using visitors in a generic 
algorithm requires even more precautions. We summarize these visitor-related 
issues in the following motivating example, followed by an illustration of a 
pattern-matching solution to the same problem enabled with our library.

\subsection{Motivating Example}
\label{sec:xmpl}

While comparing generic programming facilities available to functional and 
imperative languages (mainly Haskell and C++), Dos Reis and J\"arvi present the 
following example in Haskell describing a sum functor\cite{DRJ05}:

\begin{lstlisting}[language=Haskell,keepspaces]
data Either a b = Left a | Right b
@\halfline@
eitherLift :: (a -> c) -> (b -> d) -> Either a b -> Either c d
eitherLift f g (Left  x) = Left  (f x)
eitherLift f g (Right y) = Right (g y)
\end{lstlisting}

\noindent
In simple words, the function \codehaskell{eitherLift} above takes two functions and an 
object and depending on the actual type constructor the object was created with, 
calls first or second function on the embedded value, encoding the result 
correspondingly.

Its equivalent in C++ is not straightforward. Idiomatic, type-safe, handling of 
discriminated unions in C++ typically assumes use of the \emph{Visitor Design Pattern}\cite{DesignPatterns}, 
which in this case amounts to 25 lines of ``boiler plate code'' plus 14 lines 
of the specific functionality.

\begin{lstlisting}
template <class X, class Y> class Either;
template <class X, class Y> class Left;
template <class X, class Y> class Right;
@\halfline@
template <class X, class Y>
struct EitherVisitor {
    virtual void visit(const  Left<X,Y>&) = 0;
    virtual void visit(const Right<X,Y>&) = 0;
};
@\halfline@
template <class X, class Y>
struct Either {
    virtual @$\sim$@Either() {}
    virtual void accept(EitherVisitor<X,Y>& v) const = 0;
};
@\halfline@
template <class X, class Y>
struct Left : Either<X,Y> {
    const X& x;
    Left(const X& x) : x(x) {}
    void accept(EitherVisitor<X,Y>& v) const { v.visit(*this); }
};
@\halfline@
template <class X, class Y>
struct Right : Either<X,Y> {
    const Y& y;
    Right(const Y& y) : y(y) {}
    void accept(EitherVisitor<X,Y>& v) const { v.visit(*this); }
};
\end{lstlisting}

\noindent
The code above defines the necessary parameterized data structures as well as a 
correspondingly parameterized visitor class capable of introspecting it at 
run-time. The authors agree with us \emph{``The code has a fair amount of 
boilerplate to simulate pattern matching...''}\cite{DRJ05} The actual 
implementation of \codehaskell{lift} in C++ now amounts to declaring and 
invoking a visitor:

\begin{lstlisting}
template <class X, class Y, class S, class T>
const Either<S,T>& eitherLift(const Either<X,Y>& e, S f(X), T g(Y))
{
    typedef S (*F)(X);
    typedef T (*G)(Y);
    struct Impl : EitherVisitor<X,Y> {
        F f;
        G g;
        const Either<S,T>* value;
        Impl(F f, G g) : f(f), g(g), value() {}
        void visit(const Left<X,Y>& e) {
            value = left<S,T>(f(e.x));
        }
        void visit(const Right<X,Y>& e) {
            value = right<S,T>(g(e.y));
        }
    };
    Impl vis(f, g);
    e.accept(vis);
    return *vis.value;
}
\end{lstlisting}

\noindent
The same function expressed with our pattern-matching facility seems to be much 
closer to the original Haskell definition:

\begin{lstlisting}[keepspaces,columns=flexible]
template <class X, class Y, class S, class T>
const Either<S,T>* lift(const Either<X,Y>& e, S f(X), T g(Y))
{
    Match(e)
      Case(( Left<X,Y>), x) return  left<S,T>(f(x));
      Case((Right<X,Y>), y) return right<S,T>(g(y));
    EndMatch
}
\end{lstlisting}

\noindent
This is also as fast as the visitor solution, but unlike the visitors based 
approach it neither requires \code{EitherVisitor}, nor any of the injected 
\code{accept} member-functions. We do require binding definitions though to be 
able to bind variables \code{x} and \code{y}:

%@\footnote{We need to take the first argument in parentheses to avoid interpretation of comma in template argument list by the preprocessor}@
%\footnote{Definitions of obvious functions \code{left} and \code{right} have 
%been ommitted in both cases.}

\begin{lstlisting}[keepspaces,columns=flexible]
template <class X, class Y> 
    struct bindings<Left<X,Y>>  { CM(0, Left<X,Y>::x); };
template <class X, class Y> 
    struct bindings<Right<X,Y>> { CM(0,Right<X,Y>::y); };
\end{lstlisting}

\noindent
Note that these binding definitions are made once for all possible instantiations 
with the use of partial template specialization in C++ and would not be needed 
if we implemented pattern matching in a compiler rather than a library.

\subsection{Summary}

The contributions of the paper are twofold and can be summarized as following:

\begin{itemize}
\setlength{\itemsep}{0pt}
\setlength{\parskip}{0pt}
\item We present techniques based on memoization (\textsection\ref{sec:copc}) and 
class precedence list (\textsection\ref{sec:cotc}) that can be used to implement 
type switching efficiently based on the run-time type of the argument.

  \begin{itemize}
  \setlength{\itemsep}{0pt}
  \setlength{\parskip}{0pt}
  \item The techniques come close and often outperform its de facto contender -- 
        visitor design pattern -- without sacrificing extensibility (\textsection\ref{sec:eval}).
  \item They work in the presence of multiple inheritance, including repeated and 
        virtual inheritance, as well as in generic code (\textsection\ref{sec:vtblmem}).
  \item The solution is open by construction (\textsection\ref{sec:poets}), 
        non-intrusive, and avoids the control inversion typical for visitors.
  \item It applies to polymorphic (\textsection\ref{sec:vtp}-\ref{sec:vtblmem}) and 
        tagged (\textsection\ref{sec:cotc}) class hierarchies through a unified  
        syntax~\cite{AP}.
  \item Our memoization device (\textsection\ref{sec:memdev}) generalizes to 
        other languages and can be used to implement type switching 
        (\textsection\ref{sec:vtblmem}), type testing 
        (\textsection\ref{sec:poets},\cite[\textsection 4.7]{TR}), predicate dispatch 
        (\textsection\ref{sec:memdev}), and multiple dispatch 
        (\textsection\ref{sec:cc}) efficiently.
  \item We list conditions under which virtual table pointers, commonly used in 
        C++ implementations, uniquely identify the exact subobject within the 
        most derived type (\textsection\ref{sec:vtp}).
  \item We also build an efficient cache indexing function for virtual table 
        pointers that minimizes the amount of conflicts 
        (\textsection\ref{sec:sovtp},\ref{sec:moc},\cite[\textsection 4.3.5]{TR}).
  \end{itemize}
\item We present a functional style pattern matching for C++ built as a library 
      employing the above technique. Our solution:
  \begin{itemize}
  \item Is open, non-intrusive and avoids the control inversion typical for visitors.
  \item Can be applied retroactively to any polymorphic or tagged class hierarchy.
  \item Provides a unified syntax for various encodings of extensible 
        hierarchycal datatypes in C++.
  \item Generalizes the controversial n+k patterns by leaving semantic choices to the user.
  \item Supports a limited form of views.
  \item Is simpler to use than conventional object-oriented or union-based alternatives.
  \item Improves performance compared to alternatives in real applications.
  \end{itemize}
\end{itemize}

\noindent
Our technique can be used in a compiler and/or library setting to implement 
facilities that depend on dynamic type or run-time properties of objects: e.g. 
type switching, type testing, pattern matching, predicate dispatch, 
multi-methods etc. We also look at different approaches to endoding algebraic 
data types in C++ and present a unified pattern-matching syntax that works 
uniformly with all of them. 
We generalize Haskell's n+k patterns\cite{haskell98} to any invertible operations. 
Semantics issues that typically accompany n+k pattern are handled transparently 
by forwarding the problem into the concepts domain, thanks to the fact that we 
work in a library setting. We also provide support for views in a form that 
resembles extractors in Scala. 

A practical benefit of our solution is that it can be used right away with any 
compiler with a decent support of C++0x without requiring the installation of 
any additional tools or preprocessors. The solution is a proof of concept that 
sets a minimum threshold for the performance, brevity, clarity and usefulness of 
a language solution for open type switching in C++.

The rest of this paper is structured as following. In Section~\ref{sec:bg}, we 
present evolution of pattern matching in different languages, presenting 
informally through example commonly used terminology and semantics of various 
pattern-matching constructs. Section~\ref{sec:adt} presents various approaches 
that are taken in C++ to encoding algebraic data types.
Sections~\ref{sec:syn} and~\ref{sec:sem} describe the syntax and semantics of 
our pattern matching facilities. Sections~\ref{sec:slv} and~\ref{sec:view} discuss 
approach taken by our library in handling generalized n+k patterns and views. 
Section~\ref{sec:impl} discusses techniques that makes type switching, used as a 
back-bone of the match statement, efficient, while section~\ref{sec:eval} 
provides its performance evaluation against some common alternatives. 
Section~\ref{sec:rw} discusses related work, and section~\ref{sec:cc} concludes 
by discussing some future directions and possible improvements.

\section{Pattern Matching by Example} %%%%%%%%%%%%%%%%%%%%%%%%%%%%%%%%%%%%%%%%%%
\label{sec:bg}

Pattern matching is an abstraction mechanism that provides syntax very close to 
mathematical notations. It allows the user tersely describe a (possibly 
infinite) set of values accepted by the pattern. A \emph{pattern} represents a 
predicate on values, and is usually more concise and readable than equivalent 
imperative code. An interesting peculiarity about patterns is that a given 
pattern is rarely repeated in several places in code. This context sensitivity 
makes them unattractive for code reuse through a dedicated predicate, since the 
predicate will typically only be used once.

Pattern matching is closely related to \emph{algebraic data types} and 
\emph{equational reasoning}. In ML and Haskell an \emph{Algebraic Data Type} is 
a data type each of whose values are picked from a disjoint sum of (possibly 
recursive) data types, called \emph{variants}. Each variant is marked with a 
unique symbolic constant called a \emph{constructor}. Constructors provide a 
convenient way of creating a value of its variant type as well as discriminating 
among variants through pattern matching. Consider a simple expression language:
\begin{center}
$exp$ \is{} $val$ \Alt{} $exp+exp$ \Alt{} $exp-exp$ \Alt{} $exp*exp$ \Alt{} $exp/exp$
\end{center}

\noindent
This can be represented in OCaml as an algebraic data type:

\begin{lstlisting}[language=Caml,keepspaces,columns=flexible]
type expr = Value  of int 
          | Plus   of expr * expr 
          | Minus  of expr * expr 
          | Times  of expr * expr 
          | Divide of expr * expr ;;
\end{lstlisting}

\noindent
The set of values described by such an algebraic data type is defined 
inductively as the least set closed under constructor functions of its variants.
A simple evaluator of such expressions can be defined:

\begin{lstlisting}[language=Caml,keepspaces,columns=flexible]
let rec eval e = match e with 
            Value  v     -> v 
          | Plus  (a, b) -> (eval a) + (eval b) 
          | Minus (a, b) -> (eval a) - (eval b)
          | Times (a, b) -> (eval a) * (eval b) 
          | Divide(a, b) -> (eval a) / (eval b) ;;
\end{lstlisting}

\noindent
There are two critical differences between algebraic data types and classes in 
object-oriented languages: definition of an algebraic data type is \emph{closed}, 
while its variants are \emph{disjoint}. Closedness means that once we have listed 
all the variants a given algebraic data type may have we cannot extend it with 
new variants without modifying its definition. Disjointedness means that a value 
of an algebraic data type belongs to exactly one of its variants. Neither is 
the case in object-oriented languages. Classes are \emph{extensible},
since new variants can be added through subclassing, as well as 
\emph{hierarchical}, since variants are not necessarily disjoint and can form 
subtyping relation between themselves. The above algebraic data type can be 
encoded in C++ as following:

\begin{lstlisting}[columns=flexible]
struct Expr { virtual @$\sim$@Expr(){}};
struct Value  : Expr { int value; };
struct Plus   : Expr { Expr* e1; Expr* e2; }; ...
struct Divide : Expr { Expr* e1; Expr* e2; };
\end{lstlisting}

\noindent
while a similar evaluator in our SELL is almost as terse as OCaml code:

\begin{lstlisting}[columns=flexible]
int eval(const Expr* e)
{
  Match(e) {
    Case(Value,  n)    return n;
    Case(Plus,   a, b) return eval(a) + eval(b); 
    Case(Minus,  a, b) return eval(a) - eval(b);
    Case(Times,  a, b) return eval(a) * eval(b); 
    Case(Divide, a, b) return eval(a) / eval(b);
  } EndMatch
}
\end{lstlisting}

\noindent
An expression \code{e} used as an argument of a \emph{matching expression} or a 
\emph{match statement}, is usually referred to as \emph{subject}, and its type 
as a \emph{subject type}. We will also refer to the type of the value expected 
by a given pattern as a \emph{target type}.

\emph{Polymorphic variants} in OCaml\cite{garrigue-98} and \emph{open data 
types} in Haskell\cite{LohHinze2006} allow addition of new variants later. 
These extensions are simpler however as they maintain the 
disjointedness property: open data types do not introduce any subtyping relation, 
while the subtyping relation on polymorphic variants relates anonymous 
combinations of variants and not the variants themselves. In contrast, 
the \emph{nominative subtyping} of object-oriented languages does not maintain 
the disjointness property, making objects effectively belong to multiple classes. 

Closedness of algebraic data types is particularly useful for reasoning about 
programs by case analysis and allows the compiler to perform an automatic 
\emph{incompleteness} check -- test of whether a given match statement 
covers all possible cases. Similar reasoning about programs involving extensible 
data types is more involved as we are dealing with potentially open set of 
variants. \emph{Completeness} check in such scenario reduces to checking presence 
of a case that handles the static type of the subject. Absence of such a case,
however, does not necessarily imply incompleteness, only potential incompleteness, 
as the answer will depend on the actual set of variants available at run-time.

A related notion of \emph{redundancy} checking arises from the 
tradition of using \emph{first-fit} strategy in pattern matching. It warns the 
user of any \emph{case clause} inside a match statement that will 
never be entered because of a preceding one being more general. Object-oriented 
languages, typically prefer \emph{best-fit} strategy because it is not prone 
to errors where semantics of a statement might change depending on the ordering 
of preceding definitions. 

The patterns used in functions \codeocaml{eval} and \codehaskell{eitherLift} to 
identify and decompose a concrete variant of an algebraic data types are 
generally called \emph{tree patterns} or \emph{constructor patterns}. Their 
analog in object-oriented languages is often referred to as \emph{type pattern} 
since it may involve type testing and type casting. Special cases of tree patterns  
are \emph{list patterns} and \emph{tuple patterns} that make use of special list 
and tuple constructors \codehaskell{:} and \codehaskell{(,,...,)}.

Pattern matching can also be applied to built-in types.
In Haskell, a factorial can be defined by equations:

\begin{lstlisting}[language=Haskell]
factorial 0 = 1
factorial n = n * factorial (n-1)
\end{lstlisting}

\noindent
Here 0 in the left hand side of the first \emph{equation} is an example of a 
\emph{value pattern} that will only match when the actual argument passed is 0. 
The \emph{variable pattern} \codehaskell{n} in the left hand side of the second 
equation will match any value, \emph{binding} variable \codehaskell{n} to it in 
the right hand side of equation. The \emph{wildcard pattern} \codehaskell{_}  
will match any value, neither binding it to a variable nor even obtaining it. 
Value patterns, variable patterns and wildcard patterns are  
generally called \emph{primitive patterns}. Patterns like variable and wildcard 
patterns that never fail to match are called \emph{irrefutable}, in contrast to 
\emph{refutable} patterns like value patterns, which may fail to match.

In Haskell 98\cite{Haskell98Book} factorial could alternatively be defined as:

\begin{lstlisting}[language=Haskell]
factorial 0 = 1
factorial (n+1) = (n+1) * factorial n
\end{lstlisting}

\noindent
The \codehaskell{(n+1)} pattern in the left hand side of equation is an example of 
\emph{n+k pattern}. According to its informal semantics ``Matching an $n+k$ 
pattern (where $n$ is a variable and $k$ is a positive integer literal) against 
a value $v$ succeeds if $v \ge k$, resulting in the binding of $n$ to $v-k$, and 
fails otherwise''\cite{haskell98}. n+k patterns were introduced into Haskell to 
let users express inductive functions on natural numbers in much the same way as 
functions defined through case analysis on algebraic data types. Besides 
succinct notation, this could facilitate automatic proof of 
termination of such functions by the compiler.
However, numerous debates over semantics and usefulness of n+k patterns
resulted in their removal from the Haskell 
2010 standard\cite{haskell2010}. Generalization of n+k patterns, called 
\emph{application patterns} has been studied by Oosterhof\cite{OosterhofThesis}. 
Application patterns essentially treat n+k patterns as equations, while matching 
against them attempts to solve or validate the equation.

Our library language supports generalized n+k patterns in a different form 
(\textsection\ref{sec:slv}). We do not restrict ourselves with equational view 
of the n+k patterns, but allow the user to specify suitable semantics.
In our library, we can define fast computation of Fibonacci numbers like this:

\begin{lstlisting}[keepspaces]
int fib(int n)
{
    variable<int> m;
    Match(n) {
      When(1)         return 1;     
      When(2)         return 1;
      When(2*m)     return sqr(fib(m+1)) - sqr(fib(m-1));
      When(2*m+1) return sqr(fib(m+1)) + sqr(fib(m));
    } EndMatch
}
\end{lstlisting}

\noindent
A \emph{guard} 
is a predicate attached to a pattern that may make use of the variables bound in 
it. The result of its evaluation will determine whether the case clause and the 
body associated with it will be \emph{accepted} or \emph{rejected}.

In OCaml, we can define rules for rewriting terms in 
our $exp$ language with the help of guards:

\begin{lstlisting}[language=Caml,keepspaces,columns=flexible]
let collect e = match e with
      Plus(Times(e1,e2), Times(e3,e4)) when e1 = e3 -> Times(e1, Plus(e2,e4))
    | Plus(Times(e1,e2), Times(e3,e4)) when e2 = e4 -> Times(Plus(e1,e3), e4)
    | e -> e ;;
\end{lstlisting}

\noindent
Attempting to write the first case clause as \codeocaml{Plus(Times(e,e2), 
Times(e,e4))} would have been relying on \emph{equivalence patterns}. 
Neither OCaml nor Haskell support such patterns, but
Miranda\cite{Miranda85} and Tom\cite{Moreau:2003} do.

The example above illustrates yet another common pattern-matching facility -- 
\emph{nesting of patterns}. Constructor patterns composed 
solely of (distinct) variables are called \emph{simple pattern}s
and others are called \emph{nested pattern}s.
Nested checks are hard to handle using the visitor design pattern, as they are often 
too context-dependent to extract them into a dedicated visitor. 
In such cases, users typically prefer \emph{type tests} and \emph{type 
casts}. Our library handles such cases using nested patterns:
\begin{lstlisting}
const Expr* collect(const Expr* e)
{
  const Expr *e1, *e2, *e3, *e4;
  Match(e) {
    When(match<Plus>(match<Times>(e1,e2),match<Times>(e3 |= e1==e3,e4))) 
        return new Times(e1, new Plus(e2,e4));
    When(match<Plus>(match<Times>(e1,e2),match<Times>(e3,e4 |= e2==e4))) 
        return new Times(new Plus(e1,e3), e4);
    When() 
        return e;
  } EndMatch
}
\end{lstlisting}

\noindent
Decomposing algebraic data types through pattern matching has an important 
drawback that was originally spotted by Wadler\cite{Wadler87}: they expose 
concrete representation of an abstract data type, which conflicts with the 
principle of \emph{data abstraction}. To overcome the problem he proposed the 
notion of \emph{views} that represent conversions between different 
representations that are implicitly applied during pattern matching. As an 
example, imagine polar and cartesian representations of complex numbers. A user 
might choose polar representation as a concrete representation for the abstract 
data type \codeocaml{complex}, treating cartesian representation as view or vice 
versa:\footnote{We use the syntax from Wadler's original paper for this example}

\begin{lstlisting}[language=Haskell,columns=flexible]
complex ::= Pole real real
view complex ::= Cart real real
  in  (Pole r t) = Cart (r * cos t) (r * sin t)
  out (Cart x y) = Pole (sqrt(x^2 + y^2)) (atan2 x y)
\end{lstlisting}

\noindent
The operations then might be implemented in whatever representation is the most 
suitable, while the compiler will implicitly convert representation if needed:

\begin{lstlisting}[language=Haskell,columns=flexible]
  add  (Cart x1 y1) (Cart x2 y2) = Cart (x1 + x2) (y1 + y2)
  mult (Pole r1 t1) (Pole r2 t2) = Pole (r1 * r2) (t1 + t2)
\end{lstlisting}

\noindent
The idea of views were later adopted in various forms in several languages: 
Haskell\cite{views96}, Standard ML\cite{views98}, Scala (in the form of 
\emph{extractors}\cite{EmirThesis}) and F$\sharp$ (under the name of 
\emph{active patterns}\cite{Syme07}). We demonstrate our support of views in 
\textsection\ref{sec:view}.

\section{Pattern Matching SELL} %%%%%%%%%%%%%%%%%%%%%%%%%%%%%%%%%%%%%%%%%%%%%%%%
\label{sec:pm}

A Semantically Enhanced Library Language is not a language of its own, but 
rather a sub-language embedded into another language -- \Cpp{} in our case. The 
sub-language still has the facilities we typically associate with programming 
languages e.g. syntax, semantics, type system, but they are constrained by
the host language. A particular sub-language can often be implemented in 
different host languages, which is why it is important to describe it 
independently of its host. We thus shall abstract from describing the 
exact syntax and semantics of host-language features that are well understood and documented 
elsewhere~\cite{C++11}.

\subsection{Syntax}
\label{sec:syn}

\begin{figure}[h]
\centering
\begin{tabular}{rcll}
\Rule{Match Statement}     & $M$       & \is{}  & \code{Match(}$e$\code{)} $\{ \left[C s^*\right]^* \}$ \code{EndMatch} \\
\Rule{Case Clause}         & $C$       & \is{}  & \code{Qua(}$T\left[,\varpi\right]^*$\code{)}\Alt{}\code{When(}$\varpi^*$\code{)}\Alt{}\code{Case(}$T\left[,x\right]^*$\code{)}\Alt{}\code{Otherwise(}$x^*$\code{)} \\
\Rule{Target Expression}   & $T$       & \is{}  & $\tau$ \Alt{} $l$ \\
\Rule{Layout}              & $l$       & \is{}  & $c^{\mathsf{int}}$ \\
\Rule{Match Expression}    & $m$       & \is{}  & $\pi(e)$ \\
\Rule{Extended Pattern}    & $\varpi$  & \is{}  & $\pi$ \Alt{} $c$ \Alt{} $x$ \\
\Rule{Pattern}             & $\pi$     & \is{}  & $\mu$ \Alt{} $\varrho$ \Alt{} $\eta$ \Alt{} $\chi$ \Alt{} $\varsigma$ \Alt{} $\_$ \\
\Rule{Constructor Pattern} & $\mu$     & \is{}  & \code{match<}$\tau\left[,l\right]$\code{>(}$\varpi^*$\code{)} \\
\Rule{Guard Pattern}       & $\varrho$ & \is{}  & $\pi \models \xi$ \\
\Rule{n+k Pattern}         & $\eta$    & \is{}  & $\xi$ \\
\Rule{Variable Pattern}    & $\chi^{\mathsf{variable}\langle\tau\rangle}$   \\
\Rule{Value Pattern}       & $\varsigma^{\mathsf{value}\langle\tau\rangle}$ \\
\Rule{Wildcard Pattern}    & $\_^{\mathsf{wildcard}}$                       \\
\Rule{Lazy Expression}     & $\xi$     & \is{}  & $\varsigma$ \Alt{} $\chi$ \Alt{} $\xi \oplus c$ \Alt{} $c \oplus \xi$ \Alt{} $\ominus \xi$ \Alt{} $(\xi)$ \Alt{} $\xi \oplus \xi$ \Alt{} $\varphi(\xi^*)$ \\
\Rule{Lazy Function}       & $\varphi^{\xi^*\rightarrow \xi}$ \\
\Rule{Unary Operator}      & $\ominus$ & $\in$  & $\lbrace*,\&,+,-,!,\sim\rbrace$ \\
\Rule{Binary Operator}     & $\oplus$  & $\in$  & $\lbrace*,/,\%,+,-,\ll,\gg,\&,\wedge,|,<,\leq,>,\geq,=,\neq,\&\&,||\rbrace$ \\
\Rule{Type-Id}             & $\tau$    &        & \Cpp{}\cite[\textsection A.7]{C++11} \\
\Rule{Statement}           & $s$       &        & \Cpp{}\cite[\textsection A.5]{C++11} \\
\Rule{Expression}          & $e^\tau$  &        & \Cpp{}\cite[\textsection A.4]{C++11} \\
\Rule{Constant-Expression} & $c^\tau$  &        & \Cpp{}\cite[\textsection A.4]{C++11} \\
\Rule{Identifier}          & $x^\tau$  &        & \Cpp{}\cite[\textsection A.2]{C++11} \\
\end{tabular}
\caption{Abstract syntax of our pattern-matching SELL}
\label{syntax}
\end{figure}

Figure~\ref{syntax} presents the abstract syntax of our pattern-matching SELL. It presents 
the syntax embedded into the \Cpp{} without sacrificing 
the clarity of presentation. The idea is to show which interactions are possible 
within our SELL, while leaving the details of their implementation to 
\textsection\ref{sec:impl}. Where the specific technique we use to achieve such 
interactions crucially depends on the types of the entities involved,
we mention their type in the superscript. This dependence on the 
type system of the host language was also the reason why we chose abstract 
syntax over traditional EBNF. We make use 
of few non-terminals from the host language in order to put our constructs into 
context.

\emph{Match statement} is an analog of a switch statement with patterns as case 
clauses. Similar control structures exist in many programming languages and 
date back to at least Simula's Inspect statement~\cite{Simula67}.
In a library-based solution, we require it to be closed with a dedicated 
\code{EndMatch} macro to ensure proper nesting.

We support four kinds of \emph{case clauses}: \code{Qua}, \code{When}, 
\code{Case}, and \code{Otherwise}.
The distinction between them is only important for the library 
implementation and can trivially be inferred in a compiler solution.
A \code{Qua}-clause is the most general clause taking an  
expression that identifies the target type as well as a list of extended 
patterns.
A \code{Qua}-clause permits nested patterns, but requires all the 
variables used in the patterns to be explicitly pre-declared. \code{Case}-clause 
only accepts simple patterns, conveniently introducing all the variables into the 
clause's scope. 
A \code{When}-clause takes only patterns while its target type is 
the subject type.
An \code{Otherwise} clause is an irrefutable clause that is 
semantically equivalent to \code{Case}-clause with subject type used as a target 
type. When used it should be the last clause of the match statement.

A \emph{Target expression} is used by case clauses as either a target type or 
a constant value, representing \emph{layout}. \emph{Layout} is an enumerator 
that the user may use to define alternative bindings for the same class. They are 
discussed in \textsection\ref{sec:bnd}. The use of layout as target 
expression is only allowed for union encoding of algebraic data types 
(\textsection\ref{sec:unisyn}), in which case the library assumes the target 
type to be the subject type.

A \emph{Match expression} is an inline version of the match statement with 
a single \code{Qua}-clause. Applying a pattern to a subject checks whether the 
subject matches the pattern.

\emph{Pattern} summarizes all the patterns supported by the library. 
\emph{Extended pattern} indicates contexts in which our library implicitly 
permits the use constants as \emph{value patterns} and regular \Cpp{} variables as 
\emph{variable patterns}. The library recognizes them and transforms into 
$\varsigma$ and $\chi$ respectively.
A \emph{Constructor pattern} takes a target type, an optional layout and a list of 
nested sub-patterns.
\emph{Guard patterns} are composed from a pattern and a condition separated with 
\code{|=}.
We chose operator \code{|=} because of its low precedence 
allows most other operators to be used inside the 
condition without parenthesis. The right operand of \code{|=} is allowed to make use of any 
variables bound in the left operand. When used on arguments of a constructor 
pattern, it is also allowed to make use of any variables bound by preceding 
argument positions. 
\emph{n+k patterns} are a subset of \emph{lazy expressions} for which the user has 
provided \emph{solvers} -- overloaded functions defining semantics of matching a 
value against an expression(\textsection~\ref{sec:slv}).
\emph{Variable patterns} refers to variables whose \Cpp{} type is \code{variable<T>} for 
any given type \code{T}.
A \emph{Value pattern} is almost never declared explicitly, 
but is implicitly introduced by the library in the contexts where $c$ is 
accepted.
A \emph{Wildcard pattern} is represented with a constant of type 
\code{wildcard}.
A \emph{Lazy expression} refers to lazily evaluated expressions introduced by our SELL, 
as opposed to eagerly evaluated expressions, directly supported by \Cpp{}. The use 
of $c$ indicates contexts in which constants can be used as lazy expressions and 
is similarly replaced with $\varsigma$. \emph{Lazy function} represents 
functions that can participate in lazy evaluations. Such functions have to be 
declared in certain way and are discussed in \textsection\ref{sec:slv}. 

\emph{Binary operator} and \emph{unary operator} name a subset of \Cpp{} operators we 
make use of and provide support for in our pattern-matching library. 
The remaining syntactic categories refer to non-terminals in the \Cpp{} grammar 
bearing the same name.

\subsection{Typing Rules}

\begin{figure}[h]
\begin{mathpar}

\inferrule[T-Var]
{}
{\Gamma\vdash \chi : \Variable{T}}

\inferrule[T-Value]
{}
{\Gamma\vdash \varsigma : \Value{T}}

\inferrule[T-Wildcard]
{}
{\Gamma\vdash \_ : \Wildcard}

\inferrule[T-Unary]
{\Gamma\vdash \xi : E}
{\Gamma\vdash \ominus \xi : \ExprU{F_\ominus}{E} }

\inferrule[T-Binary]
{\Gamma\vdash \xi_1 : E_1 \\ \Gamma\vdash \xi_2 : E_2}
{\Gamma\vdash \xi_1 \oplus \xi_2 : \ExprB{F_\oplus}{E_1}{E_2} }

%\inferrule[T-Binary-Const-Left]
%{\Gamma\vdash \xi : E \\ \Gamma\vdash c : T}
%{\Gamma\vdash c \oplus \xi : \ExprB{F_\oplus}{\Value{T}}{E} }

%\inferrule[T-Binary-Const-Right]
%{\Gamma\vdash \xi : E \\ \Gamma\vdash c : T}
%{\Gamma\vdash \xi \oplus c : \ExprB{F_\oplus}{E}{\Value{T}} }

\inferrule[T-Function]
{\Gamma\vdash \xi_1 : E_1 \\ \cdots \\ \Gamma\vdash \xi_k : E_k}
{\Gamma\vdash \varphi(\xi_1,\cdots,\xi_k) : \ExprK{F_\varphi}{E_1}{E_k} }

\inferrule[T-Guard]
{\Gamma\vdash \pi : E_1 \\ \Gamma\vdash \xi : E_2}
{\Gamma\vdash \pi \models \xi : \Guard{E_1}{E_2} }

\inferrule[T-Constructor]
{\Gamma\vdash \varpi_1 : E_1 \\ \cdots \\ \Gamma\vdash \varpi_k : E_k}
{\Gamma\vdash \mathsf{match}\langle T\left[,l\right]\rangle(\varpi_1,\cdots,\varpi_k) : \Cnstr{T\left[,l\right]}{E_1}{E_k} }

%\inferrule[T-Extended-Pattern]
%{ \varpi = \pi \\ \Gamma\vdash \pi : E}
%{\Gamma\vdash \varpi : E}

%\inferrule[T-Extended-Value]
%{ \varpi = c \\ \Gamma\vdash c : T}
%{\Gamma\vdash \varpi : \Value{T}}

%\inferrule[T-Extended-Var]
%{ \varpi = x \\ \Gamma\vdash x : T}
%{\Gamma\vdash \varpi : \Variable{T}}

\end{mathpar}
\caption{Typing rules for our pattern-matching SELL}
\label{typing}
\end{figure}

Figure~\ref{typing} shows rules we use to type expressions in our SELL. The 
types presented are not necessarily the exact \Cpp{} types we use to encode them, 
but we keep the correspondence as close as possible to reflect the actual 
implementation. We use the following type constructors, indicated with their 
arity: $\CWildcard^0$, $\CValue^1$, $\CVariable^1$, $\CExpr^{1+n}$, $\CGuard^2$, 
$\CCnstr^{1+n}$. We assume that type variables $T_i$ range over any \Cpp{} types, 
while $E_i$ only range over types marked with these type constructors, to which 
we refer as \emph{SELL-types}.

The judgments are of the traditional form $\Gamma\vdash \varpi : E$ that can be 
interpreted as given a typing environment $\Gamma$, an extended pattern $\varpi$ is 
given a SELL-type $E$. $\Gamma$ represents the typing context of the \Cpp{} 
compiler with the allowance for our simplified representation of SELL-types.
Types $F_\oplus$, $F_\ominus$ and $F_\varphi$ are described in greater details 
in \textsection\ref{sec:sem}, while for the purpose of typing they can be 
interpreted as types that uniquely identify operations $\oplus$, $\ominus$ and 
$\varphi$ respectively.

To avoid confusion we would like to point out that syntactic categories $\chi$, 
$\varsigma$ and $\_$ are defined as objects of \Cpp{} types \code{value<T>}, 
\code{variable<T>} and \code{wildcard}, while here we type them with SELL-types 
$\Value{T}$, $\Variable{T}$ and $\Wildcard$. Internally these types are the same 
of course.

\subsection{Semantics}
\label{sec:sem}

We use natural semantics\cite{Kahn87} to describe the semantics of our 
pattern-matching extension. Because our SELL can be customized in a number of 
ways, we make use of several semantic functors that let the user define the 
semantics of the following operations:

\begin{compactitem}
\setlength{\itemsep}{0pt}
\setlength{\parskip}{0pt}
\item Type casting: $F_{dc}(\tau,v)$
\item Lazily evaluated functions: $F_\oplus,F_\ominus,F_\varphi$
\item Structural decomposition: $\Delta_i^{\tau,l}$
\item Algebraic decomposition: $\mathsf{solve}(\eta,v)$
\end{compactitem}

\noindent
The type of the subject used in pattern matching is not always the same as the 
type that a given pattern expects. The library in such a case may need to 
perform type casting of the subject, which may involve but is not limited to 
down-casting, up-casting, cross-casting or conversion. Depending on the types 
involved, such casting can be performed in different ways, which is why we 
abstract from a concrete semantics of such an operation with functor $F_{dc}$. 
We use the notation $F_{dc}(\tau,v)$ to refer to the result of casting value $v$ 
to target type $\tau$, which may result in a dedicated value $\nullptr$ that 
indicates impossibility of such a cast. We discuss various implementations of 
such a functor in \textsection\ref{sec:unisyn}.

Every function $\varphi$ that the user would like to be able to call lazily 
requires definition of a functor $F_\varphi$ that defines the semantics of such 
operation on any given argument types. The library defines such semantic objects 
$F_\oplus$ and $F_\ominus$ for every binary operation $\oplus$ and unary 
$\ominus$ it supports. The user is responsible for defining semantic functor 
$F_\varphi$ for every function $\varphi$ she would like to be able to evaluate 
lazily or use in a generalized n+k pattern. We show how to define such functors 
in \textsection\ref{sec:impl}, while here we use the notation $F(v_1,\cdots,v_k)$ 
to refer to the value representing the result of applying such a functor to 
values $v_1,\cdots,v_k$.

Each variant of an algebraic data type in a functional language has exactly one 
constructor, which makes it ideally suitable for structural decomposition of the 
type with pattern matching. Classes in \Cpp{} are allowed to have multiple 
constructors, which is why we need a mechanism that would let the user specify 
structural decomposition of a class. We do this with the help of bindings 
(\textsection\ref{sec:bnd}) represented here with a functor $\Delta_i^{\tau,l}$. 
We use the notation $\Delta_i^{\tau,l}(v)$ to refer to the value representing 
the $i^{th}$ component in layout $l$ of the structural decomposition of a value 
$v$ of type $\tau$.

Lastly, we let the user define the exact meaning of matching a value $v$ against 
an expression $\eta$ by case analysis on the structure of $\eta$. The exact 
details of defining such algebraic decomposition are given in 
\textsection\ref{sec:slv}, while here we use the notation $\mathsf{solve}(\eta,v)$ 
to refer to a boolean value indicating whether the generalized n+k pattern 
$\eta$ was accepted (true) or rejected (false).

We model the run-time environment of our SELL as a map $\Sigma: \chi\rightarrow T$ 
since our variables $\chi$ either hold a value of type \code{T} or refer to another 
variable of that type. In addition to meta-variables we have seen already, 
meta-variables $u,v$ and $b^{bool}$ range over values.

%\subsubsection{Evaluation Rules}
%\label{sec:eval}

\begin{figure}[h]
\begin{mathpar}

\inferrule[E-Value]
{\varsigma = \Value{\tau}(v)}
{\varsigma \lazyevals v}

\inferrule[E-Var]
{\chi = \Variable{\tau}(v)}
{\chi \lazyevals v}

\inferrule[E-Unary]
{\xi \lazyevals v}
{\ominus \xi \lazyevals F_\ominus(v)}

\inferrule[E-Binary]
{\xi_1 \lazyevals v_1 \\ \xi_2 \lazyevals v_2}
{\xi_1 \oplus \xi_2 \lazyevals F_\oplus(v_1,v_2)}

%\inferrule[E-Binary-Const-Left]
%{\xi \lazyevals v}
%{\xi \oplus c \lazyevals F_\oplus(v,c)}

%\inferrule[E-Binary-Const-Right]
%{\xi \lazyevals v}
%{c \oplus \xi \lazyevals F_\oplus(c,v)}

\inferrule[E-Function]
{\xi_1 \lazyevals v_1 \\ \cdots \\ \xi_k \lazyevals v_k}
{\varphi(\xi_1,\cdots,\xi_k) \lazyevals F_\varphi(v_1,\cdots,v_k)}

\end{mathpar}
\caption{Evaluation rules}
\label{evaluation}
\end{figure}

Figure~\ref{evaluation} shows the evaluation rules used to evaluate lazy expressions 
that our SELL introduces. The judgments are of the form $\Sigma\vdash \xi \lazyevals v$ 
stating that lazy expression $\xi$ evaluates to a value $v$ in a run-time 
environment $\Sigma$. We do not mention the run-time environment in the rules 
for brevity since the evaluation does not modify it.

%\subsubsection{Semantics of Matching Expressions}
%\label{sec:semme}

\begin{figure}[h]
\begin{mathpar}
\inferrule[P-Wildcard]
{}
{\Sigma\vdash \_(v_e) \evals \True,\Sigma}

\inferrule[P-Value]
{\varsigma \lazyevals u}
{\Sigma\vdash \varsigma^\tau(v_e) \evals (u=v),\Sigma}

\inferrule[P-Variable]
{u=F_{dc}(\tau,v_e)}
{\Sigma\vdash \chi^{\tau}(v_e) \evals (u \neq \nullptr{}),\Sigma[\chi\leftarrow u]}

\inferrule[P-n+k-Pattern]
{\Sigma\vdash \mathsf{solve}(\xi,v_e) \evals v,\Sigma'}
{\Sigma\vdash \xi(v_e) \evals v,\Sigma'}

\inferrule[P-Guard]
{\Sigma\vdash \pi(v_e) \evals b_\pi,\Sigma' \\ \Sigma'\vdash \xi \lazyevals b_\xi}
{\Sigma\vdash (\pi \models \xi)(v_e) \evals (b_\pi \wedge b_\xi),\Sigma'}

\inferrule[P-Constructor-Nullptr]
{F_{dc}(\tau,v_e)=\nullptr{}}
{\Sigma\vdash (\mathsf{match}\langle\tau\left[,l\right]\rangle(\varpi_1,...,\varpi_k))(v_e) \evals \False,\Sigma}

\inferrule[P-Constructor-Reject]
{ u=F_{dc}(\tau,v_e) \\
 \Sigma_1    \vdash \varpi_1(\Delta_1^{\tau,l}(u))         \evals \True, \Sigma_2 \\ \cdots \\
 \Sigma_{i-1}\vdash \varpi_{i-1}(\Delta_{i-1}^{\tau,l}(u)) \evals \True, \Sigma_i \\
 \Sigma_i    \vdash \varpi_i(\Delta_i^{\tau,l}(u))         \evals \False,\Sigma_{i+1}
}
{\Sigma\vdash (\mathsf{match}\langle\tau\left[,l\right]\rangle(\varpi_1,...,\varpi_k))(v_e) \evals \False,\Sigma_{i+1}}

\inferrule[P-Constructor-Accept]
{ u=F_{dc}(\tau,v_e) \\
 \Sigma_1    \vdash \varpi_1(\Delta_1^{\tau,l}(u)) \evals \True, \Sigma_2 \\ \cdots \\
 \Sigma_k    \vdash \varpi_k(\Delta_k^{\tau,l}(u)) \evals \True, \Sigma_{k+1}
}
{\Sigma\vdash (\mathsf{match}\langle\tau\left[,l\right]\rangle(\varpi_1,...,\varpi_k))(v_e) \evals \True,\Sigma_{k+1}}

\end{mathpar}
\caption{Semantics of matching expressions}
\label{exprsem}
\end{figure}

The rule set in Figure~\ref{exprsem} deals with pattern application $\pi(e)$, 
which essentially performs matching of a pattern $\pi$ against an expression 
$e$. To avoid dealing with the \Cpp{} semantics, we assume that the expression $e$ 
has already been evaluated to a value $v_e$. Our judgments are thus of the 
form $\Sigma\vdash \pi(v_e) \evals v,\Sigma'$ that can be interpreted as 
following: given an environment $\Sigma$ and a value $v_e$ representing the 
result of evaluating subject expression $e$ according to the \Cpp{} semantics, 
pattern application $\pi(v_e)$ results in value $v$ and environment $\Sigma'$. 

%\subsubsection{Semantics of Match Statement}
%\label{sec:semms}

\begin{figure}[h]
\begin{mathpar}
\inferrule[Match-True]
{ v_e \neq \nullptr \\
 \Sigma      \vdash_{v_e} C_1    \evals \False,\Sigma_1     \\ \cdots \\
 \Sigma_{i-2}\vdash_{v_e} C_{i-1}\evals \False,\Sigma_{i-1} \\
 \Sigma_{i-1}\vdash_{v_e} C_i    \evals \True, \Sigma_i
}
{\Sigma\vdash \mathsf{Match}(v_e) \{ \left[C_i \vec{s}_i\right]^*_{i=1..n} \} \mathsf{EndMatch} \evals i,\Sigma_i}

\inferrule[Match-False]
{ v_e \neq \nullptr \\
 \Sigma      \vdash_{v_e} C_1    \evals \False,\Sigma_1 \\ \cdots \\
 \Sigma_{n-1}\vdash_{v_e} C_n    \evals \False,\Sigma_n
}
{\Sigma\vdash \mathsf{Match}(e) \{ \left[C_i \vec{s}_i\right]^*_{i=1..n} \} \mathsf{EndMatch} \evals 0,\Sigma_n}

\inferrule[Qua]
{\Sigma \vdash \mathsf{match}\langle\tau,l\rangle(\vec{\varpi})(v_e) \evals b,\Sigma' }
{\Sigma \vdash_{v_e} \mathsf{Qua}(\tau,\vec{\varpi}) \evals b,\Sigma'[\mathsf{matched}^\tau\rightarrow F_{dc}(\tau,v_e)]}

\inferrule[When]
{\Sigma \vdash_{v^\tau_e} \mathsf{Qua}(\tau,\vec{\varpi}) \evals b,\Sigma'}
{\Sigma \vdash_{v^\tau_e}     \mathsf{When}(\vec{\varpi}) \evals b,\Sigma'}

\inferrule[Case]
{\Delta_i^\tau : \tau \rightarrow \tau_i, i=1..k \\
 \Sigma[x_i^{\tau_i}\rightarrow\nullptr]_{i=1..k} \vdash_{v_e} \mathsf{Qua}(\tau,x_1,...,x_k) \evals u,\Sigma' }
{\Sigma \vdash_{v_e} \mathsf{Case}(\tau,x_1,...,x_k) \evals u,\Sigma'}

\inferrule[Otherwise]
{\Sigma \vdash_{v^\tau_e} \mathsf{Case}(\tau,\vec{x}) \evals u,\Sigma'}
{\Sigma \vdash_{v^\tau_e} \mathsf{Otherwise}(\vec{x}) \evals u,\Sigma'}
\end{mathpar}
\caption{Semantics of match-statement}
\label{stmtsem}
\end{figure}

The rule set in Figure~\ref{stmtsem} describes the semantics of a \emph{match 
statement}. In order to avoid dealing with the semantics of the \Cpp{} statements, 
we define the semantics of a match-statement to be the index of the matching case 
clause and the run-time environment right before the clause's statement, or $0$ 
if none of the clauses matched. The judgments are thus of the form 
$\Sigma\vdash M \evals v,\Sigma'$ for match statement, and are slightly extended 
for case clauses $\Sigma\vdash_{v_e} C \evals b,\Sigma'$ with value $v_e$ of a 
subject that is passed along from the match statement onto the clauses.

The rules essentially describe the first-fit strategy for evaluating the clauses.
Evaluation of a \code{Qua}-clause is equivalent to evaluation of a corresponding 
match-expression on a constructor pattern. Successful match will introduce into 
the local scope of the clause a variable \code{matched} bound to the subject 
properly casted to the target type $\tau$. Evaluation of \code{When}-clause 
amounts to evaluation of a corresponding \code{Qua}-clause with target type 
being the subject type. Evaluation of \code{Case}-clauses amounts to evaluation 
of \code{Qua}-clauses in the environment extended with variables passed as 
arguments to the clause. Evaluation of default clause amounts to evaluating a 
corresponding \code{Case}-clause with target type being the subject type.


\section{Implementation} %%%%%%%%%%%%%%%%%%%%%%%%%%%%%%%%%%%%%%%%%%%%%%%%%%%%%%%
\label{sec:impl}

We begin by explaining the traditional object-oriented approach to implementing 
first class patterns, which is based on run-time compositions through 
interfaces, and point to its problems and shortcomings. We then explain our 
approach based on compile-time composition through concepts. Finally, we detail 
how the patterns fit into and extend our open type switch statement~\cite{TS12}. 

\subsection{Patterns as Objects}
\label{sec:pao}

The idea of implementing patterns through objects is not new and has been 
explored before in different languages~\cite{Visser06matchingobjects,geller2010pattern,FuncCSharp,Grace2012}.
Variations of it differ in where bindings are stored and what is returned as a 
result of the match, but in its most basic form it consists of the following 
interface:

\begin{lstlisting}[keepspaces,columns=flexible]
struct object { // root of the object class hierarchy
  virtual bool operator==(const object&) const = 0;
};
@\halfline@
struct pattern { // pattern interface
  virtual bool match(const object&) = 0;
};
\end{lstlisting}

\noindent
A particular kind of pattern then implements \code{pattern} interface 
accordingly to its logic. A value pattern, for example, will look as following:

\begin{lstlisting}[keepspaces,columns=flexible]
struct value : pattern {
  value(const object& obj) : m_obj(obj) {}
  virtual bool match(const object& obj) 
    { return m_obj == obj; }
  const object& m_obj;
};
\end{lstlisting}

\noindent
The dual approach of passing a pattern to an object has also been explored, 
notably by Newspeak~\cite{geller2010pattern}, however its default implementation 
(which is rarely overridden in practice) dispatches back to the first solution. 
Regardless of whether the emphasis is made on the pattern or on the object, however,
the solution suffers from the same bottleneck -- the cost of a virtual function 
call(s) and the cost of identifying the dynamic type of either the subject or 
the pattern. We demonstrate in \textsection\ref{sec:patcmp}, the overhead 
incurred by this solution is incomparably larger than the overhead of our 
\emph{patterns as expression templates} approach.

While the approach is open to new patterns and pattern combinators (the patterns 
are composed at run-time by holding references to other pattern objects), it has 
some design problems. For example, mismatch in the type of the subject and the 
type accepted by the pattern can only be detected at run-time, while in 
languages with built-in support of pattern matching it is typically detected at 
type-checking phase. The approach may also unnecessarily clutter the code by 
requiring lots of similar boilerplate code be written. For example, modeling n+k 
patterns requires additional interface for evaluating the \code{expression}. 
With it, we have a dilemma of whether \code{expression} should be derived from 
\code{pattern}, \code{pattern} from \code{expression}, or neither of those. 
Independently of the choice, implementation of pattern combinators will require 
that the class of the combinator conditionally derives from \code{pattern}, 
\code{expression} or both depending on which of these interfaces its arguments 
implement. On one hand, this will require a separate implementation of the 
combinator for each of the cases, while on the other it makes the combinators 
dependent on something that was only needed to implement n+k patterns.

%To quantify the overhead somewhat, we reimplemented the factorial function from 
%\textsection\ref{sec:cpppat} using object patterns and timed a million 
%computations of factorial on arguments ranging from 0 to 10. Depending on the 
%argument, the approach based on object patterns was 12-22 times slower than 
%factorial based on \emph{Mach7}. Note that for this experiment we took extra care to 
%not allocate patterns or intermediary objects on the heap, made sure the bodies 
%of all virtual functions were also available for inlining since we composed 
%objects on the stack and thus their complete types were known. We also used a 
%faster \code{typeid}-check instead of a slower \code{dynamic_cast} to ensure the 
%safety of unpacking an object. Finally, we repeated the experiment while removing 
%the safety check altogether (assuming the argument will be of the correct 
%dynamic type) and could reduce the overhead to 6.74-18 times, which is still too 
%costly to be considered a viable solution for a modern \Cpp{} use. We show in 
%\textsection\ref{sec:patcmp} that \emph{Mach7} patterns produce code that is only few 
%percentage points slower than manualy hand-crafted code without patterns. 

\subsection{Patterns as Expression Templates}
\label{sec:pat}

Patterns in \emph{Mach7} are also represented as objects, however their 
composition happens at compile time and is based on \Cpp{} concepts. 
\term{Concept} is the \Cpp{} community's long-established term for a set of 
requirements for template parameters. Concepts were not included in \Cpp{}11, 
but techniques for emulating them with 
\code{enable_if}~\cite{jarvi:03:cuj_arbitrary_overloading} have been in use for 
a while. In this work, we use the notation for \term{template constraints} -- a 
lighter version of the concepts proposal~\cite{N3580}. Template constraints are 
briefly explained in Appendix~\ref{sec:prelim}.

There are two main constraints on which the entire library is built: 
\code{Pattern} and \code{LazyExpression}.

\begin{lstlisting}
template <typename P> constexpr bool Pattern() {
  return Copyable<P>
      && is_pattern<P>::value
      && requires (typename S, P p, S s) {
           bool = { p(s) };
           AcceptedType<P,S>;
         };
}
\end{lstlisting}

\noindent
\code{Pattern} constraint is the analog of \code{pattern} interface from the 
\emph{patterns as objects} solution. Any class satisfying this constraint can 
interact with other patterns of the library and \code{Match}-statement. Patterns 
can be passed as arguments of a function, which is why the constraint subsumes a
\code{Copyable} constraint. Implementation of pattern combinators requires the 
library to overload certain operators on all the types satisfying the \code{Pattern}
constraint. To avoid overloading these operators for types that satisfy the 
requirements accidentally, \code{Pattern} constraint is a semantic constraint 
and classes that claim to satisfy it have to explicitly state this by specializing  
\code{is_pattern<P>} trait. The constraint introduces also some syntactic 
requirements, described by the \code{requires} clause. In particular, patterns 
require presence of an application operator that serves as an analog of 
\code{pattern::match(const object&)} interface method in the \emph{patterns as 
objects} approach; there is no analog of the \code{object} interface, however. 
The \code{Pattern} constraint itself does not impose any restrictions on the 
type of the subject \code{S}. Patterns like wildcard pattern will leave this 
type completely unrestricted, while other patterns may require it to satisfy 
certain constraints, model a given concept, inherit from a certain type, etc.
Application operator will typically return a value of type \code{bool} 
indicating whether the pattern is \subterm{pattern}{accepted} on a given subject 
(\code{true}) or \subterm{pattern}{rejected} (\code{false}). For convenience reasons, 
application operator is allowed to return any type that is convertible to 
\code{bool} instead, e.g. a pointer to a casted subject, which is useful in 
emulating the support of \subterm{pattern}{as-patterns}.

Most of the patterns are applicable only to subjects of a given \subterm{type}{expected type} 
or types convertible to it. This is the case, for example, with value and  
variable patterns, where the expected type is the type of the underlying value, 
as well as with constructor pattern, where the expected type is the type of the 
user-defined type it decomposes. Some patterns, however, do not have a single 
expected type and may work with subjects of many unrelated types. A wildcard 
pattern, for example, can accept values of any type without involving a 
conversion. To account for this, the \code{Pattern} constraint requires presence of 
a type alias \code{AcceptedType}, which given a pattern of type \code{P} and 
a subject of type \code{S} returns an expected type \code{AcceptedType<P,S>} 
that will accept subjects of type \code{S} with no or minimum conversions. 
By default, the alias is defined in terms of a nested type-function 
\code{accepted_type_for} as following:

\begin{lstlisting}
template<typename P, typename S>
  using AcceptedType = P::accepted_type_for<S>::type;
\end{lstlisting}

\noindent
The wildcard pattern defines \code{accepted_type_for} to be an identity 
function, while variable and value patterns define it to be their underlying 
type. Here is an example of how variable pattern satisfies the \code{Pattern} 
constraint:

%struct wildcard {
%  template <typename S>
%  struct accepted_type_for { typedef S type; };
%  template <typename S> 
%  bool operator()(const S&) const noexcept 
%    { return true; }
%};
%@\halfline@
%template <typename T>
%struct value {
%  template <typename S> 
%  struct accepted_type_for { typedef T type; };
%  bool operator()(const T& t) const noexcept 
%    { return m_value == t; }
%  T m_value;
%};
%@\halfline@
\begin{lstlisting}
template <Regular  T>
struct var {
  template <typename> 
    struct accepted_type_for { typedef T type; };
  bool operator()(const T& t) const 
    { m_value = t; return true; }
  template <Regular  S> 
  bool operator()(const S& s) const
    { m_value = s; return m_value == s; }
  mutable T m_value;
};
@\halfline@
template <Regular  T> struct is_pattern<var<T>> 
  { static const bool value = true; };
\end{lstlisting}

%Each of our six pattern kinds implements the application operator according to 
%the semantics presented in Figure~\ref{exprsem}. The application operator's 
%result has to be convertible to bool; \code{true} indicates a successful match. 
%A class might have several overloads of the above operator that distinguish 
%cases of interest. We summarize the requirements on template parameters of each 
%of our pattern in Figure~\ref{xt-reqs}.
%
%\begin{figure}[h]
%\centering
%\begin{tabular}{llll}
%{\bf Pattern}       & {\bf Parameters}          & {\bf Argument of application operator U}         \\ \hline
%\code{wildcard}     & --                        & --                                               \\
%\code{value<T>}     & \code{Regular<T>}         & \code{Convertible<U,T>}                          \\
%\code{variable<T>}  & \code{Regular<T>}         & \code{Convertible<U,T>}                          \\
%\code{expr<F,E...>} & \code{LazyExpression<E>}  & \code{Convertible<U,expr<F,E...>::result_type>}  \\
%\code{guard<E1,E2>} & \code{LazyExpression<Ei>} & any type accepted by \code{E1::operator()}       \\
%\code{ctor<T,E...>} & \code{Polymorphic<T>}     & \code{Polymorphic<U>} for open encoding          \\
%                    & \code{Object<T>}          & \code{is_base_and_derived<U,T>} for tag encoding \\
%\end{tabular}
%\caption{Requirements on parameters and argument type of an application operator}
%\label{xt-reqs}
%\end{figure}

\noindent
Note that for semantic or efficiency reasons a pattern may have several overloads 
of application operator as in the example above. The first one is used when no 
conversion is required and thus the pattern is guaranteed to be accepted, while 
the second may involve a possibly narrowing conversion, which is why we check 
that the values compare equal after assignment. Similarly, for type checking 
reasons, \code{accepted_type_for} may and typically will provide several partial 
or full specializations to limit the set of acceptable subjects. For example, 
\subterm{pattern combinator}{address combinator} can only be applied to subjects of pointer types. Its 
implementation manifests this by deriving unrestricted case of the type function 
\code{accepted_type_for} from \code{invalid_subject_type<S>}. This will trigger 
a static assertion when its associated type \code{type} gets instantiated, 
resulting in a compile-time error that states that a given subject type \code{S} 
cannot be used as an argument of the address pattern. The second case of the 
type function indicates through partial specialization of class templates that 
for any subject of a pointer type \code{S*}, the accepted type is going to be a 
pointer to the type accepted by the argument pattern \code{P1} of the address 
combinator.

\begin{lstlisting}
template <Pattern P1>
struct address
{ // ...
  template <typename S> 
    struct accepted_type_for : invalid_subject_type<S> {};
  template <typename S> struct accepted_type_for<S*> {
    typedef typename P1::template 
      accepted_type_for<S>::type* type;
  };
  template <typename S>
    bool operator()(const S* s) const 
      { return s && m_p1(*s); }
  P1 m_p1;
};
\end{lstlisting}

\noindent
Checking whether a given subject type can be accepted is inherently late and 
happens at instantiation time of the nested \code{accepted_type_for} type 
function and possibly parameterized application operator. For this reason, 
pattern's implementation may have to provide a set of overloads of the 
application operator that will be able to accept all possible outcomes of 
\code{accepted_type_for<S>::type} on any valid subject type \code{S}.

Guard and n+k patterns, as well as equivalence combinator and potentially some 
new user-defined patterns, depend on capturing the structure (term) of lazily 
evaluated expressions. All such expressions must satisfy the 
\code{LazyExpression} constraint:

\begin{lstlisting}
template <typename E> constexpr bool LazyExpression() {
  return Copyable<E> 
      && is_expression<E>::value
      && requires (E e) {
           ResultType<E>;
           ResultType<E> == { eval(e) };
           ResultType<E> { e };
         };
}
@\halfline@
template<typename E> using ResultType = E::result_type;
\end{lstlisting}

\noindent
The constraint is again semantic and the classes claiming to satisfy it must 
assert it through \code{is_expression} trait. Template alias \code{ResultType<E>} 
is defined to return expression's associated type \code{result_type}, which 
defines the type of the result of a lazily evaluated expression. Any class 
satisfying \code{LazyExpression} constraint must also provide an implementation 
of function \code{eval} that evaluates the result of the expression. Conversion 
to the \code{result_type} should call \code{eval} on the object in order to 
allow the use of lazily evaluated expressions in the contexts where their 
eagerly computed value is expected: e.g. non-pattern matching context of the 
right hand side of the \code{Case}-clause. Class \code{var<T>}, for example, 
models concept \code{LazyExpression} as following:

\begin{lstlisting}
template <Regular T>
struct var {
  // ... definitions from before
  typedef T result_type;
  friend const result_type& eval(const var& v) 
    { return v.m_value; }
  operator result_type() const { return eval(*this); }
};
\end{lstlisting}

\noindent
To capture the structure of an expression, the library employs a commonly used 
technique called ``expression templates''~\cite{Veldhuizen95expressiontemplates, 
vandevoorde2003c++}. It captures the structure of expression through the type, 
which for binary addition may look as following:

\begin{lstlisting}[keepspaces,columns=flexible]
template <LazyExpression E1, LazyExpression E2>
struct plus {
  E1 m_e1; E2 m_e2; // subexpressions
  plus(const E1& e1, const E2& e2) : m_e1(e1), m_e2(e2) {}
  typedef decltype(std::declval<E1::result_type>() 
                 + std::declval<E2::result_type>()
                  ) result_type;
  friend result_type eval(const plus& e) 
    { return eval(e.m_e1) + eval(e.m_e2); }
  friend plus operator+(const E1& e1, const E2& e2) noexcept 
    { return plus(e1,e2); }
};
\end{lstlisting}

\noindent
The user of the library never sees this definition, instead she implicitly 
creates its objects with the help of overloaded \code{operator+} on any 
\code{LazyExpression} arguments. The type itself models \code{LazyExpression} 
concept as well so that the lazy expressions can be composed. Notice that all 
the requirements of the concept are implemented in terms of the requirements 
on the types of the arguments. The key point to efficiency of expression 
templates is that all the types in the final expression are known at compile 
time, while all the function calls are trivial and fully inlined. Use of new 
\Cpp{}11 features like move constructors and perfect forwarding allows us to 
ensure further that no temporary objects will ever be created at run-time and 
that the evaluation of the expression template will be as efficient as a hand 
coded function.

In general, an \term{expression template} is an algebra $\langle T_C,\{f_1,f_2,...\}\rangle$ 
defined over the set $T_C = \{\tau~|~\tau \models C\}$ of all the types $\tau$ 
modeling a given concept $C$. Operations $f_i$ allow one to compose new types  
modeling concept $C$ out of existing ones. In this sense, the types of all lazy 
expressions in \emph{Mach7} stem from a set of few possibly parameterized basic 
types like \code{var<T>} and \code{value<T>} (which model \code{LazyExpression}) 
by applying type functors \code{plus}, \code{minus} ... etc. to them. Every type 
in the resulting family then has a function \code{eval} defined on it that 
returns a value of the associated type \code{result_type}. Similarly, the types 
of all the patterns stem from a set of few possibly parameterized patterns like 
\code{wildcard}, \code{var<T>}, \code{value<T>}, \code{C<T>} etc. by applying to 
them pattern combinators like \code{conjunction}, \code{disjunction}, 
\code{equivalence}, \code{address} etc. The user is allowed to extend both 
algebras with either basic expressions and patterns or functors and combinators. 

Sets $T_{LazyExpression}$ and $T_{Pattern}$ have non-empty intersection, which 
slightly complicates the matter. Basic types \code{var<T>} and \code{value<T>} 
belong to both families and so are some of the combinators: e.g. 
\code{conjunction}. Since we can only have one overloaded \code{operator&&} for 
a given combination of argument types, we have to state conditionally whether 
requirements of \code{Pattern}, \code{LazyExpression} or both are satisfied in a 
given instantiation of \code{conjunction<T1,T2>} depending on what combination 
of these concepts the argument types \code{T1} and \code{T2} model. Concepts, 
unlike interfaces, allow modeling such behavior without multiplying 
implementations or introducing dependencies.

\subsection{Structural Decomposition}
\label{sec:bnd}

\emph{Mach7}'s support of constructor patterns \code{C<T>(P1,...,Pn)} requires the 
library to know which member of class \code{T} should be used as the subject to 
$P_1$, which should be matched against $P_2$ etc. In functional languages 
supporting algebraic data types, such decomposition is unambiguous as each 
variant has only one constructor, which is thus also used as \subterm{constructor}{deconstructor}
-- a term used by several languages~\cite{padl08,Thorn2012} to define explicitly
decomposition of a type through pattern matching. In \Cpp{}, a class may have 
several constructors, which may not even reflect the data members it contains, 
some of which may be inaccessible or unrepresentative of the class' logic etc. 
We thus require the user to be more explicit as to class' decomposition, which 
she does by specializing the library template class \code{bindings}. We decided 
not to use the name ``deconstructor'' as it is already commonly confused with 
``destructor'' in \Cpp{}. Here are the definitions required to decompose the 
lambda terms we introduced in \textsection\ref{sec:cpppat}:

\begin{lstlisting}
template <> 
  struct bindings<Var> { Members(Var::name); };
template <> 
  struct bindings<Abs> { Members(Abs::var, Abs::body); };
template <> 
  struct bindings<App> { Members(App::func, App::arg); };
\end{lstlisting}

\noindent
Variadic macro \code{Members} simply expands each of its argument into the 
following definition, demonstrated here on \code{App::func}:

\begin{lstlisting}
static inline decltype(&App::func) member@$i$@() noexcept 
  { return &App::func; }
\end{lstlisting}

\noindent
Each of such functions returns a pointer-to-member that should be bound in 
position $i$. The library applies corresponding members to the subject in order 
to obtain subjects for sub-patterns $P_1,...,P_n$. The functions get inlined so 
the code to access a member in a given position becomes exactly the same as the 
code to access that member directly. Note that binding definitions made this way 
are \emph{non-intrusive} since the original class definition is not touched. 
They also respect \emph{encapsulation} since only the public members of the 
target type will be accessible from within \code{bindings} specialization. 
Members do not have to be data members only, which can be inaccessible, but any 
of the following:

\begin{compactitem}
\setlength{\itemsep}{0pt}
\setlength{\parskip}{0pt}
\item Data member of the target type $T$
\item Nullary member-function of the target type $T$
\item Unary external function taking the target type $T$ by pointer, reference or value.
\end{compactitem}

\noindent
Binding definitions have to be written only once for a given class hierarchy and 
can be used everywhere. This is also true for parameterized classes, as can be 
seen in an example in \textsection\ref{sec:view}.


\subsection{Algebraic Decomposition}
\label{sec:slv}

Traditional approaches to generalizing n+k patterns treat matching a pattern 
$f(x,y)$ against a value $v$ as solving an equation $f(x,y)=v$~\cite{OosterhofThesis}. 
Such an interpretation is well defined when there are zero or one solutions,
but alternative interpretations are possible when there are multiple solutions. 
Instead of discussing which interpretation is the most general or appropriate, 
we propose to look at n+k patterns as a \term{notational decomposition} of 
mathematical objects. The elements of such notation are associated with 
sub-components of the matched mathematical entity, which effectively lets us 
decompose it into parts. The structure of the expression tree used in the notion
is an analog of a constructor symbol in structural decomposition, while its 
leaves are placeholders for parameters to be matched against or inferred from 
the mathematical object in question. In essence, \term{algebraic decomposition} 
is to mathematical objects what structural decomposition is to algebraic data 
types. While the analogy is somewhat ad-hoc, it resembles the situation with 
operator overloading: you do not strictly need it, but it is so syntactically 
convenient it is virtually impossible not to have it. We demonstrate the 
alternative point of view on the n+k patterns with examples.

\begin{compactitem}
\setlength{\itemsep}{0pt}
\setlength{\parskip}{0pt}
\item An expression $n/m$ is often used to decompose a rational number into 
      numerator and denominator.
\item Euler notation $a+bi$ with $i$ being an imaginary unit is used to 
      decompose a complex number into real and imaginary parts. Similarly, 
      expressions $r(cos \phi + i\mathrm{sin} \phi)$ and $re^{i\phi}$ are used to 
      decompose it into polar form.
\item An object representing 2D line can be decomposed with slope-intercept form 
      $mX+c$, linear equation form $aX+bY=c$ or two-points form 
      $(Y-y_0)(x_1-x_0)=(y_1-y_0)(X-x_0)$.
\item An object representing polynomial can be decomposed for a specific degree: 
      $a_0$, $a_1X^1+a_0$, $a_2X^2+a_1X^1+a_0$ etc.
\item An element of a vector space can be decomposed along some sub-spaces of 
      interest. For example a 2D vector can be matched against $(0,0)$, $aX$, 
      $bY$, $aX+bY$ to separate the general case from those when one or both
      components of vector are $0$.
\end{compactitem}

\noindent
Expressions $i$, $X$ and $Y$ in these examples are not variables, but named 
constants of some dedicated type that lets the expression be generically 
decomposed into orthogonal parts. Note also that linear equation and two-point 
form for decomposing lines already include an equality sign, which makes it 
hard to give them semantics in an equational approach. It turns out that the 
equational approach can be generically expressed in our framework for many 
interesting cases of interest.

Applying equational approach to floating-point arithmetic creates even more 
problems. Even when the solution is unique, it may not be representable by 
a given floating-point type and thus not satisfy the equation. Once we settle 
for an approximation, we open ourselves to even more decompositions that become 
possible with our approach.

\begin{compactitem}
\setlength{\itemsep}{0pt}
\setlength{\parskip}{0pt}
\item Matching $n/m$ with integer variables $n$ and $m$ against a floating-point 
      value can be given semantics of finding the closest fraction to the 
      value.
\item Matching an object representing sampling of some random variable against
      expressions like $Gaussian(\mu,\sigma^2)$, $Poisson(\lambda)$ or 
      $Binomial(n,p)$ can be seen as distribution fitting. 
\item Any curve fitting in this sense becomes an application of pattern 
      matching. Precision in this case can be a global constant or explicitly 
      passed parameter of the matching expression.
\end{compactitem}

%\noindent
%We can make several observations from these examples:

%\begin{compactitem}
%\setlength{\itemsep}{0pt}
%\setlength{\parskip}{0pt}
%\item We might need to have the entire expression available to us in order to 
%      decompose its parts.
%\item Matching the same expression can have different meanings depending on 
%      types of objects composing the expression and the expected result. 
%\item An algorithm to decompose a given expression may depend on the types of 
%      objects in it and the type of the result. 
%\end{compactitem}

%\subsubsection{Solvers}

\noindent
The user of our library defines the semantics of decomposing a value of a given 
type \code{S} against an expression of shape \code{E} by overloading a function: 

\begin{lstlisting}
template <LazyExpression E, typename S> 
bool solve(const E&, const S&);
\end{lstlisting}

\noindent
The first argument of the function takes an expression template representing a 
term we are matching against, while the second argument represents the expected 
result. The following example defines a generic solver for multiplication by a 
constant:

\begin{lstlisting}
template <LazyExpression E, typename T> 
    requires Field<E::result_type>()
bool solve(const mult<E,value<T>>& e, const E::result_type& r)
    { return solve(e.m_e1,r/eval(e.m_e2)); }
@\halfline@
template <LazyExpression E, typename T>
    requires Integral<E::result_type>()
bool solve(const mult<E,value<T>>& e, const E::result_type& r) 
{
    T t = eval(e.m_e2);
    return r%t == 0 && solve(e.m_e1,r/t);
}
\end{lstlisting}

\noindent
The first overload is only applicable when the type of the result of the 
sub-expression models the \code{Field} concept. In this case, we can rely on the 
presence of a unique inverse and simply call division without any additional 
checks. The second overload uses integer division, which does not guarantee the 
unique inverse, and thus we have to verify that the result is divisible by the 
constant first. This last overload combined with a similar solver for addition 
of integral types is everything the library needs to know to properly handle the 
definition of the \code{fib} function from \textsection\ref{sec:cpppat}. It also 
demonstrates how an equational approach can be generically implemented for a 
number of expressions.

A generic solver capable of decomposing a complex value using the Euler 
notation is very easy to define by fixing the structure of expression:

\begin{lstlisting}[keepspaces]
template <LazyExpression E1, LazyExpression E2> 
    requires SameType<E1::result_type,E2::result_type>()
bool solve(
    const plus<mult<E1,value<complex<E1::result_type>>>,E2>& e, 
    const complex<E1::result_type>& r);
\end{lstlisting}

\noindent
As we mentioned in \textsection\ref{sec:cpppat}, the template facilities of 
\Cpp{} resemble pattern-matching facilities of other languages. Here, we 
essentially use these compile-time patterns to describe the structure of the 
expression this solver is applicable to: $e_1*c+e_2$ with types of $e_1$ and 
$e_2$ being the same as type on which a complex value $c$ is defined. The actual 
value of the complex constant $c$ will not be known until run-time, but assuming 
its imaginary part is not $0$, we will be able to generically obtain the values 
for sub-expressions.

Our approach is largely possible due to the fact that the library only serves as 
an interface between expressions and functions defining their semantics and 
algebraic decomposition. The fact that the user explicitly defines the variables 
she would like to use in patterns is also a key as it lets us specialize not 
only on the structure of the expression, but also on the types involved. 
Inference of such types in functional languages would be hard or impossible as the 
expression may have entirely different semantics depending on the types of 
arguments involved. Concept-based overloading simplifies significantly the case 
analysis on the properties of types, making the solvers generic and composable.
The approach is also viable as expressions are decomposed at compile-time and 
not at run-time, letting the compiler inline the entire composition of solvers. 

An obvious disadvantage of this approach is that the more complex expression 
becomes, the more overloads the user will have to provide to cover all 
expressions of interest. The set of overloads will also have to be made 
unambiguous for any given expression, which may be challenging for novices. An 
important restriction of this approach is its inability to detect multiple uses 
of the same variable in an expression at compile time. This happens because 
expression templates remember the form of an expression in a type, so use of two 
variables of the same type is indistinguishable from the use of the same 
variable twice. This can be worked around by giving different variables 
(slightly) different types or making additional checks as to the structure of 
expression at run-time, but that will make the library even more verbose or 
incur a significant run-time overhead.

\subsection{Views}
\label{sec:view}

Any type $T$ may have an arbitrary number of \term{bindings} associated with it, 
which are specified by varying the second parameter of the \code{bindings} 
template -- \term{layout}. The layout is a non-type template parameter of an 
integral type that has a default value and is thus omitted most of the time.
Support of multiple bindings through layouts in our library effectively enables 
a facility similar to Wadler's \subterm{pattern}{views}\cite{Wadler87}. Consider:

\begin{lstlisting}
enum { cartesian = default_layout, polar }; // Layouts
@\halfline@
template <typename T> 
  struct bindings<std::complex<T>>
    { Members(std::real<T>,std::imag<T>); };
template <typename T> 
  struct bindings<std::complex<T>, polar>
    { Members(std::abs<T>,std::arg<T>); };
@\halfline@
template <typename T> 
  using Cartesian = view<std::complex<T>>;
template <typename T> 
  using Polar     = view<std::complex<T>, polar>;
@\halfline@
  std::complex<double> c; double a,b,r,f;
  Match(c)
    Case(Cartesian<double>>(a,b)) ... // default layout
    Case(    Polar<double>>(r,f)) ... // view for polar layout
  EndMatch
\end{lstlisting}

\noindent
The \Cpp{} standard effectively enforces the standard library to use Cartesian 
representation\cite[\textsection26.4-4]{C++11}, which is why we choose the 
\code{Cartesian} layout to be the default. We then define bindings for each 
layout and introduce template aliases (an analog of typedefs for parameterized 
classes) for each of the views. \emph{Mach7} class \code{view<T,l>} binds together a 
target type with one of its layouts, which can be used everywhere where an 
original target type was expected.

The important difference from Wadler's solution is that our views can only be 
used in a match expression and not as a constructor or arguments of a function 
etc.

\subsection{Match Statement}
\label{sec:matchstmt}

\code{Match}-statement presented in this paper extends the efficient type switch 
for \Cpp{}~\cite{TS12}. The core of this extension amounts to ability to handle 
multiple subjects (both polymorphic and non-polymorphic) 
(\textsection\ref{sec:multiarg}) as well as the ability to accept patterns in 
case clauses (\textsection\ref{sec:patcases}).

\subsubsection{Multi-argument Type Switching}
\label{sec:multiarg}

The core of the proposal for efficient type switching was based on the fact that 
virtual table pointers (vtbl-pointers) on one hand uniquely identify subobjects 
in the object, while on the other hand they are perfect for hashing. The optimal 
hash function $H_{kl}^V$ built for a set of virtual table pointers $V$ seen by a 
type switch was chosen by varying parameters $k$ and $l$ to minimize the 
probability of conflict. Parameter $k$ was representing the logarithm of the 
size of cache, while parameter $l$ was representing the number of smallest bits 
to ignore.

We considered two different approaches to extending that solution to $N$ 
arguments. The first approach was based on maintaining an $N$-dimensional 
table indexed by independent $H_{k_il_i}^{V_i}$ maintained for each of the 
arguments $i$. The second approach was to aggregate the information from 
multiple vtbl-pointers into a single hash in a hope the hashing would still 
maintain its favorable properties. The first approach requires amount of memory 
proportional to $O(|V|^N)$ regardless of how many different combinations of 
vtbl-pointers came through the statement. The second approach requires the 
amount of memory linear in the number of vtbl-pointer combinations seen, which 
in the worst case becomes the same $O(|V|^N)$. The first approach requires 
lookup in $N$ caches, with each lookup being a subject to potential collisions; 
the second approach requires non-trivial computations to aggregate $N$ 
vtbl-pointers into a single hash value and may result in potentially more 
collisions in comparison to the first approach. Our experience of dealing with 
multiple dispatch in \Cpp{} suggests that we rarely see all combinations of 
types coming through a given multi-method in real-world applications. With this 
in mind, we did not expect all combination of types come through a given 
\code{Match}-statement and thus preferred the second solution, which grows 
linearly in memory with the number of combinations seen.

In number theory, \emph{Morton order} (aka \emph{Z-order}) is a function that 
maps multidimensional data to one dimension while preserving locality of the 
data points~\cite{Morton66}. A Morton number of an $N$-dimensional coordinate 
point is obtained by interleaving the binary representations of all coordinates.
The original one-dimensional hash function $H_{kl}^V$ applied to arguments $v \in V$ 
was producing hash values in a tight range $[0..2^k[$ where $k \in [K,K+1]$ for 
$2^{K-1} < |V| \leq 2^K$. The produced values were close to each other, which 
helped improve the performance of cache due to locality of references. The 
idea is thus to use Morton order on these hash values and not on the original 
vtbl-pointers in order to maintain the locality of references. To do this, we 
still maintain a single parameter $k$ reflecting the size of cache, however we 
keep $N$ parameters $l_i$ -- an optimal offset for argument $i$.

Consider a set $V^N = \{\tpl{v_1^1,...,v_1^N},...,\tpl{v_n^1,...,v_n^N}\}$ of 
$N$-dimensional tuples representing the set of vtbl-pointer combinations coming 
through a given \code{Match}-statement. As with one-dimensional case, we 
restrict the size $2^k$ of the cache to be not larger than twice the closest 
power of two greater or equal to $n=|V^N|$: i.e. $k \in [K,K+1]$, where 
$2^{K-1} < |V^N| \leq 2^K$. For a given $k$ and offsets $l_1,...,l_N$ a hash 
value of a given combination $\tpl{v^1,...,v^N}$ is defined as 
$H_{kl_1...l_N}(\tpl{v^1,...,v^N})=\mu(\frac{v^1}{2^{l_1}},...,\frac{v^N}{2^{l_N}}) \mod 2^k$, 
where function $\mu$ returns Morton number (bit interleaving) of $N$ numbers.
 
Similar to one-dimensional case, we vary parameters $k,l_1,...,l_N$ in 
their finite and small domains to obtain an optimal hash function 
$H^{V^N}_{kl_1...l_N}$ by minimizing the probability of conflict on values from 
$V^N$. Unlike the one-dimensional case, we do not try to find the optimal 
parameters every time we reconfigure the cache. Instead, we only try to improve 
the parameters to render fewer conflicts in comparison to the number of conflicts 
rendered by the current configuration. This does not prevent us from eventually 
converging to the same optimal parameters, which we do over time, but is 
important for maintaining the amortized complexity of the access constant. 
%Observe that the domain of each parameter of the optimal hash function 
%$H^{V^N}_{kl_1...l_N}$ only grows since $V^N$ only grows, while any cache 
%configuration is also a valid cache configuration in a larger cache, rendering 
%the same number of conflicts.
We demonstrate in \textsection\ref{sec:morton} that similarly to one-dimensional 
case such hash function produces little collisions on real-world class 
hierarchies, while is simple enough to compute to compete with alternatives 
dealing with multiple dispatch.

%In practice, the library does not consider all $N$ arguments of a given 
%\code{Match}-statement, but only the $M$ polymorphic arguments ($M \leq N$). It 
%then builds an efficient type switch based on those $M$ arguments. The type 
%switch guarantees efficient dispatch to the first case clause that can possibly 
%handle a given combination of arguments based on the subset of only polymorphic 
%arguments. The patterns are then tried sequentially. The underlying type switch 
%uses pattern's type-function \code{accepted_type_for<Si>} instantiated with the 
%subject type $Si$ of a given argument $i$ in order to obtain the target type 
%requested by the pattern in that position.

\subsubsection{Support for Patterns}
\label{sec:patcases}

Given a statement \code{Match(x_1,...,x_N)}, the library introduces several 
names into the scope of the statement: e.g. number of arguments $N$, subject 
types \code{subject_type_i} (defined as \code{decltype(x_i)} modulo type 
qualifiers), number of polymorphic arguments $M$ etc. When $M > 0$ it also 
introduces the necessary data structures to implement efficient type 
switching~\cite{TS12}. Only the $M$ arguments whose \code{subject_type_i} are 
polymorphic will be used for fast type switching.

For each case clause \code{Case(p_1,...,p_N)} the library ensures that the 
number of arguments to the case clause $N$ matches the number of arguments to 
the \code{Match} statement, and that the type \code{P_i} of every expression 
\code{p_i} passed as its argument models the \code{Pattern} concept. 
Initially we allowed case clauses to accept less than $N$ patterns, assuming the 
missing patterns to be the wildcard, however, brittleness of the macro system 
made us reconsider this. The problem is that macro system is blind to \Cpp{} 
syntax and template instantiation like \code{A<B,C>} used in a pattern will be 
treated by the preprocessor as 2 macro arguments. This resulted in errors that 
were hard for the users to comprehend.
For each \code{subject_type_i} it then introduces \code{target_type_i} into the 
scope of the case clause, defined as the result of evaluating type function 
\code{P_i::accepted_type_for<subject_type_i>}. This is the type the pattern 
expects as an argument on the subject of type \code{subject_type_i} (\textsection\ref{sec:pat}), 
which is used by the type switching mechanism to properly cast the subject if necessary. 
The library then introduces names \code{match_i} of type \code{target_type_i&} 
bound to properly casted subjects and available to the user in the right-hand 
side of the case clause in case of a successful match. The qualifiers applied to 
the type of \code{match_i} reflect the qualifiers applied to the type of subject 
\code{x_i}. Finally, the library generates the code that sequentially checks 
each pattern on properly casted subjects, making the clause's body conditional:

\begin{lstlisting}
if (p_1(match_1) && ... && p_N(match_N)) { /* body */ }
\end{lstlisting}

\noindent
When type switching is not involved, the generated code implements the naive 
backtracking strategy, which is known to be inefficient as it can produce 
redundant computations~\cite[\textsection 5]{Cardelli84}. More efficient 
algorithms for compiling pattern matching have been developed 
since~\cite{Augustsson85,Maranget92,Puel93,OPM01,Maranget08}. While these 
algorithms cover most of the typical kinds of patterns, they are not pattern agnostic 
as they make assumptions about semantics of concrete patterns. A library-based 
approach to pattern matching is agnostic of the semantics of any given 
user-defined pattern. The interesting research question in this context would 
be: what language support is required to be able to optimize first class 
patterns. While we do not address this question in its generality, our solution 
makes a small step in that direction.

The main advantage from using pattern matching in \emph{Mach7} comes from the fast type 
switching weaved into the \code{Match}-statement. It effectively skips case 
clauses that will definitely be rejected because their target types are not 
subtypes of subjects' dynamic types. This, of course, is only applicable to 
polymorphic arguments, for non-polymorphic arguments the matching is done 
naively with cascade of conditional statements.

\section{Related Work} %%%%%%%%%%%%%%%%%%%%%%%%%%%%%%%%%%%%%%%%%%%%%%%%%%%%%%%%%
\label{sec:rw}

Language support for pattern matching was first introduced for string 
manipulation in SNOBOL\cite{SNOBOL64}. SNOBOL4 had patterns as first-class data 
types providing operations of concatenation and alternation\cite{SNOBOL71}. The 
first reference to a modern pattern-matching constructs seen in functional 
languages is usually attributed to Burstall's work on structural 
induction\cite{Burstall69provingproperties}. Pattern matching was further 
developed by the functional programming community, most notably 
Hope\cite{BMS80}, ML\cite{ML90}, Miranda\cite{Miranda85} and 
Haskell\cite{Haskell98Book}. In the context of object-oriented programming, 
pattern matching has been first explored in Pizza\cite{Odersky97pizzainto} and 
Scala\cite{Scala2nd,EmirThesis}.

There are two main approaches to compiling pattern-matching code: the first is 
based on \emph{backtracking automata} and was introduced by Augustsson\cite{Augustsson85}, 
the second is based on \emph{decision trees} and was first described by 
Cardelli\cite{Cardelli84}, though he attributes the technique to Dave MacQueen 
and Gilles Kahn in their implementation of the Hope compiler \cite{BMS80}.
Backtracking approach usually generates smaller code, while decision tree 
approach produces faster code by ensuring that each primitive test is only 
performed once. With respect to matching a single expression our library 
approach follows the naive backtracking approach, however our match statement is 
based on a highly efficient type switching technique we developed\cite{TypeSwitch} 
that outperforms similar solutions based on decision trees or visitor design pattern.

Tom is a pattern-matching compiler that can be used together with Java, C or 
Eiffel to bring a common pattern matching and term rewriting syntax into the 
languages\cite{Moreau:2003}. It works as a preprocessor that transforms 
syntactic extensions into imperative code in the target language. Tom is quite 
transparent as to the concrete target language used and can potentially be 
extended to other target languages besides the three supported now.
Tom's  goals differ from ours in aiming to be a
tree-transformation language similar to Stratego/XT, XDuce and others. 
Tom's approach is prone to general problems of any preprocessor based 
solution\cite[\textsection 4.3]{SELL}. In particular, it is part of a dedicated toolchain.
Our library approach avoids that and lets us employ the C++ semantics within 
patterns: e.g. our patterns work directly on underlying user-defined data 
structures, avoiding abstraction penalties. The tight integration with 
the language semantics makes our patterns first-class citizens that can be 
composed and passed to other functions. 

Prop is another language extension that brings pattern matching into 
C++~\cite{Prop96}. This extension is not focused on pattern 
matching, but is intended for building high performance 
compiler and language transformation systems. It supports value-, variable-, 
wildcard-, constructor-, nested-, as-, type- and numerous sequence patterns.

Functional C\# is similar to our approach in trying to bring pattern matching 
into the C\# as a library\cite{FuncCSharp}. The approach uses lambda functions 
and chaining of method calls to create a structure that is then interpreted at 
run-time for the first successful predicate. The approach supports a form of 
active patterns, simple n+k patterns, list and tuple patterns as well as type 
patterns (without structural decomposition). 
However, an approach based on sequential type tests 
scales very poorly for match statements with more than two case clauses, making 
it unreasonably slower than the visitor design pattern~\cite{TypeSwitch}. Besides, the approach 
seems to be ill suited for tests involving nesting of patterns.

When the class hierarchy is fixed, one can design a pattern language that involves 
semantic notions represented by the hierarchy. Pirkelbauer devised a pattern 
language for Pivot capable of representing various entities in a C++ program using syntax very close to the C++ itself. 
Interestingly, the patterns were translated with a tool into a set of visitors 
implementing the underlying pattern-matching semantics\cite{PirkelbauerThesis}. 
Earlier, Cook et al used expression templates to implement a query language for 
Pivot's class hierarchy~\cite{iql04}. Our current work is the result of a series 
of experimental designs. The library approach was essential to provide 
relatively quick turnaround for experiments and for maintaining and improving 
performance for our applications.

\section{Conclusions and Future Work} %%%%%%%%%%%%%%%%%%%%%%%%%%%%%%%%%%%%%%%%%%
\label{sec:cc}

We present a pattern-matching library for \Cpp{} that provides fairly standard
pattern-matching facilities. Our solution is non-intrusive and can be 
retroactively applied. %to any polymorphic or tagged  
%class hierarchy. It also provides a uniform notation to these different 
%encodings of algebraic and extensible hierarchical data types in \Cpp{}.
The library provides efficient and expressive matching on multiple subjects and 
compares to multiple dispatch alternatives in terms of both time and space.

We generalize n+k patterns to arbitrary expressions by letting the user define 
the exact semantics of such patterns. Our approach is more general than traditional approaches 
as it does not require an
equational view of such patterns. It also avoids hardcoding the 
exact semantics of n+k patterns into the language. 

We used the library to rewrite existing code that relied heavily on the 
visitor design pattern.
Our pattern matching code was much shorter (both source and object code), 
simpler, easier to maintain, comprehend, and faster. 
This confirmed our view of the visitor pattern as a clever workaround,
rather than a good solution to a fundamental problem.
The library approach was essential 
for experimentation in the context of real programs and for delivering 
performance comparable with or superior to conventional techniques in the 
context of industrial compilers.

The work presented here continues our research on pattern matching for 
\Cpp{}~\cite{TS12}. We plan to further experiment with other kinds of patterns, 
including those defined by the user, look at the interaction of patterns with 
other facilities in the language and the standard library, and make views less 
ad hoc. For example, standard containers in \Cpp{} do not have the implicit 
recursive structure present in data types of functional languages, and viewing 
them as such with views would incur significant overheads. We will experiment 
with very general patterns as first-class citizens.

Our generalization of n+k patterns depends on the properties of types involved 
in the expression. This should let us experiment not only with generic 
functions, but also with their generic inversions in the form of solvers. As 
more \Cpp{}11 features become available in compilers it will also be interesting to 
look at how use of these features affects the ease of use, performance, 
readability, writability and debugging of the library and the user code that 
uses it.

%In the nearest future, we would like to make our library to be safe and efficient 
%in a multi-threaded environment. 

%From Morten Rhiger:
%New languages are often constructed by piling new features on top of an existing language's de?nition and by integrating these features in the existing language's implementation. However, it is a sign of expressiveness if new features can be implemented within an
%existing language without changing its de?nition.
%Short of macros, functional languages such as Haskell and Standard ML require new
%features to be expressed in terms of typed higher-order functions. We have demonstrated
%how to extend - or, in Guy Steele's terminology, to "grow" (Steele Jr., 1999) - Haskell
%with our own statically typed implementation of pattern matching and we have shown how
%to extend this framework with patterns not currently supported by Haskell.


\bibliographystyle{abbrv}
\bibliography{mlpatmat}
\end{document}
