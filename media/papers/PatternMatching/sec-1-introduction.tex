\section{Introduction} %%%%%%%%%%%%%%%%%%%%%%%%%%%%%%%%%%%%%%%%%%%%%%%%%%%%%%%%%
\label{sec:intro}

Conventional object-oriented programming practice suggests \emph{Visitor Design 
Pattern}~\cite{DesignPatterns} for principled traversal of structured data such 
as those manipulated by compilers, GUI frameworks, etc. However, our experience 
with various C++ front-ends and program analysis frameworks~\cite{Pivot09,Phoenix,Clang} 
have been rather unsatisfactory. We found visitors unsuitable to express our 
application logic, surprisingly hard to teach students, and slow. We found 
dynamic casts in many places, often nested. This suggested that users preferred 
shorter, cleaner, and more direct codes to visitors, even at the high cost in 
performance. Most of the dynamic cast usage mimicked the moral equivalent of 
pattern matching as found in modern functional programming languages.

Pattern matching is an abstraction mechanism popular in the functional
programming community, and recently adopted by several object-oriented
programming languages such as Scala~\cite{Scala2nd}, Fortress~\cite{RPS10},
and dialects of Java~\cite{Odersky97pizzainto,Liu03jmatch:iterable,HydroJ2003}, 
C++\cite{Prop96} and Eiffel~\cite{Moreau:2003}. It is relatively easy to add 
support for pattern matching in a new language. However, introducing one into an
existing language in widespread use is a challenge. The obvious utility of the 
feature is easily compromised by the need to fit into the language's syntax, 
semantics, and tool chains. A prototype implementation requires more effort than 
for an experimental language and is harder to get into use because mainstream 
users are unwilling to try non-portable, non-standard, and unoptimized features. 

To balance utility and effort we took the Semantically Enhanced Library Language 
(SELL) approach\cite{SELL} that advocates for extending a general-purpose 
programming language with a library, aided by a tool. With pattern matching, we 
(somewhat surprisingly) managed to avoid external tool support by relying on 
some pretty nasty macro hacking to provide a conventional and convenient syntax.
%interface to an efficient library implementation.
%By efficient, we mean about as fast as functional languages for closed cases and
%much better than code generated for visitor patterns by commercial optimizers 
%for open cases\cite{TypeSwitch}.

\subsection{Summary}

We present a functional-style pattern matching for C++ built as a library. Our solution:

\begin{compactitem}
\setlength{\itemsep}{0pt}
\setlength{\parskip}{0pt}
  \item Makes patterns a first-class citizens in the language.
  \item Is open as the user can introduce new kinds of patterns into the library.
  \item Is non-intrusive and can be retroactively applied to existing data types (\textsection\ref{sec:bnd}).
  \item Provides a unified syntax for various encodings of extensible 
        hierarchical datatypes in C++ (\textsection\ref{sec:unisyn}).
  \item Generalizes the controversial n+k patterns in a non-equational way by 
        leaving the choice of semantics to the user (\textsection\ref{sec:slv}). 
  \item Supports a limited form of views (\textsection\ref{sec:view}).
  \item Is simpler than conventional object-oriented or union-based alternatives.
%  \item Improves performance compared to alternatives in real applications~\cite{TypeSwitch}.
\end{compactitem}

\noindent
Our library solution sets a minimum threshold for performance, brevity, clarity 
and usefulness of a language solution for pattern matching in C++. It provides 
full functionality, so we can experiment with use of pattern matching in C++ and 
with language alternatives. A practical benefit of our solution is that it can 
be used right away given current support of C++11 (e.g., GCC and Microsoft C++)  
without additional tool support.

\subsection{Motivating Example}
\label{sec:xmpl}

While comparing generic programming facilities in functional and 
imperative languages (mainly Haskell and C++), Dos Reis and J\"arvi
present a Haskell sum functor\cite{DRJ05}:

\begin{lstlisting}[language=Haskell,keepspaces]
data Either a b = Left a | Right b
eitherLift :: (a -> c) -> (b -> d) -> Either a b -> Either c d
eitherLift f g (Left  x) = Left  (f x)
eitherLift f g (Right y) = Right (g y)
\end{lstlisting}

\noindent
The function \codehaskell{eitherLift} takes two functions and an 
object and depending on the actual type constructor the object was created with, 
calls first or second function on the embedded value, encoding the result 
correspondingly.

Its equivalent in C++ is not straightforward. Idiomatic, type-safe, handling of 
discriminated unions in C++ uses the \emph{Visitor Design Pattern}\cite{DesignPatterns}, 
which in this case amounts to 25 lines of ``boiler plate code'' plus 14 lines 
of the specific functionality. Using our 
pattern-matching facility, we get logically close to the original Haskell definition:

\begin{lstlisting}[keepspaces,columns=flexible]
template<class X,class Y> class Either{ virtual @$\sim$@Either(){}};
template<class X,class Y> class Left :Either<X,Y>{const X& x; Left (const X&);};
template<class X,class Y> class Right:Either<X,Y>{const Y& y; Right(const Y&);};
@\halfline@
template <class X,class Y,class S,class T>
const Either<S,T>* eitherLift(const Either<X,Y>& e, S f(X), T g(Y))
{
    Match(e) {
      Case(( Left<X,Y>), x) return new  Left<S,T>(f(x));
      Case((Right<X,Y>), y) return new Right<S,T>(g(y));
    } EndMatch
}
\end{lstlisting}

\noindent
To make the example fully functional, our library-based approach requires us to 
be explicit about binding of members in each matching position:

\begin{lstlisting}[keepspaces,columns=flexible]
template<class X,class Y> struct bindings<Left<X,Y>> {CM(0, Left<X,Y>::x);};
template<class X,class Y> struct bindings<Right<X,Y>>{CM(0,Right<X,Y>::y);};
\end{lstlisting}

\noindent
These two definitions can be avoided through the use of less elegant syntax
or through a compiler implementation.
They are introduced retroactively however, and made once for all 
possible instantiations using C++ partial template specialization. 

The syntax is provided without any external tool support. Instead we rely on a 
few C++11 features\cite{C++11}, template meta-programming, and macros. As shown 
in our companion paper~\cite{TypeSwitch}, it runs about as fast as its OCaml and 
Haskell equivalents and up to 80\% faster (depending on the usage scenario, 
compiler and underlying hardware) than a handcrafted C++ code using the visitor 
design pattern. The companion paper concentrated solely on the efficient 
implementation of an open type switch statement. This work extends the 
\code{Match}-statement introduced there to any kinds of patterns and 
concentrates on library composition, expressiveness and ease of use.

The rest of the paper is structured as following. Section~\ref{sec:bg} goes 
over some terminology and compares examples written with our library and in 
other languages. Section~\ref{sec:pm} goes over the syntax and semantics of our 
SELL independently of our host language -- C++. Section~\ref{sec:impl} discusses 
some implementation details that make our extension possible. 
Section~\ref{sec:rw} discusses related work, while section~\ref{sec:cc} 
concludes by discussing some future directions and possible improvements.
